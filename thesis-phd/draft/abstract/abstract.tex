{\fontfamily{lmr}\fontseries{b}\fontshape{n}\large\selectfont%
\begin{center}Abstract\end{center}}

Observations during the last three decades have confirmed that
the universe momentarily expands at an accelerated rate, which is 
assumed to be driven by dark energy whose origin remains unknown.  
The minimal manner of modelling dark energy is to include 
a positive cosmological constant in Einstein's equations, whose 
solution in vacuum is de Sitter space. This indicates that the 
large-scale kinematics of spacetime is approximated by the de 
Sitter group $SO(1,4)$ rather than the Poincar\'e 
group~$ISO(1,3)$. 

In this thesis we take this consideration to heart and conjecture 
that the group governing the local kinematics of physics is the 
de Sitter group, so that the amount to which it is a deformation 
of the Poincar\'e group depends pointwise on the value of 
a nonconstant cosmological function. With the objective of 
constructing such a framework we study the Cartan geometry in 
which the model Klein space is at each point a de Sitter space 
for which the combined set of pseudoradii forms a nonconstant 
function on spacetime. We find that the torsion receives 
a contribution that is not present for a cosmological constant. 
Invoking the theory of nonlinear realizations we extend the class 
of symmetries from the Lorentz group $SO(1,3)$ to the enclosing 
de Sitter group. Subsequently, we find that the geometric 
structure of teleparallel gravity--- a description for the 
gravitational interaction physically equivalent to general 
relativity--- is a nonlinear Riemann--Cartan geometry.

This finally inspires us to build on top of a de Sitter--Cartan 
geometry with a cosmological function a generalization of 
teleparallel gravity that is consistent with a kinematics locally 
regulated by the de Sitter group. The cosmological function is 
given its own dynamics and naturally emerges nonminimally coupled 
to the gravitational field in a manner akin to teleparallel dark 
energy models or scalar-tensor theories in general relativity.  
New in the theory here presented, the cosmological function gives 
rise to a kinematic contribution in the deviation equation for 
the world lines of adjacent free-falling particles. While having 
its own dynamics, dark energy manifests itself in the local 
kinematics of spacetime.

\vspace{2\baselineskip}
\noindent
\textbf{Keywords:} dark energy, de Sitter special relativity, 
teleparallel gravity, cosmological function, Cartan geometry.

\vspace{\baselineskip}
\noindent
\textbf{Subjects:} gravitation, cosmology, special relativity.
