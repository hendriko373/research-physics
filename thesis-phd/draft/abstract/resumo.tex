\begin{otherlanguage}{portuguese}
{\fontfamily{lmr}\fontseries{b}\fontshape{n}\large\selectfont%
\begin{center}Resumo\end{center}}

Observações realizadas nas últimas três décadas confirmaram que 
o universo se encontra em um estado de expansão acelerada. Essa 
ace\-le\-ração é atribuída à presença da chamada energia escura, 
cuja origem permanece desconhecida. A maneira mais simples de se 
modelar a energia escura consiste em introduzir uma constante 
cosmológica positiva nas equações de Einstein, cuja solução no 
vácuo é então dada pelo espaço de de Sitter. Isso, por sua vez, 
indica que a cinemática subjacente ao espaço-tempo deve ser 
aproximadamente governada pelo grupo de de Sitter $SO(1,4)$, 
e não pelo grupo de Poincaré $ISO(1,3)$.

Nesta tese, adotamos tal argumento como base para a conjectura de 
que o grupo que governa a cinemática local é o grupo de de 
Sitter, com o desvio em relação ao grupo de Poincaré dependendo 
ponto-a-ponto do valor de um termo cosmológico variável. Com 
o propósito de desenvolver tal formalismo, estudamos a geometria 
de Cartan na qual o espaço modelo de Klein é, em cada ponto, um 
espaço de de Sitter com o conjunto de pseudo-raios definindo uma 
função não-constante do espaço-tempo.  Encontramos que o tensor 
de torção nessa geo\-metria adquire uma contribuição que não está 
presente no caso de uma constante cosmológica.  Fazendo uso da 
teoria das realizações não-lineares, estendemos a classe de 
simetrias do grupo de Lorentz $SO(1,3)$ para o grupo de de 
Sitter. Em seguida, verificamos que a estrutura da gravitação 
telepa\-ra\-lela--- uma teoria gravitacional equivalente 
à relatividade geral--- é uma geometria de Riemann-Cartan não 
linear.

Inspirados nesse resultado, construímos uma generalização da 
gravitação teleparalela sobre uma geometria de de Sitter--Cartan 
com um termo cosmológico dado por uma função do espaço-tempo, 
a qual é consistente com uma cinemática localmente governada pelo 
grupo de de Sitter. A função cosmológica possui sua própria 
dinâmica e emerge naturalmente acoplada não-minimalmente ao campo 
gravitacional, analogamente ao que ocorre nos modelos 
telaparalelos de energia escura ou em teorias de gravitação 
escalares-tensoriais. Característica peculiar do modelo aqui 
desenvolvido, a função cosmológica fornece uma contribuição para 
o desvio geodésico de partículas adjacentes em queda livre.  
Embora tendo sua própria dinâmica, a energia escura manifesta-se 
como um efeito da cinemática local do espaço-tempo.

\end{otherlanguage}
