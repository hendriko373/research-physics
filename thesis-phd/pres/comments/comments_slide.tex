\documentclass[10pt,a4paper]{article}

\usepackage{lmodern}
\usepackage[utf8]{inputenc}
\usepackage[T1]{fontenc}

\usepackage[english]{babel}
\usepackage[left=3cm,right=3cm,top=3cm,bottom=3cm]{geometry}

\begin{document}

\noindent
\textbf{slide:} \underline{present-day accelerated expansion}

\begin{itemize}
  \item No a-priori assumptions on dynamical model / Cosmographic 
    analysis [Matt Visser]
  \item Concerning the Hubble law, one has to check the 
    convergence radius of the Taylor series, and whether finite 
    truncations give good approximations. Such problems can be 
    solved by considering different length parameters and/or 
    transformations of the redshift parameter.
\end{itemize}

\noindent
\textbf{slide:} \underline{Klein geometry}

\begin{itemize}
  \item Before we discuss the essentials of Cartan geometry, 
    which incorporates gravity, let us first look to Klein 
    geometry, which describes special relativity from a Lie 
    theoretic point of view.
  \item There are of course a lot of examples of homogeneous 
    spaces, but given our area of interest, the most relevant are 
    given by Minkowski and de Sitter space.
  \item Felix Klein (1872) understood that each homogeneous 
    geometry was characterized by a continuous group of 
    transformations which connect any two points in the space.
  \item The isotropy group of a point is the subgroup that leaves 
    that point fixed.
  \item This way one can see that the homogeneous space is given 
    by the space of cosets.
  \item Special relativities are characterized by Klein geometry, 
    and the shift to a Lie group description makes it manifest 
    that the space of special relativity becomes a de Sitter 
    space when the kinematic group is deformed to a de Sitter 
    group.
  \item But what kind of geometry do we need to incorporate 
    gravity in the description?
\end{itemize}
\end{document}
