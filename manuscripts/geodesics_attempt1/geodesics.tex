\documentclass[10pt]{article}

%Package declarations
%--------------------
\usepackage{setspace}
\setstretch{1.2}
\usepackage[a4paper]{geometry}
\geometry{left=3.5cm,right=3.5cm,top=3cm,bottom=3cm}

\usepackage[english]{babel}
\usepackage{color}

\usepackage{../Tex_files/preamble_one}

\numberwithin{equation}{section}

\author{}
\title{Geodesics: an attempt}
\date{}

\begin{document}
\maketitle

\section{Introduction}

Consider a homogeneous manifold $M = G/H = \{gH | g \in G\}$.  
Assume that $M$ is reductive, i.e.\ there exists a subspace 
$\mfrak{p} \simeq T_p M \in \mfrak{g}$ such that $\mfrak{g} = 
\mfrak{h} \oplus \mfrak{p}$ and $[\mfrak{h},\mfrak{p}] \subset 
\mfrak{p}$. Let $o = H$ be the origin of $M$. The left action of 
$G$ on itself induces a left action of $G$ on $M$, that is
%
\begin{displaymath}
	\lambda_g : M \rightarrow M : g_0 H \mapsto g g_0 H
\end{displaymath}
This action is transitive since the left action of $G$ on itself 
is transitive. Lastly, consider the set of right invariant vector 
fields $\{X_a\} (a = 1\ldots \dim G)$, which span the Lie algebra 
$\mfrak{g}$. They generate the left action of $G$ on itself, 
i.e., if $a_t$ is the integral curve of $\epsilon^a X_a$ through 
$e$, $a_t = \exp (t \epsilon^a X_a)$.

Then let us define the geodesics $\gamma(t)$ through $o \in M$ as 
the curves obtained by considering the left action $G$ on $o$,
%
\begin{equation}
	\gamma(t) = a_t(o) = \exp(t \epsilon^a X_a) o
\end{equation}

The tangent vector field to $\gamma(t)$ is left invariant under 
the 1-parameter group $a_t$, as we show now. Consider
this tangent vector field $\dot{\gamma}$
%
\begin{displaymath}
	\dot{\gamma}(t)f = \tfrac{d}{dt} f(\gamma(t))
\end{displaymath}
where $f$ is an arbitrary function along $\gamma(t)$. Left 
translating $\dot{\gamma}$ then results in
%
\begin{displaymath}
	(\lambda_{a_s \ast} \dot{\gamma}(t))f = \dot{\gamma}(t)(f\circ 
	\lambda_{a_s}) = \tfrac{d}{dt} f(a_{s+t}(o)) = 
	\dot{\gamma}(s+t)
\end{displaymath}

\section{An attempt: de Sitter spacetime}

Inspired by the introduction, we define a geodesic through $p$ as 
a curve which is of the form $\gamma(t) = a_t (p)$. This implies 
that parallel transport is defined through left translation, 
canonically given for de Sitter spacetime.

Let $Y=Y^\mu(x) \pd_\mu$ be a vector field defined on some open 
subset of $dS$. We translate it from $x$ to $x' = g(x)$ where $g 
= \exp(\epsilon^a \xi_a)$. For $f$ an arbitrary function on $dS$, 
one has
%
\begin{displaymath}
	Y^\mu(x)\pd_\mu \mapsto (\lambda_g Y)^\mu(x')\pd_\mu' f
	= Y^\nu(x)\frac{\pd x'^\mu}{\pd x^\nu} \pd_\mu' f
\end{displaymath}

We are interested in finding differential equations to be 
fulfilled by geodesics, hence we consider infinitesimal left 
translations. i.e.\
%
\begin{equation}
	(\lambda_g Y)^\mu(x + \Delta x) = Y^\lambda \pd_\lambda (x^\mu 
	+ \Delta x^\rho \delta^a_\rho \xi^\mu_a(x))
\end{equation}

A covariant derivative is then defined in the usual way, that is
%
\begin{displaymath}
	\nabla_\nu Y = \lim_{\Delta x^\nu} \tfrac{1}{\Delta x^\nu} 
	(Y^\mu(x + \Delta x) - (\lambda_g Y)^\mu(x + \Delta x))\pd_\mu
\end{displaymath}
%
which is found to be equal to
%
\begin{equation}
	\nabla_\nu Y = (\pd_\nu Y^\mu(x) - Y^\lambda \pd_\lambda 
	\xi^\mu_\nu(x)) \pd_\mu
\end{equation}
It is easily verified that the covariant derivative of $Y$ in the 
direction of $Z$ is given by
%
\begin{equation}
	\nabla_Z Y = Z^\nu \nabla_\nu Y = Z^\nu (\pd_\nu Y^\mu(x) - 
	Y^\lambda \pd_\lambda \xi^\mu_\nu(x)) \pd_\mu
\end{equation}
%

This leaves us in a position where we can define geodesics: 
\emph{geodesics are curves $\gamma(t)$ such that its tangent 
	vector field is parallel transported along the curve, under 
	left de Sitter translations}. Let $x^\mu(t)$ be the coordinates 
of $\gamma(t)$. Since the defining differential equation for a 
geodesic is $\nabla_{\dot{\gamma}(t)} \dot{\gamma}(t) \equiv 0$, 
we have in components that
%
\begin{equation}\label{eq:geod_ltrans_dS}
	\ddot{x}^\mu - \dot{x}^\nu \dot{x}^\lambda \pd_\lambda 
	\xi^\mu_a \delta^a_\nu = 0
\end{equation}

Before we compare this result with the usual geodesic equation, 
i.e.\ when parallel transport is defined w.r.t.\ the Levi-Civita 
connection, we make \eqref{eq:geod_ltrans_dS} explicit in 
stereographic coordinates. Therefore we calculate the derivative 
of $\xi^\mu_a = -\mathfrak{s} \delta^\mu_a - (4 l^2)^{-2} 
(2\eta_{a\rho} x^\rho x^\mu - \sigma^2 \delta^\mu_a)$ w.r.t.\ the 
coordinates, that is
%
\begin{displaymath}
	\pd_\lambda \xi^\mu_a = -\tfrac{1}{4l^2} (2\eta_{a\lambda}x^\mu 
	+ 2\eta_{a\rho}x^\rho\delta^\mu_\lambda - 
	2\eta_{\lambda\sigma}x^\sigma\delta^\mu_a)
\end{displaymath}
%
Substituting for \eqref{eq:geod_ltrans_dS}, then gives
%
\begin{equation}
	\ddot{x}^\mu + \tfrac{1}{4l^2} \dot{x}^\nu \dot{x}^\lambda 
	\eta_{\nu\lambda} x^\mu = 0
\end{equation}
According to our definition of parallel transport through (left) 
de Sitter translations, these are the differential equations in 
stereographic coordinates, whose solutions are geodesics on 
$(A)dS$.

\section{Comparison with the standard defintion}

To compare with the usual definition of geodesics in 
pseudo-Riemannian spacetimes, let us remind that the defining 
differential equations are given by
%
\begin{equation}
	\ddot{x}^\mu + \dot{x}^\nu \dot{x}^\lambda 
	\Gamma^\mu_{\lambda\nu} = 0
\end{equation}
where, for $(A)dS$ and using stereographic coordinates, the 
Levi-Civita connection is locally expressed as
%
\begin{displaymath}
	\Gamma^\mu_{\lambda\nu} = (\delta^\mu_\nu \delta^\sigma_\lambda
	+ \delta^\mu_\lambda \delta^\sigma_\nu - \eta^{\mu\sigma} 
	\eta_{\lambda\nu}) \pd_\sigma \ln |\Omega|
\end{displaymath}
and $\Omega(x) = (1+\tfrac{\mathfrak{s}\sigma^2}{4l^2})^{-1}$.

Some algebra then gives us the explicit differential equations,
%
\begin{equation}
	\ddot{x}^\mu - \tfrac{\mathfrak{s}\Omega}{4l^2} (4\dot{x}^\mu 
	\dot{x}^\sigma \eta_{\sigma\kappa} x^\kappa - \dot{x}^\nu 
	\dot{x}^\lambda \eta_{\lambda\nu} x^\mu) = 0
\end{equation}

\end{document}
