\documentclass[10pt]{article}

%Package declarations
%--------------------
\usepackage[tbtags]{amsmath}
\usepackage{amsfonts}
\usepackage{amssymb}

\usepackage[a4paper]{geometry}
\geometry{left=3cm,right=3cm,top=3cm,bottom=3cm}

\usepackage[english]{babel}
\usepackage{color}

%User defined commands
%---------------------
\newcommand{\mcal}{\mathcal}
\newcommand{\mfrak}{\mathfrak}
\newcommand{\mbb}{\mathbb}
\newcommand{\mrm}{\mathrm}
\newcommand{\pd}{\partial}
\newcommand{\sfr}{\mathfrak{s}}
\newcommand{\blankline}{\vspace{\baselineskip}}
\newcommand{\spc}{\ }
\newcommand{\red}{\color{red}}
\newcommand{\blu}{\color{blue}}

\title{De Sitter relativity and stereographic coordinates: 
	introduction}
\author{}
\date{}

\begin{document}
\maketitle

\section{de Sitter space time: stereographic coordinates}

Let $E^{1,4}$ be the five-dimensional flat spacetime with 
Lorentzian signature.
De Sitter spacetime is the hypersurface defined by
\begin{equation}\label{eq:dS_embedded}
	\eta_{AB}\chi^A\chi^B = \eta_{\mu\nu}\chi^{\mu}\chi^{\nu} + 
	\sfr (\chi^4)^2 = \sfr l^2
\end{equation}
where $A=0\ldots4$ and $\mu=0\ldots3$.

In fact, this hypersurface both describes a de Sitter spacetime 
$dS$ as well as an anti de Sitter spacetime $AdS$, depending on 
the value of $\sfr \equiv \eta_{44}$. Table \ref{tab:signConv} 
gives a resum\'e for these values when different sign conventions 
are used. In the following we will denote these surfaces as a de 
Sitter spacetime, however the treatment will be equally valid for 
anti de Sitter spacetimes as well through the free parameter 
$\sfr$.
%
\begin{table}[h]
	\centering
	\begin{tabular}{c|c c}
	& $\sfr = -1$ & $\sfr = +1$ \\
	\hline
	$\eta_{AB} = (+,-)$ & $dS$ & $AdS$ \\
	$\eta_{AB} = (-,+)$ & $AdS$ & $dS$
	\end{tabular}
	\caption{Sign convention resum\'e}
	\label{tab:signConv}
\end{table}
  
A useful coordinate system parametrizing $dS(1,4)$ is given by
stereographic coordinates. This chart is defined in the following 
way. Consider
the north pole $(N)$ at $(0,0,0,0,l)$ and the hyperplane $\chi^4 
= -l$, which
contains the south pole of $dS(1,4)$. We project a point $P$ of 
de Sitter
spacetime onto a point $p$ of the hyperplane by taking the line 
through $N$ and
$P$ which gives the projected point $p$, there where the line 
meets the
hyperplane.

Since $N$, $P$ and $p$ always lie on a straight line we have
\begin{displaymath}
\frac{\chi(P)-\chi(N)}{\chi(p)-\chi(N)} = n
\end{displaymath}
where $n$ will depend on the considered projection line. The 
projected point $p$
lies in the subspace $E^{(1,3)}$ (i.e. $\chi^4=-l$), with 
coordinates $x^\mu$
where $(\mu=0\ldots 3)$. Hence, we have
\begin{displaymath}
\frac{\chi^\mu-0}{x^\mu-0} = n,\quad \frac{\chi^4-l}{-l-l}=n
\end{displaymath}
such that we find
\begin{align}
\chi^\mu &= n x^\mu \\
\chi^4 &= l(1-2n) \label{eq:chi4_stereo}
\end{align}

Remembering the fact that we consider points $P$ on the $dS(1,4)$ 
we have to add
the defining equation \eqref{eq:dS_embedded} for this 
hypersurface, namely
\begin{equation}\label{eq:chi4_dS}
(\chi^{4})^{2} = l^2 - \sfr n^2\eta_{\mu\nu}x^\mu x^\nu
\end{equation}
Substituting \eqref{eq:chi4_stereo} and \eqref{eq:chi4_dS} for 
$\chi^4$ we find
a defining equation for $n$ in function of the other four 
coordinates, i.e.\footnote{We define $\sigma^2=\eta_{\mu\nu}x^\mu 
	x^\nu$.}
\begin{equation}
n^2(4l^2 + \sfr \sigma^2)-4nl^2 = 0
\end{equation}
which has two real solutions, given by $n=0$ (for the projection 
of the north pole) and
\begin{equation}
	n = \left(1 + \sfr\frac{\sigma^2}{4l^2}\right)^{-1} =: 
	\Omega(x)
\end{equation}
which is the factor used in projecting all other points.

The stereographic coordinates $x^\mu$ are then given by
\begin{align}
	\chi^\mu &= \Omega(x)x^\mu \\
	\chi^4 &= -l\Omega(x)\left(1-\sfr\frac{\sigma^2}{4l^2}\right)
\end{align}
which is defined for all points on $dS(1,4)$ but the north pole.

The inverse transformations are easily found to be
\begin{equation}
	x^\mu = \Omega^{-1}\chi^\mu \quad \mrm{with} \quad \Omega = 
	-\frac{1}{2} \left( \frac{\chi^4}{l} - 1 \right)
\end{equation}

\section{Geometric quantities}

In this section some objects will be derived, describing the 
geometric structure of de Sitter spacetime.

\paragraph{Metric tensor}
First we calculate the line element on $dS$ in terms of 
stereographic coordinates.  This line element is induced from the 
flat metric $\eta_{AB}$ on $E^{(1,4)}$, hence
\begin{equation}
	ds^2 = \eta_{AB}d\chi^A d\chi^B \quad \text{with} \quad
	\chi^4 = \pm(l^2 -\sfr \eta_{\mu\nu} \chi^\mu 
	\chi^\nu)^{\frac{1}{2}}
\end{equation}

Our task now is to calculate the differentials appearing in this 
line element in terms
of the stereographic coordinates $x^\mu$. It will be useful to 
use the following
relations between $\Omega$, $\sigma$ and its differentials
\begin{displaymath}
	\sigma^2 = \sfr 4l^2 \frac{1 - \Omega}{\Omega}\, ,\quad
	d(\sigma^2)= -\sfr\frac{4l^2}{\Omega^2}d\Omega
\end{displaymath}
substituting the Cartesian coordinates for the stereographic 
ones, we find
\begin{align} \eta_{\mu\nu}d\chi^\nu d\chi^\nu
	&=\eta_{\mu\nu}d(\Omega(x)x^\mu)d(\Omega(x)x^\nu) \\
	&=d\Omega^2 \sigma^2 + \Omega d\Omega d(\sigma)^2 + \Omega^2 
	\sigma^2
\end{align}
and
\begin{equation}
	(d\chi^4)^2 = d(l - 2l\Omega)^2 = 4l^2 d\Omega^2
\end{equation}

Combining these results we find that the line element of de 
Sitter spacetime
\begin{equation}
	ds^2 = \Omega^2(x)\eta_{\mu\nu}dx^\mu dx^\nu
	\label{eq:dS_le_stereo}
\end{equation}
which shows that the Sitter spacetime is conformally flat.
\blankline

Although this line element is conformally flat, de Sitter 
spacetime does not have the
same causal structure as Minkowski spacetime. [Question: What is 
the explanation for
this apparent paradox?] [Possible answer: the transformation is 
singular at $\sigma^2
= 4l^2$. In fact this three-dimensional hypersurface is the image 
of past and future
null infinity, since for these points $\Omega$ goes to infinity 
and we have that the $\chi^\mu$ go to infinity as 
$\mcal{O}(\Omega)$. From the equation of $dS$ it follows that 
$\chi^4$ goes to infinity, at the same rate. For these points $l$ 
is essentially zero, such that its tangent space is the lightcone 
of the embedding Minkowski space. Hence, we are considering 
null-like infinity, which turns out to be three-dimensional. Note 
also that the transformation does not make a difference between 
such a point at infinity and its spacetime inversion, such that 
these two points are represented by the same stereographic 
coordinate.]

\paragraph{Christoffel symbol}
Given the metric tensor, one obtains the corresponding 
Christoffel symbol
\begin{displaymath}
	\Gamma^{\rho}_{\spc\mu\nu} = \frac{1}{2}g^{\rho\sigma} \left( 
		\pd_\mu g_{\nu\sigma} + \pd_\nu g_{\mu\sigma} - \pd_\sigma 
		g_\mu\nu \right)
\end{displaymath}
A rather short calculation shows that the symbol is given by
\begin{equation}
	\Gamma^{\rho}_{\spc\mu\nu} = \left( \delta_\nu^\rho 
		\delta_\mu^\sigma + \delta_\mu^\rho \delta_\nu^\sigma - 
		\eta^{\rho\sigma} \eta_{\mu\nu} \right) \pd_\sigma \ln 
	|\Omega(x)|
\end{equation}

\paragraph{Riemann tensor}
The Riemann curvature tensor can be defined through the 
Christoffel symbol as
\begin{displaymath}
	R^\mu_{\spc\nu\rho\sigma} = \pd_\rho 
	\Gamma^{\mu}_{\spc\nu\sigma} - \pd_\sigma 
	\Gamma^{\mu}_{\spc\nu\rho} + \Gamma^{\mu}_{\spc\alpha\rho} 
	\Gamma^{\alpha}_{\spc\nu\sigma} + 
	\Gamma^{\mu}_{\spc\alpha\sigma} \Gamma^{\alpha}_{\spc\nu\rho}
\end{displaymath}
After a somewhat lengthier calculation one finds the familier 
curvature tensor for a maximally symmetric space
\begin{equation}
	R^\mu_{\spc\nu\rho\sigma} = \frac{\mfrak{s}}{l^2} \left( 
		\delta_\rho^\mu g_{\sigma\nu} - \delta_\sigma^\mu g_{\nu\rho} 
	\right)
\end{equation}

\paragraph{Ricci tensor}
Contracting the first with the third index of the Riemann tensor, 
one defines the Ricci tensor
\begin{equation}
	R_{\mu\nu} = \frac{3\mfrak{s}}{l^2}g_{\mu\nu}
\end{equation}

\paragraph{Ricci scalar}
Finally, upon contracting the two indices of the Ricci tensor, 
the Ricci scalar is found, namely
\begin{equation}
	R = \frac{12 \mfrak{s}}{l^2}
\end{equation}



\section{de Sitter space as a homogeneous space}

De Sitter spacetime can be constructed as a pseudo-Riemannian 
symmetric space $dS = SO(1,4)/\mcal{L}$.\footnote{[{\blu Doubt}] 
	The involution defining a Cartan decomposition is probably the 
	spacetime inversion - algebraically the generators $L_{\mu4}$ 
	span the eigenspace corresponding to -1.} This directly implies 
it is a homogeneous space, transitive under the elements 
generated by the Lie algebra 
$\mfrak{p}=\mfrak{so}(1,4)\bmod\mfrak{so}(1,3)$.  It is also a 
principal bundle $P(\mcal{L},dS)$ where the fibers are the 
Lorentz rotations.

This identification between de Sitter spacetime and its Lie group 
of isometries gives us some interesting tools to investigate its 
structure. For example, it is possible to derive geometric 
properties of the de Sitter spacetime from the Cartan-Killing 
form defined on the de Sitter Lie algebra. It also gives one a 
natural way to find the limiting spacetimes when varying the 
cosmological constant. Indeed, through an In\"on\"u-Wigner 
contraction of the Lie algebra the corresponding 
pseudo-Riemannian spacetime can be constructed.

\paragraph{de Sitter spacetime - finite $\Lambda$}
Since de Sitter spacetime can be identified with the homogeneous 
space
\begin{equation}
	dS(1,4) \equiv \frac{SO(1,4)}{\mcal{L}}
\end{equation}
it is transitive under the elements generated by (the basis 
elements of $\mfrak{so}(1,4) \bmod \mfrak{so}(1,3)$)
\begin{equation}
	L_{\mu 4} = \sfr l P_\mu + \frac{1}{4l} K_\mu
\end{equation}

\paragraph{Minkowski spacetime - zero $\Lambda$} $(l\rightarrow 
\infty)$
Performing an adequate In\"on\"u-Wigner contraction of the de 
Sitter group (through its algebra) one finds the Poincar\'e group 
$\mcal{P}$, which is the semidirect product of the Lorentz group 
and ordinary translations. Hence, the de Sitter spacetime has 
been contracted to
\begin{equation}
	M \equiv \frac{\mcal{L}\rtimes \mcal{T}}{\mcal{L}}
\end{equation}
a homogeneous space transitive under ordinary translations 
(generated by $P_\mu$), Minkowski spacetime.

\paragraph{Conic spacetime - infinite $\Lambda$} $(l\rightarrow 
0)$
In this limit one contracts $SO(4,1)$ to the conformal Poincar\'e 
group $\mcal{Q}$, the semidirect product of the Lorentz group and 
the special conformal group. The de Sitter space then becomes a 
(homogeneous) cone-spacetime
\begin{equation}
	N \equiv \frac{\mcal{L} \rtimes \mcal{C}}{\mcal{L}}
\end{equation}
which is transitive under special conformal transformations 
(generated by $K_\mu$).

\paragraph{{\blu Remark}}
Transitivity on spacetime is not longer obtained through ordinary 
translations, but through a combination of translations and 
special conformal transformations (in stereographic coordinates).  
This will have deep implications. A direct consequence is that 
the notion of differentiation will change; indeed, ordinary 
differentiation of functions on manifolds is based on an 
underlying movement of translations, i.e.
\begin{equation}
	\partial_\mu f(x^\mu) \equiv \lim_{\varepsilon \rightarrow 0}
	\frac{1}{\varepsilon}\left[f(x^\mu + 
		\delta^\mu_{(\nu)}\varepsilon^\nu) - f(x^\mu) \right]
\end{equation}

The tangent space at a point $p$ of a manifold $M$ is the space 
spanned by a maximal set of linearly independent differential 
operators at the given point.  Usually these diferential 
operators are the ordinary derivatives, the generators of 
translations. If the manifold $M$ is pseudo-Riemannian, the 
tangent space obtained in this way will be flat. [Through what 
argument? Work out.  Probably it has to do with a submanifold 
generated by translations in the embedding manifold, which has 
the same signature as M.]

However, we could span the tangent space with a different set of 
linearly independent differential operators $\hat{\pd}$
\begin{equation}
	\hat{\pd}_\mu f(x^\mu)\equiv \lim_{\varepsilon \rightarrow 0} 
	\frac{1}{\varepsilon}\left[f(x^\mu + 
		\xi^\mu_{(\nu)}(x)\varepsilon^\nu) - f(x^\mu) \right]
\end{equation}
which would be a space transitive under the elements generated by 
$\hat{\pd}$.

If one takes the differential operators to be the de Sitter 
translations $L_{\mu4}$, the tangent space at some manifold $M$, 
would be given by de Sitter spacetime $dS$. Note that a de Sitter 
spacetime, would be locally de Sitter - as a Minkowski spacetime 
is locally Minkowski. It gives a very precise meaning on how the 
equivalence principle changes in a de Sitter general relativity - 
and how it is the only change being made. General relativity 
still presumes that spacetime is a four-dimensional manifold with 
Lorentzian signature, where the metric is a solution to 
Einstein's equations. However, the modified equivalence principle 
states that one always can find a coordinate system in which the 
laws of physics are locally given by the laws of de Sitter 
special relativity.  Indeed, the tangent space (the local 
structure) at a spacetime point is a de Sitter spacetime. \hfill 
{\blu $\blacksquare$}

\section{Spacetime inversion: duality between $M$ and $N$}

Let us begin with introducing the \emph{spacetime inversion} on 
stereographic coordinates
%
\begin{equation}
	\label{eq:st_inversion}
	x^\mu \rightarrow \bar{x}^\mu \equiv -\frac{x^\mu}{\sigma^2}
\end{equation}
%
where as usual, $\sigma^2 = \eta_{\mu\nu}x^\mu x^\nu$. A first 
interesting question that may be posed is how this quantity 
behaves under the inversion \eqref{eq:st_inversion}. It is 
helpful to note that the notation $\eta_{\mu\nu}x^\mu x^\nu$ was 
introduced as a shorthand for $-(x^0)^2 + (x^i)^2$, rather than 
having a geometric meaning. For example, $\sigma^2$ in general is 
\emph{not} the invariant length of $x^\mu$. Therefore, it is 
plausible to assume the following transformation behaviour of 
$\sigma^2$ under the spacetime inversion, that is
%
\begin{equation}
	\sigma^2 \rightarrow \bar{\sigma}^2 = \eta_{\mu\nu}\bar{x}^\mu 
	\bar{x}^\nu = \frac{\eta_{\mu\nu}x^\mu x^\nu}{\sigma^4} = 
	\frac{1}{\sigma^2}
\end{equation}
%
[Hence, the $\eta_{\mu\nu}$ in $\sigma^2$ should \emph{not} be 
thought of as the Minkowski tensor and considered constant under 
the inversion---the assumption is important for finding the 
inverse of the inversion and the consequences thereof].
The inverse transformation of \eqref{eq:st_inversion} is then 
easily found to be
%
\begin{equation}
	\label{eq:inv_st_inversion}
	\bar{x}^\mu \rightarrow x^\mu = 
	-\frac{\bar{x}^\mu}{\bar{\sigma}^2}
\end{equation}
%
where, as discussed, $\bar{\sigma}^2 = \eta_{\mu\nu}\bar{x}^\mu 
\bar{x}^\nu$.

Given these facts, it is possible to find how the vector fields 
$P_\mu$ and $K_\mu$ transform. More specifically, invoking the 
chain rule for partial differentiation and 
\eqref{eq:inv_st_inversion} one calculates,
%
\begin{displaymath}
	\begin{split}
		\bar{P}_\mu &= \frac{\pd}{\pd\bar{x}^\mu}
		= \frac{\pd x^\mu}{\pd\bar{x}^\mu} \frac{\pd}{\pd x^\nu}\\
		&= -(\bar{\sigma}^{-2} \delta^\nu_\mu - \bar{x}^\nu
			\bar{\sigma}^4 2\eta_{\rho\sigma}\delta^\rho_\mu 
			\bar{x}^\sigma) \frac{\pd}{\pd x^\nu} \\
		&= (2\eta_{\mu\rho}x^\rho x^\nu - \sigma^2
			\delta^\nu_\mu) \frac{\pd}{\pd x^\nu}
		= K_\mu
	\end{split}
\end{displaymath}
and
\begin{displaymath}
	\begin{split}
		\bar{K}_\mu &= (2\eta_{\mu\rho} \bar{x}^\rho
			\bar{x}^\lambda - \bar{\sigma}^2 \delta^\lambda_\mu)
			\bar{\pd}_{\lambda} \\
		&= (2\eta_{\mu\rho}\frac{x^\rho x^\lambda}{\sigma^4}
			-\sigma^{-2}\delta^\lambda_\mu) (2\eta_{\lambda\sigma} 
			x^\sigma x^\nu -\sigma^2 \delta^\nu_\lambda) \pd_{\nu} \\
		&= (4\eta_{\mu\rho}x^\rho x^\nu \sigma^{-2}
			- 2\eta_{\mu\rho} x^\rho x^\nu \sigma^{-2} - 
			2\eta_{\mu\sigma} x^\sigma x^\nu \sigma^{-2} + 
			\delta^\nu_\mu) \pd_{\nu} \\
		&= \frac{\pd}{\pd x^\mu} = P_\mu
	\end{split}
\end{displaymath}
%
Furthermore, the vector fields $L_{\mu\nu}$ generating Lorentz 
transformations transform as
%
\begin{displaymath}
	\begin{split}
		\bar{L}_{\mu\nu} &= \eta_{\mu\lambda}\bar{x}^\lambda
			\bar{\pd}_\nu - \eta_{\nu\lambda}\bar{x}^\lambda 
			\bar{\pd}_\mu \\
		&= - \eta_{\mu\lambda}x^\lambda\sigma^{-2}
			(2\eta_{\nu\sigma}x^\sigma x\rho - \sigma^2 
			\delta^\rho_\nu) \pd_\nu - [\mu \leftrightarrow \nu] \\
		&= \eta_{\mu\lambda} x^\lambda \pd_\nu - \eta_{\nu\lambda}
			x^\lambda \pd_\mu = L_{\mu\nu}
	\end{split}
\end{displaymath}

One concludes that translations and special conformal 
transformations are interchanged under a spacetime inversion, 
while Lorentz rotations are unchanged, that is
%
\begin{equation}
	\label{eq:duality_PK}
	\begin{split}
		P_\mu &\rightarrow \bar{P}_\mu = K_\mu \\
		K_\mu &\rightarrow \bar{K}_\mu = P_\mu \\
		L_{\mu\nu} &\rightarrow \bar{L}_{\mu\nu} = L_{\mu\nu}
	\end{split}
\end{equation}
%
Because of the above alluded homogeneous character of $M$ and 
$N$, these transformation properties directly imply the 
interchange of $M$ and $N$ under a spacetime inversion 
\eqref{eq:inv_st_inversion}. In this sense, they are said to be 
dual to eachother and the duality relation is given through 
\eqref{eq:duality_PK}.

Given this duality we might wonder what happens with the 
Minkowski line element under a spacetime inversion. Since $M$ 
goes into $N$ under this transformation, the resulting line 
element may be interpreted as the invariant line element on $N$.  
Therefore,
%
\begin{displaymath}
	dx^\mu \rightarrow d\bar{x}^\mu = d(-x^\mu \sigma^{-2})
		= -\frac{dx^\mu}{\sigma^2} + \frac{x^\mu}{\sigma^4} 
		2\eta_{\mu\nu}dx^\mu x^\nu
\end{displaymath}
%
which we use for calculating
%
\begin{align*}
	\eta_{\mu\nu} d\bar{x}^\mu d\bar{x}^\nu
	&= \eta_{\mu\nu}(-\sigma^{-2}dx^\mu + 2\sigma^{-4}x^\mu 
	\eta_{\rho\sigma} dx^\rho x^\sigma) (-\sigma^{-2}dx^\nu + 
	2\sigma^{-4}x^\nu \eta_{\alpha\beta} dx^\alpha x^\beta) \\
	\begin{split}
		&=\eta_{\mu\nu}\sigma^{-4} dx^\mu dx^\nu - 2\eta_{\mu\nu} 
		\sigma^{-6}dx^\mu x^\nu \eta_{\alpha\beta} dx^\alpha 
		x^\beta\\ &\qquad
		-2\eta_{\mu\nu} \sigma^{-6}dx^\nu x^\mu \eta_{\rho\sigma} 
		dx^\rho x^\sigma +4\sigma^{-8} x^\mu x^\nu \eta_{\rho\sigma} 
		\eta_{\alpha\beta} dx^\rho x^\sigma dx^\alpha x^\beta
	\end{split}\\
	&= \eta_{\mu\nu}\sigma^{-4} dx^\mu x^\nu
\end{align*}
%
We thus have the following behaviour of the line element $ds^2$ 
under a spacetime inversion,
%
\begin{equation}
	ds^2 = \eta_{\mu\nu} dx^\mu dx^\nu \rightarrow d\bar{s}^2 = 
	\eta_{\mu\nu} d\bar{x}^\mu d\bar{x}^\nu =	
	\frac{\eta_{\mu\nu}}{\sigma^4} dx^\mu dx^\nu
\end{equation}
A simpler [and equivalent?] derivation would be $\sigma^2 
\rightarrow \sigma^{-2} = \sigma^{-4}\eta_{\mu\nu}x^\mu x^\nu$.
Note that we did not consider any transformation on the Minkowski 
matrix $\eta_{\mu\nu}$. Of course, if one would consider the 
spacetime inversion as a diffeomorphism under which the Minkowski 
matrix transforms as a tensor, the line element would be a 
scalar.

Next we show that the above introduced matrix $\eta_{\mu\nu} 
\sigma^{-4}$ is invariant under special conformal 
transformations. A special conformal transformation (SCT) is 
composed of an inversion, followed by a translation and another 
inversion. Therefore, we first calculate the transformation 
behaviour of $\eta_{\mu\nu}$ under a spacetime inversion. To make 
it clear: now we do consider it as a tensor, i.e.\ the inversion 
acts once for each index. The reason is that the object 
$\eta_{\mu\nu}\sigma^{-4}$ is to be interpreted as the 
\emph{metric tensor} on $N$. As will be shown, it is invariant 
under SCTs (and LTs) and therefore can be considered defining the 
conformal Poincar\'e group as the group leaving
$\eta_{\mu\nu}\sigma^{-4}$ invariant. Because a second rank 
tensor transforms as
%
\begin{displaymath}
	\eta_{\mu\nu} \rightarrow \frac{\pd x^\alpha}{\pd\bar{x}^\mu} 
	\frac{\pd x^\beta}{\pd\bar{x}^\nu} \eta_{\alpha\beta}
\end{displaymath}
and since $\pd x^\alpha/\pd \bar{x}^\mu = 2\eta_{\mu\rho} x^\rho 
x^\alpha - \sigma^2 \delta^\alpha_\mu$, one finds
%
\begin{align*}
	\bar{\eta}_{\mu\nu} &= (2\eta_{\mu\sigma} x^\sigma x^\alpha - 
	\sigma^2 \delta^\alpha_\mu) (2\eta_{\nu\rho} x^\rho x^\beta - 
	\sigma^2 \delta^\beta_\nu) \eta_{\alpha\beta} \\
	&= 4\eta_{\mu\sigma} \eta_{\nu\rho} x^\sigma x^\rho - 2 
	\sigma^2 \eta_{\mu\sigma} \eta_{\alpha\nu} x^\sigma x^\alpha - 
	2 \sigma^2 \eta_{\mu\beta} \eta_{\nu\rho} x^\beta x^\rho + 
	\sigma^4 \eta_{\mu\nu}
\end{align*}
Interpreted as a second rank tensor, $\eta_{\mu\nu}$ thus 
transforms under a spacetime inversion according to
%
\begin{equation}
	\eta_{\mu\nu} \rightarrow \bar{\eta}_{\mu\nu} = \sigma^4 
	\eta_{\mu\nu}
\end{equation}

Before calculating the behaviour of $\eta_{\mu\nu}\sigma^{-4}$ 
under special conformal transformations, let us remind that 
$\sigma^2$ does not behave as a scalar under spacetime 
inversions. By the argument given in the beginning of the 
section, it is mapped into $\sigma^{-2}$. Applying the first 
inversion, we have
%
\begin{displaymath}
	\frac{\eta_{\mu\nu}}{\sigma^4} \rightarrow \sigma^8 
	\eta_{\mu\nu}
\end{displaymath}
The second operation in the special conformal transformation is 
an ordinary translation. {\blu Since the first inversion mapped 
	the conic space into Minkowski space, the object $\sigma^8 
	\eta_{\mu\nu}$ is a scalar under translations.} Therefore, we 
can directly go to the third part of the special conformal 
transformation: another inversion. This maps us again into the 
conic spacetime and the metric tensor is mapped accordingly as
%
\begin{displaymath}
	\sigma^8 \eta_{\mu\nu} \rightarrow 
	\frac{\eta_{\mu\nu}}{\sigma^4}
\end{displaymath}
This shows that $\sigma^{-4} \eta_{\mu\nu}$ is invariant under 
arbitrary special conformal transformations. [The above 
discussion makes us guess that its invariance on $N$ is in some 
sence ``dual'' to the invariance of the Minkowski metric on the 
dual space $M$.]






%\bibliographystyle{plain}
%\bibliography{$RS/References/All.bib}
\end{document}
