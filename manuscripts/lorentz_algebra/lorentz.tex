\documentclass[10pt]{amsart}

\usepackage[english]{babel}
\usepackage{amsmath,amsfonts,amssymb,amsthm}
\usepackage[all]{xy}
\usepackage{setspace}
\setstretch{1.15}
\usepackage[a4paper]{geometry}
\geometry{left=3cm,right=3cm,top=3cm,bottom=3cm}
\usepackage{cite}
\usepackage{bbold} %Symbol identity matrix

\usepackage{thmtools}
\declaretheorem[numberwithin=section]{definition}
\declaretheorem[numberwithin=section]{proposition}
\let\proof\relax
\declaretheorem[style=remark,numbered=no,qed=\qedsymbol]{proof}

\numberwithin{equation}{section}
\usepackage{color}

%\usepackage{/Volumes/DATA/research/drafts/Tex_files/preamble_one}
\usepackage{../Tex_files/preamble_one}

\title{The special orthogonal algebras and groups}
\author{Hendrik}
\date{\today}

\begin{document}

\begin{abstract}
	We bundle our conventions on the special orthogonal groups and 
	algebras and on some of its special cases, e.g.~the Lorentz 
	algebra.
\end{abstract}

\maketitle

\section{The general case}

\subsection{Definition}

The special orthogonal group $SO(p,q)$ can be defined as a set of 
matrices acting on $\mathbb{R}^{p,q}$ that leave invariant the 
metric tensor $\eta$ of signature $(p,q)$. In this defining $n = 
p+q$-dimensional representation, elements $\Lambda$ of $SO(p,q)$ 
are the non-singular matrices that satisfy 
%
\begin{equation}\label{eq:def_so(p,q)}
	\eta_{cd}\Lambda^c_{~a}\Lambda^d_{~b} = \eta_{ab}~;
	\quad \det\Lambda = 1~.
\end{equation}
Only the component connected to the identity element will be 
considered. This implies that any element of the group can be 
obtained by exponentiation of some element of the Lie algebra 
$\mathfrak{so}(p,q)$, the tangent vectors at the identity.  
Consider therefore the elements infinitesimally close to this 
identity, that is $\Lambda^a_{~b} = \delta^a_b + \omega^a_{~b}$.  
Since $\Lambda^a_{~b}$ satisfies \eqref{eq:def_so(p,q)}, one 
finds for the elements $\omega_{ab} \in \mathfrak{so}(p,q)$ that 
$\omega_{ab} = -\omega_{ba}$.\footnote{Indices are lowered or 
	raised by contraction with $\eta$.} This implies that the 
algebra is of dimension $n(n-1)/2$.

Let us denote by $M_{ab}$ the set of generators of 
$\mathfrak{so}(p,q)$ in some representation, i.e.~they act as 
matrices $[M_{ab}]^{A}_{~B}$ on some $d$-dimensional vector 
space.\footnote{The indices $ab$ a-priori are not representation 
	indices; rather they enumerate the generators of the algebra.} 
Any element of the special orthogonal group can be obtained by 
exponentiation of some element of the algebra. For a given 
representation $D$ of the group this means
%
\begin{equation}
	D(\omega) = \exp(\tfrac{i}{2} \omega^{ab} M_{ab}) \equiv 
	\exp(\tfrac{i}{2}\omega\cdot M)~,
\end{equation}
where one half has been introduced because of double counting 
(antisymmetry in $ab$). The numbers $\omega_{ab}$ are the group 
parameters and as mentioned before, the generators are the 
elements tangent to the identity
%
\begin{displaymath}
	M_{ab} = -i \left.\frac{\pd}{\pd \omega^{ab}}\right|_0 
	D(\omega)~.
\end{displaymath}
An element of $\mathfrak{so}(p,q)$ is then given by 
$\tfrac{i}{2}\omega\cdot M$ for some choice of $\omega$.

As an example, the $n$-dimensional or defining representation is 
recovered if the generators are chosen to be $[M_{cd}]^a_{~b} = 
-i(\eta_{db} \delta^a_c - \eta_{cb} \delta^a_d)$. A generic 
element of the algebra is denoted by $\tfrac{i}{2} 
\omega^{cd}[M_{cd}]^a_{~b} = \omega^a_{~b}$, of course recovering 
our previous result.  Furthermore, it explains the double use of 
the letter $\omega$ for both group parameters and elements of the 
Lie algebra in the defining representation. 

\subsection{Commutation relations}

Characteristic for a Lie algebra are its commutations relations.  
In the following the latter are calculated for an arbitrary 
representation of $\mathfrak{so}(p,q)$. To do this one considers 
two arbitrary Lorentz transformations $\Lambda$ and 
$\tilde{\Lambda}$ and the composition $\Lambda 
\circ\tilde{\Lambda} \circ\Lambda^{-1} 
\circ\tilde{\Lambda}^{-1}$. This group element equals the 
identity if and only if elements of $SO(p,q)$ commute. On the 
other hand, since the orthogonal group is not Abelian, such a 
sequence generally will not give the identity. The generating 
vector of this composition will be the definition of the 
commutator. Consider%
\begin{equation}\label{eq:comm_group}
	D(\omega) \circ D(\tilde{\omega}) \circ D(\omega)^{-1} \circ 
	D(\tilde{\omega})^{-1} = D(\omega\, \tilde{\omega}\, 
	\omega^{-1}\, \tilde{\omega}^{-1})
\end{equation}
for an arbitrary representation.
To evaluate this, it is sufficient to consider group elements up 
to second order in the group parameters, that is
%
\begin{gather*}
	\Lambda = \delta^a_b + \omega^a_{~b} + \tfrac{1}{2} 
	\omega^a_{~c}\omega^c_{~b} \\
	D(\omega) = \mathbb{1} + \tfrac{i}{2}\omega\cdot M + 
	\tfrac{1}{2}(\tfrac{i}{2}\omega\cdot M)^2
\end{gather*}
from which one is able to compute
%
\begin{displaymath}
	D(\omega) \circ D(\tilde{\omega}) \circ D(\omega)^{-1} \circ 
	D(\tilde{\omega})^{-1} = \mathbb{1} + (\tfrac{i}{2})^2 
	[\omega\cdot M,\tilde{\omega}\cdot M] + \mathcal{O}(3)
\end{displaymath}
and
\begin{displaymath}
	\Lambda \circ\tilde{\Lambda} \circ\Lambda^{-1} 
	\circ\tilde{\Lambda}^{-1} = \delta^a_b + 
	\omega^a_{~c}\tilde{\omega}^c_{~b} - 
	\tilde{\omega}^a_{~c}\omega^c_{~b} + \mathcal{O}(3)~.
\end{displaymath}
Combing these results together with \eqref{eq:comm_group} we get
%
\begin{displaymath}
	(\tfrac{i}{2})^2 \omega^{ab}\tilde{\omega}^{cd}[M_{ab},M_{cd}] 
	= \tfrac{i}{2}( \omega^a_{~c}\tilde{\omega}^{cb} - 
	\tilde{\omega}^a_{~c}\omega^{cb})M_{ab}
	= \tfrac{i}{2}\omega^{ab}\tilde{\omega}^{cd} (\eta_{bc}M_{ad} 
	- \eta_{ad}M_{cb})~.
\end{displaymath}
This result does not depend yet on any information, typical for 
the orthogonal group. Recall that for the latter $M_{ab} = 
-M_{ba}$ so that
%
\begin{displaymath}
	\tfrac{i}{2}\omega^{ab}\tilde{\omega}^{cd}[M_{ab},M_{cd}] = 
	\tfrac{1}{2}\omega^{ab}\tilde{\omega}^{cd} (\eta_{bc}M_{ad} - 
	\eta_{bd}M_{ac} - \eta_{ad}M_{cb} + \eta_{ac}M_{db})~.
\end{displaymath}
Since \eqref{eq:comm_group} should be true for any group 
parameters considered, we find the commutation relations for 
$\mathfrak{so}(p,q)$, namely
%
\begin{equation}\label{eq:comm_rel_so(p,q)}
	-i[M_{ab},M_{cd}] = \eta_{ac}M_{bd} - \eta_{ad}M_{bc} + 
	\eta_{bd}M_{ac} - \eta_{bc}M_{ad}~.
\end{equation}
	
\section{The de Sitter algebra}

In this section the specific case where the orthogonal algebra is 
denoted by $\mathfrak{so}(1,4)$ is discussed in more detail.  
Naturally, its commutation relations are those described 
in~\eqref{eq:comm_rel_so(p,q)}.\footnote{We let $A = 0\ldots 4$ 
	and $a = 0\ldots 3$.} The de Sitter algebra is symmetric, 
which means that there is a reductive splitting 
$\mathfrak{so}(1,4) = \mathfrak{so}(1,3) \oplus \mathfrak{p}$ so 
that the adjoint action of the Lorentz subalgebra on 
$\mathfrak{p}$ is an automorphism. Explicitely this goes as 
follows. Denote by $M_{AB}$ a set of basis elements of the de 
Sitter algebra. Let then $\mathfrak{so}(1,3) = 
\mathrm{span}\{M_{ab}\}$ and $\mathfrak{p} = 
\mathrm{span}\{M_{a4}\}$. Introducing a length scale $P_a \equiv 
l^{-1} M_{a4}$, the commutation 
relations~\eqref{eq:comm_rel_so(p,q)} are rewritten as
%
\begin{equation}
\label{eq:comm_relations_so(1,4)}
\begin{split}
	-i[M_{ab},M_{cd}] &= \eta_{ac}M_{bd} - \eta_{ad}M_{bc} + 
	\eta_{bd}M_{ac} - \eta_{bc}M_{ad} \\
	-i[M_{ab},P_c] &= \eta_{ac}P_b- \eta_{bc}P_a\\
	-i[P_a,P_b] &= \frac{\mathfrak{s}}{l^2}M_{ab}~,
  \end{split}
\end{equation}
from which it becomes manifest that the algebra is symmetric.
\end{document}

