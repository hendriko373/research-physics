\documentclass[11pt]{article}

%Load preamble files
\usepackage{../../Tex_files/standard}
\usepackage{../../Tex_files/preamble_one}
%\usepackage{showframe} %show frame borders

\usepackage{mathtools}

\usepackage[all]{xy}

\title{de Sitter Teleparallel Gravity}
\author{Hendrik}
\date{\today}

\begin{document}

\maketitle

\begin{abstract}
	In this document we construct de Sitter Teleparallel Gravity.
\end{abstract}

\section{de Sitter Teleparallel Gravity}

\subsection{Introduction}

In this section we require that the $\mathfrak{h}$-valued part of 
the Cartan curvature $\bar{F}$ vanishes. In other words, the 
geometry outlined in the last section should at all times satisfy 
the following condition, namely
\begin{equation}
	\label{eq:cond_Rnonlin0}
	\bar{R} \equiv 0~.
\end{equation}
From the discussion on transformation behavior in 
\verb+nonlinear.pdf+  it is clear that this condition is 
consistent with the geometry, i.e.~invariant under $G$-gauge 
transformations. This construction may result in the 
mathemathical structure of Teleparallel Gravity in the 
corresponding limit, i.e.~a diverging length scale $l(x) \to 
\infty$ at any point in spacetime. In that case, and because of 
the naturality of the given Cartan geometry together with a 
vanishing curvature~\eqref{eq:cond_Rnonlin0}, the thus obtained 
geometry could be seen as the generalization of Teleparallel 
Gravity, where the local kinematics are those governed by the de 
Sitter algebra.

\blankline
Let us begin by taking a closer look at the condition of 
vanishing curvature, given in 
Eq.\verb+~\eqref{eq:cond_Rnonlin0},nonlinear.odf+.  Combining 
this requirement with 
Eq.\verb+~\eqref{eq:nonlin_curv},nonlinear.pdf+, one finds that
\begin{displaymath}
	R^{ab}_{~~\mu\nu} = \frac{\cosh z - 1}{\xi^2} \xi^c (\xi^a 
	R^b_{~c\mu\nu} - \xi^b R^a_{~c\mu\nu}) + \frac{\sinh z}{l^2 z} 
	(\xi^a T^b_{~\mu\nu} - \xi^b T^a_{~\mu\nu})~.
\end{displaymath}
This expression can be contracted with $\xi$, which results in
\begin{equation}
	\cosh z\, \xi^c R^a_{~c\mu\nu} = \frac{\sinh z}{l^2 z} (\xi^a 
	\xi_b	T^b_{~\mu\nu} - \xi^2 T^a_{~\mu\nu})~.
\end{equation}
Substituting this equations into the torsion $\bar{T}$, see 
Eq.\verb+~\eqref{eq:nonlin_tors},nonlinear.pdf+, one obtains
\begin{equation}
	\label{eq:Tnonlinear_Rnl0}
	\bar{T}^a_{~\mu\nu} = \frac{1}{\cosh z} T^a_{~\mu\nu} + 
	\bigg(1-\frac{1}{\cosh z}\bigg) \frac{\xi^a\xi_b 
		T^b_{~\mu\nu}}{\xi^2}~.
\end{equation}
Contracting both sides with $\xi$ additionally shows 
that
\begin{displaymath}
	\xi_a \bar{T}^a_{~\mu\nu} = \xi_a T^a_{~\mu\nu}~.
\end{displaymath}
\begin{remark}
	It is interesting to have a look at the limiting situations 
	for a vanishing, respectively diverging cosmological constant.  
	In the case of $l(x) \to \infty$, $z$ vanishes and 
	from~\eqref{eq:Tnonlinear_Rnl0} it is found that
	\begin{displaymath}
		\lim_{\Lambda \to 0} \bar{T}^a_{~\mu\nu} = T^a_{~\mu\nu}~,
	\end{displaymath}
	while on the other hand for $l(x) \to 0$, $z$ diverges and
	\begin{displaymath}
		\lim_{\Lambda \to \infty} \bar{T}^a_{~\mu\nu} = 
		\frac{\xi^a\xi_b T^b_{~\mu\nu}}{\xi^2}~.
		\qedhere
	\end{displaymath}
\end{remark}

Subsequently let us investigate the additional restriction of a 
vanishing torsion, i.e.\footnote{To be clear: the condition of a 
	vanishing curvature is \emph{not} relaxed.}
\begin{equation}
	\label{eq:cond_Tnonlin0}
	\bar{T} = 0~.
\end{equation}
From Eq.~\eqref{eq:Tnonlinear_Rnl0} and observing that 
$\xi\cdot\bar{T} = \xi\cdot T$ one infers that $T$ vanishes. An 
obvious choice of gauge corresponding to such a geometry is $e = 
0$, that is
\begin{displaymath}
	e = 0 \quad\Rightarrow\quad \bar{T} = 0~.
\end{displaymath}
On the other hand, a vanishing torsion does not necessarily imply 
that $e$ is equal to zero. To find the most general $e$ 
consistent with the condition~\eqref{eq:cond_Tnonlin0}, it is 
worthwhile to note that the latter is invariant under local gauge 
transformations and spacetime diffeomorphisms.  These 
transformations are the most general at hand and their effect on 
$e$ will exhaust its values, corresponding to a vanishing 
torsion.  Since $e$ transforms in a homogeneous way under both 
spacetime diffeomorphisms and local Lorentz transformations, 
these will leave $e=0$ invariant. Therefore it is sufficient to 
consider the action on $e$ due to de Sitter transvections 
$\exp(i\alpha\cdot P)$ solely.  From the transformation 
rule\verb+~\eqref{eq:transvec_fin_Ah}+ for $e$ one finds that 
$e^a = 0$ transforms into
\begin{equation}
	\label{eq:e_trivial}
	e'^a = -\frac{\sinh z}{z} (d\alpha^a + \omega^a_{~b}\alpha^b) + 
	\frac{dl}{l} \alpha^a + \bigg( \frac{\sinh z}{z} - 1 \bigg) 
	\frac{\alpha_b d\alpha^b \alpha^a}{\alpha^2}~.
\end{equation}
The vierbein $\bar{e}'$ Lorentz rotates according 
to
\begin{multline*}
	\bar{e}'^a = \cosh z' e'^a - (\cosh z' -1) \frac{\xi'_b e'^b 
		\xi'^a}{\xi'^2}
	\\
	+ \frac{\sinh z'}{z'}(d\xi'^a + \omega'^a_{~b}\xi'^b) - 
	\frac{dl}{l} \xi'^a - \bigg( \frac{\sinh z'}{z'} - 1 \bigg) 
	\frac{\xi'_b d\xi'^b \xi'^a}{\xi'^2}~,
\end{multline*}
while the torsion $\bar{T}'^a$ remains zero. We call a vierbein 
$\bar{e}$ \emph{trivial} if and only if $e$ is of the 
form~\eqref{eq:e_trivial}. In the case of a trivial vierbein, it 
is clear that some gauge transformation will render $e = 0$. But 
the relevant action is given by de Sitter transvections, which 
correspond to a shift in the section $\xi$.
Since $\xi$ is arbitrary, it is then without loss of generality 
to assume that a trivial vierbein is of the form
\begin{equation}
	\bar{e}^a = \frac{\sinh z}{z}(d\xi^a + \omega^a_{~b}\xi^b) - 
	\frac{dl}{l} \xi^a - \bigg( \frac{\sinh z}{z} - 1 \bigg) 
	\frac{\xi_b d\xi^b \xi^a}{\xi^2}~.
\end{equation}
Hence, the vanishing of torsion entails the triviality of the 
vierbein. Conversely, in case the vielbein is trivial, the 
torsion will be equal to zero.


\subsection{Equations of motion for a particle}

Given the vielbein $\bar{e}$, it is possible to construct a line 
element on spacetime that is invariant under local de Sitter 
transformations.  The quadratic line element is defined as
\begin{displaymath}
	d\tau^2 = \bar{e}\ind{^a_\mu} \bar{e}\ind{_{a\nu}} dx^\mu 
	dx^\nu~,
\end{displaymath}
from which the square root is extracted, resulting in
\begin{equation}
	d\tau = \bar{u}_a \bar{e}^a~.
\end{equation}
In the last expression the nonlinear four-velocity has been 
introduced, which is given by
\begin{displaymath}
	\bar{u}^a = \bar{e}\ind{^a_\mu} u^\mu~.
\end{displaymath}

The line element has the dimension of length, implying that a 
possible action for the worldline $x^\mu(\tau)$ of a particle 
with mass $m$ equals
\begin{equation}
	\label{eq:action_particle}
	\mathcal{S} = -mc \int_{\tau_1}^{\tau_2} d\tau
		= -mc \int_{\tau_1}^{\tau_2} \bar{u}_a \bar{e}^a~.
\end{equation}
The action attains an extremum for the worldline being the 
physical one. This means that the equations of motion correspond 
to $\delta \mathcal{S} = 0$, where an infinitisemal variation of 
the worldline $x^\mu(\tau) \to x^\mu + \delta x^\mu(\tau)$ is 
considered. Under this deviation, the 
action~\eqref{eq:action_particle} varies according to
\begin{displaymath}
	\delta\mathcal{S} = -mc \int_{\tau_1}^{\tau_2} \delta\bar{u}_a 
	\bar{e}^a + \bar{u}_a \delta\bar{e}^a
	= -mc \int_{\tau_1}^{\tau_2} \bar{u}_a \delta\bar{e}^a~.
\end{displaymath}
After a rather lengthy calculation, which we wrote down in 
Appendix~\eqref{ssec:dSTG_var_ue}, one finds
\begin{displaymath}
	\delta\mathcal{S} = mc \int_{\tau_1}^{\tau_2} d\tau \delta 
	x^\mu \bigg\{ \bar{e}\ind{^a_\mu} \bigg( 
	\frac{d\bar{u}_a}{d\tau} - \bar{\omega}\ind{^b_{a\rho}} 
	\bar{u}\ind{_b} u^\rho + u^\rho \frac{\pd_\rho l}{l} \bar{u}_a 
	\bigg) - \bar{T}\ind{^a_{\mu\rho}} \bar{u}\ind{_a} u^\rho - 
	\frac{\pd_\mu l}{l} \bigg\}~.
\end{displaymath}
This quantity should vanish for an arbitrary variation, a 
condition that leads to the equations of motion:
\begin{displaymath}
	u^\rho \bar{D}_\rho (l \bar{u}^a) = l\bar{e}^{a\mu} 
	\bigg(\bar{T}\ind{^b_{\mu\rho}} \bar{u}\ind{_b} u^\rho + 
	\frac{\pd_\mu l}{l} \bigg)~,
\end{displaymath}
where $\bar{D} \equiv d + \bar{\omega}$ is the covariant 
derivative with respect to the spin connection $\bar{\omega}$.
The equations of motion can be rewritten in the form
\begin{equation}
	\label{eq:eom_particle_dSTG}
	u^\rho \bar{D}_\rho \bar{u}^a = \bar{e}\ind{^{a\mu}} 
	\bar{T}\ind{^b_{\mu\rho}} \bar{u}\ind{_b} u^\rho + 
	(\bar{e}\ind{^{a\mu}} - \bar{u}^a u^\mu) \frac{\pd_\mu l}{l}~.
\end{equation}

It is interesting to have a closer look at this equation. First 
note that in the appropriate limit of a vanishing cosmological 
function $(l \to \infty)$, the  equation of motion of 
Teleparallel Gravity for a spinless particle in a gravitational 
field is recovered \cite{aldrovandi:2012tele}. Similar to the 
equation there, we still have a force equation at hand in which 
both the terms on the right-hand side are indeed genuine 
relativistic forces, being orthogonal to the four-velocity 
$\bar{u}^a$. The first force is the obvious generalization to the 
given geometry of the gravitational force in ordinary 
Teleparallel gravity. The second force term however is new, and 
will be noticeable only in spacetime regions where the 
cosmological function varies relativily strong. Observe that the 
operator $\bar{e}\ind{^{a\mu}} - \bar{u}^a u^\mu$ is a projector, 
since
\begin{displaymath}
	(\bar{e}\ind{^{b\rho}} - \bar{u}^b u^\rho) \bar{e}_{a\rho} 
	(\bar{e}\ind{^{a\mu}} - \bar{u}^a u^\mu) = \bar{e}\ind{^{b\mu}} 
	- \bar{u}^b u^\mu~.
\end{displaymath}

\subsection{Field equations}

In this subsection we look for the equations of motions that 
specify for the geometry in de Sitter Teleparallel Gravity.  In a 
first attempt, the free action is the one given by replacing $T 
\to \bar{T}$ in the action that describes free Poincar\'e 
Teleparallel Gravity {\blu [Citations]}. It is thus proposed that  
\begin{equation}
	\label{eq:action_dSTG}
	\mathcal{S} = \frac{c^3}{16\pi G} \int \mathrm{Tr}~ \bar{T} 
	\wedge \star \bar{T} = \frac{c^3}{16\pi G} \int d^4 x \, 
	\bar{e}\, \mathcal{L}~,
\end{equation}
where
\begin{equation}
	\label{eq:lagrangian_dSTG}
	\mathcal{L} = \tfrac{1}{4} \bar{T}\ind{^a_{\mu\nu}} 
	\bar{T}\ind{_a^{\mu\nu}} + \tfrac{1}{2} 
	\bar{T}\ind{^a_{\mu\nu}} \bar{T}\ind{^{b\mu}_\lambda} 
	\bar{e}\ind{_a^\lambda} \bar{e}\ind{_b^\nu} - 
	\bar{T}\ind{^a_{\mu\nu}} \bar{T}\ind{^{b\mu}_\lambda} 
	\bar{e}\ind{_a^\nu} \bar{e}\ind{_b^\lambda}~.
\end{equation}
The corresponding field equations are found by 
extremizing~\eqref{eq:action_dSTG} with respect to the vierbein 
$\bar{e}\ind{^a_\mu}$, i.e.~
\begin{displaymath}
\begin{split}
	0 = \delta\mathcal{S} &= \int d^4x \, \delta\bar{e} \,
	\mathcal{L} + \int d^4x \, \bar{e} \, \delta\mathcal{L} \\
	&= \int d^4x \, \bar{e} \, \bar{e}\ind{_a^\mu} \mathcal{L} 
	\delta\bar{e}\ind{^a_\mu} + \int d^4x \, \bar{e} \bigg( 
	\frac{\pd \mathcal{L}}{\pd \bar{e}\ind{^a_\mu}} \delta 
	\bar{e}\ind{^a_\mu} + \frac{\pd \mathcal{L}}{\pd \pd_\rho 
		\bar{e}\ind{^a_\mu}} \delta \pd_\rho \bar{e}\ind{^a_\mu} 
	\bigg)~.
\end{split}
\end{displaymath}
After invoking Stokes' theorem together with the assumption that 
the fields go to zero when approaching infinity,\footnote{{\blu 
		Is it the fields that go to zero that legitimate the 
		omitting of the boundary terms, or is it the vanishing of 
		the variation?}} the equations of motion are
\begin{equation}
	\pd_\mu \bigg( \bar{e} \frac{\pd \mathcal{L}}{\pd \pd_\mu 
		\bar{e}\ind{^a_\nu}} \bigg) - \bar{e} \frac{\pd 
		\mathcal{L}}{\pd \bar{e}\ind{^a_\nu}} - \bar{e}\, 
	\bar{e}\ind{_a^\nu} \mathcal{L} = 0~.
\end{equation}
For the given Lagrangian it is shown in 
Appendix~\ref{app:field_eqs} that these equations reduce to
\begin{displaymath}
	\pd_\mu (\bar{e}\, \bar{W}\ind{_a^{\mu\nu}}) - \bar{e}\, 
	\bar{\omega}\ind{^b_{a\mu}} \bar{W}\ind{_b^{\mu\nu}} + \bar{e} 
	\frac{\pd_\mu l}{l} \bar{W}\ind{_a^{\mu\nu}} + \bar{e}\, 
	\bar{T}\ind{^b_{\mu a}} \bar{W}\ind{_b^{\mu\nu}} - \bar{e}\, 
	\bar{e}\ind{_a^\nu} \mathcal{L} = 0~,
\end{displaymath}
where we introduced the notation
\begin{equation}
	\bar{W}\ind{_a^{\mu\nu}} \equiv \bar{T}\ind{_a^{\mu\nu}} + 
	\bar{T}\ind{^{\nu\mu}_a} - \bar{T}\ind{^{\mu\nu}_a} - 
	2\bar{e}\ind{_a^\nu} \bar{T}\ind{^{\lambda\mu}_\lambda} + 
	2\bar{e}\ind{_a^\mu} \bar{T}\ind{^{\lambda\nu}_\lambda}~.
\end{equation}

The field equations can be rewritten in a manifestly covariant 
form as
\begin{equation}
	\label{eq:field_eqs_dSTG}
	\bar{D}_\mu (\bar{e}\, \bar{W}\ind{_a^{\mu\nu}}) + \bar{e} 
	\frac{\pd_\mu l}{l} \bar{W}\ind{_a^{\mu\nu}} + \bar{e}\, 
	\bar{t}\ind{_a^\nu} = 0~,
\end{equation}
where we denoted the expression
\begin{displaymath}
	\bar{t}\ind{_a^\nu} = \bar{T}\ind{^b_{\mu a}} 
	\bar{W}\ind{_b^{\mu\nu}} - \bar{e}\ind{_a^\nu} \mathcal{L}~.
\end{displaymath}
Note further that
\begin{displaymath}
	\bar{D}_\nu \bar{D}_\mu (\bar{e}\, \bar{W}\ind{_a^{\mu\nu}})
	= \frac{1}{2} [\bar{D}_\nu, \bar{D}_\mu] (\bar{e}\, 
	\bar{W}\ind{_a^{\mu\nu}}) 
	= \frac{1}{2} \bar{e}\, \bar{B}\ind{_a^b_{\nu\mu}} 
	\bar{W}\ind{_b^{\mu\nu}}~.
\end{displaymath}
For the given geometry one has that $\bar{B}^{ab} = -l^{-2} 
\bar{e}^a \wedge \bar{e}^b$ so that
\begin{displaymath}
	\bar{D}_\nu \bar{D}_\mu (\bar{e}\, \bar{W}\ind{_a^{\mu\nu}})
	= \frac{1}{l^2} \bar{e}\, \bar{e}\ind{_{a\mu}} 
	\bar{e}\ind{^b_\nu} \bar{W}\ind{_b^{\mu\nu}}
	= \frac{1}{l^2} \bar{e}\, \bar{W}\ind{_{ba}^b}~.
\end{displaymath}
Since $\bar{W}\ind{_{ba}^b} = -4 \bar{T}\ind{_{ba}^b}$ and 
because one can infer from the second Bianchi identity that the 
trace of $\bar{T}$ vanishes, one concludes that
\begin{equation}
	\bar{D}_\nu \bar{D}_\mu (\bar{e}\, \bar{W}\ind{_a^{\mu\nu}})
	= 0~.
\end{equation}
It should be emphasized that this result is particular to the 
geometry at hand. More specifically is it a consequence of the 
condition $\bar{R} \equiv 0$. From the field 
equations~\eqref{eq:field_eqs_dSTG} one observes that
\begin{equation}
	\bar{D}_\mu(\bar{e}\, \bar{t}\ind{_a^\mu}) = \frac{\pd_\mu 
		l}{l} \bar{e}\, \bar{t}\ind{_a^\mu}~.
\end{equation}


\newpage
\appendix
\section{de Sitter Teleparallel Gravity: intermediate results}

In this section we work out to some extend, results related to de 
Sitter Teleparallel gravity that were used in the main body of 
the text.

\subsection{Variation of $\int \bar{u}_a \bar{e}^a$}
\label{ssec:dSTG_var_ue}

In this calculation the variation of $\int \bar{u}_a \bar{e}^a$ 
will be verified.  To begin with let us rewrite the expression 
for a non-trivial vierbein, i.e.
\begin{multline*}
	\bar{e}^a = \cosh z\, e^a - (\cosh z -1) \frac{\xi_b e^b 
		\xi^a}{\xi^2} \\
		+ \frac{\sinh z}{z}(d\xi^a + \omega^a_{~b}\xi^b) - 
		\frac{dl}{l} \xi^a - \bigg( \frac{\sinh z}{z} - 1 \bigg) 
		\frac{\xi_b d\xi^b \xi^a}{\xi^2}~.
\end{multline*}
We compute:
\begin{displaymath}
\begin{split}
	\int \bar{u}_a &\delta\bar{e}^a \\
	&= \int \bar{u}_a \bigg\{ \delta\!\cosh z\, e^a + \cosh z\, 
	\delta e\ind{^a_\rho} dx^\rho + \cosh z\, e\ind{^a_\rho} 
	\delta dx^\rho - \delta\!\cosh z \frac{\xi_b e^b \xi^a}{\xi^2} 
	\\
	&\nla{} + (\cosh z - 1) \bigg[ \frac{2\delta\xi}{\xi} 
	\frac{\xi_b e^b \xi^a}{\xi^2} - \frac{\delta\xi_b e^b 
		\xi^a}{\xi^2} -\frac{\xi\ind{_b} \delta e\ind{^b_\rho} 
		\xi^a}{\xi^2} dx^\rho - \frac{\xi_b e^b \delta\xi^a}{\xi^2} 
	\\
	&\nla{} - \frac{\xi\ind{_b} e\ind{^b_\rho} \xi^a}{\xi^2} 
	\delta dx^\rho \bigg] + \delta \Big( \frac{\sinh z}{z} \Big) ( 
	d\xi^a + \omega\ind{^a_b}\xi^b )
	+ \frac{\sinh z}{z} (\delta d\xi^a + \delta 
	\omega\ind{^a_{b\rho}} \xi^b dx^\rho
	\\
	&\nla{} + \omega\ind{^a_b} \delta\xi^b + 
	\omega\ind{^a_{b\rho}} \xi^b \delta dx^\rho) + \frac{\delta l 
		dl}{l^2} \xi^a - \frac{\delta dl}{l} \xi^a - 
	\frac{1}{l}\delta\xi^a - \delta \Big( \frac{\sinh z}{z} \Big) 
	\frac{\xi_b d\xi^b \xi^a}{\xi^2}
	\\
	&\nla{} + \Big( \frac{\sinh z}{z} -1 \Big) \bigg[ 
	\frac{2\delta\xi}{\xi} \frac{\xi_b d\xi^b \xi^a}{\xi^2} - 
	\frac{\delta\xi_b d\xi^b \xi^a}{\xi^2} - \frac{\xi_b \delta 
		d\xi^b \xi^a}{\xi^2} - \frac{\xi_b d\xi^b \xi^a}{\xi^2} 
	\bigg]\bigg\}
\end{split}
\end{displaymath}
For any function on $M$, note that $df \to df + d\delta f$ so 
that $\delta (df) = d(\delta f)$, i.e.
\begin{equation}
	[\delta,d] f = 0~.
\end{equation}
The following variations also are useful:
\begin{gather}
	\delta z = \delta (l^{-1} \xi) = - l^{-2} \delta l\, \xi + 
	l^{-1} \delta\xi~, \\
	\delta \xi = \xi^{-1} \xi_a \delta \xi^a~.
\end{gather}
The variation is assumed to vanish at the endpoints of the 
particle's worldline, so that a total derivative over a term 
containing $\delta x^\rho$ integrates to zero. One first 
integrates by parts the terms containing variations of the 
differentials $\delta dx^\rho = d\delta x^\rho$, after which the 
boundary integrals render zero. Doing so, one obtains
\begin{displaymath}
\begin{split}
	\int \bar{u}_a &\delta\bar{e}^a \\
	&= \int \bigg[ -d\bar{u}\ind{_a} \bigg\{ \cosh z\, 
	e\ind{^a_\mu} - (\cosh z - 1) \frac{\xi\ind{_b} e\ind{^b_\mu} 
		\xi^a}{\xi^2} + \frac{\sinh z}{z}(\pd_\mu \xi^a + 
	\omega\ind{^a_{b\mu}} \xi^b)
	\\
	&\nla{} - \frac{\pd_\mu l}{l} \xi^a - \Big( \frac{\sinh z}{z} 
	- 1 \Big) \frac{\xi_b \pd_\mu \xi^b \xi^a}{\xi^2} \bigg\} 
	\delta x^\mu + \bar{u}_a \delta x^\mu dx^\rho \bigg\{ \bigg[ 
	\frac{\sinh z}{z} \omega\ind{^a_{b\rho}} \Big( \pd_\mu \xi^b
	\\
	&\nla{} - \xi^b \frac{\xi_c \pd_\mu \xi^c}{\xi^2} \Big) + 
	\cosh z\, ( \pd_\rho \xi^a + \omega\ind{^a_{b\rho}} \xi^b )  
	\frac{\xi_c \pd_\mu \xi^c}{\xi^2} - \pd_\rho \xi^a \frac{\xi_b 
		\pd_\mu \xi^b}{\xi^2} - \cosh z \Big( \pd_\rho \xi^a
	\\
	&\nla{} + \omega\ind{^a_{b\rho}} \xi^b - \xi^a \frac{\xi_b 
		\pd_\rho \xi^b}{\xi^2} \Big) \frac{\pd_\mu l}{l} + \pd_\rho 
	\xi^a \frac{\pd_\mu l}{l} - 2(\cosh z - 1) \xi^a \frac{\xi_b 
		\pd_\rho \xi^b}{\xi^2} \frac{\xi\ind{_c} 
		e\ind{^c_\mu}}{\xi^2}
	\\
	&\nla{} + (\cosh z - 1) \pd_\rho \xi^a \frac{\xi\ind{_b} 
		e\ind{^b_\mu}}{\xi^2} + (\cosh z - 1) \xi^a \frac{\pd_\rho 
		\xi\ind{_b} e\ind{^b_\mu}}{\xi^2} + z\sinh z \frac{\xi_c 
		\pd_\mu \xi^c}{\xi^2} \Big( e\ind{^a_\rho}
	\\
	&\nla{} - \xi^a \frac{\xi\ind{_b} e\ind{^b_\rho}}{\xi^2} \Big) 
	- z \sinh z \Big( e\ind{^a_\rho} - \xi^a \frac{\xi\ind{_b} 
		e\ind{^b_\rho}}{\xi^2} \Big) \frac{\pd_\mu l}{l} \bigg] - 
	\bigg[ \rho \leftrightarrow \mu \bigg] \bigg\} + \bar{u}_a 
		\delta x^\mu dx^\rho \bigg\{
	\\
	&\nla{} -\bigg[ \frac{\pd_\rho l}{l} \frac{\sinh z}{z} \Big( 
	\pd_\mu \xi^a + \omega\ind{^a_{b\mu}} \xi^b - \xi^a 
	\frac{\xi_b \pd_\mu \xi^b}{\xi^2} \Big) + \frac{\pd_\rho l}{l} 
	\frac{\pd_\mu l}{l} \xi^a + \cosh z\, \pd_\rho e\ind{^a_\mu}
	\\
	&\nla{} - (\cosh z - 1) \xi^a \frac{\xi_b \pd_\rho 
		e\ind{^b_\mu}}{\xi^2} + \frac{\sinh z}{z} \pd_\rho 
	\omega\ind{^a_{b\rho}} \xi^b	\bigg] + \bigg[ \rho 
	\leftrightarrow \mu \bigg] \bigg\} \bigg]
\end{split}
\end{displaymath}
The terms between the first pair of curly brackets is just the 
vierbein $\bar{e}\ind{^a_\mu}$, while those between the second 
pair of curly brackets equal
\begin{multline*}
	\bigg[ \bar{\omega}\ind{^a_{b\rho}} \bar{e}\ind{^b_\mu} - 
	\frac{\sinh z}{z} \omega\ind{^a_{b\rho}} \omega\ind{^b_{c\mu}} 
	\xi^c - \frac{\pd_\rho l}{l} \xi^a \frac{\xi_b \pd_\mu 
		\xi^b}{\xi^2} - \cosh z\, \omega\ind{^a_{b\rho}} 
	e\ind{^b_\mu}
	\\
	- (\cosh z - 1) \xi^a \frac{\omega\ind{_{bc\rho}} \xi^c 
		e\ind{^b_\mu}}{\xi^2} - z \sinh z\, e\ind{^a_\rho} 
	\frac{\xi\ind{_c} e\ind{^c_\mu}}{\xi^2} \bigg] - \bigg[ \rho 
	\leftrightarrow \mu \bigg]~.
\end{multline*}
This permits us to further work out
\begin{displaymath}
\begin{split}
	\int \bar{u}_a &\delta\bar{e}^a \\
	&= \int \bigg[ -d\bar{u}\ind{_a} \bar{e}\ind{^a_\mu} \delta 
	x^\mu + \bar{u}_a \delta x^\mu dx^\rho \bigg\{ \bigg[ 
	\bar{\omega}\ind{^a_{b\rho}} \bar{e}\ind{^b_\mu} - \frac{\sinh 
		z}{z} \xi^c \Big( \pd_\rho \omega\ind{^a_{c\rho}} + 
	\omega\ind{^a_{b\rho}} \omega\ind{^b_{c\mu}}
	\\
	&\nla{} + \frac{1}{l^2} e\ind{^a_\rho} e\ind{_{c\mu}} \Big) - 
	\cosh z \Big( \pd_\rho e\ind{^a_\mu} + \omega\ind{^a_{b\rho}} 
	e\ind{^b_\mu} - \frac{\pd_\rho l}{l} e\ind{^a_\mu} \Big) - (1 
	- \cosh z) \frac{\xi^a \xi_b}{\xi^2} \Big( \pd_\rho 
	e\ind{^b_\mu}
	\\
	&\nla{} + \omega\ind{^b_{c\rho}} e\ind{^c_\mu} - 
	\frac{\pd_\rho l}{l} e\ind{^b_\mu} \Big) - \frac{\pd_\rho 
		l}{l} \bigg( \cosh z\, e\ind{^a_\mu} - (\cosh z - 1) 
	\frac{\xi\ind{_b} e\ind{^b_\mu} \xi^a}{\xi^2} + \frac{\sinh 
		z}{z}(\pd_\mu \xi^a
	\\
	&\nla{} + \omega\ind{^a_{b\mu}} \xi^b)- \frac{\pd_\mu l}{l} 
	\xi^a - \Big( \frac{\sinh z}{z} - 1 \Big) \frac{\xi_b \pd_\mu 
		\xi^b \xi^a}{\xi^2} \bigg) \bigg] - \bigg[ \rho 
	\leftrightarrow \mu \bigg] \bigg\} \bigg]
	\\
	&= \int \delta x^\mu \bigg[ -d\bar{u}\ind{_a} 
	\bar{e}\ind{^a_\mu} + \bar{u}\ind{_a} 
	\bar{\omega}\ind{^a_{b\rho}} \bar{e}\ind{^b_\mu} dx^\rho - 
	\bar{u}\ind{_a} \bar{\omega}\ind{^a_{b\mu}} 
	\bar{e}\ind{^b_\rho} dx^\rho - \bar{u}_a dx^\rho \bigg( 
		\frac{\sinh z}{z} \xi^c R\ind{^a_{c\rho\mu}}
	\\
	&\nla{} + \cosh z\, T\ind{^a_{\rho\mu}} + (1 - \cosh z) 
	\frac{\xi^a \xi\ind{_b} T\ind{^b_{\rho\mu}}}{\xi^2} \bigg) - 
	\frac{\pd_\rho l}{l} \bar{e}\ind{^a_\mu} \bar{u}\ind{_a} 
	dx^\rho + \frac{\pd_\mu l}{l} \bar{e}\ind{^a_\rho} 
	\bar{u}\ind{_a} dx^\rho \bigg]
\end{split}
\end{displaymath}
In this expression one recognizes the torsion $\bar{T}$, as given 
in Eq.\verb+~\eqref{eq:nonlin_tors}+. This leads to the end of 
this calculation:
\begin{equation}
	\int \bar{u}_a \delta\bar{e}^a = \int d\tau \delta x^\mu 
	\bigg\{ - \bar{e}\ind{^a_\mu} \bigg( \frac{d\bar{u}_a}{d\tau} 
	- \bar{\omega}\ind{^b_{a\rho}} \bar{u}\ind{_b} u^\rho + u^\rho 
	\frac{\pd_\rho l}{l} \bar{u}\ind{_a} \bigg) + 
	\bar{T}\ind{^a_{\mu\rho}} \bar{u}\ind{_a} u^\rho + 
	\frac{\pd_\mu l}{l} \bigg\}~.
\end{equation}



\subsection{Variation of $\mathcal{L}$ with respect to 
	$\bar{e}$.}
\label{app:field_eqs}

In this subsection we work out some intermediary results that 
lead to the functional variation of the 
Lagrangian~\eqref{eq:lagrangian_dSTG} with respect to the 
vierbein. More precisely, we calculate the derivatives of 
$\mathcal{L}$ with respect to $\bar{e}\ind{^c_\sigma}$ and 
$\pd_\rho \bar{e}\ind{^c_\sigma}$ in turn.  
From~\eqref{eq:lagrangian_dSTG}:
\begin{displaymath}
\begin{split}
	\frac{\pd \mathcal{L}}{\pd \bar{e}\ind{^c_\sigma}}
	&= \frac{1}{4} \frac{\pd \bar{T}\ind{^a_{\mu\nu}}}{\pd 
		\bar{e}\ind{^c_\sigma}} \bar{T}\ind{_a^{\mu\nu}}
	+ \frac{1}{4} \bar{T}\ind{^a_{\mu\nu}} \frac{\pd 
		\bar{T}\ind{_a^{\mu\nu}}}{\pd \bar{e}\ind{^c_\sigma}}
	+ \frac{1}{2} \frac{\pd \bar{T}\ind{^a_{\mu\nu}}}{\pd 
		\bar{e}\ind{^c_\sigma}} \bar{T}\ind{^{b\mu}_\lambda} 
	\bar{e}\ind{_a^\lambda} \bar{e}\ind{_b^\nu}
	+ \frac{1}{2} \bar{T}\ind{^a_{\mu\nu}} \frac{\pd 
		\bar{T}\ind{^{b\mu}_\lambda}}{\pd \bar{e}\ind{^c_\sigma}} 
	\bar{e}\ind{_a^\lambda} \bar{e}\ind{_b^\nu}
	\\
	&\nla{} + \frac{1}{2} \bar{T}\ind{^a_{\mu\nu}} 
	\bar{T}\ind{^{b\mu}_\lambda} \frac{\pd( 
		\bar{e}\ind{_a^\lambda} \bar{e}\ind{_b^\nu})}{\pd 
		\bar{e}\ind{^c_\sigma}}
	- \frac{\pd \bar{T}\ind{^a_{\mu\nu}}}{\pd 
		\bar{e}\ind{^c_\sigma}} \bar{T}\ind{^{b\mu}_\lambda} 
	\bar{e}\ind{_a^\nu} \bar{e}\ind{_b^\lambda}
	- \bar{T}\ind{^a_{\mu\nu}} \frac{\pd 
		\bar{T}\ind{^{b\mu}_\lambda}}{\pd \bar{e}\ind{^c_\sigma}} 
	\bar{e}\ind{_a^\nu} \bar{e}\ind{_b^\lambda}
	\\
	&\nla{} - \bar{T}\ind{^a_{\mu\nu}} 
	\bar{T}\ind{^{b\mu}_\lambda} \frac{\pd( \bar{e}\ind{_a^\nu} 
		\bar{e}\ind{_b^\lambda})}{\pd \bar{e}\ind{^c_\sigma}}
\end{split}
\end{displaymath}
It is thus useful to consider first the following equalities:
\begin{gather*}
	\frac{\pd \bar{T}\ind{^a_{\mu\nu}}}{\pd \bar{e}\ind{^c_\sigma}}
	= \big[ \bar{\omega}\ind{^a_{c\mu}} \delta^\sigma_\nu - 
	\frac{\pd_\mu l}{l} \delta^a_c \delta^\sigma_\nu \big] - \big[ 
	\mu \leftrightarrow \nu \big]~,
	\\
	\frac{\pd g_{\rho\lambda}}{\pd \bar{e}\ind{^c_\sigma}} = 
	\frac{\pd (\bar{e}\ind{^a_\rho} \bar{e}\ind{_{a\lambda}})}{\pd 
		\bar{e}\ind{^c_\sigma}} = \bar{e}\ind{_{c\lambda}} 
	\delta^\sigma_\rho + \bar{e}\ind{_{c\rho}} 
	\delta^\sigma_\lambda~,
	\\
	\frac{\pd g^{\rho\lambda}}{\pd \bar{e}\ind{^c_\sigma}} =
	- g^{\sigma\rho} \bar{e}\ind{_c^\lambda} - g^{\sigma\lambda} 
	\bar{e}\ind{_c^\rho}~.
\end{gather*}
Subsequently it is possible to obtain
\begin{gather*}
	\frac{\pd \bar{e}\ind{_a^\lambda}}{\pd \bar{e}\ind{^c_\sigma}} 
	= - \bar{e}\ind{_a^\sigma} \bar{e}\ind{_c^\lambda}~,
	\\
	\begin{split}
	\frac{\pd \bar{T}\ind{_a^{\mu\nu}}}{\pd \bar{e}\ind{^c_\sigma}} 
	&= \big[ \eta_{ab} \bar{\omega}\ind{^b_{c\alpha}} g^{\alpha\mu} 
	g^{\sigma\nu} - \eta_{ac} \frac{\pd_\lambda l}{l} 
	g^{\lambda\mu} g^{\sigma\nu} + \bar{T}\ind{_a^{\sigma\mu}} 
	\bar{e}\ind{_c^\nu} \\
	&\nla{}+ \bar{T}\ind{_{a\lambda}^\mu} \bar{e}\ind{_c^\lambda} 
	g^{\sigma\nu} \big] - \big[ \mu \leftrightarrow \nu \big]~,
	\end{split}
	\\
	\begin{split}
	\frac{\pd \bar{T}\ind{^{b\mu}_\lambda}}{\pd 
		\bar{e}\ind{^c_\sigma}}
	&= g^{\rho\mu} \bar{\omega}\ind{^b_{c\rho}} 
	\delta^\sigma_\lambda - g^{\sigma\mu} 
	\bar{\omega}\ind{^b_{c\lambda}} - g^{\rho\mu} \frac{\pd_\rho 
		l}{l} \delta^b_c \delta^\sigma_\lambda + g^{\sigma\mu} 
	\frac{\pd_\lambda l}{l} \delta^b_c \\
	&\nla{}- \bar{T}\ind{^b_{\rho\lambda}} \bar{e}\ind{_c^\mu} 
	g^{\sigma\rho} - \bar{T}\ind{^b_{\rho\lambda}} 
	\bar{e}\ind{_c^\rho} g^{\sigma\mu}~.
	\end{split}
\end{gather*}
Substituting these equations for $\pd \mathcal{L}/\pd 
\bar{e}\ind{^c_\sigma}$, it takes some algebra to get the 
following:
\begin{equation}
	\frac{\pd \mathcal{L}}{\pd \bar{e}\ind{^c_\sigma}}
	= \bar{\omega}\ind{^a_{c\mu}} \bar{W}\ind{_a^{\mu\sigma}} + 
	\bar{T}\ind{^a_{\mu c}} \bar{W}\ind{_a^{\sigma\mu}} - 
	\frac{\pd_\mu l}{l} \bar{W}\ind{_c^{\mu\sigma}}~.
\end{equation}

It is a simpler exercise to find the derivative of the Lagrangian 
with respect to the first order derivatives of the vierbein. One 
only needs the expression
\begin{displaymath}
	\frac{\pd \bar{T}\ind{^a_{\mu\nu}}}{\pd\pd_\rho 
		\bar{e}\ind{^c_\sigma}} = \delta^\rho_\mu \delta^\sigma_\nu 
	\delta^a_c - \delta^\rho_\nu \delta^\sigma_\mu \delta^a_c~.
\end{displaymath}
This is sufficient since the derivative operator annihilates the 
metric $g_{\mu\nu} = \bar{e}\ind{^a_\mu}\bar{e}\ind{_{a\nu}}$ and 
we can freely raise and lower spacetime indices. Using this 
information, it is readily found that
\begin{equation}
	\frac{\pd \mathcal{L}}{\pd\pd_\rho \bar{e}\ind{^c_\sigma}} = 
	\bar{W}\ind{_c^{\rho\sigma}}~.
\end{equation}


\newpage
\bibliographystyle{ieeetr}
\bibliography{../../references/All}


\end{document}

