\documentclass[%
aps,
prd,
reprint,
showpacs
]{revtex4-1}

%%%%%%%%%%%%%%%%
%%%-PREAMBLE-%%%
%%%%%%%%%%%%%%%%

%AMS packages
\usepackage{mathtools} %Includes amsmath
\usepackage{amsthm}
\usepackage{amsfonts}
\usepackage{amssymb}
\allowdisplaybreaks

%Tensor package - to facilitate index placement
\usepackage{tensor}
\newcommand{\ind}{\indices}

%Font settings
\usepackage[T1]{fontenc}
\usepackage{lmodern}
\usepackage{microtype}

%Language settings
\usepackage[english]{babel}
\uchyph=0 %no hyphenation of capitalized words

%Theorem tools 
\usepackage{thmtools}
\declaretheorem[numbered=no]{definition}
\declaretheorem[numbered=no]{proposition}

%Tikz-cd - Commutative diagrams (Tikz)
\usepackage{tikz-cd}
\tikzset{%document wide settings
	%append settings to 'every label':
	/tikz/commutative diagrams/labels={font=\normalsize}
}

%User-specified commands
\def\Ad{\mathrm{Ad}}

%%%-END-PREAMBLE%%%


\begin{document}

\title{Cartan geometry of spacetimes with a nonconstant 
	cosmological function $\Lambda$}
\author{Hendrik Jennen}
\email{hjennen@ift.unesp.br}
\affiliation{%
	Instituto de F\'{\i}sica Te\'orica, UNESP-Universidade 
	Estadual Paulista, \\
	Rua Dr.~Bento Teobaldo Ferraz, 271 - Bl.~II, 01140-070, S\~ao 
	Paulo, SP,  Brazil
}
\date{\today}

\begin{abstract}
	We present the geometry of spacetimes that are tangentially 
	approximated by de~Sitter spaces whose cosmological constants 
	vary over spacetime. Cartan geometry provides one with the 
	tools to describe manifolds that reduce to a homogeneous Klein 
	space at the infinitesimal level. After briefly reviewing 
	Cartan geometry, we discuss the case in which the underlying 
	Klein space is at each point a de~Sitter space, whose combined 
	set of pseudo-radii forms a nonconstant function on spacetime.  
	We show that the torsion of such a geometry receives a 
	contribution, which is not present for a cosmological 
	constant. The structure group of the obtained de~Sitter-Cartan 
	geometry is by construction the Lorentz group~$SO(1,3)$.  
	Invoking the theory of nonlinear realizations, we extend the 
	class of symmetries to the enclosing de~Sitter 
	group~$SO(1,4)$, and compute the corresponding spin 
	connection, vierbein, curvature, and torsion.
\end{abstract}

\pacs{04.20.Cv, 02.40.-k, 98.80.-k}

\maketitle

\section{Introduction}
\label{sec:intro}

In theories of gravity, the strong equivalence principle states 
that at any spacetime point in an arbitrary gravitational field 
the laws of classical physics reduce to the corresponding laws in 
the absence of gravity, namely, to their special relativistic 
form, when they are considered on a sufficiently small region 
around the point in 
question~\cite{Weinberg:1972gc,DiCasola:2013nep}.  This implies 
that in the presence of a gravitational field spacetime 
$\mathcal{M}$ is locally approximated by the spacetime underlying 
special relativity.  Since the laws that govern special 
relativistic physics are covariant with respect to the Poincar\'e 
group $ISO(1,3)$, the corresponding spacetime is the affine 
Minkowski space~$M$. In view of what follows, and with the risk 
of being overprecise, let us emphasize that the equivalence 
principle does not in the first place imply that the tangent 
spaces to spacetime are given by the affine Minkowski space, but 
rather that they are isomorphic to the infinitesimal structure of 
the latter.  Although finite Poincar\'e translations are not 
defined for a generic spacetime, the equivalence principle 
indicates that locally they are in one-to-one correspondence with 
infinitesimal active diffeomorphisms, for both sets generate 
translations along spacetime~\footnote{An infinitesimal 
	translation at a point $x$ of $\mathcal{M}$ is defined as an 
	element $1 + X$, where $X \in T_x\mathcal{M}$, the tangent 
	space at $x$. The physical meaning of such a translation, 
	being a Poincar\'e translation, for example, depends on the 
	Cartan connection defined on spacetime, see 
	below.}\cite{Hehl:1976grs}.
Mathematically speaking, at any point $x$ in spacetime there is a 
$1$-form $T_x\mathcal{M} \to \mathfrak{t} = 
\mathfrak{iso}(1,3)/\mathfrak{so}(1,3)$, which is called the 
vierbein, and where $\mathfrak{t}$ is the algebra of Poincar\'e 
translations. The vierbein pulls back or \emph{solders} the 
geometric and algebraic structure of $\mathfrak{t}$ to spacetime.  
For example, the Minkowski metric on $\mathfrak{t}$ gives way to 
a metric of the same signature on $\mathcal{M}$, from which it 
follows that the vierbein can be chosen to be an orthonormal 
frame---an idealized observer---on spacetime.  Due to the 
equivalence principle, Lorentz transformations of these observers 
constitute a symmetry and are therefore elements of the structure 
group of the geometry, which in turn leads to the introduction of 
a spin connection. Since the vierbein is valued in 
$\mathfrak{p}$, it transforms as a four-vector under local 
Lorentz rotations, an observation that complies with its 
interpretation of representing an idealized observer at any given 
point.

The right mathematical framework for the setting just outlined is 
due to Elie Cartan~\cite{Cartan:1926gh}, in which the 
$\mathfrak{so}(1,3)$-valued spin connection and the
$\mathfrak{t}$-valued vierbein are combined into an
$\mathfrak{iso}(1,3)$-valued Cartan connection, thereby defining 
a Riemann-Cartan geometry~\cite{Hehl:1976grs}. It is explained 
comprehensibly in~\cite{Wise:2010sm} how the 
$\mathfrak{iso}(1,3)$-valued connection gives a prescription for 
\emph{rolling without slipping} the affine Minkowski space along 
the integral curves of vector fields on spacetime.
It is indeed the central idea behind Cartan geometry, to be 
reviewed in Sec.~\ref{sec:overview_cartan_geo}, that a 
homogeneous model space is generalized to a nonhomogeneous space, 
for which the local structure is algebraically isomorphic to the 
one of the model space~\cite{sharpe1997diff_geo}, and where the 
degree of nonhomogeneity is quantified by the presence of 
curvature and torsion.  In the manner thus explained, the choice 
for a Riemann-Cartan geometry, i.e.,~a Cartan connection valued 
in the algebra $\mathfrak{iso}(1,3)$, to describe spacetimes 
underlying theories of gravity is implied by the equivalence 
principle, together with the assumption that the local kinematics 
are governed by the Poincar\'e group.

When the $\mathfrak{iso}(1,3)$-valued connection of a 
Riemann-Cartan geometry is replaced by a Cartan connection that 
is valued in the de~Sitter algebra $\mathfrak{so}(1,4)$, 
spacetime is locally approximated by de~Sitter space $dS$ in 
place of the affine Minkowski space. Such a structure will be 
called a de~Sitter-Cartan spacetime in the remainder of this 
work. In fact, since the vierbein is valued in the space of 
de~Sitter transvections $\mathfrak{p} = 
\mathfrak{so}(1,4)/\mathfrak{so}(1,3)$, it solders the algebraic 
structure of $dS$ to the tangent bundle of spacetime. The 
Cartan-Killing metric on $\mathfrak{p}$ is once again the 
diagonal Minkowski matrix, so that a Lorentzian metric is 
introduced on $\mathcal{M}$.  Different from a Riemann-Cartan 
geometry, however, is that translations in spacetime are 
generated by de~Sitter transvections, because the space of 
infinitesimal translations at any point $x$, i.e., 
$T_x\mathcal{M}$, is isomorphic to $\mathfrak{p}$. This implies 
that the commutator of infinitesimal translations is proportional 
to a Lorentz rotation.  The constant of proportionality is 
essentially the cosmological constant of the tangent de~Sitter 
spaces~\cite{Wise:2010sm}, see also Sec.~\ref{sec:lin_dSC_geo}.  
It is then sensible to identify this geometric cosmological 
constant with the dark energy on spacetime. Such an 
interpretation is in concordance with the MacDowell-Mansouri 
model for gravity~\cite{MacDowell1977}. In this model, the 
fundamental field  is indeed a $\mathfrak{so}(1,4)$-valued Cartan 
connection, for which the action is equivalent, up to topological 
terms, with the Palatini action for general relativity in the 
presence of a cosmological 
constant~\cite{Wise:2010sm,Westman:2012xk}. Whatever may be its 
nature, the cosmological constant thus shows up in the kinematics 
of spacetime, where it is a measure for the noncommutativity of 
infinitesimal spacetime translations.

At any point in a de~Sitter-Cartan spacetime, the cosmological 
constant is related to a length scale defined in the commutation 
relations of the de~Sitter transvections. Therefore, it is rather 
straightforward to generalize to geometries in which this length 
scale becomes a nonconstant function on spacetime. In 
Sec.~\ref{sec:lin_dSC_geo}, we claim some originality for 
constructing a de~Sitter-Cartan geometry that provides spacetime 
with a \emph{cosmological function} $\Lambda(x)$, which in 
general does not satisfy $d\Lambda = 0$. We shall see that a 
nonconstant $\Lambda$ gives rise to a new term in the expression 
for the torsion of the de~Sitter-Cartan geometry.
The cosmological function could possibly serve to model dark 
energy that changes along space and time, in which way it might 
give an alternative description for---or interpretation of---one 
of the models for time-evolving dark 
energy~\cite{Peebles:2003cc,Copeland:2006de}. Naturally, an 
adequate action for gravity will have to be defined, which 
determines whether a spacetime-dependent cosmological function is 
possible and how its value at any point can be accounted for. In 
the present paper, we discuss the kinematics only and will not be 
concerned with the important issue of specifying for the function 
$\Lambda$ dynamically. Of course, since Einstein's equations 
render a constant $\Lambda$, general relativity is not 
satisfactory and a generalization will have to be tried for. In 
view of this problem, we argue that it may be helpful to discuss 
the geometry of spacetimes with a generic cosmological function, 
for it reveals what quantities are available to construct 
possible geometric actions. 

The organization of this paper is as follows. In 
Sec.~\ref{sec:overview_cartan_geo}, an attempt to review Klein 
and Cartan geometry concisely is undertaken. This will clarify 
why Cartan connections are adequate to describe the geometry of 
theories of gravity, in favor of the better known 
Ehresmann connections. The specific case of a de~Sitter-Cartan 
spacetime is reviewed in Sec.~\ref{sec:lin_dSC_geo}, and is 
extended to incorporate a nonconstant cosmological 
function~$\Lambda$.  From our discussion in 
Sec.~\ref{sec:overview_cartan_geo}, it furthermore will become 
clear that a $\mathfrak{so}(1,4)$-valued Cartan connection with 
structure group $SO(1,3)$ can be obtained from a 
$\mathfrak{so}(1,4)$-valued Ehresmann connection, for which the 
structure group $SO(1,4)$ is reduced to its Lorentz subgroup.  
This observation will clarify how the broken symmetries can be 
recovered, while preserving the well-defined existence of a spin 
connection and vierbein, if the Cartan connection is defined 
through an Ehresmann connection that is realized nonlinearly. It 
will be the subject of Sec.~\ref{sec:nonlin_dSC_geo} to construct 
such a $SO(1,4)$-invariant de~Sitter-Cartan geometry, again for a 
nonconstant cosmological function. We conclude in 
Sec.~\ref{sec:conclusion}.
 
\section{Cartan connections and the infinitesimal symmetry of 
	spacetime}
\label{sec:overview_cartan_geo}

Let $H$ and $G$ be Lie groups and assume there is an injective 
group homomorphism between the same, namely,
\begin{equation}
	\label{eq:inclHG}
	i : H \to G,
\end{equation}
for which $i(H)$ is a subgroup of $G$. Furthermore, let $i(H)$ be 
isomorphic to $H$, in which case it is said that $H \subset G$ is 
a subgroup. In general, the inclusion~\eqref{eq:inclHG} is 
possible in different ways, hence the short exact sequence
\begin{displaymath}
\begin{tikzcd}
	e \rar & H \rar{i} & G \rar & \dfrac{G}{i(H)} \rar & e
\end{tikzcd}
\end{displaymath}
is not canonically given. As any inclusion is supposed to be 
isomorphic to $H$, they are naturally isomorphic to each other and 
one usually denotes each of the inclusions $i(H)$ by $H$, a habit 
we shall follow.  There is nevertheless a reason to have put 
emphasis on the existence of different, albeit isomorphic, 
inclusions~\eqref{eq:inclHG}.

These mathematically equivalent subgroups in $G$ may have 
different interpretations from a concrete, physical point of 
view.  An example of interest to us is found in the so-called 
Klein geometries \cite{sharpe1997diff_geo,Klein:1893}.  These 
geometries describe homogeneous spaces from a Lie group-theoretic 
perspective, roughly as follows. Let $G$ be the symmetry group of 
a homogeneous space $S$ and fix an arbitrary point $\xi \in S$.  
The subgroup of elements in $G$ that leave $\xi$ invariant is 
denoted by $H_\xi$ and is called the isotropy group of the point 
in question. On account of the transitivity of $S$, there exists 
an isomorphism between $S$ and $G/H_\xi$, by identifying a point 
$\zeta = g(\xi)$ in the former with an element $gH_\xi$ in the 
latter. It is important to note that although it is necessary to 
choose a $\xi$ in order to establish the isomorphism, the choice 
in itself is nonetheless completely arbitrary. If another point 
$\xi' = a(\xi)$ were chosen, $S$ could similarly be identified 
with $G/H_{\xi'}$.  The group $H_{\xi'}$ is related to the 
isotropy group of $\xi$ through the adjoint action, i.e.,
\begin{displaymath}
	H_{\xi'} = \Ad(a)(H_\xi) = a H_\xi a^{-1}.
\end{displaymath}
This is true for any two points in $S$, so that the isotropy 
groups of different points are isomorphic to each other. In this 
way, they form different but equivalent 
inclusions~\eqref{eq:inclHG} of a group $H$ in $G$. Examples are 
given for the affine group in~\cite{petti:2006a} and the 
de~Sitter group in $D$ dimensions in~\cite{gibbons:2009b}, of 
which in particular the latter for $D=4$ is of interest for the 
sections below.

A \emph{Klein geometry} is then unambiguously denoted by such a 
pair $(G,H)$ for which the homogeneous space $G/H$ is connected 
\cite{sharpe1997diff_geo}. Let $\mathfrak{g}$ and $\mathfrak{h}$ 
be the Lie algebras of $G$ and $H$, respectively.  Throughout the 
text it is assumed that $\mathfrak{g}$ is reductive, to wit, 
there is a direct sum decomposition as vector spaces
\begin{equation}
	\label{eq:red_split}
	\mathfrak{g} = \mathfrak{h} \oplus \mathfrak{p},
\end{equation}
which is $\Ad(H)$-invariant, thereby implying that 
$[\mathfrak{h},\mathfrak{h}] \subseteq \mathfrak{h}$ and 
$[\mathfrak{h},\mathfrak{p}] \subseteq \mathfrak{p}$. Apart from 
the homogeneous space $S = G/H$, there are two more structures 
associated with a Klein geometry that are worth mentioning in 
view of what will follow \cite{sharpe1997diff_geo}.
The natural projection of $G$ onto its right cosets gives rise to 
a principal bundle $\pi : G \to G/H$ with typical fibre $H$.  
Secondly, there is a $\mathfrak{g}$-valued differential form on 
the bundle space, that is to say, the Maurer-Cartan form 
$\omega_G$ of the Lie group $G$. Being a property of generic 
Maurer-Cartan forms, it follows that its exterior covariant 
derivative vanishes \cite{kob1996found}, i.e.,
\begin{equation}
	\label{eq:MC_eq}
	d\omega_G + \tfrac{1}{2}[\omega_G,\omega_G] = 0.
\end{equation}

The Lie-theoretic point of view on homogeneous spaces offered by 
Klein geometry is generalized to nonhomogeneous manifolds in 
Cartan geometry~\cite{Cartan:1926gh}. A given Cartan geometry is 
said to be modeled on some Klein geometry, since the base 
manifold associated with the former is tangentially approximated 
by the homogeneous space of the latter. Moreover, if the Cartan 
geometry is flat, the manifold is locally the same as the 
corresponding model space. In the following paragraphs, we give a 
short review of the subject. A detailed mathematical treatment 
can be found in~\cite{sharpe1997diff_geo,Alekseevsky:1995cc}, 
while the articles~\cite{Wise:2010sm,Westman:2012xk,Wise:2009fu} 
are very helpful in developing an intuition that goes along with 
the mathematics of Cartan geometry.

Let $P(\mathcal{M},H)$ be a principal bundle $\pi : P \to \mathcal{M}$ with typical 
fibre $H$. A Cartan geometry modeled on the pair 
$(\mathfrak{g},\mathfrak{h})$ is obtained by introducing a 
\emph{Cartan connection} on $P$, which is defined as 
follows~\cite{sharpe1997diff_geo}:
\begin{definition}
	\label{def:cartan_conn}
	A Cartan connection $A$ is a $\mathfrak{g}$-valued 1-form on 
	$P$ that satisfies the conditions:
	\begin{itemize}
		\item[(i)] for each $p \in P$, the linear map $A_p : 
			T_pP \to \mathfrak{g}$ is an isomorphism;
		\item[(ii)] $A(\zeta_X) = X$, for each fundamental vector 
			field $\zeta_X$ corresponding to $X \in \mathfrak{h}$;
		\item[(iii)] $R_h^\ast A = \mathrm{Ad}(h^{-1}) \cdot 
			A$ for each $h \in H$.
	\end{itemize}
\end{definition}
\noindent
From the first of these properties one is able to conclude that 
$\mathcal{M}$ and $G/H$ are equal in dimension. This is an important 
observation, for it lies at the heart of the usefulness of Cartan 
connections to describe theories of gravity. The isomorphism 
hints at the existence of an object---the vielbein---that takes 
the geometry of the homogeneous space to the tangent structure of 
spacetime $\mathcal{M}$.

The Cartan curvature of the connection $A$ is defined as its 
exterior covariant derivative, i.e., the $\mathfrak{g}$-valued 
2-form
\begin{equation}
	\label{eq:def_Ccurvature}
	F = dA + \tfrac{1}{2} [A,A].
\end{equation}
The Cartan curvature is strictly horizontal in the sense that it 
annihilates vertical vector fields on the principal bundle.  This 
is a consequence of the second defining property of the Cartan 
connection, which implies that $A$ restricts to the Maurer-Cartan 
form $\omega_H$ along the fibres of 
$P$~\cite{sharpe1997diff_geo}. If the curvature of a Cartan 
connection vanishes, the corresponding geometry is said to be 
$\emph{flat}$. An example of a flat Cartan geometry modeled on 
$(\mathfrak{g},\mathfrak{h})$ is the Klein geometry $(G,H)$.  
Indeed, it can be verified rather easily that the Maurer-Cartan 
form on the principal bundle $\pi : G \to G/H$ satisfies the 
defining properties of a Cartan connection. That it is a flat 
connection follows from its structural equation~\eqref{eq:MC_eq}.  
Conversely, the spacetime $\mathcal{M}$ of a flat Cartan geometry 
modeled on $(\mathfrak{g},\mathfrak{h})$ is in the neighborhood 
of any point of $\mathcal{M}$ isomorphic to the Klein geometry 
$G/H$, see~\cite{sharpe1997diff_geo}. The curvature $F$ is thus a 
measure for the nonhomogeneity of $\mathcal{M}$, in comparison 
with the perfect homogeneity of the Klein space $G/H$.

The third defining property of a Cartan connection determines the 
$H$-equivariance of $A$ and $F$ under local $H$-transformations.  
Because the Cartan connection and its curvature are valued in a 
reductive Lie algebra, it is sensible to consider their 
projections according to the splitting~\eqref{eq:red_split}, 
which we denote by
\begin{equation}
	\label{eq:red_split_AF}
	A = A_\mathfrak{h} + A_\mathfrak{p}
	\quad\text{and}\quad
	F = F_\mathfrak{h} + F_\mathfrak{p}.
\end{equation}
Naturally, all projections are $H$-equivariant. The objects 
$A_\mathfrak{p}$, $F_\mathfrak{h}$ and $F_\mathfrak{p}$ are 
horizontal, while the $\mathfrak{h}$-valued differential form 
$A_\mathfrak{h}$ is an Ehresmann 
connection~\cite{Alekseevsky:1995cc}. If the Lie algebra 
$\mathfrak{g}$ is symmetric, i.e.,~it is reductive and 
$[\mathfrak{p},\mathfrak{p}] \subseteq \mathfrak{h}$, the 
definition~\eqref{eq:def_Ccurvature} allows one to express 
$F_\mathfrak{h}$ and $F_\mathfrak{p}$ in terms of 
$A_\mathfrak{h}$ and $A_\mathfrak{p}$, namely,
\begin{subequations}
	\label{eqs:curv_tors_expl}
\begin{align}
	\label{eq:curv_expl}
	F_\mathfrak{h} &= dA_\mathfrak{h} + \tfrac{1}{2} 
	[A_\mathfrak{h}, A_\mathfrak{h}]  + \tfrac{1}{2}
	[A_\mathfrak{p},A_\mathfrak{p}],
	\\
	\label{eq:tors_expl}
	F_\mathfrak{p} &= dA_\mathfrak{p} + 
	[A_\mathfrak{h},A_\mathfrak{p}].
\end{align}
\end{subequations}
The $\mathfrak{h}$-valued part $F_\mathfrak{h}$ is called the 
curvature of the geometry. Note that this in general is not the 
same as the exterior covariant derivative of the Ehresmann 
connection~$A_\mathfrak{h}$, which is given by $dA_\mathfrak{h} + 
\tfrac{1}{2} [A_\mathfrak{h}, A_\mathfrak{h}]$ only. The 
$\mathfrak{p}$-component of the Cartan curvature is called the 
\emph{torsion} of the geometry.

For the reason that the decompositions~\eqref{eq:red_split_AF} 
are invariant under the action of local $H$-transformations, the 
corresponding projections of $A$ and $F$ are well-defined 
geometric objects. In the next section, for example, a Cartan 
geometry for the de~Sitter algebra will be constructed, in which 
the spin connection and vierbein, and the curvature and torsion 
do not mix up under local Lorentz transformations. On the other 
hand, under local $G$-transformations the $\mathfrak{h}$- and 
$\mathfrak{p}$-valued parts of the Cartan connection and 
curvature will transform into each other. In case the local 
symmetry group is wished to be extended to $G$, the thus defined 
objects do not have a geometric meaning, as their construction is 
not $G$-invariant. The existence of a spin connection and 
vierbein is nonetheless mandatory to describe theories of 
gravity, so that it is necessary to maintain the 
decompositions~\eqref{eq:red_split_AF} while extending the 
symmetry group from~$H$ to~$G$.

To understand how these at first sight conflicting objectives can 
be reconciled, it is convenient to view a Cartan connection on a 
principal $H$-bundle as an Ehresmann connection on a principal 
$G$-bundle, for which the symmetry group $G$ is broken to its 
subgroup $H$, see also~\cite{Wise:2011-sym.br,gibbons:2009b}.  
Mathematically speaking, this corresponds to a reduction of the 
principal $G$-bundle to the principal $H$-bundle. In view of 
this, it is useful to point to the existence of the following 
proposition, a proof of which can be found in,
e.g.,~\cite{husemoller:1966fibre}.
\begin{proposition}
	A principal $G$-bundle $Q$ is reducible to a principal 
	$H$-bundle $P$ if and only if the associated bundle $Q[S] = Q 
	\times_G S$ of homogeneous spaces $S \cong G/H$ admits a 
	globally defined section.
\end{proposition}
\noindent
Locally, the section $\xi$ alluded to in the proposition singles 
out a point $\xi(x)$ in each of the homogeneous spaces $S_x$ over 
the points $x \in \mathcal{M}$. As a result, the symmetry group 
$G$ of $S_x$ is broken down to $H_{\xi(x)}$. This restriction of 
the structure group to $H_{\xi(x)} \cong H$ essentially 
constitutes the reduction of $Q$ to $P$.  Concomitantly, an 
Ehresmann connection on $Q$ gives way to a Cartan connection on 
the reduced $H$-bundle $P$~\cite{sharpe1997diff_geo}, for which 
the reductive splittings~\eqref{eq:red_split_AF} make sense. The 
resulting geometry is once more the above considered Cartan 
geometry on a principal $H$-bundle. From the construction just 
discussed, however, it is obvious that the symmetries of $G$ that 
are broken upon reduction are merely hidden. This is because the 
section $\xi$ that has been singled out for the reduction is 
arbitrary, very much similar to the arbitrariness mentioned of in 
the above discussed Klein geometries.  Different but equivalent 
sections are related by $G$-transformations that are not 
everywhere in $H$, i.e.,~they are connected through the broken 
symmetries. The latter can be incorporated in the principal 
$H$-bundle by nonlinearly realizing them as local 
$H$-transformations, see~\cite{Stelle:1979va,stelle.west:1980ds} 
and Sec.~\ref{sec:nonlin_dSC_geo}. Since the broken symmetries 
are realized as elements of $H$, local $G$-invariance is 
restored, and at the same time there is a meaningful reductive 
decomposition of the nonlinear Cartan connection and curvature, 
i.e.,~the presence of a spin connection and vierbein, as well as 
curvature and torsion tensors. These objects will be constructed 
in Sec.~\ref{sec:nonlin_dSC_geo} on $SO(1,4)$-invariant 
de~Sitter-Cartan geometry, in a way that is compatible with a 
nonconstant cosmological function on spacetime.

\section{de~Sitter-Cartan geometry with a cosmological function}
\label{sec:lin_dSC_geo}

A de~Sitter-Cartan geometry is the Cartan geometry that is 
modeled on $(\mathfrak{so}(1,4),SO(1,3))$ and thus consists of a 
principal Lorentz bundle $P(\mathcal{M},SO(1,3))$ over spacetime, 
on which is defined a $\mathfrak{so}(1,4)$-valued Cartan 
connection $A$. This connection provides spacetime with the 
information that it is tangentially approximated by de~Sitter 
spaces, see~Sec.~\ref{sec:overview_cartan_geo}. We shall 
construct a de~Sitter-Cartan geometry, in which these tangent 
de~Sitter spaces have cosmological constants that are not 
required to be the same along spacetime.  As a consequence, the 
thus obtained geometry describes a spacetime on which a 
cosmological function $\Lambda(x)$ is defined from the onset.  We 
shall see that the possibility for spacetime-dependent internal 
de Sitter pseudo-radii gives rise to a contribution to the 
torsion of the geometry, which disappears if $\Lambda$ is 
constant. 

In the preceding section, the Cartan connection was defined on 
the principal bundle $P$. As for the remainder of the text, on 
the other hand, it is understood that this connection is pulled 
back to a connection on spacetime by some local section $\sigma : 
U \subset \mathcal{M} \to P$. We prefer not to introduce new 
symbols and therefore recycle notation by denoting the pulled 
back connection $\sigma^\ast A$ also by $A$. The same is true for 
its exterior covariant derivative $F$.

Under the action of a local $SO(1,3)$-transformation, the 
de~Sitter-Cartan connection transforms according 
to~\cite{sharpe1997diff_geo}
\begin{equation}
	A \mapsto \Ad(h(x)) ( A + d ),
\end{equation}
which is the natural nonhomogeneous transformation behavior for 
connections. The $1$-form $A$ is valued in the de~Sitter algebra 
$\mathfrak{so}(1,4)$, which is characterized by the commutation 
relations
\begin{gather}
\label{eq:comm_relations_so(1,4)}
\begin{aligned}
	-i[M_{ab},M_{cd}] &= \eta_{ac}M_{bd} - \eta_{ad}M_{bc} + 
	\eta_{bd}M_{ac} - \eta_{bc}M_{ad}, \\
	-i[M_{ab},P_c] &= \eta_{ac}P_b- \eta_{bc}P_a, \\
	-i[P_a,P_b] &= -l^{-2}M_{ab}.
  \end{aligned}
\end{gather}
The convention for the Minkowski metric is $\eta_{ab} = 
(+,-,-,-)$, while we parametrize an element of 
$\mathfrak{so}(1,4)$ by $\tfrac{i}{2} \lambda^{ab} M_{ab} + i 
\lambda^a P_a$.
The generators of de Sitter transvections are defined by $P_a = 
M_{a4}/l$, where $l$ is an \emph{a-priori} arbitrary length scale 
that effectively determines the cosmological constant of the 
corresponding Klein geometry $dS = SO(1,4)/SO(1,3)$, 
namely,~\cite{Wise:2010sm}
\begin{equation}
	\label{eq:rel.Ll}
	\Lambda = \frac{3}{l^2}.
\end{equation}

At any point in spacetime, the Cartan connection is valued in a 
copy of the de~Sitter algebra, thereby fixing the cosmological 
constant of the tangent de~Sitter space. Because we wish the set 
of these cosmological constants to constitute a generic 
nonconstant function of spacetime, the length scales $l(x)$ 
should equally be considered spacetime-dependent.  In the 
following paragraphs, we have a look at what this assumption 
implies for the geometry.

From the brackets~\eqref{eq:comm_relations_so(1,4)} one concludes 
that the algebra is symmetric. The reductive splitting 
schematically reads as
\begin{equation}
	\label{eq:red_split_so(1,4)}
	\mathfrak{so}(1,4) = \mathfrak{so}(1,3) \oplus \mathfrak{p},
\end{equation}
where $\mathfrak{so}(1,3) = \mathrm{span}\{M_{ab}\}$ is the 
Lorentz subalgebra and $\mathfrak{p} = \mathrm{span}\{P_a\}$ the 
subspace of infinitesimal de~Sitter transvections, or de~Sitter 
translations.
The corresponding decompositions~\eqref{eq:red_split_AF} of the 
Cartan connection and curvature are denoted by
\begin{equation}
	\label{eq:red_split_AF_dSC}
	A = \tfrac{i}{2} A^{ab} M_{ab} + iA^a P_a
	\quad\text{and}\quad
	F = \tfrac{i}{2} F^{ab} M_{ab} + iF^a P_a,
\end{equation}
from which one concludes that $A^a$ and $F^a$ have the dimension 
of length. The $\mathfrak{so}(1,3)$-valued $1$-form $A^{ab}$ is 
an Ehresmann connection for local Lorentz 
transformations~\cite{Alekseevsky:1995cc}, i.e.,~a spin 
connection, while the forms $A^a$ constitute a vierbein. To any 
tangent vector $X^\mu$ at some event in spacetime, which singles 
out a direction at that point, it relates an infinitesimal 
de~Sitter translation $A\ind{^a_\mu} X^\mu$ in $\mathfrak{p}$.  
As we have mentioned in the introductory section, this gives a 
precise meaning to the statement that motion in a 
de~Sitter-Cartan spacetime is generated by de~Sitter 
translations, and that spacetime is locally approximated by a 
de~Sitter space. Let us emphasize once more that the 
decompositions~\eqref{eq:red_split_AF_dSC} are well defined, 
since local Lorentz transformations leave the reductive 
splittings invariant.  Due to the presence of the spin connection 
and vierbein, it is possible to define local Lorentz and 
diffeomorphism covariant differentiation, as well as a metric 
structure on spacetime; see~\cite{Ortin:2004}, for example.

Given the commutation
relations~\eqref{eq:comm_relations_so(1,4)}, one is able to 
compute the curvature~$F^{ab}$ and the torsion~$F^{a}$ in terms 
of the spin connection and vierbein. According to the 
expressions~\eqref{eqs:curv_tors_expl}, it follows that
\begin{subequations}
	\label{eqs:curvtors_dSC}
\begin{align}
	\label{eq:curv_dSC}
	F^{ab} &= dA^{ab} + A\ind{^a_c} \wedge A^{cb} + \frac{1}{l^2} 
	A^a \wedge A^b \\
	\nonumber
	&= d_A A^{ab} + \frac{1}{l^2} A^a \wedge A^b,
	\\
	\label{eq:tors_dSC}
	F^a &= dA^a + A\ind{^a_b} \wedge A^b - \frac{1}{l} dl \wedge 
	A^a 
	\\
	\nonumber
	&= d_A A^a - \frac{1}{l} dl \wedge A^a.
\end{align}
\end{subequations}
In these equations, the exterior covariant derivatives of the 
spin connection and vierbein with respect to the spin connection 
are denoted by $d_A A^{ab}$ and $d_A A^a$, respectively.  Some 
further remarks concerning these results are in place.  Note that 
in the limit of an everywhere diverging length scale $l$, or in 
other words, an everywhere vanishing cosmological constant, the 
expressions~\eqref{eqs:curvtors_dSC}
reduce to the curvature~$d_A A^{ab}$ and torsion~$d_A A^a$ for a 
Riemann-Cartan geometry~\cite{Ortin:2004}. In the generic case, 
however, the curvature and torsion are not given by the exterior 
covariant derivatives of the spin connection and vierbein. The 
extra term in~\eqref{eq:curv_dSC} represents the curvature of the 
local de Sitter space. This contribution is present because the 
commutator of two infinitesimal de~Sitter transvections equals an 
element of the Lorentz algebra. In addition, there is a new term 
in the expression~\eqref{eq:tors_dSC} for the torsion if the 
length scale is a nonconstant function.  This term comes about as 
follows.  Remember from equation~\eqref{eq:tors_expl} that 
torsion is the $\mathfrak{p}$-valued $2$-form $dA_\mathfrak{p} + 
[A_\mathfrak{h},A_\mathfrak{p}]$. The first term in this 
expression really means
\begin{displaymath}
	dA_\mathfrak{p} = d(i A^a P_a) 
	= i\, dA^a P_a - i \bigg(\frac{dl}{l} \wedge A^a \bigg) P_a,
\end{displaymath}
since $P_a = M_{a4}/l$ and $l$ is generally not a constant 
function on $\mathcal{M}$.  By use of the 
relation~\eqref{eq:rel.Ll} between $l$ and the cosmological 
function $\Lambda$, the last term of the torsion can be expressed 
as
\begin{displaymath}
	- d\ln l \wedge A^a = \tfrac{1}{2} d\ln\Lambda \wedge A^a,
\end{displaymath}
which shows that this contribution depends on the relative 
infinitesimal change of the cosmological function along 
spacetime, rather than on its absolute change.

Although the curvature and torsion have contributions that are 
not there for a Riemann-Cartan geometry, the Bianchi identities 
are unchanged, as they are given by
\begin{subequations}
	\label{eqs:lin_Bianchi}
	\begin{align}
	\label{eq:lin_Bianchi1}
	d_A \circ d_A A^{ab} 
	%= d\circ d_A A^{ab} + A\ind{^a_c} \wedge d_A A^{cb} + 
	%A\ind{^b_c} \wedge d_A A^{ac}
	&\equiv 0,
	\\
	\label{eq:lin_Bianchi2}
	d_A \circ d_A A^a + A^b \wedge d_A A\ind{_b^a} 
	%= d\circ d_A A^a + A\ind{^a_b} \wedge d_A A^b + A^b \wedge 
	%d_A A\ind{_b^a}
	&\equiv 0,
	\end{align}
\end{subequations}
where $d_A$ is the exterior covariant derivative with respect to 
the spin connection.

The de~Sitter-Cartan geometry hitherto discussed describes a 
spacetime that is approximated at any point $x$ by a de~Sitter 
space of cosmological constant $\Lambda(x)$. The transformations 
that are defined for this geometry are local Lorentz 
transformations and spacetime diffeomorphisms.  In general, the 
latter are unphysical in the sense that they just relabel 
spacetime coordinates, whereas the former have physical 
significance in that they relate at any point the reference 
frames of different idealized observers, see 
also~\cite{Edelstein:2006a}.
In contrast, generic local $SO(1,4)$-transformations are not in 
the structure group of symmetries for the de~Sitter-Cartan 
geometry. This does not come as a surprise, since the Cartan 
connection~$A$ can be seen as a $SO(1,4)$ Ehresmann connection, 
for which the structure group is restricted to the subgroup 
$SO(1,3)$.  This breaking of symmetry is necessary to render the 
decompositions~\eqref{eq:red_split_AF_dSC} invariant under the 
action of the structure group $SO(1,3)$.  Hence, the definitions 
of the spin connection, vierbein, curvature, and torsion are 
consistent with local Lorentz rotations, but not with local 
$SO(1,4)$-transformations.  For example, a local infinitesimal 
\emph{pure} de~Sitter translation $1 + i\epsilon\cdot P$ leads to 
the following variations of the spin connection and vierbein,
\begin{align*}
	\delta_\epsilon A^{ab} &= \frac{1}{l^2}(\epsilon^a A^b - 
	\epsilon^b A^a),
	\\
	\delta_\epsilon A^a &= -d\epsilon^a - A\ind{^a_b}\epsilon^b + 
	\frac{dl}{l}\epsilon^a,
\end{align*}
while for the curvature and torsion it is found that 
\begin{displaymath}
	\delta_\epsilon F^{ab} = \frac{1}{l^2}(\epsilon^a F^b - 
	\epsilon^b F^a)
	\quad\text{and}\quad
	\delta_\epsilon F^a = -\epsilon^b F\ind{^a_b}.
\end{displaymath}
It is manifest that the objects $A^{ab}$, $A^a$, $F^{ab}$, and 
$F^a$ are well defined, only up to local Lorentz transformations.
Therefore, if a theory of gravity is to be invariant under local
de~Sitter transformations, the geometric objects defined through 
the decompositions~\eqref{eq:red_split_AF_dSC} of the $SO(1,4)$ 
Ehresmann connection and curvature, corresponding to a 
restriction of the structure group to $SO(1,3)$, do not lead to 
the necessary structure. It is the subject of the next section to 
preserve the presence of these geometric objects, but whose 
definition will be consistent, not only with local 
$SO(1,3)$-transformations, but also with elements of the 
encompassing de~Sitter group~$SO(1,4)$.


\section{$\textbf{SO(1,4)}$-invariant de~Sitter-Cartan geometry 
	with a cosmological function}
\label{sec:nonlin_dSC_geo}

In the previous section, the definitions for the spin connection 
and vierbein, as well as for the curvature and torsion, are not 
consistent with theories of gravity in which local de~Sitter 
invariance is demanded. We have pointed out repeatedly that this 
is due to $\mathfrak{so}(1,4)$'s being reductive with respect to 
the Lorentz group only. Nevertheless is it possible to extend the 
structure group from $SO(1,3)$ to $SO(1,4)$, while preserving a 
well-defined reductive splitting of the Cartan connection and its 
curvature, if they are realized nonlinearly.

The formalism of nonlinear realizations was developed to 
systematically study spontaneous symmetry breaking in 
phenomenological field 
theory~\cite{Coleman:1969sm,Callan:1969sn,Volkov:1973vd}, in 
which linearly transforming irreducible multiplets become 
nonlinear but reducible realizations, when the symmetry group is 
broken to one of its subgroups. The realization is reducible 
because the broken symmetries are realized through elements of 
the given subgroup, see below.

A Cartan connection on a principal Lorentz bundle may be thought 
of as an Ehresmann connection on a principal $SO(1,4)$-bundle 
that is reduced to the $SO(1,3)$-bundle. At the end of 
Sec.~\ref{sec:overview_cartan_geo} it was argued that this is 
essentially a symmetry breaking process, for the reason that it 
corresponds to singling out a section $\xi$ of the associated 
bundle of tangent de Sitter spaces, thereby reducing the 
structure group $G$ to $H_\xi$~\cite{Wise:2011-sym.br}.  Stelle 
and West~\cite{stelle.west:1980ds} invoked nonlinear realizations 
to extend the symmetry group of the MacDowell-Mansouri action for 
general relativity from the Lorentz to the (anti-)de~Sitter 
group. In their work, the internal de~Sitter spaces were assumed 
to have the same curvature at all spacetime points. Here, this 
assumption is relaxed and we construct a geometry for which the 
tangent de~Sitter spaces have pseudo-radii that vary along 
spacetime. To begin with, let us review the necessary facts on 
nonlinear realizations for the de~Sitter group, see 
also~\cite{Zumino1977189,wess:1992ss}.

Within some neighborhood of the identity, an element $g$ of 
$SO(1,4)$ can uniquely be represented in the form
\begin{displaymath}
	g = \exp(i\xi\cdot P) \tilde{h},
\end{displaymath}
with $\tilde{h} \in SO(1,3)$ and $\xi\cdot P = \xi^a P_a$.
The $\xi^a$ parametrize the coset space $SO(1,4)/SO(1,3)$ so that 
they constitute a coordinate system for de~Sitter space. This 
parametrization allows us to define the action of $SO(1,4)$ on 
de~Sitter space by
\begin{displaymath}
	%\label{eq:left_action_group}
	g_0 \exp(i\xi\cdot P) = \exp(i\xi'\cdot P)h'~;
	\quad
	h' = \tilde{h}'\tilde{h}^{-1},
\end{displaymath}
where $\xi'=\xi'(g_0,\xi)$ and $h'=h'(g_0,\xi)$ are in general 
nonlinear functions of the indicated variables. In case $g_0 = 
h_0$ is an element of $SO(1,3)$, the action is linear since
\begin{displaymath}
	\exp(i\xi'\cdot P) = h_0\exp(i\xi\cdot P)h_0^{-1}~;
	\quad
	h' = h_0,
\end{displaymath}
and the transformation of $\xi$ is given explicitly by
\begin{displaymath}
	h_{0}: i\xi\cdot P \mapsto i\xi'\cdot P = i\xi\cdot 
	\mathrm{Ad}(h_0)(P).
\end{displaymath}
If on the other hand $g_0 = \exp(i\alpha\cdot P)$ is a pure 
de~Sitter translation, the coordinates $\xi$ change according to 
\begin{displaymath}\label{eq:left_action_coset}
	\exp(i\alpha\cdot P)\exp(i\xi\cdot P) = \exp(i\xi'\cdot P)h'.
\end{displaymath}
For an infinitesimal translation $1 + i\epsilon\cdot P$ this 
becomes
\begin{multline}
	\label{eq:nonlin_trafo_inf}
	\exp(-i\xi\cdot P) i\epsilon\cdot P \exp(i\xi\cdot P) \\
	- \exp(-i\xi\cdot P) \delta\!\exp(i\xi\cdot P) = \tfrac{i}{2} 
	\delta h \cdot M,
\end{multline}
where $\tfrac{i}{2} \delta h\cdot M = h'-1 \in 
\mathfrak{so}(1,3)$ and $\delta h\cdot M = \delta h^{ab}M_{ab}$.
Equation~\eqref{eq:nonlin_trafo_inf} determines the 
variations~$\delta \xi^a$ and $\delta h^{ab}$ that are generated 
by the considered infinitesimal de~Sitter translation. They are 
given by~\cite{stelle.west:1980ds}
\begin{align}
	\label{eq:inf_tr_xi}
	\delta\xi^a &= \epsilon^a + \Big(\frac{z\cosh z}{\sinh z} - 
	1\Big) \bigg(\epsilon^a - \frac{\xi^a \epsilon_b 
		\xi^b}{\xi^2}\bigg), \\
	\label{eq:inf_tr_h}
	\delta h^{ab} &= \frac{1}{l^2} \frac{\cosh z - 1}{z\sinh z} 
	(\epsilon^a\xi^b - \epsilon^b\xi^a),
\end{align}
where we made use of the notation $z = l^{-1} \xi$ and $\xi = 
(\eta_{ab} \xi^a \xi^b)^{1/2}$.

Subsequently, let $\psi$ be a field that belongs to some linear 
representation $\sigma$ of $SO(1,4)$. Given a local section of 
the associated bundle of homogeneous de~Sitter spaces, i.e.,~$\xi: 
U \subset \mathcal{M} \to U \times dS$, the corresponding 
nonlinear field is constructed pointwise as 
\begin{equation}
	\label{eq:def_nonlinear_field}
	\bar{\psi}(x) = \sigma(\exp(-i\xi(x)\cdot P))\psi(x).
\end{equation}
Under a local de~Sitter transformation $g_0$, it rotates only 
according to its $SO(1,3)$-indices, namely,
\begin{equation}
	\label{eq:trafo_nonlin_field}
	\bar{\psi}'(x) = \sigma(h'(\xi,g_0)) \bar{\psi}(x).
\end{equation}
It is manifest that the irreducible linear representation~$\psi$ 
has given way to a nonlinear and reducible 
realization~$\bar{\psi}$.  Note that upon restricting the action 
of $SO(1,4)$ to the Lorentz subgroup, the 
transformation~\eqref{eq:trafo_nonlin_field} remains linear.  Let 
us next make use of this general framework to construct a 
nonlinear de~Sitter-Cartan geometry.  More precisely, we realize 
a linear $SO(1,4)$ Ehresmann connection nonlinearly, which leads 
to a $\mathfrak{so}(1,4)$-valued Cartan connection that 
transforms with respect to its $SO(1,3)$-indices only, when acted 
upon by local de~Sitter transformations.

In concordance with the 
prescription~\eqref{eq:def_nonlinear_field} to construct 
nonlinear realizations, the nonlinear $\mathfrak{so}(1,4)$-valued 
Cartan connection is defined as~\cite{stelle.west:1980ds}
\begin{equation}
	\label{eq:A_nonlin}
	\bar{A} = \mathrm{Ad}(\exp(-i\xi\cdot P))(A + d).
\end{equation}
Under local de~Sitter transformations, the field $\bar{A}$ 
transforms according to
\begin{displaymath}
	\bar{A} \mapsto \Ad(h'(\xi,g_0))(\bar{A}+d),
\end{displaymath}
which is the correct compatibility law for a local 
de~Sitter-Cartan connection on a principal Lorentz 
bundle~\cite{sharpe1997diff_geo}. Because elements of $SO(1,4)$ 
are nonlinearly realized as elements of $SO(1,3)$, the reductive 
decomposition $\bar{A}_\mathfrak{h} + \bar{A}_\mathfrak{p}$, with 
$\mathfrak{h} = \mathfrak{so}(1,3)$, is invariant under local 
de~Sitter transformations.  It is then sensible to define the 
spin connection and vierbein through these projections as $\omega 
= \bar{A}_\mathfrak{h}$ and $e = \bar{A}_\mathfrak{p}$, 
respectively, which form a reducible multiplet for the de~Sitter 
group.

The spin connection $\omega$ and vierbein $e$ can be expressed in 
terms of the section $\xi$ and the projections $A_\mathfrak{h}$ 
and $A_\mathfrak{p}$ of the linear $SO(1,4)$ connection.  These 
relations follow from~\eqref{eq:A_nonlin}, in which the different 
objects appear according to
\begin{multline*}
	\tfrac{i}{2} \omega^{ab} M_{ab} + i e^a P_a \\
	= \Ad(\exp(-i\xi\cdot P)) \Big( \tfrac{i}{2} A^{ab} M_{ab} + i 
	A^a P_a + d \Big).
\end{multline*}
The right-hand side of this equation has to be worked out and 
separated in two parts, where one is valued in the Lorentz 
algebra $\mathfrak{so}(1,3)$ and a second takes values in the 
subspace of transvections $\mathfrak{p}$. The former terms 
constitute the spin connection, while the latter realize the 
vierbein. To carry out the computation we utilize the techniques 
of~\cite{Zumino1977189,stelle.west:1980ds}, explained in their 
appendices. Basically, one expands the adjoint action of the 
exponential as a power series in the adjoint action of its 
generating element $-i\xi\cdot P$. The latter is just the Lie 
commutator and is given explicitly
in~\eqref{eq:comm_relations_so(1,4)}. We find
\begin{widetext}
\begin{subequations}
\label{eqs:nonlin_spin_vier}
\begin{align}
\label{eq:nonlin_spinconn}
	\omega^{ab} &= A^{ab} - \frac{\cosh z - 1}{l^2 z^2} \big[ 
	\xi^a (d\xi^b + A\ind{^b_c} \xi^c) - \xi^b (d\xi^a + 
	A\ind{^a_c} \xi^c) \big] - \frac{\sinh z}{l^2 z} (\xi^a A^b - 
	\xi^b A^a),
\\
\label{eq:nonlin_vierbein}
	e^a &= A^a + \frac{\sinh z}{z}( d\xi^a + A\ind{^a_b} \xi^b) - 
	\frac{dl}{l} \xi ^a %\\
	+ (\cosh z - 1) \bigg( A^a - \frac{\xi^b A_b \xi^a}{\xi^2} 
	\bigg) - \bigg( \frac{\sinh z}{z} - 1 \bigg) \frac{\xi^b 
		d\xi_b \xi^a}{\xi^2}.
\end{align}
\end{subequations}
\end{widetext}
These expressions are almost identical to the corresponding 
objects found by Stelle and West~\cite{stelle.west:1980ds}. The 
difference to note is that we have a new term in the 
expression~\eqref{eq:nonlin_vierbein} for the vierbein, namely, 
$-l^{-1} dl\,\xi^a$. This term is present in the given geometry 
for it is possible that the internal de Sitter spaces are 
characterized by cosmological constants that are not necessarily 
equal along spacetime. More precisely, one has to take into 
account the possibility that the in $\mathfrak{p}$ defined length 
scale is a nonconstant function, see~Sec.~\ref{sec:lin_dSC_geo}.  
On the other hand, the results of~\cite{stelle.west:1980ds} 
specialized for the case that the local de Sitter spaces have the 
same pseudo-radius at any point in spacetime. The extra 
contribution is proportional to the dimensionless factor $l^{-1} 
dl \sim \Lambda^{-1} d\Lambda$, which becomes relevant only if 
the variation is relatively vast.  When $l$ is a constant 
function, one naturally recovers the results 
of~\cite{stelle.west:1980ds}. 

Upon the action of local de Sitter transformations, the linear 
curvature $F$ rotates in the adjoint representation. Therefore, 
one deduces that the nonlinear Cartan curvature $\bar{F}$ is 
equal to the exterior covariant derivative of the nonlinear 
connection, i.e.,
\begin{equation}
\label{eq:F_nonlin}
	\bar{F} = \Ad(\exp(-i\xi\cdot P))(F) = d\bar{A} + 
	\tfrac{1}{2}[\bar{A},\bar{A}],
\end{equation}
which naturally complies with the structure of a Cartan geometry.
The nonlinear Cartan curvature is a $\mathfrak{so}(1,4)$-valued 
$2$-form on spacetime, which we decompose once again according 
to~$\bar{F} = \bar{F}_\mathfrak{h} + \bar{F}_\mathfrak{p}$. Since 
$\bar{F}$ transforms---in general, nonlinearly---with elements of 
$SO(1,3)$, the reductive splitting is invariant under local 
de~Sitter transformations.  Similarly to our discussion on the 
nonlinear connection $\bar{A}$, the covariant nature of the 
decomposition suggests that $\bar{F}_\mathfrak{h}$ and 
$\bar{F}_\mathfrak{p}$ must be considered the genuine curvature 
and torsion of the Cartan geometry, which are denoted by $R$, 
respectively $T$. The definition~\eqref{eq:F_nonlin} implies that
\begin{multline*}
	\tfrac{i}{2} R^{ab} M_{ab} + i T^a P_a \\
	= \Ad(\exp(-i\xi\cdot P)) \Big( \tfrac{i}{2} F^{ab} M_{ab} + i 
	F^a P_a \Big),
\end{multline*}
from which one is able to express the curvature and torsion in 
terms of $\xi$, $F_\mathfrak{h}$ and $F_\mathfrak{p}$. Indeed, by 
computation of the adjoint action as a series of nested 
commutators, we are led to the results
\begin{subequations}
\label{eqs:nonlin_curv_tors}
\begin{align}
	\nonumber
	R^{ab} &= F^{ab} - \frac{\cosh z - 1}{l^2 z^2}\, \xi^c (\xi^a 
	F\ind{^b_c} -  \xi^b F\ind{^a_c}) \\
	\label{eq:nonlin_curv}
	&\hspace{3cm} - \frac{\sinh z}{l^2 z} (\xi^a F^b - \xi^b F^a),
\\
\label{eq:nonlin_tors}
	T^a &= \frac{\sinh z}{z} \xi^b F\ind{^a_b} + \cosh z\, F^a + 
	(1 - \cosh z) \frac{\xi_b F^b \xi^a}{\xi^2}.
\end{align}
\end{subequations}
From~\eqref{eq:F_nonlin} it furthermore follows that
\begin{displaymath}
	R^{ab}
	%= d\omega^{ab} + \omega\ind{^a_c} \wedge \omega^{cb} + 
	%\frac{1}{l^2} e^a \wedge e^b
	= d_\omega \omega^{ab} + \frac{1}{l^2} e^a \wedge e^b
	\quad\text{and}\quad
	T^a
	%&= de^a + \omega\ind{^a_b} \wedge e^b - \frac{1}{l} dl \wedge 
	%e^a
	= d_\omega e^a - \frac{1}{l} dl \wedge e^a.
\end{displaymath}
These equations, which express the curvature and torsion in terms 
of the spin connection and vierbein, are the ones expected for a 
Cartan geometry.
Because the exterior covariant derivative of $\bar{F}$ is always 
zero, there are two Bianchi identities that are formally the same 
as those given by~\eqref{eqs:lin_Bianchi},~i.e.,
\begin{displaymath}
	d_\omega \circ d_\omega \omega^{ab} \equiv 0
	\quad\text{and}\quad
	d_\omega \circ d_\omega e^a + e^b \wedge d_\omega 
	\omega\ind{_b^a} \equiv 0.
\end{displaymath}

When the section $\xi$ is gauge-fixed along spacetime, and for 
convenience at any point is chosen to be the origin of the 
tangent de~Sitter spaces, i.e.,~$\xi^a(x) = 0$, all the 
expressions reduce to those of Sec.~\ref{sec:lin_dSC_geo}.  This 
is to be expected, because the broken symmetries are not 
considered, and the geometry is described simply by a $SO(1,4)$ 
Ehresmann connection for which only 
$SO(1,3)$-transformations---the isotropy group of $\xi^a 
=0$---are taken into account. This has precisely been the way in 
which the de~Sitter-Cartan geometry of Sec.~\ref{sec:lin_dSC_geo} 
was set up. In this section, on the other hand, the section $\xi$ 
has been left arbitrary, thereby recovering local $SO(1,4)$ 
invariance.  The elements of $SO(1,4)/SO(1,3)$ act nonlinearly 
with elements of $SO(1,3)$, so that the 
$\mathfrak{so}(1,4)$-valued connection $\bar{A}$ is also a Cartan 
connection on a principal Lorentz bundle.

Finally, let us remark that if the fields $A^{ab}$ and $A^a$ can 
be made to vanish everywhere, so that also $F^{ab}$ and $F^a$ are 
equal to zero, it follows that
\begin{displaymath}
	R^{ab} = 0 \quad\text{and}\quad T^a = 0.
\end{displaymath}
This shows that the nonhomogeneity of $\mathcal{M}$ is encoded in 
$A$ and $F$, and naturally independent of the section $\xi$.

\section{Conclusions and outlook}
\label{sec:conclusion}

In this work we have generalized the geometric framework of 
de~Sitter-Cartan spacetimes with a cosmological constant to the 
case of a nonconstant cosmological function $\Lambda$. A 
de~Sitter-Cartan spacetime consists of a principal Lorentz bundle 
over spacetime, on which is defined a $\mathfrak{so}(1,4)$-valued 
Cartan connection. It accounts for a spin connection and 
vierbein, as well as for their curvature and torsion, whereas 
spacetime is locally approximated by de~Sitter spaces. The 
cosmological constants of these tangent de~Sitter spaces are 
determined by a length scale, defined in the translational part 
of $\mathfrak{so}(1,4)$. By letting this length scale depend 
arbitrarily on the spacetime point in Sec.~\ref{sec:lin_dSC_geo}, 
we obtained a de~Sitter-Cartan geometry that accommodates a 
cosmological function by construction. Most importantly, it was 
shown that a nonconstant $\Lambda$ gives rise to an extra 
contribution in the expression for the torsion, in which the 
cosmological function appears through its logarithmic derivative.  
In the limit $\Lambda \to 0$ one recovers the well-known 
Riemann-Cartan spacetime with arbitrary curvature and torsion.

The structure group of this Cartan geometry is given by 
$SO(1,3)$.  Moreover, the definitions for the different geometric 
objects are not consistent with the action of generic elements of 
the encompassing de~Sitter group. However, by nonlinearly 
realizing a $SO(1,4)$-connection and its curvature in 
Sec.~\ref{sec:nonlin_dSC_geo}, elements of $SO(1,4)/SO(1,3)$ were 
realized nonlinearly as elements of the Lorentz group, and thus 
included in the structure group of the Cartan geometry. This 
allowed for $SO(1,4)$-covariant definitions for the spin 
connection and vierbein, and likewise for the curvature and 
torsion. Again, we generalized this framework to include a 
nonconstant cosmological function. 

We reason that such a generalized framework with nonconstant 
$\Lambda$ can be of use to construct theories of gravity, for it 
extends the geometric meaning of a cosmological constant to the 
case of a spacetime-dependent function. The cosmological function 
quantifies the lack of commutation of two infinitesimal spacetime 
translations, and therefore manifests itself in the local 
kinematics on spacetime.  It is then an interesting question to 
pose and investigate, if and how it is possible to cook up 
actions for theories of gravity by making use of the geometric 
ingredients discussed in this work, and which can account for a 
spacetime-dependent dark energy.  Doing so, one would obtain a 
link between the dynamical character of $\Lambda$ and its 
kinematical implications.

Another point of interest comes about upon noting that, when the 
de~Sitter algebra is contracted to the Poincar\'e algebra, 
namely, when $l \to \infty$ in the commutation 
relations~\eqref{eq:comm_relations_so(1,4)}, the geometric 
objects of Sec.~\ref{sec:nonlin_dSC_geo} reduce to those of 
teleparallel gravity~\cite{aldrovandi:2012tele,Arcos:2005ec}.  
This observation suggests that the geometry of spacetime that 
underlies teleparallel gravity is described by a Riemann-Cartan 
geometry (with vanishing curvature), for which the Poincar\'e 
translations are realized nonlinearly as elements of $SO(1,3)$.  
In fact, from~\eqref{eq:inf_tr_h} one sees that the nonlinear 
element of the Lorentz algebra, which corresponds to an 
infinitesimal Poincar\'e translation with parameters 
$\epsilon^a$, vanishes, for
\begin{displaymath}
	\delta h^{ab} = \lim_{l \to \infty} \frac{1}{l^2} \frac{\cosh 
		z - 1}{z\sinh z} (\epsilon^a\xi^b - \epsilon^b\xi^a) = 0.
\end{displaymath}
One then concludes that any Poincar\'e translation is trivially 
realized by the identity transformation, a property that is 
relied upon in the interpretation of teleparallel gravity as a 
gauge theory for the Poincar\'e translations. Given the knowledge 
that the geometric structure of teleparallel gravity is such a 
Riemann-Cartan spacetime, the de~Sitter-Cartan geometry of 
Sec.~\ref{sec:nonlin_dSC_geo} might be the right framework to 
generalize teleparallel gravity to a theory that is invariant 
under local $SO(1,4)$-transformations, in place of the elements 
of the Poincar\'e group.


\begin{acknowledgments}

The author would like to thank J.~G.~Pereira for helpful 
discussions and suggestions. He also gratefully acknowledges 
financial support by CAPES.

\end{acknowledgments}



%%%%%%%%%%%%%%%%
% BIBLIOGRAPHY %
%%%%%%%%%%%%%%%%

%merlin.mbs apsrev4-1.bst 2010-07-25 4.21a (PWD, AO, DPC) hacked
%Control: key (0)
%Control: author (8) initials jnrlst
%Control: editor formatted (1) identically to author
%Control: production of article title (-1) disabled
%Control: page (0) single
%Control: year (1) truncated
%Control: production of eprint (0) enabled
\begin{thebibliography}{30}%
\makeatletter
\providecommand \@ifxundefined [1]{%
 \@ifx{#1\undefined}
}%
\providecommand \@ifnum [1]{%
 \ifnum #1\expandafter \@firstoftwo
 \else \expandafter \@secondoftwo
 \fi
}%
\providecommand \@ifx [1]{%
 \ifx #1\expandafter \@firstoftwo
 \else \expandafter \@secondoftwo
 \fi
}%
\providecommand \natexlab [1]{#1}%
\providecommand \enquote  [1]{``#1''}%
\providecommand \bibnamefont  [1]{#1}%
\providecommand \bibfnamefont [1]{#1}%
\providecommand \citenamefont [1]{#1}%
\providecommand \href@noop [0]{\@secondoftwo}%
\providecommand \href [0]{\begingroup \@sanitize@url \@href}%
\providecommand \@href[1]{\@@startlink{#1}\@@href}%
\providecommand \@@href[1]{\endgroup#1\@@endlink}%
\providecommand \@sanitize@url [0]{\catcode `\\12\catcode `\$12\catcode
  `\&12\catcode `\#12\catcode `\^12\catcode `\_12\catcode `\%12\relax}%
\providecommand \@@startlink[1]{}%
\providecommand \@@endlink[0]{}%
\providecommand \url  [0]{\begingroup\@sanitize@url \@url }%
\providecommand \@url [1]{\endgroup\@href {#1}{\urlprefix }}%
\providecommand \urlprefix  [0]{URL }%
\providecommand \Eprint [0]{\href }%
\providecommand \doibase [0]{http://dx.doi.org/}%
\providecommand \selectlanguage [0]{\@gobble}%
\providecommand \bibinfo  [0]{\@secondoftwo}%
\providecommand \bibfield  [0]{\@secondoftwo}%
\providecommand \translation [1]{[#1]}%
\providecommand \BibitemOpen [0]{}%
\providecommand \bibitemStop [0]{}%
\providecommand \bibitemNoStop [0]{.\EOS\space}%
\providecommand \EOS [0]{\spacefactor3000\relax}%
\providecommand \BibitemShut  [1]{\csname bibitem#1\endcsname}%
\let\auto@bib@innerbib\@empty
%</preamble>
\bibitem [{\citenamefont {Weinberg}(1972)}]{Weinberg:1972gc}%
  \BibitemOpen
  \bibfield  {author} {\bibinfo {author} {\bibfnamefont {S.}~\bibnamefont
  {Weinberg}},\ }\href@noop {} {\emph {\bibinfo {title} {Gravitation and
  cosmology: principles and applications of the general theory of
  relativity}}}\ (\bibinfo  {publisher} {Wiley},\ \bibinfo {address} {New
  York},\ \bibinfo {year} {1972})\BibitemShut {NoStop}%
\bibitem [{\citenamefont {Di~Casola}\ \emph {et~al.}(2013)\citenamefont
  {Di~Casola}, \citenamefont {Liberati},\ and\ \citenamefont
  {Sonego}}]{DiCasola:2013nep}%
  \BibitemOpen
  \bibfield  {author} {\bibinfo {author} {\bibfnamefont {E.}~\bibnamefont
  {Di~Casola}}, \bibinfo {author} {\bibfnamefont {S.}~\bibnamefont {Liberati}},
  \ and\ \bibinfo {author} {\bibfnamefont {S.}~\bibnamefont {Sonego}},\
  }\href@noop {} {\  (\bibinfo {year} {2013})},\ \Eprint
  {http://arxiv.org/abs/1310.7426} {arXiv:1310.7426 [gr-qc]} \BibitemShut
  {NoStop}%
%%CITATION = ARXIV:1310.7426;%%
\bibitem [{Note1()}]{Note1}%
  \BibitemOpen
  \bibinfo {note} {An infinitesimal translation at a point $x$ of $\protect
  \mathcal {M}$ is defined as an element $1 + X$, where $X \in T_x\protect
  \mathcal {M}$, the tangent space at $x$. The physical meaning of such a
  translation, being a Poincar\'e translation, for example, depends on the
  Cartan connection defined on spacetime, see below.}\BibitemShut {Stop}%
\bibitem [{\citenamefont {Hehl}\ \emph {et~al.}(1976)\citenamefont {Hehl},
  \citenamefont {Von Der~Heyde}, \citenamefont {Kerlick},\ and\ \citenamefont
  {Nester}}]{Hehl:1976grs}%
  \BibitemOpen
  \bibfield  {author} {\bibinfo {author} {\bibfnamefont {F.~W.}\ \bibnamefont
  {Hehl}}, \bibinfo {author} {\bibfnamefont {P.}~\bibnamefont {Von Der~Heyde}},
  \bibinfo {author} {\bibfnamefont {G.~D.}\ \bibnamefont {Kerlick}}, \ and\
  \bibinfo {author} {\bibfnamefont {J.~M.}\ \bibnamefont {Nester}},\ }\href
  {\doibase 10.1103/RevModPhys.48.393} {\bibfield  {journal} {\bibinfo
  {journal} {Rev. Mod. Phys.}\ }\textbf {\bibinfo {volume} {48}},\ \bibinfo
  {pages} {393} (\bibinfo {year} {1976})}\BibitemShut {NoStop}%
%%CITATION = RMPHA,48,393;%%
\bibitem [{\citenamefont {Cartan}(1926)}]{Cartan:1926gh}%
  \BibitemOpen
  \bibfield  {author} {\bibinfo {author} {\bibfnamefont {E.}~\bibnamefont
  {Cartan}},\ }\href {\doibase 10.1007/BF02629755} {\bibfield  {journal}
  {\bibinfo  {journal} {Acta Math.}\ }\textbf {\bibinfo {volume} {48}},\
  \bibinfo {pages} {1} (\bibinfo {year} {1926})}\BibitemShut {NoStop}%
\bibitem [{\citenamefont {Wise}(2010)}]{Wise:2010sm}%
  \BibitemOpen
  \bibfield  {author} {\bibinfo {author} {\bibfnamefont {D.~K.}\ \bibnamefont
  {Wise}},\ }\href {\doibase 10.1088/0264-9381/27/15/155010} {\bibfield
  {journal} {\bibinfo  {journal} {Class. Quantum Grav.}\ }\textbf {\bibinfo
  {volume} {27}},\ \bibinfo {pages} {155010} (\bibinfo {year} {2010})},\
  \Eprint {http://arxiv.org/abs/gr-qc/0611154} {arXiv:gr-qc/0611154}
  \BibitemShut {NoStop}%
%%CITATION = GR-QC/0611154;%%
\bibitem [{\citenamefont {Sharpe}(1997)}]{sharpe1997diff_geo}%
  \BibitemOpen
  \bibfield  {author} {\bibinfo {author} {\bibfnamefont {R.~W.}\ \bibnamefont
  {Sharpe}},\ }\href@noop {} {\emph {\bibinfo {title} {Differential Geometry:
  {C}artan's Generalization of Klein's Erlangen Program}}}\ (\bibinfo
  {publisher} {Springer},\ \bibinfo {address} {New York},\ \bibinfo {year}
  {1997})\BibitemShut {NoStop}%
\bibitem [{\citenamefont {MacDowell}\ and\ \citenamefont
  {Mansouri}(1977)}]{MacDowell1977}%
  \BibitemOpen
  \bibfield  {author} {\bibinfo {author} {\bibfnamefont {S.~W.}\ \bibnamefont
  {MacDowell}}\ and\ \bibinfo {author} {\bibfnamefont {F.}~\bibnamefont
  {Mansouri}},\ }\href {\doibase 10.1103/PhysRevLett.38.1376,
  10.1103/PhysRevLett.38.739} {\bibfield  {journal} {\bibinfo  {journal} {Phys.
  Rev. Lett.}\ }\textbf {\bibinfo {volume} {38}},\ \bibinfo {pages} {739}
  (\bibinfo {year} {1977})}\BibitemShut {NoStop}%
%%CITATION = PRLTA,38,739;%%
\bibitem [{\citenamefont {Westman}\ and\ \citenamefont
  {Zlosnik}(2012)}]{Westman:2012xk}%
  \BibitemOpen
  \bibfield  {author} {\bibinfo {author} {\bibfnamefont {H.~F.}\ \bibnamefont
  {Westman}}\ and\ \bibinfo {author} {\bibfnamefont {T.~G.}\ \bibnamefont
  {Zlosnik}},\ }\href@noop {} {\  (\bibinfo {year} {2012})},\ \Eprint
  {http://arxiv.org/abs/1203.5709} {arXiv:1203.5709 [gr-qc]} \BibitemShut
  {NoStop}%
%%CITATION = ARXIV:1203.5709;%%
\bibitem [{\citenamefont {Peebles}\ and\ \citenamefont
  {Ratra}(2003)}]{Peebles:2003cc}%
  \BibitemOpen
  \bibfield  {author} {\bibinfo {author} {\bibfnamefont {P.~J.~E.}\
  \bibnamefont {Peebles}}\ and\ \bibinfo {author} {\bibfnamefont
  {B.}~\bibnamefont {Ratra}},\ }\href {\doibase 10.1103/RevModPhys.75.559}
  {\bibfield  {journal} {\bibinfo  {journal} {Rev. Mod. Phys.}\ }\textbf
  {\bibinfo {volume} {75}},\ \bibinfo {pages} {559} (\bibinfo {year} {2003})},\
  \Eprint {http://arxiv.org/abs/astro-ph/0207347} {arXiv:astro-ph/0207347}
  \BibitemShut {NoStop}%
%%CITATION = ASTRO-PH/0207347;%%
\bibitem [{\citenamefont {Copeland}\ \emph {et~al.}(2006)\citenamefont
  {Copeland}, \citenamefont {Sami},\ and\ \citenamefont
  {Tsujikawa}}]{Copeland:2006de}%
  \BibitemOpen
  \bibfield  {author} {\bibinfo {author} {\bibfnamefont {E.~J.}\ \bibnamefont
  {Copeland}}, \bibinfo {author} {\bibfnamefont {M.}~\bibnamefont {Sami}}, \
  and\ \bibinfo {author} {\bibfnamefont {S.}~\bibnamefont {Tsujikawa}},\ }\href
  {\doibase 10.1142/S021827180600942X} {\bibfield  {journal} {\bibinfo
  {journal} {Int. J. Mod. Phys. D}\ }\textbf {\bibinfo {volume} {15}},\
  \bibinfo {pages} {1753} (\bibinfo {year} {2006})},\ \Eprint
  {http://arxiv.org/abs/hep-th/0603057} {arXiv:hep-th/0603057} \BibitemShut
  {NoStop}%
%%CITATION = HEP-TH/0603057;%%
\bibitem [{\citenamefont {Klein}(1893)}]{Klein:1893}%
  \BibitemOpen
  \bibfield  {author} {\bibinfo {author} {\bibfnamefont {F.}~\bibnamefont
  {Klein}},\ }\href@noop {} {\bibfield  {journal} {\bibinfo  {journal} {Bull.
  New York Math. Soc.}\ }\textbf {\bibinfo {volume} {2}},\ \bibinfo {pages}
  {215} (\bibinfo {year} {1893})},\ \bibinfo {note}
  {\url{http://projecteuclid.org/euclid.bams/1183407629}}\BibitemShut {NoStop}%
\bibitem [{\citenamefont {Petti}(2006)}]{petti:2006a}%
  \BibitemOpen
  \bibfield  {author} {\bibinfo {author} {\bibfnamefont {R.~J.}\ \bibnamefont
  {Petti}},\ }\href {\doibase 10.1088/0264-9381/23/3/012} {\bibfield  {journal}
  {\bibinfo  {journal} {Class. Quantum Grav.}\ }\textbf {\bibinfo {volume}
  {23}},\ \bibinfo {pages} {737} (\bibinfo {year} {2006})}\BibitemShut
  {NoStop}%
%%CITATION = CQGRD,23,737;%%
\bibitem [{\citenamefont {Gibbons}\ and\ \citenamefont
  {Gielen}(2009)}]{gibbons:2009b}%
  \BibitemOpen
  \bibfield  {author} {\bibinfo {author} {\bibfnamefont {G.~W.}\ \bibnamefont
  {Gibbons}}\ and\ \bibinfo {author} {\bibfnamefont {S.}~\bibnamefont
  {Gielen}},\ }\href {\doibase 10.1088/0264-9381/26/13/135005} {\bibfield
  {journal} {\bibinfo  {journal} {Class. Quantum Grav.}\ }\textbf {\bibinfo
  {volume} {26}},\ \bibinfo {pages} {135005} (\bibinfo {year} {2009})},\
  \Eprint {http://arxiv.org/abs/0902.2001} {arXiv:0902.2001 [gr-qc]}
  \BibitemShut {NoStop}%
%%CITATION = ARXIV:0902.2001;%%
\bibitem [{\citenamefont {Kobayashi}\ and\ \citenamefont
  {Nomizu}(1996)}]{kob1996found}%
  \BibitemOpen
  \bibfield  {author} {\bibinfo {author} {\bibfnamefont {S.}~\bibnamefont
  {Kobayashi}}\ and\ \bibinfo {author} {\bibfnamefont {K.}~\bibnamefont
  {Nomizu}},\ }\href@noop {} {\emph {\bibinfo {title} {Foundations of
  Differential Geometry}}},\ Vol.~\bibinfo {volume} {1}\ (\bibinfo  {publisher}
  {Wiley-Interscience},\ \bibinfo {address} {New York},\ \bibinfo {year}
  {1996})\ \bibinfo {note} {reprint of the 1963 original}\BibitemShut {NoStop}%
\bibitem [{\citenamefont {Alekseevsky}\ and\ \citenamefont
  {Michor}(1995)}]{Alekseevsky:1995cc}%
  \BibitemOpen
  \bibfield  {author} {\bibinfo {author} {\bibfnamefont {D.~V.}\ \bibnamefont
  {Alekseevsky}}\ and\ \bibinfo {author} {\bibfnamefont {P.~W.}\ \bibnamefont
  {Michor}},\ }\href@noop {} {\bibfield  {journal} {\bibinfo  {journal} {Publ.
  Math. Debrecen}\ }\textbf {\bibinfo {volume} {47}},\ \bibinfo {pages} {349}
  (\bibinfo {year} {1995})},\ \Eprint {http://arxiv.org/abs/math/9412232}
  {arXiv:math/9412232 [math.DG]} \BibitemShut {NoStop}%
\bibitem [{\citenamefont {Wise}(2009)}]{Wise:2009fu}%
  \BibitemOpen
  \bibfield  {author} {\bibinfo {author} {\bibfnamefont {D.~K.}\ \bibnamefont
  {Wise}},\ }\href {\doibase 10.3842/SIGMA.2009.080} {\bibfield  {journal}
  {\bibinfo  {journal} {SIGMA}\ }\textbf {\bibinfo {volume} {5}},\ \bibinfo
  {pages} {080} (\bibinfo {year} {2009})},\ \Eprint
  {http://arxiv.org/abs/0904.1738} {arXiv:0904.1738 [math.DG]} \BibitemShut
  {NoStop}%
%%CITATION = ARXIV:0904.1738;%%
\bibitem [{\citenamefont {Wise}(2012)}]{Wise:2011-sym.br}%
  \BibitemOpen
  \bibfield  {author} {\bibinfo {author} {\bibfnamefont {D.~K.}\ \bibnamefont
  {Wise}},\ }\href {\doibase 10.1088/1742-6596/360/1/012017} {\bibfield
  {journal} {\bibinfo  {journal} {J. Phys.: Conf. Ser.}\ }\textbf {\bibinfo
  {volume} {360}},\ \bibinfo {pages} {012017} (\bibinfo {year} {2012})},\
  \Eprint {http://arxiv.org/abs/1112.2390} {arXiv:1112.2390 [gr-qc]}
  \BibitemShut {NoStop}%
%%CITATION = ARXIV:1112.2390;%%
\bibitem [{\citenamefont {Husem{\"o}ller}(1966)}]{husemoller:1966fibre}%
  \BibitemOpen
  \bibfield  {author} {\bibinfo {author} {\bibfnamefont {D.}~\bibnamefont
  {Husem{\"o}ller}},\ }\href@noop {} {\emph {\bibinfo {title} {Fibre
  Bundles}}}\ (\bibinfo  {publisher} {Springer},\ \bibinfo {address} {New
  York},\ \bibinfo {year} {1966})\BibitemShut {NoStop}%
\bibitem [{\citenamefont {Stelle}\ and\ \citenamefont
  {West}(1979)}]{Stelle:1979va}%
  \BibitemOpen
  \bibfield  {author} {\bibinfo {author} {\bibfnamefont {K.~S.}\ \bibnamefont
  {Stelle}}\ and\ \bibinfo {author} {\bibfnamefont {P.~C.}\ \bibnamefont
  {West}},\ }\href {\doibase 10.1088/0305-4470/12/8/003} {\bibfield  {journal}
  {\bibinfo  {journal} {J. Phys. A}\ }\textbf {\bibinfo {volume} {12}},\
  \bibinfo {pages} {L205} (\bibinfo {year} {1979})}\BibitemShut {NoStop}%
%%CITATION = INSPIRE-146713;%%
\bibitem [{\citenamefont {Stelle}\ and\ \citenamefont
  {West}(1980)}]{stelle.west:1980ds}%
  \BibitemOpen
  \bibfield  {author} {\bibinfo {author} {\bibfnamefont {K.~S.}\ \bibnamefont
  {Stelle}}\ and\ \bibinfo {author} {\bibfnamefont {P.~C.}\ \bibnamefont
  {West}},\ }\href {\doibase 10.1103/PhysRevD.21.1466} {\bibfield  {journal}
  {\bibinfo  {journal} {Phys. Rev. D}\ }\textbf {\bibinfo {volume} {21}},\
  \bibinfo {pages} {1466} (\bibinfo {year} {1980})}\BibitemShut {NoStop}%
\bibitem [{\citenamefont {{Ort\'{\i}n}}(2004)}]{Ortin:2004}%
  \BibitemOpen
  \bibfield  {author} {\bibinfo {author} {\bibfnamefont {T.}~\bibnamefont
  {{Ort\'{\i}n}}},\ }\href@noop {} {\emph {\bibinfo {title} {Gravity and
  Strings}}}\ (\bibinfo  {publisher} {Cambridge University Press},\ \bibinfo
  {address} {Cambridge},\ \bibinfo {year} {2004})\BibitemShut {NoStop}%
%%CITATION = INSPIRE-648696;%%
\bibitem [{\citenamefont {Edelstein}\ and\ \citenamefont
  {Zanelli}(2006)}]{Edelstein:2006a}%
  \BibitemOpen
  \bibfield  {author} {\bibinfo {author} {\bibfnamefont {J.~D.}\ \bibnamefont
  {Edelstein}}\ and\ \bibinfo {author} {\bibfnamefont {J.}~\bibnamefont
  {Zanelli}},\ }\href {\doibase 10.1088/1742-6596/33/1/008} {\bibfield
  {journal} {\bibinfo  {journal} {J. Phys.: Conf. Ser.}\ }\textbf {\bibinfo
  {volume} {33}},\ \bibinfo {pages} {83} (\bibinfo {year} {2006})},\ \Eprint
  {http://arxiv.org/abs/hep-th/0605186} {arXiv:hep-th/0605186} \BibitemShut
  {NoStop}%
%%CITATION = HEP-TH/0605186;%%
\bibitem [{\citenamefont {Coleman}\ \emph {et~al.}(1969)\citenamefont
  {Coleman}, \citenamefont {Wess},\ and\ \citenamefont
  {Zumino}}]{Coleman:1969sm}%
  \BibitemOpen
  \bibfield  {author} {\bibinfo {author} {\bibfnamefont {S.~R.}\ \bibnamefont
  {Coleman}}, \bibinfo {author} {\bibfnamefont {J.}~\bibnamefont {Wess}}, \
  and\ \bibinfo {author} {\bibfnamefont {B.}~\bibnamefont {Zumino}},\ }\href
  {\doibase 10.1103/PhysRev.177.2239} {\bibfield  {journal} {\bibinfo
  {journal} {Phys. Rev.}\ }\textbf {\bibinfo {volume} {177}},\ \bibinfo {pages}
  {2239} (\bibinfo {year} {1969})}\BibitemShut {NoStop}%
%%CITATION = PHRVA,177,2239;%%
\bibitem [{\citenamefont {Callan}\ \emph {et~al.}(1969)\citenamefont {Callan},
  \citenamefont {Coleman}, \citenamefont {Wess},\ and\ \citenamefont
  {Zumino}}]{Callan:1969sn}%
  \BibitemOpen
  \bibfield  {author} {\bibinfo {author} {\bibfnamefont {C.~G.}\ \bibnamefont
  {Callan}, \bibfnamefont {Jr.}}, \bibinfo {author} {\bibfnamefont
  {S.}~\bibnamefont {Coleman}}, \bibinfo {author} {\bibfnamefont
  {J.}~\bibnamefont {Wess}}, \ and\ \bibinfo {author} {\bibfnamefont
  {B.}~\bibnamefont {Zumino}},\ }\href {\doibase 10.1103/PhysRev.177.2247}
  {\bibfield  {journal} {\bibinfo  {journal} {Phys. Rev.}\ }\textbf {\bibinfo
  {volume} {177}},\ \bibinfo {pages} {2247} (\bibinfo {year}
  {1969})}\BibitemShut {NoStop}%
%%CITATION = PHRVA,177,2247;%%
\bibitem [{\citenamefont {Volkov}(1973)}]{Volkov:1973vd}%
  \BibitemOpen
  \bibfield  {author} {\bibinfo {author} {\bibfnamefont {D.~V.}\ \bibnamefont
  {Volkov}},\ }\href@noop {} {\bibfield  {journal} {\bibinfo  {journal} {Fiz.
  Elem. Chastits At. Yadra}\ }\textbf {\bibinfo {volume} {4}},\ \bibinfo
  {pages} {3} (\bibinfo {year} {1973})},\ \bibinfo {note} {[Sov. J. Part.
  Nucl.]}\BibitemShut {NoStop}%
%%CITATION = FECAA,4,3;%%
\bibitem [{\citenamefont {Zumino}(1977)}]{Zumino1977189}%
  \BibitemOpen
  \bibfield  {author} {\bibinfo {author} {\bibfnamefont {B.}~\bibnamefont
  {Zumino}},\ }\href {\doibase 10.1016/0550-3213(77)90211-5} {\bibfield
  {journal} {\bibinfo  {journal} {Nucl. Phys.}\ }\textbf {\bibinfo {volume}
  {B127}},\ \bibinfo {pages} {189} (\bibinfo {year} {1977})}\BibitemShut
  {NoStop}%
\bibitem [{\citenamefont {Wess}\ and\ \citenamefont
  {Bagger}(1992)}]{wess:1992ss}%
  \BibitemOpen
  \bibfield  {author} {\bibinfo {author} {\bibfnamefont {J.}~\bibnamefont
  {Wess}}\ and\ \bibinfo {author} {\bibfnamefont {J.}~\bibnamefont {Bagger}},\
  }\href@noop {} {\emph {\bibinfo {title} {Supersymmetry and Supergravity}}}\
  (\bibinfo  {publisher} {Princeton University Press},\ \bibinfo {address}
  {Princeton},\ \bibinfo {year} {1992})\BibitemShut {NoStop}%
\bibitem [{\citenamefont {Aldrovandi}\ and\ \citenamefont
  {Pereira}(2012)}]{aldrovandi:2012tele}%
  \BibitemOpen
  \bibfield  {author} {\bibinfo {author} {\bibfnamefont {R.}~\bibnamefont
  {Aldrovandi}}\ and\ \bibinfo {author} {\bibfnamefont {J.~G.}\ \bibnamefont
  {Pereira}},\ }\href@noop {} {\emph {\bibinfo {title} {Teleparallel Gravity:
  An Introduction}}}\ (\bibinfo  {publisher} {Springer},\ \bibinfo {address}
  {Dordrecht},\ \bibinfo {year} {2012})\BibitemShut {NoStop}%
\bibitem [{\citenamefont {Arcos}\ and\ \citenamefont
  {Pereira}(2004)}]{Arcos:2005ec}%
  \BibitemOpen
  \bibfield  {author} {\bibinfo {author} {\bibfnamefont {H.~I.}\ \bibnamefont
  {Arcos}}\ and\ \bibinfo {author} {\bibfnamefont {J.~G.}\ \bibnamefont
  {Pereira}},\ }\href {\doibase 10.1142/S0218271804006462} {\bibfield
  {journal} {\bibinfo  {journal} {Int. J. Mod. Phys. D}\ }\textbf {\bibinfo
  {volume} {13}},\ \bibinfo {pages} {2193} (\bibinfo {year} {2004})},\ \Eprint
  {http://arxiv.org/abs/gr-qc/0501017} {arXiv:gr-qc/0501017} \BibitemShut
  {NoStop}%
%%CITATION = GR-QC/0501017;%%
\end{thebibliography}%

%\bibliography{bib.cosm_funct.bib}
\end{document}
