\documentclass[10pt]{article}

%Package declarations
%--------------------
\usepackage{amsmath,amsfonts,amssymb}
\usepackage{setspace}
\setstretch{1.1}
\usepackage[a4paper]{geometry}
\geometry{left=3.5cm,right=3.5cm,top=3cm,bottom=3cm}

\usepackage[english]{babel}
\usepackage{color}



\title{Cosmology}
\date{\today}
\author{Hendrik}


\begin{document}
\maketitle

\begin{abstract}
	This is to be a rough survey of introductory concepts in 
	Cosmology.  \end{abstract}

\section{Kinematics due to the Cosmological Principle}

From the Cosmological Principle, which is the assumption that 
space is homogeneous and isotropic on large scales, one is lead 
to the Friedmann-Lema\^itre-Robertson-Walker metric,
%
\begin{equation}
	ds^2 = -dt^2 + a^2(t) d\Sigma_3
\end{equation}
%
where $d\Sigma_3$ is a three-dimensional space of constant 
curvature. It can be expressed as
[Expand this by argumentating the form; fundamental observers, 
etc. See ... ]
%
\begin{equation}
	d\Sigma_3 = d\chi^2 + S^2(\chi)(d\theta^2 + \sin^2 \theta 
	d\phi^2)
\end{equation}
%
\begin{equation}
	S(\chi) =
	\begin{cases}
		\sin\chi & \text{positive spatial curvature}\\
		\chi & \text{no spatial curvature}\\
		\sinh\chi & \text{negative spatial curvature}
	\end{cases}
\end{equation}
Note that the \emph{comoving} coordinates---the coordinates of 
fundamental observers---$\chi,\theta,\phi$ are constants in time. 
These observers will move relative to eachother because their 
seperation distances change due to the time dependence of the 
scale factor $a(t)$, but their coordinate distance is 
independepent of time.

Due to the timedependent scale factor in the FRW--metric, 
identical photons which are emitted at different cosmic time will 
generally have different energies when observed on earth. This 
can be understood by noting that because of the changing scale 
factor, the distance between two wavecrests will increase. The 
relative change in wavelength is called the redshift factor $z$ 
and is given by
%
\begin{equation}
	z = \frac{\Delta \lambda}{\lambda_e} = \frac{a(t_o)}{a(t_e)}-1
\end{equation}
Note that for an increasing scale factor photons are redshifted 
$(z>0)$, while for a decreasing factor photons are blueshifted 
$(z<0)$.

The scale factor $a(t)$ is not known at all cosmic times. One 
therefore starts with a Taylor series expansion around 
``now'',\footnote{Note that if photons are emitted very close to 
	now, that this also implies that the emitting object is nearby 
	in space.  Of course what is considered close is to be defined 
	with respect to the overall scale. This is also reflected in 
	taking a taylor series w.r.t.\ a-priori dimensionful 
	quantities. For example, when expanding around now, the 
	quantity $\delta t = t - t_o$ is really $t - t_o / t_o$.} i.e.  
$t_o$.  Doing so one finds,
%
\begin{displaymath}
	\frac{a(t)}{a(t_o)} = 1 - \frac{\dot{a}(t_o)}{a(t_o)}(t_o - t) 
	+ \tfrac{1}{2} \ddot{a}(t_o)(t_o -t)^2 + O(3)
\end{displaymath}
%
The Hubble parameter $H(t)$ and decelaration parameter $q(t)$ are 
then defined as
%
\begin{align}
	H(t) &\equiv \frac{\dot{a}(t)}{a(t)} \\
	q(t) &\equiv -\frac{\ddot{a}(t)a(t)}{\dot{a}^2(t)}
\end{align}
%
Hence, the scale parameter can be recast in the form
%
\begin{equation}
	\frac{a(t)}{a(t_o)} = 1 - H_o(t_o-t) + \tfrac{1}{2} q_o H^2_o 
	(t_o-t)^2 + O(3)
\end{equation}

It is a same exercise to find the redshift of a photon emitted at 
time $t$, as a Taylor expansion around $t_0$;
%
\begin{equation}
	z(t) = H_o(t_o-t) + H^2_o(1+\tfrac{1}{2}q_o) (t_o-t)^2 + O(3)
\end{equation}
%
For nearby objects, the cosmological redshift is $z \approx H_o 
(t_o-t)$. Since for such objects $(t_o-t)$ is approximately equal 
to the proper distance between us and the object ($t$ being the 
time the photon was emitted), one has \emph{Hubble's Law}, that 
is
%
\begin{equation}
	z \approx H_o d_P
\end{equation}

\paragraph{Proper distance}

A photon travels along a null geodesic such that along its 
worldline, one always has $dt = \pm a(t)d\chi$. The proper 
distance between two fundamental observers at time $t_o$ is given 
by
\begin{equation}
	d_P \equiv \int_{0}^{\chi} a(t_o) d\chi
\end{equation}
%
Let $t_e$ be the time at which a photon was emitted at $\chi = 
\chi_e$ that has arrived now at the earth. One can rewrite the 
proper distance then as
%
\begin{equation}
	d_P = a(t_o) \int_{t_e}^{t_o} \frac{dt}{a(t)}
\end{equation}
%
Using the relation $dt = -H^{-1}(1+z)^{-1}dz$, this integral may 
be rewritten as
%
\begin{equation}
	d_P = - \int_{z}^0 \frac{dz}{H(z)} = \int_0^z \frac{dz}{H(z)}
\end{equation}
%
Which is the proper distance between us (as a fundamental 
observer) and another fundamental observer seen at redshift $z$.

\paragraph{Luminosity distance}

Let $L$ be the \emph{absolute luminosity} radiated by a 
fundamental observer $(\chi_e)$ and let $f$ be the \emph{observed 
	flux}. The \emph{luminosity distance} is the distance defined 
as
%
\begin{equation}
	d_L = \left( \frac{L}{4\pi f} \right)^{1/2}
\end{equation}
%
The observed flux will depend on the large scale geometry of 
spacetime.  Indeed, it is the observed luminosity divided by the 
area of a sphere surrounding the emitting object. The observed 
luminosity will be redshifted twice: once because the photons are 
redshifted and once because the arrival rate will be affected by 
the same rate--the latter can be understood by imagining that the 
space \emph{between} photons is scaled with a factor $1+z$. Since 
the area of a sphere is given by $4\pi a^2(t_o)S^2(\chi_e)$, one 
finds that the observed flux is given by
%
\begin{displaymath}
	f = \frac{L(1+z)^{-2}}{4 \pi a^2(t_o)S^2(\chi_e)}
\end{displaymath}
%
One concludes that the luminosity distance to an object at 
coordinate $\chi$ and redshift $z$ is given by
%
\begin{equation}
	d_L = a(t_o)S(\chi)(1+z)
\end{equation}
%
Note that for flat spacelike sections $S(\chi) = \chi$ and one 
finds
that
%
\begin{displaymath}
	d_L = d_P (1+z) = (1+z)\int_0^z \frac{d\bar{z}}{H(\bar{z})}
\end{displaymath}

\section{Dynamics due to general relativity}

%\bibliographystyle{plain}
%\bibliography{cosm_bib.bib}
\end{document}
