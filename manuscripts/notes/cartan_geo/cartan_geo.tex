\documentclass[11pt]{amsart}

%Load preamble files
\usepackage{../../Tex_files/standard}
\usepackage{../../Tex_files/preamble_one}

\usepackage[all]{xy}
\usepackage{breqn}

\title{Cartan geometry}
\author{Hendrik}
\date{\today}

\begin{document}

\begin{abstract}
	In this document, Cartan geometry is reviewed and discussed 
	explicitly for a geometry based on the Klein geometry of a de 
	Sitter spacetime. Then, we try to generalize for a spacetime 
	depending cosmological function, and check consistency with 
	the Bianchi identities. This leads to postulating modified 
	Einstein's equations.
\end{abstract}

\maketitle

\section{Introduction}

In de Sitter special relativity it is postulated that in the 
absence of gravitational fields, all physical laws are to be 
invariant under de Sitter transformations, that is the 
pseudo--orthogonal group $SO(4,1)$ [{\blu REF}]. Of course, this 
is a generalization of ordinary special relativity in the sense 
that the vanishing cosmological constant is allowed to take a 
non-zero and positive value. Which value one should be using is a 
question to be answered outside special relativity, more 
precisely in an adequately adapted general relativistic theory of 
gravity.\footnote{At first sight it might seem contradictary to 
	postulate de Sitter special relativity in the absence of 
	gravity, while its cosmological constant is to be determined 
	by the general relativistic theory. {\blu The resolution will 
		be found in allowing for a cosmological function on 
		spacetime in the presence of gravity. Note also that the 
		special case $\Lambda = 0$ is still allowed.}}  
Conceptually, this adaptation will find its expression in 
modifying the equivalence principle in such a way to be 
consistent with de Sitter special relativity [{\blu REF}]. The 
equivalence principle may then be formulated as: \emph{at every	
	spacetime point in the presence of gravity it is possible to 
	change reference frame, so that the laws of nature take the 
	same form as in de Sitter special relativity}. This brings up 
the image of spacetime to be described locally by de Sitter 
spaces of possibly varying cosmological constant. Clearly, this 
image is not allowed by the standard formulation of general 
relativity in terms of a Riemannian manifold with a torsionless 
affine metric connection, where the local Minkowskian nature is 
exactly what the equivalence principle for ordinary general 
relativity prescribes. The question then arises, how to account 
for the modified equivalence principle?

An answer may be found in Cartan geometry [{\blu REF to original 
	work}]. This may be illustrated by the following commutative 
diagram, which is adapted from the generic one given in
\cite{sharpe1997diff_geo,Wise:2010sm}.

%\begin{displaymath}
%	\xymatrix@!C=4cm{
%		*+\txt\footnotesize{Minkowskian\\geometry}
%				\ar[r]^-{\mathcal{P} \to SO(4,1)}
%				\ar[d]^-{R_{\mu\nu\rho\sigma} \neq 0}
%				&*+\txt\footnotesize{de Sitter\\Klein geometry}
%				\ar[d]^-{\hat{R}_{\mu\nu\rho\sigma} \neq 0}
%			\\
%		*+\txt\footnotesize{Lorentzian\\geometry}
%				\ar[r]^-{\mathcal{P} \to SO(4,1)}
%				&*+\txt\footnotesize{Cartan geometry\\modelled on dS}
%		}
%\end{displaymath}
Going from left to right 


\section{From Klein to Cartan geometry}

\subsection{Klein geometry}

\subsection{Cartan geometry}

Let us start by writing down the formal definition of a Cartan 
geometry as given in \cite{sharpe1997diff_geo}.

\begin{definition}[Cartan geometry]
	\label{def:cartan_geo}
	A \textbf{Cartan geometry $(P,\tilde{\omega})$ modeled on 		
		$(\mathfrak{g},\mathfrak{h})$} consists of a principal 
	bundle $P(M,H)$ together with a $\mathfrak{g}$-valued 1-form 
	$\tilde{\omega}$ on $P$, satisfying the following properties:
	\begin{itemize}
		\item[(i)] for each $u \in P$, the linear map 
			$\tilde{\omega}_u : T_uP \to \mathfrak{g}$ is an 
			isomorphism;
		\item[(ii)] $\tilde{\omega}(A^\dagger) = A$, for each 
			fundamental vector field $A^\dagger$ corresponding to $A 
			\in \mathfrak{h}$;
		\item[(iii)] $R_h^\ast \tilde{\omega} = 
			\mathrm{Ad}_{h^{-1}} \tilde{\omega}$ for each $h \in H$.
	\end{itemize}
	The 1-form $\tilde{\omega}$ is called a \textbf{Cartan 
		connection}.
\end{definition}

Since $\tilde{\omega} : TP \to \mathfrak{g}$ defines an 
isomorphism at any $u \in P$, it follows that $\dim M = \dim 
G/H$.  Furthermore, the second property implies that the Cartan 
connection $\tilde{\omega}$ restricts to the Maurer-Cartan form 
$\omega_H$ on the fibers of $P$ \cite{sharpe1997diff_geo}.

The \textbf{curvature} $\tilde{\Omega}$ of a Cartan connection is 
the $\mathfrak{g}$-valued 2-form on $P$ defined by
\begin{displaymath}
	\tilde{\Omega} = d\tilde{\omega} + 
	\frac{1}{2}[\tilde{\omega},\tilde{\omega}]~.
\end{displaymath}
This 2-form is horizontal, in the sense that it vanishes when any 
of its arguments is tangent to the fiber. This is understood by 
noting that $\tilde{\omega}$ restricts to $\omega_H$ on the 
fibers for which the curvature is just the structure equation 
\cite{kob1996found}.

The \textbf{torsion} $\Theta$ of the Cartan connection is a 
$\mathfrak{g}/\mathfrak{h}$-valued 2-form on $P$ obtained by 
composing the curvature form with the canonical projection 
$\mathfrak{g} \to \mathfrak{g}/\mathfrak{h}$, that is 
%\begin{displaymath}
%	\xymatrix{
%		T^{(2,0)}P
%			\ar[r]^-{\tilde{\Omega}}
%			\ar@/_15pt/[rr]_-{\Theta}
%				&\mathfrak{g}\ar[r] &\mathfrak{g}/\mathfrak{h}
%	}~.
%\end{displaymath}

By taking the exterior derivative of the curvature, one finds the 
\textbf{Bianchi identity}
\begin{displaymath}
	d\tilde{\Omega} + [\tilde{\omega},\tilde{\Omega}] = 0~.
\end{displaymath}

\subsection{Reductive Cartan geometry}

A Cartan geometry $(P,\tilde{\omega})$ modeled on $(G,H)$ is 
reductive if there is an $H$-module decomposition $\mathfrak{g} = 
\mathfrak{h} \oplus \mathfrak{p}$, i.e.~a splitting of 
$\mathfrak{g}$ in $\mathrm{Ad}(H)$-invariant subspaces.  
Corresponding to this decomposition, one can break up the Cartan 
connection in an $\mathfrak{h}$-valued and a 
$\mathfrak{p}$-valued part
%
\begin{displaymath}
	\tilde{\omega} = \omega + \theta~;\quad
	\omega \in \Omega^1(P,\mathfrak{h}),~
	\theta \in \Omega^1(P,\mathfrak{p})~.
\end{displaymath}
Since the splitting is reductive, the 1-form $\omega$ is an 
Ehresmann connection on $P(M,H)$. This follows from the 
definition\footnote{More precisely, from the second property.} of 
a Cartan connection and because the reductive splitting implies 
$R^\ast_h \omega = \mathrm{Ad}_{h^{-1}}\omega$. The 
$\mathfrak{p}$-valued 1-form $\theta$ is called the 
\textbf{coframe field}.

In the same manner, the curvature form $\tilde{\Omega}$ can be 
written as
%
\begin{displaymath}
	\tilde{\Omega} = \hat{\Omega} + \Theta~;\quad
	\hat{\Omega} \in \Omega^2(P,\mathfrak{h}),~
	\Theta \in \Omega^2(P,\mathfrak{p})~.
\end{displaymath}
%
Seperating the different contributions to $\tilde{\Omega}$ for a 
reductive Lie algebra $\mathfrak{g}$: 
%
\begin{displaymath}
	\tilde{\Omega} = \underbrace{d\omega + 
		\tfrac{1}{2}[\omega,\omega]}_{\mathfrak{h}\text{-valued}}
	+ \underbrace{d\theta + 
		[\omega,\theta]}_{\mathfrak{p}\text{-valued}} + 
	\underbrace{\tfrac{1}{2}[\theta,\theta]}_{\mathfrak{g}\text{-valued}}
\end{displaymath}
In the case that $\mathfrak{g}$ is a \emph{symmetric Lie 
	algebra}, $[\theta,\theta] \in \Omega^2(P,\mathfrak{h})$ so 
that
\begin{displaymath}
	\hat{\Omega} = \Omega + \tfrac{1}{2}[\theta,\theta]~,\quad
	\Theta = d\theta + [\omega,\theta]\quad
	(\text{symmetric}~\mathfrak{g})~.
\end{displaymath}
Here, $\Omega = d\omega + \tfrac{1}{2}[\omega,\omega]$ is the 
curvature of the Ehresmann connection $\omega$. A reductive 
Cartan geometry for which $\Theta$ vanishes is said to be 
\textbf{torsion-free}.

Going on in the same way, the Bianchi identity is decomposed in a 
$\mathfrak{h}$-valued and a $\mathfrak{p}$-valued part. This 
results in two Bianchi identities:
%
\begin{alignat*}{2}
	&d^2_\omega\theta + [\theta,\Omega] = 0&\qquad\quad&(\text{1st 
		identity})~,\\
	&d_\omega \Omega = 0&&(\text{2nd identity})~.
\end{alignat*}
Here, $d_\omega \cdot \equiv d\cdot + [\omega,\cdot]$ denotes the 
exterior covariant derivative. 

\section{de Sitter geometry}

In this section, we apply to above formalism for the a Cartan 
geometry modeled on a de Sitter spacetime, which we will denote 
by a \emph{de Sitter Cartan geometry}. In this case, the model 
Klein geometry is $(\mathfrak{o}(4,1)/\mathfrak{o}(3,1))$, which 
is well known to be symmetric:
%
\begin{equation}
	\label{eq:comm_relations_SO(4,1)}
  \begin{split}
		[L_{ab},L_{kl}] &= \eta_{al}L_{bk} + \eta_{bk}L_{al} - 
		\eta_{ak}L_{bl} - \eta_{bl}L_{ak} \\
		[L_{a4},L_{kl}] &= \eta_{al}L_{4k} - \eta_{ak}L_{4l} \\
		[L_{a4},L_{b4}] &= \eta_{44}L_{ba}
  \end{split}
\end{equation}
where indices run from $0$ to $3$. The generators $L_{ab}$ span 
$\mathfrak{h} = \mathfrak{o}(3,1)$, while $L_{a4}$ span 
$\mathfrak{p}$, the generators of de Sitter translations.  
Introducing some length scale $l$ for the translations, that is 
let $P_a\equiv L_{4a}/l$, the commutation relations 
\eqref{eq:comm_relations_SO(4,1)} can re-expressed as
%
\begin{equation}
	\label{eq:comm_relations_SO(4,1)_length}
  \begin{split}
		[L_{ab},L_{kl}] &= \eta_{al}L_{bk} + \eta_{bk}L_{al} - 
		\eta_{ak}L_{bl} - \eta_{bl}L_{ak} \\
		[L_{kl},P_a] &= \eta_{al}P_k- \eta_{ak}P_l\\
		[P_a,P_b] &= \frac{\eta_{44}}{l^2}L_{ba}
  \end{split}
\end{equation}

A de Sitter Cartan geometry then consists of the bundle of 
orthonormal frames\linebreak[4] $OM(M,SO(3,1))$ over spacetime 
$M$ and a Cartan connection of type 
$(\mathfrak{o}(4,1),\mathfrak{o}(3,1))$ on $OM$, that locally on 
$M$ can be expanded as\footnote{The same symbol is used for a 
	Cartan connection on $P$ and its pulled back version on $M$.}
%
\begin{equation}
	\label{eq:cartan_conn_dS}
	\tilde{\omega}_\mu dx^\mu = 
	\tfrac{1}{2}\omega^{ab}_{~~\mu}L_{ab} dx^\mu + \theta^a_{~\mu} 
	P_a dx^\mu~.
\end{equation}

The curvature $2$-form on the other hand, is expanded 
as\footnote{The local versions of $\hat{\Omega}$ and $\Omega$ are 
	given the symbols $\hat{R}$ and $R$, respectively.}
%
\begin{displaymath}
	\hat{R} = \tfrac{1}{4} \hat{R}^{ab}_{~~\mu\nu} L_{ab}\, dx^\mu 
	\wedge dx^\nu~.
\end{displaymath}
Given the definition of $\hat{R} = d\omega + 
\tfrac{1}{2}[\omega,\omega] + \tfrac{1}{2}[\theta,\theta]$ and 
due to the commutation relations 
\eqref{eq:comm_relations_SO(4,1)_length}, the curvature of the 
Cartan connection is equal to
%
\begin{displaymath}
	\begin{split}
		\hat{R}^{ab}_{~~\mu\nu} &= \pd_\mu \omega^{ab}_{~~\nu} - 
		\pd_\nu \omega^{ab}_{~~\mu} + 
		\omega^a_{~c\mu}\omega^{cb}_{~~\nu} - 
		\omega^a_{~c\nu}\omega^{cb}_{~~\mu} - l^{-2} \eta_{44} 
		(\theta^a_{~\mu}\theta^b_{~\nu} - 
		\theta^a_{~\nu}\theta^b_{~\mu}) \\
		&= R^{ab}_{~~\mu\nu} - 
		l^{-2}\eta_{44}(\theta^a_{~\mu}\theta^b_{~\nu} - 
		\theta^a_{~\nu}\theta^b_{~\mu})
	\end{split}
\end{displaymath}
where $R^{ab}_{~~\mu\nu}$ is the curvature of the spin 
connection. The second contribution to the curvature is due to 
the commutation relations of the translations. This term is a 
crucial difference with the usual affine connection (modeled on 
Minkowski space). In the latter the corresponding commutator 
vanishes and the curvature of the metric affine connection equals 
the curvature of the spin connection. Note that a \emph{flat} 
geometry $(\hat{R} = 0)$ with respect to the Cartan connection 
implies a constantly curved geometry with respect to the spin 
connection. {\blu [Work out argument. In fact, maybe it is a 
	better idea to first translate objects into spacetime language 
	and then discuss this matter.]}

The torsion 2-form is locally given by
%
\begin{displaymath}
	T = \tfrac{1}{2} T^a_{~\mu\nu} P_a dx^\mu \wedge dx^\nu~.
\end{displaymath}
Since the torsion equals $T = d\theta + [\omega,\theta]$, it 
follows that
%
\begin{displaymath}
	T^a_{~\mu\nu} = \pd_\mu\theta^a_{~\nu} - 
	\pd_\nu\theta^a_{~\mu} + \omega^a_{~b\mu}\theta^b_{~\nu} - 
	\omega^a_{~b\nu}\theta^b_{~\mu}~.
\end{displaymath}
As one would expect, the torsion equals the one associated with a 
metric affine connection---since the commutation relations 
involved are identical.

The Bianchi identities for this geometry are as follows. The 
first identity is locally expressed as\footnote{The covariant 
	derivative with respect to the spin connection $\omega$ is 
	denoted by $D$.}
%
\begin{displaymath}
	D_\rho T^a_{\mu\nu} + D_\nu T^a_{\rho\mu} + D_\mu 
	T^a_{\nu\rho} = \theta^b_{~\rho}R^a_{~b\mu\nu} + 
	\theta^b_{~\nu}R^a_{~b\rho\mu} + 
	\theta^b_{~\mu}R^a_{~b\nu\rho}~,
\end{displaymath}
while the second identity is written out as
%
\begin{displaymath}
	D_\rho R^a_{~b\mu\nu} + D_\nu R^a_{~b\rho\mu} + D_\mu 
	R^a_{~b\nu\rho} = 0~.
\end{displaymath}




\section{de Sitter geometry with spacetime dependent length 
	scale}

In this section we try to generalize de Sitter Cartan geometry 
where the length scale, which was introduced in the de Sitter 
translations, is allowed to vary over spacetime. The reason for 
this, is that this length scale fixes the local cosmological 
constant of the tangent de Sitter space \cite{Wise:2010sm}. A 
spacetime dependent length scale then induces a cosmological 
function.  Of course, whether this can be made consistent 
with---possibly modified---Einstein's equations remains to be 
shown.

More precisely, the generalization is done by $l \to l(x)$ in 
\eqref{eq:cartan_conn_dS}, so that
%
\begin{equation}
	\tilde{\omega}_\mu dx^\mu = 
	\tfrac{1}{2}\omega^{ab}_{~~\mu}L_{ab} dx^\mu + 
	l^{-1}(x)\theta^a_{~\mu} L_{4a} dx^\mu~.
\end{equation}
Of course the commutation relations are not to be affected by 
this change, since de Sitter Klein geometry is still to be the 
model.  What is affected by changing the length scale is the 
size---pseudo-radius---of the corresponding de Sitter space, see 
e.g.~\cite{Wise:2010sm}.

The resulting curvature and torsion tensors are then
%
\begin{displaymath}
	\hat{R}^{ab}_{~~\mu\nu} = R^{ab}_{~~\mu\nu} - l^{-2}(x) 
	\eta_{44}(\theta^a_{~\mu}\theta^b_{~\nu} - 
	\theta^a_{~\nu}\theta^b_{~\mu})
\end{displaymath}
and\footnote{Note, $T = \tfrac{1}{2}l^{-1}(x) 
	T^a_{~\mu\nu}L_{4a}dx^\mu\wedge dx^\nu$.}
\begin{multline}
	l^{-1}(x) T^a_{~\mu\nu} = l^{-1}(x) (\pd_\mu\theta^a_{~\nu} - 
	\pd_\nu\theta^a_{~\mu} + \omega^a_{~b\mu}\theta^b_{~\nu} - 
	\omega^a_{~b\nu}\theta^b_{~\mu}) \\
	- l^{-2}(x)( \pd_\mu l(x) \theta^a_{~\nu} - \pd_\nu l(x) 
	\theta^a_{~\mu})~.
\end{multline}
We conclude that the torsion is affected by a spacetime dependent 
length scale. {\blu [This does possibly make sense.  Torsion is a 
	measure for the amount the point of tangency of our model 
	spacetime (dS) with spacetime $M$ has been ``translated", 
	after moving along a closed path on $M$ \cite{Wise:2010sm}.  
	This amount will be influenced rather directly if this model 
	spacetime has a changing cosmological constant!]}.  On the 
other hand, the curvature $(\hat{R})$ has changed in a rather 
passive way. It now shows that our spacetime is locally 
\emph{flat} for a constant spin curvature $(R)$ for the given 
length scale.  Finally, the spin curvature is unchanged.  {\blu 
	[Again a possible interpretation is that the amount to which 
	our frame has rotated after moving along a closed path is 
	independent of the length scale of our model spacetime.]}

\section{Einstein's equations}

In case we consider a de Sitter Cartan geometry with vanishing 
torsion, it seems we are led to the Einstein's equations, for 
which the left hand side would take the form {\blu [Check this!]}
%
\begin{equation}
	G_{\mu\nu} + \Lambda(x)g_{\mu\nu}~.
\end{equation}

\bibliographystyle{plain}
\bibliography{../../References/All.bib}
\end{document}
