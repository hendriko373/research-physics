\documentclass[10pt,reqno]{amsart}

%Package declarations
%--------------------

\usepackage[english]{babel}
\usepackage{color}

\usepackage{amsmath,amsfonts,amssymb,amsthm}
\usepackage[all]{xy}
\usepackage{setspace}
\setstretch{1.10}
\usepackage[a5paper]{geometry}
\geometry{left=1.8cm,right=1.8cm,top=3cm,bottom=3cm}

\usepackage{thmtools}
\declaretheorem[numberwithin=section]{definition}
\declaretheorem[numberwithin=section]{proposition}
\let\proof\relax
\declaretheorem[style=remark,numbered=no,qed=\qedsymbol]{proof}

\usepackage[latin1]{inputenc}
\usepackage[T1]{fontenc}
\usepackage{concmath}
\usepackage{fix-cm}

\usepackage{../Tex_files/preamble_one}
%\newcommand{\mcA}{\mcal{A}} %Local connection 1-from
%\newcommand{\mfg}{\mfrak{g}} %Lie algebra

\numberwithin{equation}{section}

\author{Hendrik}
\title{Differential geometry: a survey}
\date{\today}

\begin{document}
\begin{abstract}
In this file notes on differential geometry in general will be 
put; we will try to be sufficiently abstract and leave physical 
applications for other notes.  However, it should be done in such 
a way that these applications are fairly easy obtained.
\end{abstract}

\maketitle

\noindent Books used: \cite{nakahara2003geometry}, 
\cite{o1983semi}, \cite{kob1996found}, 
\cite{ruben1995introduction}.

\tableofcontents

\newpage

\section{Preliminaries}

\subsection{Topological spaces}

\begin{definition}[Topological space]
	Let $S$ be a set and denote by $\mcal{J} = \{U_i | i \in I\}$ a 
	collection of subsets of $S$. The pair $(S,\mcal{J})$ is called 
	a \emph{topological space} if the following is true.
	\begin{itemize}
		\item[(1)] Both the empty set and $S$ are in $\mcal{J}$, 
			i.e.\ $\emptyset, S \in \mcal{J}$.
		\item[(2)] Any (possibly infinite) union of subsets of $S$ 
			which are in $\mcal{J}$ is also in $\mcal{J}$, i.e.\ 
			$\cup_{j} U_j \in \mcal{J}$,
			where $j \in J \subset I$.
		\item[(3)] Any finite intersection of subsets of $S$ which 
			are in $\mcal{J}$ is also in $\mcal{J}$, i.e.\ $\cap_{j} 
			U_j \in \mcal{J}$, where $j \in J \subset I$.
	\end{itemize}
\end{definition}
%
The $U_i$ are called \emph{open sets} and $\mcal{J}$ gives a 
\emph{topology} to $S$. If $\mcal{J}$ is the collection of all 
subsets of $S$, it follows directly from the definition that it 
is a topology---the \emph{discrete} topology. Another extreme 
example is the topology given by $\emptyset$ and $S$ only, which 
is called the trivial topology.
%

\begin{example}
	The open intervals $(a,b)$ on $\mbb{R}$ define a topology. Note 
	that if one required an infinite intersection of open sets to 
	be an open set, the open intervals would not give a topology to 
	$\mbb{R}$, indeed for example
  %
	\begin{displaymath}
		\bigcap_n^\infty \left( -\frac{1}{n}, 1-\frac{1}{n} \right) = 
		[0,1]
	\end{displaymath}
  %
	The closed intervals $[a,b]$ do not define a topology as an 
	infinite union of them is generally not closed.
\end{example}
%

Let us call the elements of $S$ points. A subset $N$ of $S$ is a 
\emph{neighborhood} of a point $x$ if it contains an open set 
$U_i$ to which $x$ belongs. If $N$ is open it is an \emph{open 
	neighborhood}. 

If for any arbitrary pair of points of a topological 
space\footnote{We will abuse terminology: let us refer to $S$ 
	also as the topological space, although strictly speaking the 
	term should be reserved for the pair $(S,\mcal{J})$.} there 
exist disjoint neighborhoods, the topological space is a 
\emph{Hausdorff space}.

\begin{example}
	The real line with the usual topology is a Hausdorff space.  
	Indeed for any two points $x_1$ and $x_2$, choose neighborhoods 
	$(x_i-\delta,x_i+\delta)$ with $\delta \leqslant |x_1 - 
	x_2|/2$.
\end{example}

\subsection{Homotopy}

\begin{definition}[Paths and loops]
	Let $S$ be a topological space and denote the interval $[0,1]$ 
	by $I$. A \emph{path} is a continuous map $\mu : I \to S : t 
	\mapsto \mu_t$, with initial point $p = \mu_0$ and end point 
	$q = \mu_1$.
	If $p = \mu_0 = \mu_1$, then $\mu$ is called a \emph{loop} 
	with \emph{base point} $p$.
\end{definition}
A rather trivial but nonetheless important case is the 
\emph{constant path}, i.e.\ $c_t = p$ $(0 \leq t \leq 1)$. One 
endows the set of paths in $S$ with an algebraic structure by 
introducing the following operation.
%
\begin{definition}[Product and inverse]
	Let $\mu$ and $\tau$ be paths in $S$, for which $\mu_1 = 
	\tau_0$. The \emph{product} $\mu\cdot\tau$ ($\tau$ after 
	$\mu$) is the path in $S$ defined by
	\begin{equation}
		(\mu\cdot\tau)_t \equiv
		\begin{cases}
			\mu_{2t} & ;\quad 0 \leq t \leq \tfrac{1}{2}~, \\
			\tau_{2t-1} & ;\quad \frac{1}{2} \leq t \leq 1~.
		\end{cases}
	\end{equation}
	The \emph{inverse path} $\mu^{-1}$ of $\mu$ is the path 
	defined by
	\begin{equation}
		\mu^{-1}_t \equiv \mu_{1-t}~.
	\end{equation}
\end{definition}
Note that the inverse path $\mu^{-1}$ begins at the end point and 
ends at the begin point of $\mu$. It follows that 
$\mu\cdot\mu^{-1}$ is not a constant path.

\begin{definition}[Homotopy]
	Let $\mu$ and $\tau$ be loops at $p$ in $S$. The loops $\mu$ 
	and $\tau$ are said to be \emph{homotopic} with each other if 
	there exists a \emph{continuous} mapping $f : I \times I \to 
	S$ such that $f(t,0) = \mu_t$ and $f(t,1) = \tau_t$, while 
	$f(0,s) = f(1,s) = p$. One calls $f$ a homotopy between $\mu$ 
	and $\tau$ and writes $\mu \sim \tau$.
\end{definition}
Intuitively, two loops are homotopic if they can be deformed 
smoothly into each other.

\begin{proposition}
	The homotopy relation is an equivalence relation.
\end{proposition}
\begin{proof}
	\emph{Reflectivity}: $\mu \sim \mu$. Any loop is homotopic 
	with itself; choose $f(t,s) \equiv \mu_t$.
	\emph{Symmetry}: $\mu \sim \tau \Rightarrow \tau \sim \mu$.  
	Let $f$ be the homotopy $\mu \sim \tau$. Then $g : I \times I 
	\to S$ defined by $g(t,s) \equiv f(t,s-1)$ is a homotopy $\tau 
	\sim \mu$.
	\emph{Transitivity}: $\mu \sim \tau$ and $\tau \sim \rho
	\Rightarrow \mu \sim \rho$. Let $f$ be the homotopy between 
	$\mu$ and $\tau$, and $g$ the homotopy between $\tau$ and 
	$\rho$. Then
	\begin{displaymath}
		h(t,s) =
		\begin{cases}
			f(t,2s) &;\quad 0 \leq s \leq \tfrac{1}{2} \\
			g(t,2s-1) &;\quad \tfrac{1}{2} \leq s \leq 1
		\end{cases}
	\end{displaymath}
	is a homotopy $\mu \sim \rho$.
\end{proof}

Given this equivalence relation, one can define corresponding 
equivalence classes. Let $\mu$ be a loop at $p \in S$. The set of 
all loops at $p$ homotopic with $\mu$ is denoted by $[\mu]$.

%%% SECTION %%%
\section{Manifolds}

\subsection{Smooth manifolds}

Before introducing manifolds we define the concept of an 
\emph{atlas}.

\begin{definition}[Atlas]
	A differentiable atlas $\mcal{A}$ of dimension $m$ of a 
	topological space $S$ is a family of pairs 
	$\left(U_i,\phi_i\right)$---the \emph{charts}---such that
	\begin{itemize}
		\item[(a)] Each $U_i$ is an open set of $S$ and $\cup_i U_i = 
			S$
		\item[(b)] Each $\phi_i$ is a homeomorphism from of $U_i$ 
			onto an open subset of $\mbb{R}^m$
		\item[(c)] Whenever $U_i \cap U_j$ is non-empty the map 
			$\phi_j \circ \phi_i^{-1}$ from $\phi_i(U_i \cap U_j)$ to 
			$\phi_j(U_i \cap U_j)$ is 
			differentiable.\footnote{Differentiable or smooth: the 
				mapping is infinitely differentiable.}
	\end{itemize}
\end{definition}
%
The following then introduces \emph{differentiable manifolds}.
%
\begin{definition}[Manifold]
	An $m$-dimensional differentiable manifold $M$ is a Hausdorff 
	space equipped with an $m$-dimensional differentiable atlas.
\end{definition}
%
Given two manifolds $M$ and $N$, a mapping $f : M \rightarrow N$ 
is differentiable if for every chart $(U_i,\phi_i)$ of $M$ and 
every chart $(V_j,\psi_j)$ of $N$ such that $f(U_i) \subset V_j$, 
the mapping $\psi_j \circ f \circ \phi_i^{-1} : \mbb{R}^m 
\rightarrow \mbb{R}^n$ is differentiable. If the inverse mapping 
$f^{-1}$ exists and is differentiable, the mapping is called a 
\emph{diffeomorphism} and $M$ and $N$ are said to be 
diffeomorphic. They are of the same dimension. If $M$ and $N$ are 
the same manifold, the diffeomorphism is also an 
\emph{automorphism}.

A special case of a mapping is given by a curve $c: [a,b] 
\rightarrow M$, which maps a closed interval of $\mbb{R}$ into a 
manifold $M$. Another interesting example are the functions $f : 
M \rightarrow \mbb{R}$, which assign to each point of a manifold 
a real number. We will denote the algebra of smooth functions on 
$M$ by $\mfrak{F}(M)$.

\blankline
Consider then a curve $c(t)$ on a manifold such that $c(t_0) = 
p$. A \emph{tangent vector} $X_p$ to $c(t)$ at the point $p$ is 
the directional derivative of functions along the curve at $p$, 
i.e.
\begin{equation}
	X_p f \equiv \left. \frac{df(c(t))}{dt}\right|_{t=t_0}
\end{equation}
%

\begin{definition}\emph{(Tangent vector)}
	Let $p$ be a point on a manifold $M$. A tangent vector to $M$ 
	at $p$ is a real-valued function $X_p : \mfrak{F}(M) 
	\rightarrow \mbb{R}$ that is
	\begin{itemize}
		\item[(1)] $\mbb{R}$-linear, i.e. $X_p(af+bg) = aX_p(f) + 
			bX_p(g)$, and
		\item[(2)] Leibnizian, i.e. $X_p(fg) = X_p(f)g(p) + 
			f(p)X_p(g)$
	\end{itemize}
	for all $a,b \in \mbb{R}$ and $f,g \in \mfrak{F}(M)$.
\end{definition}
The set of tangent vectors at the point $p$ forms a vector space, 
the \emph{tangent space} $T_p M$.

\blankline
A \emph{vector field} $X$ on a manifold $M$ is an assignment of a 
vector $X_p$ to each point $p$ on $M$.
\begin{equation}
	(Xf)(p) = X_p(f) \quad \text{for all}\spc p \in M
\end{equation}
We call $X$ differentiable if the function $Xf$ is differentiable 
for every differentiable function $f$. Denote the set of all 
differentiable vector fields on $M$ by $\mfrak{X}(M)$. Then, 
$\mfrak{X}(M)$ forms a module over the algebra $\mfrak{F}(M)$. It 
follows immediately that it also is a vector space over the 
reals---as the latter are just constant elements in 
$\mfrak{F}(M)$. This infinite dimensional vector space also is a 
Lie algebra over the reals, as the commutator of two vector 
fields is a vector field.
%
\begin{equation}
	[X,Y]f = (X^j \pd_j Y^i - Y^j \pd_j X^i) \pd_i f
\end{equation}
%
If one has a \emph{frame} $X_a$ over $M$, i.e.\ a set of $n$ 
vector fields such that at each point $p$ they form a basis for 
$T_p M$ we can write
%
\begin{equation}\label{eq:GenStrucEqs}
	[X_a,X_b] = c_{ab}^{\spc\spc c}(p) X_c
\end{equation}
%
Note that the \emph{structure functions} depend on $p$ since 
there is no reason to assume that the commutator of two frame 
fields is a linear combination of frame fields. The commutator 
will be a generic vector field, an element of an 
infinitedimensional vector space, for which the frame fields are 
\emph{not} a basis.

\blankline
A \emph{1-form} $\omega$ on a manifold $M$ is defined as an 
$\mfrak{F}(M)$-linear mapping of the module $\mfrak{X}(M)$ into 
$\mfrak{F}(M)$.
Hence, at each point $p$ they are the duals of the tangent 
vectors and the space spanned by them is called the 
\emph{cotangent space} $T^\ast_p M$. A 1-form $\omega$ is 
differentiable if $\omega(X)$ is differentiable for all $X \in 
\mfrak{X}(M)$. The set of all 1-forms $\mfrak{X}^\ast(M)$ then 
also is a module over the algebra $\mfrak{F}(M)$. The \emph{total 
	differential} of $f \in \mfrak{F}(M)$ at $p$ is defined by the 
1-form $df_p$ for which
\begin{equation}
	(df(X))_p = X_p(f) \quad \text{for all}\spc X \in \mfrak{X}(M)
\end{equation}
%

Let $\varphi : M \rightarrow N$ be a mapping. It is possible to 
consider the following induced mappings. First let us consider 
the \emph{pullback} of a generic function $f$, that is
%
\begin{equation}
	\varphi^\ast : \mfrak{F}(N) \to \mfrak{F}(M) : f \mapsto 
	\varphi^\ast f \quad \mathrm{with} \quad \varphi^\ast f \equiv 
	f \circ \varphi~.
\end{equation}
%
This allows one to define the \emph{differential map} or 
\emph{pushforward} of a vector field, the linear mapping 
%
\begin{equation}
	\varphi_\ast : TM \rightarrow TN : X \mapsto \varphi_\ast X
\end{equation}
such that for every function $f \in \mfrak{F}(N)$, we have
\begin{equation}
	(\varphi_\ast X) f \equiv X (\varphi^\ast f)
\end{equation}
%
Given the notion of a pushforward of a vectorfield, one defines 
the pullback of a 1-form $\omega$ by
\begin{equation}
	\varphi^\ast : T^\ast N \rightarrow T^\ast M : \omega \mapsto 
	\varphi^\ast \omega
\end{equation}
such that for all vectors $X \in T_p M$
\begin{equation}
	(\varphi^\ast \omega) (X) \equiv \omega (\varphi_\ast X)
\end{equation}
%
These definitions are readily extended to incorporate the 
pushforward and pullback of respectively complete contravariant 
and covariant tensors. Indeed, we define
%
\begin{equation}
	\varphi_\ast : T_{(r,0)}M \to T_{(r,0)}N : t \mapsto 
	\varphi_\ast t
	\quad \text{with} \quad
	(\varphi_\ast t)(\omega_1,\ldots,\omega_r) \equiv 
	t(\varphi^\ast\omega_1,\ldots,\varphi^\ast\omega_r)
\end{equation}
and
%
\begin{equation}
	\varphi^\ast : T_{(0,r)}N \to T_{(0,r)}M : t \mapsto 
	\varphi^\ast t
	\quad \text{with} \quad
	(\varphi^\ast t)(X_1,\ldots,X_r) \equiv t(\varphi_\ast 
	X_1,\ldots,\varphi_\ast X_r)~.
\end{equation}

Let us consider the effect of these mappings in some coordinate 
system $x^i(p)$ on $M$ and $y^i(\varphi(p))$ on $N$. A vector 
field $X = X^i(x(p))\pd_i|_p$ transforms according to
%
\begin{displaymath}
	\begin{split}
		(\varphi_\ast X)^i(y)\pd_i|_{\varphi(p)} f &\equiv 
		X^i(x)\pd_i|_p f\circ\varphi \\
		&= X^j(x) \frac{\pd y^i}{\pd x^j} \pd_i|_{\varphi(p)} f
	\end{split}
\end{displaymath}
A 1-form $\omega = \omega_i(y(\varphi(p)))dy^i(\varphi(p))$ 
transforms as
%
\begin{displaymath}
	\begin{split}
		(\varphi^\ast \omega)_i(x)dx^i(p) (X^j(x)\pd_j|_p) &\equiv 
		\omega_i(y) dy^i(\varphi(p)) (\underbrace{X^j(x) \frac{\pd 
				y^k}{\pd x^j} \pd_k|_{\varphi(p)}}_{\phi_\ast x})\\
		(\phi^\ast\omega)_i(x)X^i(x) &= \omega_j(y) \frac{\pd 
			y^j}{\pd x^i} X^i(x)
	\end{split}
\end{displaymath}
%
These kind of calculations can be done straightforwardly for any 
completely contravariant or covariant tensor.

Note how the differential map works in the same direction as the 
mapping $\varphi$, while the pullback goes the other way around.  
Therefore, a mixed type tensor cannot be transported in the above 
defined manner. However, if the mapping $\varphi$ is a 
diffeomorphism one can extend the definition of a pushforward 
(pullback) by noting that the inverse mapping exists. Indeed, 
given the fact that the pullback \emph{pulls the tensor back}, we 
can push it along $\varphi$ by considering the pullback of the 
inverse of $\varphi$. In other words, given the diffeomorphism 
$\phi$ and the pushforward $\varphi_\ast$ of a vector, we map a 
form $\omega$ in the same direction as $\varphi$ by invoking 
$\varphi^{-1\ast} \omega$.

\begin{definition}[Differential map]
Let $\varphi : M \to M$ be a diffeomorphism. The pushforward of 
an arbitrary tensorfield, induced by the diffeomorphism $\varphi$ 
is the linear mapping
\begin{equation}
	\varphi_\ast : T_{(r,s)}M \to T_{(r,s)}M : t \mapsto 
	\varphi_\ast t
\end{equation}
such that for any set of 1-forms and vector fields
\begin{equation}
	(\varphi_\ast t)(\omega_1,\ldots,\omega_r,X_1,\ldots,X_s) 
	\equiv t(\varphi^\ast\omega_1, \ldots, \varphi^\ast\omega_r, 
	\varphi^{-1}_\ast X_1, \ldots, \varphi^{-1}_\ast X_s)~.
\end{equation}
\end{definition}
One could define also the pullback for a generic tensor field, 
but the above considerations show that $\phi^\ast = 
\phi^{-1}_\ast$ to be consistent.

Let $x^i(p)$ be a coordinate system on $M$. Under a the 
pushforward of a diffeomorphism, a generic tensor field 
transforms as (we omit the coordinate basis elements)
%
\begin{equation}
	(\varphi_\ast t)^{i_1\cdots i_r}_{j_1\cdots j_s}(x(\varphi(p))
	= t^{k_1\cdots k_r}_{l_1\cdots l_s} \frac{\pd 
		x^{i_1}(\varphi(p))}{\pd x^{k_1}(p)} \cdots \frac{\pd 
		x^{i_r}(\varphi(p))}{\pd x^{k_r}(p)} \frac{\pd 
		x^{l_1}(p)}{\pd x^{j_1}(\varphi(p))} \cdots \frac{\pd 
		x^{l_s}(p)}{\pd x^{j_s}(\varphi(p))}
\end{equation}

The resemblence with the transformation law under general 
coordinate transformations is not a coincidence. Given a 
diffeomorphism $\varphi$ and a coordinate system $x^i$, a 
coordinate transformation can be defined by pulling back the 
coordinate functions, i.e.\ $y^i(p) := (\varphi^\ast x^i)(p) = 
x^i(\varphi(p))$. Instead of performing the diffeomorphism on 
$M$---the \emph{active} point of view---one can leave $M$ as it 
is and perform the coordinate transformation by pulling back the 
coordinate functions---the \emph{passive} point of view. The 
transformation laws of tensors under change of coordinates are 
then identified with the pushforwards/pullbacks of the given 
diffeomorphism. It is then a matter of taste if one thinks of it 
as an active or passive transformation.

\blankline
Let $M$ be an $m$-dimensional manifold with atlas $\{(U_i, 
\phi_i)\}$ and $N$ an $n$-dimensional manifold with atlas 
$\{(V_i, \psi_i)\}$. The \emph{product manifold} $M \times N$ is 
defined as the manifold with atlas $\{(U_i \times V_j), 
(\phi_i,\psi_j)\}$. A point $(p,q)$ of $M \times N$ has 
coordinates $(\phi_i,\psi_j)(p,q) = (\phi_i(p),\psi_j(q)) \in 
\mathbb{R}^{m+n}$, from which it is clear that the dimension of 
$M \times N$ is $m + n$. For every $(p,q) \in M \times N$, the 
tangent space $T_{(p,q)} M \times N$ can be identified with the 
direct sum $T_p M + T_q N$ as follows. Let $X \in T_p M$ and $Y 
\in T_q N$ be the tangent vectors to the curves $\mu_t$ and 
$\nu_t$, at $p = \mu_0$ and $q = \nu_0$ respectively. We identify 
$(X,Y) \in T_p M + T_q N$ with $Z \in T_{(p,q)} M \times N$, the 
tangent vector to $(\mu_t,\nu_t)$ at $(p,q) = (\mu_0,\nu_0)$.  
Consider the mappings $\imath_q : M \to M \times N : p' \mapsto 
(p',q)$ and $\imath_p : N \to M \times N : q' \mapsto (p,q')$. It 
follows that $Z = \imath_{q\ast}X + \imath_{p\ast}Y$. To see 
this, consider
%
\begin{displaymath}
	Zf = \tfrac{d}{dt}f(\mu_t,\nu_t)|_0 = 
	\tfrac{d}{dt}f(\mu_t,q)|_0 + \tfrac{d}{dt}f(p,\nu_t)|_0 = 
	(\imath_{q\ast}X)f + (\imath_{p\ast}Y)f~.
\end{displaymath}
\begin{proposition}[Leibniz's formula]
	Let $\varphi : M \times N \to K$ be a mapping and consider the 
	compositions
	%
	\begin{align*}
		\varphi_1 &\equiv \varphi\circ\imath_q : M \to K : p' 
		\mapsto (p',q) \\
		\varphi_2 &\equiv \varphi\circ\imath_p : N \to K : q' 
		\mapsto (p,q')~.
	\end{align*}
	The differential $\varphi_\ast$ can be expressed in the 
	following manner. If $Z \in T_{(p,q)} M \times N$ corresponds 
	to $(X,Y) \in T_p M + T_q N$, then
	\begin{equation}
		\varphi_\ast(Z) = \varphi_{1\ast}(X) + \varphi_{2\ast}(Y)~.
	\end{equation}
\end{proposition}
\begin{proof}
	Since $Z = \imath_{q\ast}X + \imath_{p\ast}Y$, it follows 
	that
	\begin{displaymath}
		\varphi_\ast(Z) = \varphi_\ast(\imath_{q\ast}X + 
		\imath_{p\ast}Y) = \varphi_{1\ast}(X) + 
		\varphi_{2\ast}(Y)~.
	\end{displaymath}
\end{proof}

\subsection{Differential forms}

\begin{definition}[Differential form]
	A differential form $\omega$ of degree $r$ is a $(0,r)$-type 
	tensor field which is skew symmetric, that is
	\begin{equation}
		\omega(X_{\pi(1)},\ldots,X_{\pi(r)}) = 
		\varepsilon(\pi)\cdot \omega(X_1,\ldots,X_r)~,
	\end{equation}
	where $\pi$ is an element of the group of permutations of 
	$(1,\ldots,r)$ and $\epsilon(\pi)$ the corresponding sign.
\end{definition}
%
In other words, a differential form is a multilinear 
skew-symmetric mapping of $\mfrak{X}(M) \times \ldots \times 
\mfrak{X}(M)$ into $\mfrak{F}(M)$. We denote the module of 
differential $r$-forms on $M$ by $\Omega^r(M)$.

Given an arbitrary tensor $(0,r)$-type tensor $t$, the 
\emph{antisymmetrizer} $A$ is defined as follows,
%
\begin{equation}
	(At)(X_1,\ldots,X_r) := \frac{1}{r!} \sum_\pi \varepsilon(\pi) 
	\cdot t(X_{\pi(1)},\ldots,X_{\pi(r)})~.
\end{equation}
%
From the definition of differential forms it follows that these 
are eigenstates of $A$, as can be seen from
%
\begin{displaymath}
	\begin{split}
		(A\omega)(X_1,\ldots,X_r) &= \frac{1}{r!} \sum_\pi 
		\varepsilon(\pi)^2 \cdot \omega(X_1,\ldots,X_r) \\
		&= \omega(X_1,\ldots,X_r)~.
	\end{split}
\end{displaymath}
On the other hand, if $\omega = A\omega$ then it is a 
differential form--being completely skew-symmetric in its 
arguments. We conclude that a $(0,r)$-type tensor is a $r$-form 
if and only if it an eigenstate of the antisymmetrizer operator.  
Furthermore, since $A$ is a projection operator, i.e.~$A^2=A$, 
any $(0,r)$-tensor is mapped into a differential form under $A$.

Given the antisymmetrizer operator $A$, the \emph{exterior} or 
\emph{wedge product} of two differential forms is defined as 
$(\omega \in \Omega^r(M)~,~\xi \in \Omega^s(M))$
%
\begin{equation}
	\omega \wedge \xi \equiv \tfrac{(r+s)!}{r!s!} A(\omega \otimes 
	\xi)
\end{equation}
which is an element of $\Omega^{r+s}(M)$.

Every differential $r$-form $\omega$ may be obtained from some 
$(0,r)$-tensor $t$ by invoking $A$, that is $\omega = At$. Since, 
$A$ is a linear operator and considering a local coordinate basis 
of $M$ we have $\omega = t_{i_1\cdots i_r} A(dx^{i_1}\otimes 
\cdots dx^{i_r})$. It follows that\footnote{Note that this is 
	consistent with the definition of the wedge product, since 
	$\omega \wedge \xi \wedge \eta = \tfrac{(r+s+q)!}{r!s!q!} 
	A(\omega \otimes \xi \otimes \eta)$, for $\omega \in 
	\Omega^r(M)$, $\xi \in \Omega^s(M)$ and $\eta \in 
	\Omega^q(M)$.}
%
\begin{displaymath}
	dx^{i_1}\wedge\cdots dx^{i_r} = r!  A(dx^{i_1}\otimes\cdots 
	dx^{i_r})
\end{displaymath}
form a basis for the vector spaces $\Omega^r_p(M)$. Hence any 
differential form $\omega$ may be written as $\omega = 
\tfrac{1}{r!}\omega_{i_1\cdots i_r} dx^{i_1}\wedge\cdots 
dx^{i_r}$. Note that this is equivalent to
%
\begin{equation}
	\omega = \tfrac{1}{r!} \omega_{i_1\cdots i_r} \sum_\pi 
	\varepsilon(\pi)\cdot dx^{\pi(i_1)}\otimes\cdots dx^{\pi(i_r)} 
	= \omega_{[i_1\cdots i_r]} dx^{i_1}\otimes\cdots dx^{i_r}~.  
\end{equation}
Hence, the a differential form is just a tensor with 
antisymmetric component functions. This means also that the 
possible symmetric part in $\omega_{i_1\dots i_r}$ may be 
omitted. We state without derivation that the wedge product of 
two differential forms is expressed in local coordinates as
%
\begin{equation}
	\omega \wedge \xi = \frac{1}{r!s!} \omega_{i_1\cdots i_r} 
	\xi_{i_{r+1}\cdots \xi_{r+s}} dx^{i_1}\wedge \ldots 
	dx^{i_{r+s}}~.
\end{equation}

This explicit construction reveals some interesting properties 
regarding differential forms. For example, the dimension of the 
vector space $\Omega^r_p(M)$ is given by $\binom{m}{r}$. Also, 
there are no $r$-forms for $r > m$. Given this information the 
reader may understand that the \emph{exterior 
	algebra}\footnote{We remark that $\Omega^0(M) := 
	\mfrak{F}(M)$.}
\begin{equation}
	\Omega(M) \equiv \Omega^0(M) \oplus \cdots \oplus \Omega^m(M)
\end{equation}
of differential forms on $M$ is closed under the exterior 
product.

\begin{definition}[Exterior differentiation]
	\emph{Exterior differentiation} is the unique linear mapping
	\begin{displaymath}
		d : \Omega^{r}(M) \to \Omega^{r+1}(M)
	\end{displaymath}
	satisfying the properties:
	\begin{enumerate}
		\item for any function $f \in \Omega^0(M)$, $df$ is the 
			total differential
		\item $d \circ d \equiv 0$
		\item for any $\omega \in \Omega^r(M)$ and $\xi \in 
			\Omega^s(M)$,
			\begin{displaymath}
				d(\omega \wedge \xi) = d\omega \wedge \xi + (-1)^r 
				\omega \wedge d\xi
			\end{displaymath}
	\end{enumerate}
\end{definition}
In local coordinates the action of this operator is given by 
\begin{displaymath}
	d\omega = \frac{1}{r!} \pd_i \omega_{i_1\cdots i_r} dx^i 
	\wedge dx^{i_1} \wedge \cdots dx^{i_r}~.
\end{displaymath}

\begin{definition}[Interior product]
	The \emph{interior product} of a differential form with 
	respect to a vector field $X$ is the mapping
	%
	\begin{equation}
		i_X : \Omega^r(M) \to \Omega^{r-1}(M)
	\end{equation}
	such that
	%
	\begin{enumerate}
		\item $i_X f = 0$ where $f \in \mfrak{F}(M)$,
		\item $(i_X \omega)(X_1,\ldots,X_{r-1}) = 
			\omega(X,X_1,\ldots,X_{r-1})$ for $\omega \in 
			\Omega^r(M)$.
	\end{enumerate}
\end{definition}

In local coordinates, $(i_X\omega)_{[i_2\cdots i_r]} = 
X^i\omega_{[ii_2\cdots i_r]}$ such that
\begin{equation}
	i_X\omega = \frac{1}{(r-1)!} X^i\omega_{ii_2\cdots i_r} 
	dx^{i_2} \wedge \ldots dx^{i_r}
\end{equation}
%
This can be rewritten in a form that is useful for future 
reference. Namely, let $i_X\omega = \tfrac{1}{(r-1)!}  
X^{i_1}\omega_{i_1\cdots i_r} d\hat{x}^{i_1}\wedge 
dx^{i_2}\wedge\ldots dx^{i_r} = \tfrac{(-1)^{s-1}}{(r-1)!}  
X^{i_s}\omega_{i_1\cdots i_s\cdots i_r} dx^{i_1}\wedge \ldots 
d\hat{x}^{i_s} \wedge \ldots dx^{i_r}$, where a hat on top 
denotes that differential underneath is omitted and where we used 
the antisymmetry of $\omega$. It then follows that
%
\begin{equation}
	i_X\omega = \frac{1}{r!}\sum_s (-1)^{s-1} X^{i_s} 
	\omega_{i_1\cdots i_s \cdots i_r} dx^{i_1} \wedge \ldots 
	d\hat{x}^{i_s} \wedge \ldots dx^{i_r}~.
\end{equation}

%%%SECTION%%%
\subsection{1-parameter groups of differentiable manifolds}
%%%%%%%%%%%%%

Let $X$ be a vector field on a manifold $M$. An \emph{integral 
	curve} of $X$ is a curve $x(t)$ on $M$ such that for each $t$, 
$X_{x(t)}$ is tangent to the curve. This means that for each $t$ 
one has
%
\begin{equation}
	\frac{df(x(t))}{dt} = X_{x(t)}f \quad \mathrm{for\spc all}\spc 
	f \in \mfrak{F}(M)
\end{equation}
%
which can be rephrased in a local coordinate basis as
%
\begin{equation}\label{eq:IntCur-ode}
	\frac{dx^i(t)}{dt} \frac{\pd f}{\pd x^i} = X^i(x(t)) \frac{\pd 
		f}{\pd x^i}
\end{equation}
%
where we used the same notation for the coordinate system and the 
coordinates of the curve.\footnote{To make the distinction 
	somewhat clearer, the integral curve coordinates keep an 
	explicit dependence on the parameter in the notation.}

Solving the system of ODEs \eqref{eq:IntCur-ode} for given 
initial conditions $x_0^i = x^i(0)$ gives one a unique integral 
curve on the manifold.
%
\begin{example}
	Consider the vector field in $\mbb{R}^2$
  %
	\begin{equation}
		X = x\frac{\pd}{\pd y} - y \frac{\pd}{\pd x} 
		\quad\mathrm{i.e.}\spc X^i = (-y,x).
	\end{equation}
  %
	The components of \eqref{eq:IntCur-ode} then give two ODEs for 
	the integral curve $(x(t),y(t))$,
  %
	\begin{displaymath}
		\frac{dx(t)}{dt} = -y(t)\spc,\quad \frac{dy(t)}{dt} = x(t)
	\end{displaymath}
  %
	and we denote initial values as $x_0,y_0$.
	This system has the unique solution
  %
	\begin{equation}
		\begin{split}
			x(t) &= x_0 \cos t - y_0 \sin t \\
			y(t) &= y_0 \cos t - x_0 \sin t
		\end{split}
	\end{equation}
  %
	which is a circle centered around the origin with radius $r = 
	[(x_0)^2 + (y_0)^2 ]^{1/2}$. Hence, the infinite set of these 
	circles are the integral curves of the considered vector field.
\end{example}
%

\begin{definition}[1-parameter group of transformations]
	A 1-parameter group of transformations of $M$ is a mapping
	\begin{displaymath}
		\mbb{R} \times M \rightarrow M : (t,p) \mapsto \phi_t(p)
	\end{displaymath}
	which satisfies the conditions
	\begin{itemize}
		\item[(a)] for each $t \in \mbb{R}$, $\phi_t$ is a 
			transformation of $M$
		\item[(b)] for each $t,s \in \mbb{R}$ and $p \in M$, 
			$\phi_{t+s} = \phi_t(\phi_s)$
	\end{itemize}
\end{definition}
This abelian group of transformations naturally induces a vector 
field $X$, i.e.\ the vector field for which $\phi_t(p)$ is the 
integral curve for some point $p = \phi_0(p)$.

One can also consider a local 1-parameter group of local 
diffeomorphisms, where $t$ is in an open interval 
$(-\veps,\veps)$ and $p$ lies in an open set of $M$.
It can be proven that if a vector field $X$ is given, there is 
for each point $p$ a local 1-parameter group of local 
diffeomorphisms $\phi_t$ which induces $X$. See 
\cite{kob1996found}, pg.\ 13. The vector field $X$ then is said 
to be the generator of $\phi_t$ in the neighborhood of $p$. In 
the remainder, it will be assumed that this correspondence holds 
for ``big enough'' regions.

This induced vector field will satisfy the ODE
%
\begin{equation}\label{eq:1parODE}
	\frac{d\phi_t^i(x)}{dt} = X^i(\phi_t(x))
\end{equation}
%
and the \emph{flow}---the integral curve---asociated with the 
group $\phi_t(x)$ is generated by this vector field in the 
following manner.
%
\begin{align*}
	\phi_t^i(x) &= \sum_{n}^\infty \frac{t^n}{n!} \left( 
		\frac{d}{dt} \right)^n \phi_t^i(x)|_{t=0} \\
	&= \left[ 1 + \sum_{n=1}^\infty \frac{t^n}{n!} \left( 
			\frac{d}{dt} \right)^n\right]\phi_t^i(x)|_{t=0} \\
	&= \exp\left(t\frac{d}{dt}\right) \phi_t^i(x)|_{t=0}
\end{align*}
%
This is of course the realization on the manifold of a finite 
translation in parameter space. We can rewrite it formally as
%
\begin{equation}
	\phi_t^i(x) = \exp\left(t X \right) x^i
\end{equation}
%

Let $\phi$ be a transformation of a manifold. If a vector field 
$X$ generates the 1-parameter group $\phi_t$, then the vector 
field $\phi_\ast X$ generates the 1-parameter group $\phi\circ 
\phi_t\circ \phi^{-1}$ (see \cite{kob1996found}, pg.\ 14). From 
this it follows that $X$ is an ivariant vector field under $\phi$ 
if and only if $[\phi, \phi_t] = 0$, because $X$ and $\phi_\ast 
X$ then satisfy the same ordinary differential equation 
\eqref{eq:1parODE} and the uniqueness of its solution.

\subsection{Lie derivative}

Let $X$ and $Y$ be vector fields on a manifold M and let 
$\varphi_t$ be a 1-parameter group of transformations of $M$, 
generated by $X$.
%
\begin{definition}[Lie derivative]
	The Lie derivative of any tensor $t$ with respect to $X$ is 
	given by
   %
	\begin{equation}
		\mcal{L}_X t \equiv \lim_{\tau\to 0} \tfrac{1}{\tau} [t - 
		\varphi_{\tau\ast} t]~,
	\end{equation}
   %
	or more precisely
   %
	\begin{displaymath}
		\mcal{L}_X t = \lim_{\tau\to 0} \tfrac{1}{\tau} [t_p - 
		(\varphi_{\tau\ast} t)_p]~, \quad p \in M\spc.
	\end{displaymath}
\end{definition}

Note that for a differential form $\omega$ the Lie derivative is 
also given by
%
\begin{equation}
	\mcal{L}_X \omega = \lim_{\tau\to 0} \tfrac{1}{\tau} [\omega - 
	\varphi^\ast_{-\tau}\omega] = \lim_{\tau\to 0} \tfrac{1}{\tau} 
	[\varphi^\ast_\tau \omega - \omega]
\end{equation}

It is clear that the Lie derivative acts linearly on the algebra 
of tensor fields. Let us then show that it is also a Leibnizian 
operator, that is
%
\begin{equation}
	\mcal{L}_X(t_1 \otimes t_2) = \mcal{L}_X t_1 \otimes t_2 + t_1 
	\otimes \mcal{L}_X t_2~.
\end{equation}
Indeed, one finds
%
\begin{align*}
	\mcal{L}_X(t_1 \otimes t_2)
	&= \lim_{\tau \to 0} \tfrac{1}{\tau} [t_1 \otimes t_2 - 
	\varphi_{\tau\ast}(t_1 \otimes t_2)] \\
	&= \lim_{\tau \to 0} \tfrac{1}{\tau} [t_1 \otimes t_2 - 
	\varphi_{\tau\ast}t_1 \otimes \varphi_{\tau\ast}t_2 \\
	&= \lim_{\tau \to 0} \tfrac{1}{\tau} [t_1 - 
	\varphi_{\tau\ast}t_1] \otimes t_2 + \lim_{\tau \to 0} 
	\tfrac{1}{\tau} \varphi_{\tau\ast} t_1 \otimes [t_2 - 
	\varphi_{\tau\ast} t_2] \\
	&= \mcal{L}_X t_1 \otimes t_2 + t_1 \otimes \mcal{L}_X t_2~.
\end{align*}
%%%SECTION%%%
\section{Lie groups and Lie algebras}
%%%%%%%%%%%%%

%%%SUBSECTION%%%
\subsection{Left invariant vector fields}
%%%%%%%%%%%%%%%%

\begin{definition}[Lie group]
	A Lie group $G$ is a group which is at the same time a 
	differentiable manifold such the group composition and the 
	inverse operation are differentiable mappings. The dimension of 
	the group is the dimension of the manifold.
\end{definition}
%

Let $a$ and $g$ then be elements of a Lie group $G$. Denote the 
\emph{left translation} $L_a$ of $G$ by $a$ as the automorphism
%
\begin{equation}
	L_a : G \rightarrow G : g \mapsto ag\spc, \quad g \in G\spc.
\end{equation}
%
The associated differential map is an invertible transformation 
of vector fields on $G$, i.e.\ $L_a{}_\ast : T_g G \rightarrow 
T_{ag}G$.

This inherent automorphism acting on Lie group manifolds is a 
very interesting feature of them. Given a Lie group this 
automorphism singles out a particular class of vector fields on 
$G$, namely those ones which are invariant under left 
translations. We call them \emph{left invariant vector fields} 
and they satisfy
%
\begin{equation}\label{eq:LIvector}
	L_a{}_\ast X_g = X_{ag}\spc.
\end{equation}
%
In a local coordinate system this is rephrased as
%
\begin{equation}\label{eq:LIvectorLC}
	(L_a{}_\ast X)^i\pd_i|_{ag} f = X^i(g) \frac{\pd x^j(ag)}{\pd 
		x^i(g)}\pd_j|_{ag} f \overset{LI}{=} X^i(ag)\pd_i|_{ag} f
\end{equation}

Denote the set of left invariant vector fields by $\mfrak{g}$.  
They form a subset of the infinite dimensional Lie algebra 
$\mfrak{X}(G)$ with the usual addition and scalar multiplication.  
It inherits the bracket operation of $\mfrak{X}(G)$, under which 
$\mfrak{g}$ is closed as is implied by the following.
%
\begin{equation}
	L_a{}_\ast [X,Y]_g = [L_a{}_\ast X_g,L_a{}_\ast Y_g]
	= [X_{ag},Y_{ag}]
		= [X,Y]_{ag}
\end{equation}
%
This shows that the Lie bracket of two left invariant vector 
fields is a left invariant vector field. Hence, $\mfrak{g}$ is 
closed and forms a subalgebra of $\mfrak{X}(G)$.

As a vector space $\mfrak{g}$ is isomorphic to $T_e G$. Denote a 
generic element of $T_e G$ by $X_e$. These elements can be 
obtained by mapping a left invariant vector field $X$ to its 
value at $e$. The other direction of the isomorphism then is 
defined by the transformation $X_e \mapsto X$, as in 
\eqref{eq:LIvector}. Hence, the dimension of $\mfrak{g}$  is $n = 
\dim G$ and one refers to it as \emph{the Lie algebra of $G$}.

\begin{example}
	Consider the general linear group $Gl(n,\mbb{R})$, the group of 
	all real non-singular $(n \times n)$-matrices. They form an 
	open submanifold of $\mbb{R}^{n\cdot n}$ and a generic element 
	$g$ will be specified by the coordinates $x^i_{\spc j}(g)$. The 
	unit element $e$ has coordinates $\delta^i_j$. The left 
	translation is then given in the following terms
  %
	\begin{equation}
		(L_a g)^i_{\spc j} = x^i_{\spc j}(ag) = x^i_{\spc k}(a) 
		x^k_{\spc j}(g)
	\end{equation}

	An element in the tangent space at the identity is a vector $V 
	\in T_e G$
  %
	\begin{equation}
		V = V^i_{\spc j} \left.\frac{\pd}{\pd x^i_{\spc j}}\right|_e 
		\spc.
	\end{equation}
  %
	We use \eqref{eq:LIvector} and \eqref{eq:LIvectorLC} to 
	construct a left invariant vector field $X$ by left translating 
	$V$ over $Gl(n,\mbb{R})$,
  %
	\begin{displaymath}
		\begin{split}
			X_g = L_g{}_\ast V &= (L_g{}_\ast V)^i_{\spc j} 
			\pd_{x^i_{\spc j}}|_g \\
			&= V^i_{\spc j} \frac{\pd x^k_{\spc l}(ge)}{\pd x^i_{\spc 
					j}(e)} \pd_{x^k_{\spc l}}|_g \\
			&= V^i_{\spc j} x^k_{\spc m}(g) \frac{\pd x^m_{\spc 
					l}(e)}{\pd x^i_{\spc j}(e)} \pd_{x^k_{\spc l}}|_g \\
			&= V^i_{\spc j} x^k_{\spc m}(g) \delta^m_i \delta^l_j 
			\pd_{x^k_{\spc l}}|_g \\
			&= V^i_{\spc j} x^k_{\spc i}(g) \pd_{x^k_{\spc j}}|_g
			= (gV)^i_{\spc j} \pd_{x^i_{\spc j}}|_g
		\end{split}
	\end{displaymath}   
  %
	The vector field $X_g = (gV)^i_{\spc j} \pd_{x^i_{\spc j}}|_g$ 
	is said to be generated by $V \equiv X_e \in T_e 
	G$.\footnote{The notation $(gV)^i_{\spc j}$ is to be understood 
		as the matrix product of the coordinates of $g$ with the 
		components of $V$, i.e.\ $x^i_{\spc k}(g)V^k_{\spc j}$.}
	It is clearly left invariant since $L_a{}_\ast X_g = 
	(agV)^i_{\spc j} \pd_{x^i_{\spc j}}|_{ag} = X_{ag}$.  These 
	vector fields form the Lie algebra of $Gl(n,\mbb{R})$.  
	However, we will show that not only there is a Lie algebra 
	isomorphism between these vector fields and their vectors at 
	the identity, but that the \emph{components} of the vectors at 
	the identity also are isomorphic with $\mfrak{g}$.  Therefore, 
	we first compute the commutator of two left invarant vector 
	fields $X = (gV)^i_{\spc j} \pd_{x^i_{\spc j}}|_g$ and $Y = 
	(gW)^i_{\spc j} \pd_{x^i_{\spc j}}|_g$.
  %
	\begin{displaymath}
		\begin{split}
			[X,Y]|_g &= [x^i_{\spc j}(g)V^j_{\spc k} \pd_{x^i_{\spc 
					k}}|_g, x^a_{\spc b}(g)W^b_{\spc d} \pd_{x^a_{\spc 
					d}}|_g] \\
	&= x^i_{\spc j}(g)V^j_{\spc k} \delta^a_i \delta^k_b W^b_{\spc 
		d} \pd_{x^a_{\spc d}}|_g - [V \leftrightarrow W] \\
	&= x^a_{\spc b}(g) (V^b_{\spc j} W^j_{\spc d} - W^b_{\spc j} 
	V^j_{\spc d} \pd_{x^a_{\spc d}}|_g \\
	&= x^a_{\spc b}(g) [V,W]^b_{\spc d} \pd_{x^a_{\spc d}}|_g \\
	&= (g[V,W])^i_{\spc j} \pd_{x^i_{\spc j}}|_g
			\end{split}
	\end{displaymath}
  %
	One concludes that the commutator of two left invariant vector 
	fields is given by left translating the vector at the identity 
	with components $V^i_{\spc k}W^k_{\spc j} - W^i_{\spc 
		k}V^k_{\spc j}$.
	This is a very interesting result, as it means that the Lie 
	algebra structure can be understood completely by considering 
	the components only of the left invariant vector fields at the 
	identity. These components are \emph{arbitrary} $(n \times 
	n)$-matrices. The Lie algebra of $Gl(n\mbb{R})$ is then to be 
	understood as these matrices together with the standard matrix 
	commutator $[\cdot,\cdot]$ and is denoted by 
	$\mfrak{gl}(n,\mbb{R})$.
  
	We repeat that this is only true for matrix Lie groups. For 
	generic Lie groups, the components of the left invariant vector 
	fields at the identity do not have a structure to preserve the 
	Lie algebra commutator and one needs the \emph{vector} (field), 
	not only its components.

	In the remainder, we will both use $V$ and $V^i_{\spc j}$ to 
	denote an element of $\mfrak{gl}(n,\mbb{R})$, although in the 
	above introduced notation, $V$ is a vector of $T_e 
	Gl(n,\mbb{R})$ whereas only its components are considered an 
	element of the matrix Lie algebra $\mfrak{gl}(n,\mbb{R})$.
\end{example}



%%%SUBSECTION%%%
\subsection{Frames and structure equations}
%%%%%%%%%%%%%%%%

Let $E_a (a = 1\ldots n)$ be a set of linearly independent left 
invariant vector fields on $G$. They form a basis for $\mfrak{g}$ 
and at each point $g$ they are a basis for the tangent space $T_g 
G$, hence a frame.
    
Since $\mfrak{g}$ is closed under the Lie bracket operation, we 
can write \eqref{eq:GenStrucEqs} with constant structure 
functions
%
\begin{equation}\label{eq:LAStructEqs}
	[E_a,E_b] = c_{ab}^{\spc\spc c}E_c
\end{equation}
%
The $c$ are then referred to as \emph{the structure constants of 
	the Lie algebra}.

Let us introduce a dual basis $\theta^a$ to the basis of 
$\mfrak{g}$, i.e.\
%
\begin{equation}
	\theta^a (E_b) \equiv \delta^a_b\spc,\quad\mathrm{for\spc 
		all}\spc g\in G
\end{equation}
%
which are a basis for the set of left invariant 1-forms 
$\mfrak{g}^\ast$.  These are indeed left-invariant since
%
\begin{displaymath}
	L^\ast_g \theta^a(E_b) = \theta^a(L_g{}_\ast E_b) = 
	\theta^a(E_b)\spc
	\quad\mathrm{for\spc all}\spc E_b \in \mfrak{g}\spc.
\end{displaymath}
%
We now derive the \emph{Maurer-Cartan's structure equations}, see 
also \cite{kob1996found}, pg.\ 36.\footnote{We use another 
	convention compared with \cite{kob1996found}, where an overal 
	factor $1/2$ is present. The result, i.e.\ the Maurer-Cartan 
	equations are interestingly the same, due to another---or maybe 
	consistent?---difference in conventions: the definition of the 
	wedge product in the exterior algebra. To be complete, we 
	adhere to $\theta^d \wedge \theta^e(E_b,E_c) = 
	\theta^d(E_b)\theta^e(E_c) - \theta^d(E_d)\theta^e(E_c)$.}
%
\begin{align*}
	d\theta^a(E_b,E_c) &= \left[E_b(\theta^a(E_c)) - 
		E_c(\theta^a(E_b)) - \theta^a([E_b,E_c])
		 \right] \\
	 &= \theta^a([E_b,E_c]) \\
	 &= -c_{bc}^{\spc\spc f} \theta^a(E_f) \\
	 &= -\tfrac{1}{2} c_{de}^{\spc\spc a} 2\delta^d_{[b} 
	 \delta^e_{c]} \\
	 &= -\tfrac{1}{2} c_{de}^{\spc\spc a} \theta^d \wedge \theta^e
	(E_b,E_c)
\end{align*}
%
In the second to last step we used the fact that the structure 
constants are antisymmetric in the lower indices.
Since the found result is true for arbitrary $E_a$ we obtain the 
Maurer-Cartan's structure equations, the dual expression of 
\eqref{eq:LAStructEqs}, i.e.\
%
\begin{equation}\label{eq:MCeqs}
	d\theta^a = -\tfrac{1}{2} c_{bc}^{\spc\spc a} \theta^b \wedge 
	\theta^c
\end{equation}
%

The \emph{canonical 1-form} $\omega$ is a left invariant 
Lie-algebra valued 1-form. This means it maps elements of 
$\mfrak{g}$ into its vectors in the tangent space at $e$,
%
\begin{equation}
	\omega(X) = L_{g^{-1}}{}_\ast X
\end{equation}
%
Hence, it can be expanded as $\omega_g = E_{e,a}\cdot\theta^a_g$, 
where $\theta^a_g$ is the dual basis of $T^\ast_g G$. Indeed, let 
$X = X^a E_a$ be an element of $\mfrak{g}$. It follows that
%
\begin{displaymath}
	\omega_g(X_g) = \omega_g(X^a E_{g,a}) = X^a E_{e,b} \cdot 
	\theta^b_{g}(E_{g,a}) = E_{e,b}X^a \delta^b_a = X^a E_{e,a}
\end{displaymath}
%
By noting that $\omega \wedge \omega \equiv \tfrac{1}{2} 
[E_{e,a}, E_{e,b}] \cdot \theta^a \wedge \theta^b$, it directly 
follows from \eqref{eq:MCeqs} that
%
\begin{equation}\label{eq:MCeqsCanForm}
	d\omega + \omega \wedge \omega = 0
\end{equation}
%

\begin{example}
	Let us take a look what this means for the general linear group 
	and algebra. The canonical 1-form is to be a map $T_g 
	Gl(n,\mbb{R}) \rightarrow \mfrak{gl}(n,\mbb{R})$, sending a 
	left invariant vector field into the Lie algebra 
	$\mfrak{gl}(n,\mbb{R})$. Remember that a left invariant vector 
	field on $Gl(n,\mbb{R})$ is given by $X_g = (gV)^i_{\spc j} 
	\pd_{x^i_{\spc j}}|_g$ and that the matrix of components 
	$V^i_{\spc j}$ is the corresponding element of 
	$\mfrak{gl}(n,\mbb{R})$. We show that
  %
	\begin{equation}
		\omega^i_{\spc j} \equiv x^i_{\spc k}(g^{-1}) dx^k_{\spc 
			j}(g)
	\end{equation}
  %
	is the (matrix valued) 1-form sought after and that it 
	satisfies \eqref{eq:MCeqsCanForm}.
  %
	\begin{displaymath}
		\begin{split}
			\omega^i_{\spc j}(X_g) &= x^i_{\spc k}(g^{-1}) dx^k_{\spc 
				j}(g)
	\left[ (gV)^a_{\spc b} \pd_{x^a_{\spc b}}|_g \right] \\
	&= x^i_{\spc k}(g^{-1})x^a_{\spc c}(g) V^c_{\spc b} \delta^k_a 
	\delta^b_j \\
	&= x^i_{\spc a}(g^{-1})x^a_{\spc c}(g) V^c_{\spc j} \\
	&= \delta^i_c V^c_{\spc j} = V^i_{\spc j}
		\end{split}
	\end{displaymath}
  %
	Furthermore does it satisfy \eqref{eq:MCeqsCanForm}. Since
  %
	\begin{displaymath}
		\begin{split}
			d\omega^i_{\spc j} &= dx^i_{\spc k}(g^{-1}) dx^k_{\spc     
				j}(g) \\
			&= - x^i_{\spc m}(g^{-1}) dx^m_{\spc l}(g)
			x^l_{\spc k}(g^{-1}) dx^k_{\spc j}(g)
		\end{split}
	\end{displaymath}
  %
	and
  %
	\begin{displaymath}
		\begin{split}
			(\omega \wedge \omega)^i_{\spc j} &=
			\tfrac{1}{2} 2x^i_{\spc [m}(g^{-1}) x^l_{\spc k]}(g^{-1})
	dx^m_{\spc l}(g) \wedge dx^k_{\spc j}(g) \\
			&= x^i_{\spc m}(g^{-1}) x^l_{\spc k}(g^{-1})
	dx^m_{\spc l}(g)dx^k_{\spc j}(g)
		\end{split}
	\end{displaymath}
  %
	It then follows that
  %
	\begin{equation}
		d\omega^i_{\spc j} + (\omega \wedge \omega)^i_{\spc j} = 0
	\end{equation}
  
	The matrix valued 1-form $\omega^i_{\spc j}$ is often denoted 
	by the shorthand ``$g^{-1}dg$'', which as a canonical 1-form 
	does only make sense for matrix groups. The definition of 
	$\omega$ for matrix groups maps a left invariant vector field 
	into the \emph{components} of the corresponding vector in $T_e 
	Gl(n,\mbb{R})$.  For a generic Lie group $\omega$ maps the left 
	invariant vector field into the vector in the tangent space at 
	the identity.
\end{example}

%%%SUBSECTION%%%

\subsection{1-parameter subgroups}
%%%%%%%%%%%%%%%%

Let $\phi_t$ be a 1-parameter group of transformations of $G$, 
generated by a left invariant vector field $X \in \mfrak{g}$. 
%
\begin{equation}
	\frac{d\phi_t^i(g)}{dt} = X^i(\phi_t(g))
\end{equation}
%
Since $X$ is left invariant we have $[L_a , \phi_t] = 0$ for any 
$a \in G$.

Set $a_t \equiv \phi_t(e): \mbb{R} \rightarrow {G}$. We have that 
$a_0 = e$ and
%
\begin{displaymath}
	\begin{gathered}
		a_{t+s} = \phi_{t+s}(e) = \phi_{s+t}(e) = \phi_s \circ 
		\phi_t(e)
		= \phi_s \circ L_{a_t}(e) = L_{a_t} \circ \phi_s(e) = 
		L_{a_t}(a_s)
		= a_t a_s \spc,\\
		a_t^{-1} = \phi_t^{-1}(e) = \phi_{-t}(e) = a_{-t}
	\end{gathered}
\end{displaymath}
%
The curve $a_t$ is a \emph{1-parameter subgroup of G generated by 
	X}. The action of $\phi_t$ on $G$ is given by the right action 
of $a_t$ since
%
\begin{displaymath}
	\phi_t(g) = \phi_t \circ L_g(e) = L_g \circ \phi_t(e) = L_g 
	\circ a_t = ga_t
\end{displaymath}
%
such that $\phi_t(g) = R_{a_t}(g)$. From this it follows that the 
Lie derivative $L_X Y = [X,Y]$ on $G$ is given by
%
\begin{equation}
	L_X Y = \lim_{t\rightarrow 0} [Y - (\phi_t)_\ast Y]
	=\lim_{t\rightarrow 0} [Y - R_{a_t}{}_\ast Y]
\end{equation}
%

To conclude this section the exponential map is defined. Set 
$\exp X \equiv a_1$. From this it follows that $a_t = \exp tX$.  
Indeed, consider first the 1-parameter subgroup $a_{st}$ at the 
identity,
%
\begin{equation}
	\left.\frac{d a^i_{st}}{dt}\right|_{t=0} = s\left.\frac{d 
		a^i_u}{du}\right|_{u=0} = s X^i_e
\end{equation}
%
which implies that $a_{st}$ is generated by $sX$. But we also 
have that there exists a one-parameter group which is generated 
by $sX$,
%
\begin{equation}
	\frac{db^i_t}{dt} = sX^i_{b_t}
\end{equation}
%
By uniqueness of solutions this means that $b_t = a_{st}$. Hence, 
$\exp sX \equiv b_1 = a_s$. The desired result is found by 
replacing $s$ by $t$.

%%%SUBSECTION%%%

\subsection{The adjoint action}
%%%%%%%%%%%%%%%%

Every automorphism $\phi$ of a Lie group $G$ induces an 
automorphism $\phi_\ast$ on its Lie algebra $\mfrak{g}$. This can 
be seen as follows.  Let $X$ be an element of $\mfrak{g}$.
%
\begin{displaymath}
	L_a{}_\ast (\phi_\ast X)(f) = L_a{}_\ast X(f\circ \phi) \\
		= X(f\circ \phi) = (\phi_\ast X) (f)
\end{displaymath}
%

Consider the mapping $\mrm{adj}$ which relates to each element 
$a$ of a Lie group an automorphism,
%
\begin{equation}
	\mrm{adj} : G \rightarrow \mrm{Aut}(G) : a \mapsto
	\mrm{adj}_a \spc.
\end{equation}
%
The automorphism $\mrm{adj}_a$ introduced is called the 
\emph{inner automorphism} of $G$ and for a given element $a \in 
G$ it is defined as
%
\begin{equation}
	\mrm{adj}_a : G \rightarrow G : g \mapsto a g a^{-1}
\end{equation}
%
It is obvious that $\mrm{adj}_a = L_a \circ R_{a^{-1}} = 
R_{a^{-1}} \circ L_a$. Furthermore, $\mrm{adj}_a \circ 
\mrm{adj}_b = \mrm{adj}_{ab}$, hence $\mrm{adj}$ is a group 
homomorphism.

The automorphism $\mrm{adj}_a$ induces an automorphism 
$\mrm{adj}_a{}_\ast$ on the Lie algebra $\mfrak{g}$. Define the 
mapping $\mrm{Ad}_a$ then as the differential map 
$\mrm{adj}_a{}_\ast$ at the identity $e$ of $G$, i.e.\
%
\begin{equation}
	\mrm{Ad}_a : \mfrak{g} \rightarrow \mfrak{g} : X \mapsto 
	\mrm{Ad}_a X
\end{equation}
%
What does $\mrm{Ad}_a X$ looks like? For an element $X$ of the 
Lie algebra we have that
%
\begin{displaymath}
	\mrm{Ad}_a X = (R_{a^{-1}} \circ L_a)_\ast X = 
	R_{a^{-1}}{}_\ast X
\end{displaymath}
%
since $X$ is left invariant. Let us give a concrete example for 
matrix groups.
%
%
\begin{example}
	Let $X$ be a left invariant vector field on $Gl(n,\mbb{R})$, 
	generated by the $V^i_{\spc j}$. The adjoint action is given
  %
	\begin{equation}
		\mrm{Ad}_a X_g = (agVa^{-1})^i_{\spc j} \pd_{x^i_{\spc j}}|_g 
		\quad(= R_{a^{-1}}{}_\ast X_{ag})
	\end{equation}
  %
	One way to find the corresponding generator in 
	$\mfrak{gl}(n,\mbb{R})$ is to apply the Maurer-Cartan form to 
	$\mrm{Ad}_a X$, i.e.\
  %
	\begin{displaymath}
		\begin{split}
			\omega(\mrm{Ad}_a X)^i_{\spc j} &=
	\omega((agVa^{-1})^i_{\spc j} \pd_{x^i_{\spc j}}|_g)^i_{\spc j} 
	\\
	&= x^i_{\spc k}([aga^{-1}]^{-1}) dx^k_{\spc j}(aga^{-1}) \left[ 
		x^a_{\spc b}(ag)V^b_{\spc c} x^c_{\spc d}(a^{-1})
		\pd_{x^a_{\spc d}}|_{aga^{-1}} \right] \\
	&= x^i_{\spc l}(a) x^l_{\spc k}([ag]^{-1}) \delta^k_a
		x^a_{\spc b}(ag) V^b_{\spc c} x^c_{\spc d}(a^{-1}) \delta^d_j 
		\\
	&= x^i_{\spc l}(a) \delta^l_b V^b_{\spc c} x^c_{\spc
		d}(a^{-1}) \\
	&= x^i_{\spc k}(a) V^k_{\spc l} x^l_{\spc j}(a^{-1})
		\end{split}
	\end{displaymath}
  %
	This implies that the adjoint representation of $Gl(n,\mbb{R})$ 
	on $\mfrak{gl}(n,\mbb{R})$ is given by
  %
	\begin{equation}\label{eq:AdjRepMatrixLA}
		\mrm{Ad}_a V = aVa^{-1}
	\end{equation}

	Note that although the adjoint representation on left invariant 
	vector fields is given by $R_{a^{-1}}{}_\ast$ only, the 
	corresponding action on the components at the identity is given 
	by \eqref{eq:AdjRepMatrixLA}.
\end{example}
Note that $\mrm{Ad}$ is a mapping of $G$ into the automorphisms 
of its Lie algebra,
%
\begin{equation}
	\mrm{Ad} : G \rightarrow \mrm{Aut}(\mfrak{g}) : a \mapsto 
	\mrm{Ad}_a
\end{equation}
%
Since $\mrm{adj}$ is a group homomorphism, $\mrm{Ad}$ is a 
representation on $\mfrak{g}$, called the \emph{adjoint 
	representation} of $G$.
%

Last but not least let us consider the differential map 
$\mrm{Ad}_\ast$ at the identity and define this as $\mrm{ad}$,
%
\begin{equation}
	\mrm{ad} : \mfrak{g} \rightarrow \mrm{Der}(\mfrak{g}) : X 
	\mapsto \mrm{ad}_X
\end{equation}
%
We can now calculate how this operator works on an element $Y$ of 
$\mfrak{g}$. Let $a_t = \exp tX$ be the curve in $G$ generated by 
$X$. By definition we have
%
\begin{displaymath}
	\begin{split}
		\mrm{ad}_X Y &= (\mrm{Ad}_{\ast e})_X Y \\
		&= \lim_{t \rightarrow 0}\frac{1}{t} (\mrm{Ad}_{a_t} Y - 
		\mrm{Ad}_e Y) \\
		&= \lim_{t \rightarrow 0}\frac{1}{t} (R_{a_{t}^{-1}}{}_\ast Y 
		- Y) \\
		&= L_X Y = [X,Y]
	\end{split}
\end{displaymath}


%%%SUBSECTION

\subsection{Action of Lie groups on manifolds}
%%%%%%%%%%%%%

\begin{definition}[Lie transformation group]
\end{definition}
%
Let $G$ act on $M$ on the right and consider a left invariant 
vector field $X \in \mfrak{g}$. The \emph{fundamental vector 
	field} $X^\star$ on $M$ corresponding to $X$ is the vector 
field induced by the right action of the 1-parameter subgroup 
$a_t = \exp(tX)$ on $M$. In other words, $X^\star$ is the 
generator of $R_{a_t}$ on $M$ while $X$ is the generator of $a_t$ 
on $G$.  Hence, we have defined a mapping
%
\begin{equation}
	\rho : \mfrak{g} \rightarrow \mfrak{X}(M) : X \mapsto X^\star
\end{equation}
%

Let us define the mapping $\rho$ more carefully as follows.  
Consider first a mapping $\rho_p$ for any $p \in M$ such that
%
\begin{equation}
		\rho_p : G \rightarrow M : g \mapsto pg
\end{equation}
%
The differential map $\rho_p{}_\ast : T_g G \rightarrow T_{pg} M$ 
is at the identity given by the following. Remember that $X \in 
\mfrak{g}$ is the generator of $a_t$.
%
\begin{displaymath}
	\begin{split}
		(\rho_p{}_\ast X_e)_p f &= X_e (f \circ \rho_p) \\
		&= \frac{d}{dt}(f \circ \rho_p(a_t))|_{t=0} \\
		&= \frac{d}{dt}(f(R_{a_t}(p))|_{t=0} \\
		&= A_p^\star f
	\end{split}
\end{displaymath}
%
An element of the Lie algebra of $G$ is mapped onto the 
corresponding fundamental vector field at the point $p$ on $M$.  
The desired result is then found in defining $\rho$ implicitly by 
$(\rho X)_p \equiv \rho_p{}_\ast X = X_p^\star$ for all $p \in 
M$.

The mapping $\rho$ is a Lie algebra homomorphism, i.e.\ 
$\rho([X,Y]) = [\rho(X),\rho(Y)]$. Let us proof this proposition.
%
\begin{align*}
	[A^\star_p,B^\star_p] &= \lim_{t\rightarrow 0} \frac{1}{t} 
	\left[B_p^\star - (R_{a_t\ast} B^\star)_p \right] \\
		&= \lim_{t\rightarrow 0}\frac{1}{t} \left[(\rho B)_p - 
			R_{a_t\ast} (\rho B)_{pa_t^{-1}} \right] \\
		&= \lim_{t\rightarrow 0}\frac{1}{t} \left[\rho_{p\ast} B - 
			R_{a_t\ast}\circ \rho_{pa_t^{-1}\ast}B \right]
\end{align*}
%
To continue, note that $R_{a_t} \circ \rho_{pa_t^{-1}}(c) = 
R_{a_t}(p a_t^{-1}c) = p\spc \mrm{adj}_{a_t^{-1}}(c) = \rho_p 
\circ \mrm{adj}_{a_t^{-1}} (c)$.  Hence, we find
%
\begin{align*}
	[A^\star_p,B^\star_p] &= \lim_{t\rightarrow 0}\frac{1}{t} 
	\left[\rho_{p\ast}B - (\rho_p \circ \mrm{adj}_{a_t^{-1}})_\ast 
		B \right]\\
		&= \lim_{t\rightarrow 0}\frac{1}{t} \left[ \rho_{p\ast}B - 
			\rho_{p\ast} R_{a_t\ast} B \right] \\
		&= \rho_{p\ast} \lim_{t\rightarrow 0}\frac{1}{t} [B - 
		R_{a_t\ast}B] \\
		&= \rho_{p\ast}[A,B] = [A,B]^\star_p
\end{align*}
%
Hence, $\rho$ defines a Lie algebra homomorphism of $\mfrak{g}$ 
into $\mfrak{X}(M)$.



%%%SECTION%%%

\section{(Principal) Fibre bundles}
%%%%%%%%%%%%%

\subsection{Fibre bundles}
%%%%%%%%%%%%%

\begin{definition}[Fibre bundle]
	\label{def:fibre_bundle}
	A differentiable fibre bundle $(E,\pi,M,F,G)$ consists of the 
	following elements;
  %
	\begin{itemize}
		\item[(a)] a manifold $E$, the \emph{bundle space}; a 
			manifold $M$, the \emph{base space}; and a manifold $F$, 
			the (typical) \emph{fibre}.
		\item[(b)] a surjection $\pi : E \rightarrow M$ which maps 
			any point $u$ of the bundle space into a point $p = \pi(u)$ 
			on the base space.  Note that information is lost under 
			this \emph{projection}.  Its inverse image $\pi^{-1}(p) = 
			F_p$ is diffeomorphic to $F$ (see local trivialization) and 
			is called the \emph{fibre at $p$}.
		\item[(c)] a Lie group $G$ acting on $F$ on the left. It is 
			called the \emph{structure group} of the bundle.
		\item[(d)] a \emph{local trivialization} $(U_i,\phi_i)$ where 
			$\{U_i\}$ is an open covering of $M$ and $\phi_i$ a 
			diffeomorphism
			\begin{equation}
	\phi_i : U_i \times F \rightarrow \pi^{-1}(U_i)
			\end{equation}
			such that $\pi \circ \phi_i = \mrm{id}_{U_i}$, i.e.\ 
			$\pi\phi_i(p,f) = p$.
		\item[(e)] a set $\{t_{ij}\}$ of \emph{transition functions}, 
			defined as follows. Let $\phi_{i,p}(f) \equiv \phi_i(p,f)$ 
			be the mapping $\phi_{i,p} : F \rightarrow F_p$ which sends 
			an element $f \in F$ into an element of the fibre $F_p 
			\subset E$ at $p \in M$. On an overlapping region $U_i \cap 
			U_j \neq \emptyset$, require that
      %
			\begin{equation}
	t_{ij}(p) \equiv \phi_{i,p}^{-1} \phi_{j,p} : F \rightarrow F :
	f \mapsto t_{ij}(p)f
			\end{equation}
      %
			be an element of the structure group $G$. This implies that 
			the local trivializations $\phi_i$ and $\phi_j$ are related 
			by the map $t_{ij} : U_i \cap U_j \rightarrow G$ in the 
			sense that
      %
			\begin{equation}
	\phi_j(p,f) = \phi_i(p,t_{ij}(p)f)\spc .
			\end{equation}
	\end{itemize}
\end{definition}
%
Note that $\phi_i^{-1}$ maps $\pi^{-1}(U_i)$ into a direct 
product structure, hence a fibre bundle is \emph{locally} 
diffeomorphic to this direct product. For a given $p$ this 
defines a diffeomorphism between $F$ and $F_p$, which rectifies 
why both are being referred to as fibres.  Although a fibre 
bundle is locally a direct product, this is not to be expected 
globally. Indeed, the concept of transition functions allows one 
to construct a rich variety of manifolds. Since overlapping 
charts of $M$ will refer to a same element of the fibre bundle, 
using different elements of $F$ in their direct product 
descriptions, reconstructing a global picture forces one to 
\emph{glue}, i.e.\ identify the respective elements in $F$ 
together by using the transition functions corresponding to the 
same element in the bundle.  This spoils the direct product 
structure on a \emph{global} level, which is exactly where the 
interest of fibre bundles has to be found. Of course, this 
glueing procedure will only make sense if the transition 
functions satisfy the consistency conditions
%
\begin{subequations}
	\begin{align}
		t_{ii}(p) &= e & p \in& U_i \\
		t_{ij}(p) &= t_{ji}^{-1}(p) & p \in& U_i \cap U_j \\
		t_{ij}(p) \circ t_{jk}(p) &= t_{ik}(p) & p \in& U_i \cap U_j 
		\cap U_k \end{align}
\end{subequations}
%

The trivial limit, where $t_{ij} = e$ for all overlapping 
regions, allows one to let the local direct product structure to 
be meaningful globally.  This is nothing else then saying that 
the fibre bundle $E$ is diffeomorphic with the direct product $M 
\times F$.

Let $E$ be a fibre bundle. A \emph{cross section} of $E$ is 
mapping $\sigma$ of the base space into bundle such that $\pi 
\circ \sigma = \mrm{id}_M$, i.e.\
%
\begin{equation}
	\sigma : M \rightarrow E : p \mapsto \sigma(p) \in 
	\pi^{-1}(p)\spc.
\end{equation}
%
Note that it is because $\pi \circ \sigma$ is the identity on $M$ 
that a point gets mapped into a point on the bundle which is an 
element of its fibre $F_p$. A section may not be defined globally 
on the fibre bundle. A \emph{local section} is a section defined 
on a subset $U_i \subset M$ of the base manifold.


\subsection{Principal bundles}
%%%%%%%%%%%%%

A principal bundle $(P,\pi,M,G)$ is a fibre bundle where the 
fibre is identical with the structure group, i.e.\ $F \equiv G$.  
One defines the \emph{right action} of $G$ on the bundle $P$ as 
follows. Consider a point $u \in P$ such that $\pi(u) = p$. A 
local trivialization then gives $\phi_i^{-1}(u) = (p,g_i)$ where 
$g_i \in G$. The right action of $G$ on $\pi^{-1}(U_i)$ is then 
defined by $\phi_i^{-1}(ua) \equiv (p,g_ia)$, or equivalently
%
\begin{equation}
	ua \equiv \phi_i(p,g_i a)
\end{equation}
%
for all $a \in G, u \in F_p$. It is easily shown that the 
definition is in fact independent of the local trivialization 
chosen. For consider a point $p \in U_i \cap U_j$ and let $u$ be 
in the fibre at $p$:
%
\begin{displaymath}
	ua = \phi_j(p,g_j a) = \phi_j(p,t_{ji}(p)g_i a) = \phi_i(p,g_i 
	a)\spc.
\end{displaymath}
The right action is thus defined as a mapping $P \times G 
\rightarrow P : (u,a) \mapsto ua$, without the need of making 
reference to a specific local trivialization.

From its definition it is clear that the right action of $G$ on 
any fibre $F_p \equiv \pi^{-1}(p)$ is transitive, since $G$ acts 
transitively on itself on the right. Furthermore, the action is 
free. Let $ua = u$ for some  $u \in P$ and $a \in G$. We then 
have that $ua = \phi_i(p,g_i a) = \phi_i(p,g_i)a = u = 
\phi_i(p,g_i)$, such that $\phi_i(p,g_i a) = \phi_i(p,g_i)$.  
Since $\phi_i$ is a bijection, $g_i a = g_i$ and it follows that 
$a = e$.

\blankline
Given a local section $\sigma_i(p)$ over $U_i$, a \emph{canonical 
	local trivialization} $\phi_i : U_i \times G \rightarrow 
\pi^{-1}(U_i)$ is defined as follows.  Let $u$ be a point in the 
fibre at $p \in U_i$.  Because $G$ acts transitively on 
$\pi^{-1}(p)$, there is a unique $g_u \in G$ such that $u = 
\sigma_i(p) g_u$. Define $\phi_i$ such that $\phi_i^{-1}(u) = 
(p,g_u)$. This is the canonical local trivialization.
Note that $\phi_i^{-1}(\sigma_i(p)) = (p,e)$ and that 
$\phi_i(p,g) = \phi_i(p,e)g = \sigma_i(p)g$. If $p \in U_i \cap 
U_j$, two sections $\sigma_i(p)$ and $\sigma_j(p)$ are related by 
$\sigma_j(p) = \sigma_i(p)t_{ij}(p)$, since
%
\begin{displaymath}
	\sigma_j(p) = \phi_j(p,e) = \phi_i(p,t_{ij}(p)e) = 
	\phi_i(p,e)t_{ij}(p) = \sigma_i(p)t_{ij}(p)
\end{displaymath}
%

\begin{example}
	A widely used instance of principal bundles is the \emph{bundle 
		of linear frames}, which will be discussed in this example.

	Let $M$ be a manifold of dimension $n$. A linear frame $u$ at a 
	point $p \in M$ is an ordered basis $\{X_a\} \spc (a=1\ldots 
	n)$ of the tangent space at $p$, i.e.\ $T_pM$.  Let $L_pM$ be 
	the set of all frames at $p$. If $x^i$ is a local coordinate 
	system on $U_i \subset M$, the frame $u = \{X_a\}$ can be 
	expressed as
  %
	\begin{equation}
		X_a = X^i_{\spc a} \left. \frac{\pd}{\pd x^i}\right|_{p} 
		\spc,
	\end{equation}
  %
	where $X^i_{\spc a}$ is a non-singular matrix, since the 
	elements of a frame are linearly independent. This implies that 
	$X^i_{\spc a} \in Gl(n,\mbb{R})$ and $L_pM \simeq 
	Gl(n,\mbb{R})$.

	Let $LM$ be the set of all frames at all points $p \in M$ and 
	define $\pi : LM \rightarrow M$ as the projection which maps a 
	frame $u$ at $p$ into $p$. It is shown that this defines a 
	principal bundle as follows. Define a local trivialization 
	$\phi_i : U_i \times Gl(n,\mbb{R}) \rightarrow \pi^{-1}(U_i)$ 
	by $\phi_i^{-1}(u) = (p,X^i_a)$, i.e.\
  %
	\begin{equation}
		u = \phi_i(p, X^i_{\spc a})\spc.
	\end{equation}
  %
	This implies that $LM$ is a differentiable manifold (of 
	dimension $m + m^2$), since $M$ and $Gl(n,\mbb{R})$ are smooth 
	manifolds and $\phi_i$ is a smooth diffeomorphism. The right 
	action of the fibre $Gl(n,\mbb{R})$ on $LM$ is given as 
	follows. For $a \in Gl(n,\mbb{R})$ and $u \in LM, \pi(u) = p$,
  %
	\begin{equation}
		ua = \phi_i(p,X^i_{\spc b}) a = \phi_i(p, X^i_{\spc 
			b}x^b_{\spc a}(a)) = \phi_i(p, Y^i_{\spc a})
	\end{equation}
  %
	where the frame $\{X_i\}$ at $p$ is transformed into a new 
	frame $\{Y_a = X_b x^b_{\spc a}(a)\}$ at $p$. Note that this 
	action is free and transitive on the fibre. One concludes that 
	$LM(M,Gl(n,\mbb{R}))$ is a principal fibre bundle: the bundle 
	of linear frames.
\end{example}

\subsection{Associated bundles}

Let $P(M,G)$ be a principal bundle and $F$ a manifold on which 
$G$ acts on the left. The manifolds $G$ and $F$ do not have to be 
of the same dimension. Consider the product manifold $P \times 
F$. The right action of $G$ on $P \times F$ is defined as
%
\begin{equation}
	R_a : P \times F \rightarrow P \times F : (u,f) \mapsto 
	(ua,a^{-1}f) \spc, \quad a \in G
\end{equation}
We use this operation to define the space $E$ of equivalence 
classes as
%
\begin{equation}
	E \equiv P \times_G F \equiv \frac{P \times F}{G}\spc.
\end{equation}
Hence, the equivalence relation is given by the right action, 
i.e.\
%
\begin{displaymath}
	(u,f) \sim (ua,a^{-1}f)\spc, \quad a \in G\spc,
\end{displaymath}
and elements of $E$ are denoted by $[u,f]_\sim$.

A projection $\pi_E$ of $E$ onto $M$ is defined as
%
\begin{equation}
	\pi_E : E \rightarrow M : (u,f)_\sim \mapsto \pi(u)
\end{equation}
where $\pi(u)$ is the projection in $P$. Note that this 
definition is consistent with the equivalence relation since 
$\pi(ua) = \pi(u)$. For each $p \in M$, the set of points 
$\pi^{-1}_E(p) = \{(u,f)_\sim | \pi(u) = p\}$ is called the fibre 
of $E$ over $p$.

Let us now define a differentiable structure in $E$ such that it 
is turned into a smooth manifold. From the definition of a 
principal bundle it follows that every $p \in M$ has an open 
neighborhood $U_i$ such that it is locally a direct product, 
i.e.\ $\pi^{-1}(U_i) \simeq U_i \times G$. Locally then, the 
right action of $G$ on $P \times F$ goes as $(p,g,f) \mapsto 
(p,ga,a^{-1}f)$. Since this right action leaves the equivalent 
classes $E$ invariant and because $G$ is transitive under its own 
right action, the equivalence classes $[u,f]_\sim$ are locally 
described by a couple $(p,f)$ ($\pi(u)=p$). It follows that 
$\pi^{-1}_E(U_i) \simeq U_i \times F$. A differentiable structure 
in $E$ is obtained by demanding that $\pi^{-1}_E(U_i)$ be an open 
submanifold of $E$. 

Finally, we show that the above given local trivialization for 
$E$ has the same transition functions as the the local 
trivialization for $P$. The local trivialization for $E$ can be 
written down as (on $U_i \cap U_j \subset M$)
%
\begin{gather*}
	\psi_i^{-1}([u,f]_\sim) = (p,g,f)_\sim = (p,e,gf)_\sim \\
	\psi_j^{-1}([u,f]_\sim) = (p,t_{ji}(p)g,f)_\sim = 
	(p,e,t_{ji}(p)gf)_\sim \quad
\end{gather*}
This implies that on any overlapping region
%
\begin{equation}
	\psi_j(p,f) = \psi_i(p,t_{ij}(p)f)
\end{equation}

As such, we have constructed a fibre bundle $(E,\pi_E,M,F,G;P)$ 
associated with the principal bundle $P$. In fact, it is just a 
fibre bundle $(E,\pi,M,F,G)$ as in definition 
\ref{def:fibre_bundle}.

\blankline
Given a principal bundle $P(M,G)$ and an associated bundle $E$ 
with fibre $F$, one can define the following mapping. For each $u 
\in P$ and $f \in F$, denote by $uf$ the image of $(u,f) \in P 
\times F$ under the natural projection $P \times F \rightarrow 
E$, i.e.\ $uf = [u,f]_\sim$. One has for each $u \in P$ a mapping 
$u : F \rightarrow F_p = \pi^{-1}_E(p)$, where $p = \pi(u)$.  
Since for any $a \in G$, $[ua,f]_\sim = [u,af]_\sim$, it follows 
that $(ua)f = u(af)$.

Let $F_p = \pi^{-1}_E(p)$ and $F_q = \pi^{-1}_E(q)$, $p,q \in M$.  
An isomorphism of a fibre $F_p$ onto another fibre $F_q$ is a 
diffeomorphism that can be written as $v \circ u^{-1}$, where $u 
\in \pi^{-1}(p)$ and $v \in \pi^{-1}(q)$ are considered mappings 
of $F$ onto $F_p$ and $F_q$ respectively. An automorphism of 
$F_p$ is a mapping of the form $v \circ u^{-1}$, where $u$ and 
$v$ are in the same fibre $\pi^{-1}(p)$. Since the fibre $G_p$ is 
transitive under its own right action, $v = ua$ for some $a \in 
G$. It follows that any automorphism of $F_p$ is of the form $u 
\circ a \circ u^{-1}$, for an arbitrarily \emph{fixed} $u \in 
G_p$.  Hence, the group of automorphisms of $F_p$ is isomorphic 
with the structure group $G$.

\begin{example}
	*Tangent bundle*
\end{example}

\subsection{Reduction of principal bundles}

A \emph{homomorphism} $f : Q(M',H) \to P(M,G)$ between principal 
fibre bundles consists of a mapping $f' : Q \to P$ and a group 
homomorphism $f'' : H \to G$, such that $f'(qh) = f'(q)f''(h)$ 
for any $q \in Q$ and $h \in H$. In the remainder we denote these 
mappings all by the same letter $f$ and assume that its meaning 
is clear from the element it is acting on. Every homomorphism $f$ 
preserves the fibre structure of the bundles. For let $H_{p'} = 
\{qh~|~h \in H\}$ and $q \in H_{p'}$ arbitrary. Denote $u = f(q)$ 
where $u \in G_p$. It follows that $f(qh) = f(q)f(h)$ with $f(h) 
\in G$. Then $\pi(uf(h)) = \pi(u)$ so that $f(H_{p'}) \subset 
G_p$. Since $f$ maps each fibre of $Q$ into a fibre of $P$, this 
induces a mapping $f: M' \to M$.

The homomorphism $f$ is called an \emph{injection} if $f : Q \to 
P$ is an injection and $f : H \to G$ a group monomorphism. It 
follows that $f : M' \to M$ also is an injection. Given an 
injection, we can identify $Q \equiv f(Q)$, $H \equiv f(H)$ and 
$M' \equiv f(M')$. Then $Q(M',H)$ is said to be a 
\emph{subbundle} of $P(M,G)$. If moreover $M' = M$ and $f : M' 
\to M$ is the identity transformation of $M$, $f : Q(M,H) \to 
P(M,G)$ is said to be a \emph{reduction} of the structure group 
$G$ of $P(M,G)$ to $H$. The subbundle $Q(M,H)$ is a \emph{reduced 
	bundle} of $P(M,G)$.

\blankline
Let $H$ be a closed subgroup of $G$ and consider the coset space 
$G/H$. There is the natural left action $\tau_a : G/H \to G/H : 
gH \mapsto (ag)H$. Let then $E = P \times_G G/H$ be the 
associated bundle with standard fibre $G/H$. Furthermore, denote 
by $P/H$ the space of equivalence classes with equivalence 
relation $u \sim uh$ for $h \in H$. We identify $E$ with $P/H$ by 
mapping each element $[u,gH] = [ug,eH] \in E$ into $[ug] \in P$.  
Lastly, we define the projection $\mu : P \to E : u \mapsto u(eH) 
= [u,eH]$, where we considered the mapping $u : G/H \to 
\pi^{-1}_E(p)$ with $\pi(u) = p$. Note that the projection 
preserves fibres and that it is a submersion by the transitivity 
of $G$ on $G/H$. To conclude we summarize things in the following 
diagram:
\begin{displaymath}
	\xymatrix@R=1.5cm@C=1.3cm{
		Q(M,H)
			\ar[r]^-*{f}
			\ar[dr]^-*{\pi'}
		& P(M,G)
			\ar[r]^-*{\mu}
			\ar[d]^-*{\pi}
		& E = P/H
			\ar[dl]_-*{\pi_E}
		\\
		& M &
	}
\end{displaymath}

\begin{proposition}
	The structure group $G$ of $P(M,G)$ is reducible to a closed 
	subgroup $H$ if and only if the associated bundle $E = P 
	\times_G G/H$ admits a globally defined section $\sigma : M 
	\to E$.
\end{proposition}
\begin{proof}
	Let $f : Q(M,H) \to P(M,G)$ be a reduction of $G$ to a closed 
	subgroup $H$ and let $\mu : P \to E$ be the projection $u 
	\mapsto [u,eH]$. If $q$ and $q'$ are in the same fibre 
	$\pi'^{-1}(p)$, then there exists some $h \in H$ so that 
	$q=q'h$. One then has $\mu(f(q)) = \mu(f(q)f(h)) = \mu(f(q))$ 
	so that $\mu\circ f$ is a constant function on each fibre of 
	$Q$. This induces a mapping $\sigma : M \to E$ by $\sigma(p) = 
	\mu(f(q))$ with $q \in \pi'^{-1}(p)$. As $\pi(f(q)) = 
	\pi_E(\mu(f(q))) = p$, we have $\pi_E \circ \sigma = 
	\mathrm{id}$ and $\sigma$ is a section.

	Conversely, let $\sigma$ be a section of $E$. Denote by $Q$ 
	the subset of $P$ such that $\mu(u) = \sigma(\pi(u))$. In 
	other words, $Q = \mu^{-1}(\sigma(M))$.
\end{proof}


\section{Connections}

\subsection{Ehresmann connections}

Let $P(M,G)$ be a principal bundle, $u \in P$ and $p = \pi(u) \in 
M$, such that the fibre at $p$ is $G_p \equiv \pi^{-1}(p)$.

The \emph{vertical subspace} $V_u P$ at a point $u$ is the space 
tangent to the fibre through $u$. It is a subspace of the tangent 
space $T_u P$ at $u$. Note that given a $u \in P$ and $\pi(u) = 
p$, the fibre $G_p$ through $u$ is sweeped out by the right 
action of $G$ on $u$ (because $G_p$ is transitive under this 
action). For any 1-parameter subgroup $a_t$ of $G$, a curve 
$R_{a_t}(u)$ is traced out in the fibre $G_p$.  The generating 
vector field for the latter curve is the fundamental vector field 
$A^\star$ corresponding to the generator $A \in \mfrak{g}$ of 
$a_t$. It follows that $V_u P$ is spanned by $n = \dim G$ 
linearly independent $A^\star _u$, at any $u \in P$.  The 
vertical subspace is thus canonically given for a principal 
bundle. By definition $A^\star_{ug} = R_g{}_\ast A^\star_u$ and 
one concludes that the vertical subspace at $ug$ is given by 
right translating the vertical subspace at $u$, i.e.\ $V_{ug}P = 
R_g{}_\ast V_u P$.\footnote{This does \emph{not} imply that a 
	generic vertical vector field $X$ is right invariant, since 
	such a vector field may not be a linear combination of the 
	fundamental vector fields $A^\star$.}
Furthermore, since a curve $R_{a_t}(u)$ in the fibre $G_p$ will 
be projected into the same point $p \in M$, an element of $V_u P$ 
will be anihilated under this projection: $\pi_\ast X^V_u(f) = 
X^V_u (f \circ \pi) = 0$ because $f \circ \pi$ is a constant 
function.

As mentioned in the last paragraph, a principal fibre bundle 
canonically picks out the vertical subspace of dimension $n = 
\dim G$ from the tangent space, at any point in the bundle. This 
implies directly that at any point a subspace of dimension $m = 
\dim P - \dim G$ is left over: the \emph{horizontal subspace} 
$H_u P$. Unfortunately, the mere observation that at any point 
the horizontal subspace should be the linear complement of the 
vertical subspace does not give one a unique manner to smoothly 
assign one at each $u \in P$. Introducing extra structure into 
the picture by defining an \emph{Ehresmann 
	connection}\footnote{An Ehresmann connection will be called 
	just a connection} does allow one to have such an assignment.

\begin{definition}[Connection]\label{def:connection1}
	Let $P(M,G)$ be a principal bundle. A connection is a unique 
	seperation of the tangent space $T_u P$ at each $u \in P$ into 
	the vertical subspace $V_u P$ and the horizontal subspace $H_u 
	P$, such that
  %
	\begin{itemize}
		\item[(a)] $T_u P = V_u P \oplus H_u P$
		\item[(b)] A smooth vector field $X$ on $P$ is seperated into 
			smooth vector fields $X^V$ and $X^H$ such that $X = X^V + 
			X^H$ and for any $u \in P$, $X^V_u \in V_u P$ and $X^H_u 
			\in H_u P$
		\item[(c)] $H_{ug}P = R_g{}_\ast H_u P$ for any $u \in P, g 
			\in G$
	\end{itemize}
\end{definition}
%
Consider the pushforward of the projection, i.e.\ $\pi_\ast : T_u 
P \rightarrow T_p M$. From item (a) it follows that $\pi_\ast 
T_uP = \pi_\ast V_uP \oplus \pi_\ast H_uP$. Since $V_uP$ gets 
annihilated under this projection one concludes that $\dim 
\pi_\ast H_uP = \dim T_pM$. This means that once a connection is 
given, $H_uP$ is projected isomorphically onto $T_pM$. Let us 
mention that $\dim H_uP = \dim T_pM$, since $\dim P = \dim G + 
\dim M$.

\blankline
As we have seen, a connection basically splits up smoothly the 
tangent spaces along a principal bundle in vertical and 
horizontal subspaces. This splitting up effectively defines a 
projection of a tangent vector onto the vertical subspace. The 
latter is spanned by the fundamental vector fields corresponding 
to $\mfrak{g}$, hence isomorphic to $T_e G$. A connection thus 
defines an operator which takes an element of $T_uP$ as its  
input, mapping it into $T_e G \simeq \mfrak{g}$. This operator is 
a $\mfrak{g}$-valued 1-form on $P$ and it is called the 
\emph{connection 1-form}. Its definition is an equivalent 
starting point for discussing connections on principal bundles 
and will be used for applications.

\begin{definition}[Connection 1-form]\label{def:connection2}
	A connection 1-form $\omega$ on a principal bundle $P$ is 
	$\mfrak{g}$-valued 1-form, i.e.\ a projection of $T_uP$ onto 
	the vertical component $V_uP \simeq \mfrak{g}$, for any $u \in 
	P$; the projection property is summarized by the following 
	requirements:
  %
	\begin{itemize}
		\item[(a)] $\omega(A^\star) = A$, where $A \in \mfrak{g}$
		\item[(b)] $R_g^\ast \omega = \mrm{Ad}_{g^{-1}} \omega$, 
			where $\mrm{Ad}_{g^{-1}}$ acts on the $\mfrak{g}$ in 
			$\omega \in \mfrak{g} \otimes T^\ast P$.
	\end{itemize}
  %
	The horizontal subspace $H_uP$ is the kernel of $\omega$:
  %
	\begin{equation}\label{def:horsubspace}
		H_uP \equiv \{ X \in T_uP | \omega(X) = 0 \}
	\end{equation}
\end{definition}
%
At any point $u \in P$ and for every $X \in T_uP$, $\omega(X)$ is 
the unique $A \in \mfrak{g}$ such that $A^\star_u$ is equal to 
the vertical component of $X$.

Consider requirement (b) again. Taking a closer look at the 
equation,
%
\begin{displaymath}
	(R_g^\ast\omega)_{u}(X) = \omega_{ug}(R_g{}_\ast X_u) 
	\overset{def}{=} \mrm{Ad}_{g^{-1}}\omega_u(X) = R_g{}_\ast 
	\omega_u(X)
\end{displaymath}
%
shows that it basically tells the following. The projection of 
$R_g{}_\ast X$ in $ug \in P$ gives the same result as right 
translating the projection of $X$ in $u \in P$. This is the dual 
statement of item (c) in definition \ref{def:connection1} which 
states that the horizontal space at $ug$ is the right translation 
of the horizontal space at $u$. Indeed, the space defined in 
\eqref{def:horsubspace} satisfies $H_{ug}P = R_g{}_\ast H_uP$ as 
we prove now. Let $X_u \in H_uP$, i.e.\ $\omega_u(X) = 0$. It 
then follows that $\omega_{ug}(R_g{}_\ast X) = R_g^\ast 
\omega_{ug}(X) = R_g{}_\ast \omega_u(X) = 0$, hence $R_g{}_\ast X 
\in H_{ug}P$. In the other direction, since right translation in 
$P$ is an invertible linear map, any element in $H_{ug}P$ can be 
written as $R_g{}_\ast X$, for some $X \in H_uP$.

\blankline
In the following, the above introduced connection will be 
expressed by a family of forms defined on open subsets of the 
base manifold $M$. Therefore, let $\{U_i\}$ be an open covering 
of $M$ and consider a family of local sections $\sigma_i : U_i 
\rightarrow \pi^{-1}(U_i)\subset P$. The associated canonical 
local trivializations are isomorphisms $\phi_i$ such that 
$\phi_i(p,g) = \sigma_i(p)g$ and thus $\phi_i(p,e) = 
\sigma_i(p)$, for any $p \in M, g \in G$. Remember that 
$\sigma_j(p) = \sigma_i(p) t_{ij}(p)$, where $t_{ij} : U_i \cup 
U_j \rightarrow G$ are the transition functions. Let $\theta: 
T_gG \rightarrow T_eG$ be the Maurer-Cartan form on $G$.

Define on each non-empty intersection $U_i \cap U_j$ the 
$\mfrak{g}$-valued 1-form
%
\begin{equation}
	\theta_{ij} \equiv t_{ij}^\ast \theta : T_pM \rightarrow T_eG 
	\simeq \mfrak{g} \spc.
\end{equation}
%
To understand this definition, consider its action on an element 
$X$, i.e.\ $\theta_{ij}(X) = \theta(t_{ij}{}_\ast X)$. Since 
$t_{ij}{}_\ast: T_pM \rightarrow T_gG$ where $g = t_{ij}(p)$, $X$ 
is a vector field on $M$. The Maurer-Cartan form then left 
translates the resulting vector into the tangent space at the 
identity, i.e.\ $\theta_{ij}(X) = L_{g^{-1}}{}_\ast 
(t_{ij}{}_\ast X_p)_g$, $(g = t_{ij}(p))$.

The \emph{local connection forms} then are defined by pulling 
back the connection 1-form $\omega$ onto the open covering 
$\{U_i\}$,
%
\begin{equation}
	\mcal{A}_i \equiv \sigma_i^\ast \omega
\end{equation}
%
Its action on an element $X$ is given by $\mcal{A}_i (X) = 
\omega(\sigma_i{}_\ast X)$. Remember that $\sigma_i{}_\ast : 
T_pU_i \rightarrow T_uP$ and $u = \sigma_i(p)$. It is clear that 
$\mcal{A}_i : T_pU_i \rightarrow V_uP \simeq \mfrak{g}$ is a 
$\mfrak{g}$-valued 1-form on $U_i$, i.e.\ $\mcal{A}_i \in 
\mfrak{g} \otimes T^\ast U_i$.  Note that $\omega$ maps 
$(\sigma_i{}_\ast X)_u$ into an $A \in \mfrak{g}$ such that 
$A_u^\star$ is the vertical component of $(\sigma_i{}_\ast X)_u$.

On a region $U_i \cap U_j \neq \emptyset$, two local connection 
forms $\mcal{A}_i$ and $\mcal{A}_j$ are related by
%
\begin{equation}\label{eq:GaugeTrafo_A}
	\mcal{A}_j = \mrm{Ad}_{t_{ij}^{-1}} \mcal{A}_i + 
	\theta_{ij}\spc,
\end{equation}
%
a proof of which can be found in \cite{kob1996found}, pg.\ 66. We 
note without proof that although the connection forms 
$\mcal{A}_i$ are defined only locally on regions of the base 
manifold, the whole set of them together with the transformation 
rule \eqref{eq:GaugeTrafo_A} do capture all the information 
related to the global definition of $\omega$ on the principal 
bundle, i.e.\ the separation of the tangent bundle $TP$ in its 
vertical and horizontal components.

\begin{example}
	Consider a principal bundle $P(M,Gl(n,\mbb{R}))$. Since 
	$\mcal{A}_i \in \mfrak{g} \otimes T^\ast U_i$ and for 
	$Gl(n,\mbb{R})$ the Lie algebra $\mfrak{g} \simeq 
	\mfrak{gl}(n,\mbb{R})$, the arbitrary $n\times n$-matrices, one 
	concludes that the local connection forms will be of the form 
	$\mcal{A}^a_{i,b} = \mcal{A}^a_{i,b\mu} dx^\mu$.\footnote{The 
		``low'' Latin indices a, b, c, \dots are Lie-algebra indices.  
		The Greek indices are indices refering to the base manifold 
		coordinates.} The transition functions $t_{ij}^{-1}(p)$ are 
	elements of $Gl(n,\mbb{R})$. Since $\mrm{Ad}_{t_{ij}^{-1}}$ act 
	on the Lie algebra one has that $\mrm{Ad}_{t_{ij}^{-1}} 
	(\mcal{A}_i) \in \mrm{Ad}_{t_{ij}^{-1}} \mfrak{gl}(n,\mbb{R}) 
	\otimes T^\ast U_i$, hence (let $g(p) \equiv 
	t_{ij}^{-1}(p)$),\footnote{We will omit for simplicity the 
		$p$-dependence. However, this fact should
	not be forgotten in the following.}
  %
	\begin{equation}
		\begin{split}
			\mrm{Ad}_{g}(\mcal{A}^a_{i,b})
			&= x^a_{\spc c}(g) \mcal{A}^c_{i,d\mu} x^d_{\spc b}(g^{-1}) 
			dx^\mu \\
			&\approx g \mcal{A}_i g^{-1}
		\end{split}
	\end{equation}
  %

	Next, we compute $\theta_{ij} \equiv t_{ij}^\ast \theta \in 
	\mfrak{g} \otimes T^\star M$ explicitly. One has that $\left[
	\theta_{ij}(X_p) \right]^a_{\spc b} = \theta(t_{ij}{}_\ast 
X_p)^a_{\spc b}$, where $X_p \in T_pM$. Since
  %
	\begin{displaymath}
		t_{ij\ast} X^\mu \pd_\mu|_p f = X^\mu \pd_\mu|_p 
		f(g^{-1}(p)) = X^\mu \frac{\pd x^c_{~d}(g^{-1})}{\pd 
			x^\mu(p)} \pd_{x^c_{~d}}|_{g^{-1}} f
	\end{displaymath}
  %
  and $\theta^a_{~b} = x^a_{~c}(g) dx^c_{~b}(g^{-1})$, one finds 
  that
  %
	\begin{displaymath}
		\begin{split}
			\theta_{ij}(X_p)^a_{\spc b}
			&= x^a_{~c}(g) dx^c_{~b}(g^{-1}) \left[ X^\mu \pd_\mu 
				x^e_{~d}(g^{-1}) \pd_{x^e_{~d}}|_{g^{-1}}\right] \\
			&= x^a_{~e}(g) X^\mu \pd_\mu x^e_{~d}(g^{-1})
	\delta^e_c \delta^d_b \\
			&= x^a_{~c}(g) \pd_\mu x^c_{~b}(g^{-1})
	X^\nu dx^\mu(\pd_\nu|_p) \\
			&= x^a_{~c}(g) \pd_\mu x^c_{~b}(g^{-1})
	dx^\mu(X_p)
		\end{split}
	\end{displaymath}
  %
	From this it is clear that
  %
	\begin{equation}
		\begin{split}
			[\theta_{ij}]^a_{\spc b} &= x^a_{\spc c}(t_{ij}^{-1}(p)) 
			\pd_\mu x^c_{\spc b}(t_{ij}(p)) dx^\mu \\
			&\approx t_{ij}^{-1}(p) dt_{ij}(p)
		\end{split}
	\end{equation}

	To conclude, these results are used to rewrite eq.\ 
	\eqref{eq:GaugeTrafo_A} for a general matrix group fibre bundle 
	$P(M,Gl(n,\mbb{R}))$,
  %
	\begin{equation}
		\begin{split}
			\mcal{A}^a_{i,b\mu} dx^\mu &= x^a_{\spc c}(t_{ij}^{-1}) 
			\mcal{A}^c_{i,d\mu} x^d_{\spc b}(t_{ij}) dx^\mu + x^a_{\spc 
				c}(t_{ij}^{-1}) \pd_\mu x^c_{\spc b}(t_{ij}) dx^\mu \\
			&\approx t_{ij}^{-1} \mcal{A}_i t_{ij} + t_{ij}^{-1} 
			dt_{ij}
		\end{split}
	\end{equation}
\end{example}

\subsection{Horizontal lift and parallelism}

Let $P(M,G)$ be a principal fibre bundle. \emph{The horizontal 
	lift} of a vector field $X$ on $M$ is a vector field $X^h$ on 
$P$ which is horizontal and projects onto $X$, i.e.\ 
$\omega_u(X^h) = 0$ and $\pi_\ast(X^h_u) = X_{\pi(u)}$ for all $u 
\in P$.

Given a connection in $P$, the horizontal lift exists and is 
unique, because the projection $\pi$ gives a linear isomorphism 
between $H_uP$ and $T_uM$. The horizontal lift $X^h$ is right 
invariant, because (1) the horizontal subspace is right invariant 
and (2) the projection acts the same on the whole fibre $(\pi 
\circ R_a = \pi)$. Conversely, given a right invariant horizontal 
vector field $X^h$ on $P$, it is the lift of a vector field $X$ 
on $M$.

Let $\gamma_t$ ($0 \leq t \leq 1$) be a curve in $M$.  \emph{A 
	horizontal lift} of $\gamma_t$ is a curve $\gamma_t^h$ ($0 \leq 
t \leq 1$) in $P$ such that for any $t$ it projects onto 
$\gamma_t$ , i.e.\ $\pi(\gamma_t^h) = \gamma_t$, and its tangent 
vector field is horizontal.

The notions of the lift of a vector field and a lift of a curve 
are related to eachother. If $X^h$ on $P$ is the lift of $X$ on 
$M$, then the integral curve $u_t$ of $X^h$ through a point $u_0$ 
is a lift of the integral curve $\gamma_t$ of $X$ through 
$\gamma_0 = \pi(\gamma_0^h)$. Indeed, let $u_t$ be the integral 
curve of $X^h$ through $u_0$.  Then,
%
\begin{displaymath}
	\pi_\ast X^h_{u_t} f = \pi_\ast\tfrac{d}{dt}f(u_t) = 
	\tfrac{d}{dt} f(\pi(u_t)) \overset{!}{=} X_{\gamma_t}f = 
	\tfrac{d}{dt}f(\gamma_t)
\end{displaymath}
%
Since $X^h$ is horizontal and because the last equality implies 
$\pi(u_t) = \gamma_t$, $\gamma^h_t \equiv u_t$ is a lift of 
$\gamma_t$.  Note that the set of all horizontal lifts of a curve 
are the integral curves of the lift of the corresponding vector 
field.

Let $\gamma_t$ ($0 \leq t \leq 1$) be a curve in $M$ and let $u_0 
\in \pi^{-1}(\gamma_0)$. There exists a unique lift $\gamma_t^h$, 
such that $\gamma^h_0 = u_0$. A proof can be found in 
\cite{kob1996found}, pg.\ 69. Because the lift is unique, there 
is a unique $u_1 = \gamma^h_1 \in \pi^{-1}(\gamma_1)$. This point 
$u_1$ is called the \emph{parallel displacement along the curve 
	$\gamma^h$} of $u_0$. In this way, a mapping is defined
%
\begin{equation}
	\Gamma(\gamma^h) : G_{\gamma_0} \rightarrow G_{\gamma_1} : u = 
	\gamma^h_0 \mapsto \gamma^h_1
\end{equation}
%
Since the horizontal lift $X^h$ is right invariant, a lifted 
curve is mapped into another one of the same vector field under 
the right action of $G$. Hence, one has that
%
\begin{equation}
	\Gamma(\gamma^h) \circ R_g = R_g \circ \Gamma(\gamma^h)\spc.
\end{equation}
%
This then implies that the parallel displacement 
$\Gamma(\gamma^h)$ is an isomorphism between $G_{\gamma_0}$ and 
$G_{\gamma_1}$. Indeed, if parallel displacement and right action 
would not commute, the inverse of $\Gamma(\gamma^h)$ would not be 
a well-defined mapping.

\subsection{Curvature and the Bianchi identity}

Let $P(M,G)$ be a principal bundle and $\rho$ a representation of 
$G$ on a vector space $V$.
%
\begin{definition}[Pseudotensorial form]
	A pseudotensorial form of degree $r$ on $P$ of type $(\rho,V)$ 
	is a $V$-valued $r$-form $\phi$ on $P$ such that
  %
	\begin{equation}
		R_g^\ast \phi = \rho(g^{-1}) \phi \spc, \quad g \in G
	\end{equation}
  %
	The form $\phi$ is called tensorial if it is horizontal, i.e.\
  %
	\begin{equation}
		\phi(X_1,\ldots\,X_r) = 0
	\end{equation}
  %
	whenever at least one of the $X_i$ is vertical.
\end{definition}
%

\begin{example}\label{ex:tensorial_skewsymm}
	Let $\vphi$ be a tensorial form of degree $r$ of type 
	$(\rho,V)$ on $P$ and let $E$ be the associated bundle 
	$P\times_G V$. In this example we show that $\vphi$ can be 
	regarded as an assignment of a multilinear skew-symmetric 
	mapping
	%
	\begin{equation}\label{eq:skewsymm_tens}
		\tilde{\vphi} : T^{(r,0)}_p M \rightarrow \pi^{-1}_E(p)
	\end{equation}
	to each $p \in M$. Namely, define for $X_i \in T_pM$
	%
	\begin{equation}
		\tilde{\vphi}(X_1,\ldots,X_r) \equiv 
		u(\vphi(X^\ast_1,\ldots,X^\ast_r))
	\end{equation}
	where $u \in \pi^{-1}(p)$ and $X^\ast_i \in T_uP$ such that 
	$\pi_\ast(X^\ast_i) = X_i$. Since $\vphi$ takes its values in 
	$V$ and $u$ maps elements of $V$ into $\pi^{-1}_E(p)$,  
	$\tilde{\vphi}(X_i)$ is indeed as proposed. We now show that 
	this definition is independent of the choice $u \in 
	\pi^{-1}(p)$ and $X^\ast_i \in T_uP$. Therefore, consider $ua$ 
	with $a \in G$. We then have $\tilde{\vphi}(X_i) = 
	(ua)(\vphi_{ua}(Y^\ast_i))$. Since $\vphi$ is a tensorial form 
	we can forget about vertical components and assume $Y^\ast_i 
	\in H_{ua}P$. Horizontal subspaces are invariant under the 
	right $G$-action, so that there exists a $Z^\ast_i = 
	R_{a^{-1}\ast} Y^\ast_i$ in $H_uP$. By construction 
	$\pi_\ast(Z^\ast_i) = X_i$ and since $\pi_\ast$ constitutes an 
	isomorphism between $H_uP$ and $T_pM$, we have $Z^\ast_i = 
	X^\ast_i$ or $Y^\ast_i = R_{a\ast}X^\ast_i$. This gives us
	%
	\begin{displaymath}
		\tilde{\vphi}(X_i) = (ua)(\vphi(R_{a\ast}X^\ast_i) = 
		(ua)\rho(a^{-1})\cdot\vphi(X^\ast_i) = u(\phi(X^\ast_i))
	\end{displaymath}
	where we used that $(ua)(V) = u(aV)$.

	Conversely, given a skew-symmetric mapping 
	\eqref{eq:skewsymm_tens} for each $p \in M$, a tensorial form 
	$\vphi$ of degree $r$ of type $(\rho,V)$ can be defined by
	%
	\begin{equation}
		\vphi(X^\ast_i) \equiv 
		u^{-1}(\tilde{\vphi}(\pi_\ast(X^\ast_i)))
	\end{equation}
	It is easy to check that this is indeed the right tensorial 
	form.
\end{example}

Denote by $h : T_uP \rightarrow H_uP : X_u \mapsto hX_u \equiv 
X^H_u$ the projection operator, which maps a vector field into 
its horizontal component. If $\phi$ is a pseudotensorial $r$-form 
on $P$ of type $(\rho,V)$, then one has the following.
%
\begin{itemize}
	\item[(a)] The form $\phi h$ defined by
		\begin{equation}
			(\phi h)(X_i) \equiv \phi(hX_i)
		\end{equation}
    %
		is a tensorial $r$-form of type $(\rho,V)$.

		To prove this we first note that $R_g \circ h = h \circ R_g$.  
		Therefore,
    %
		\begin{displaymath}
			\begin{split}
	R_g^\ast(\phi h)(X_i) &= (\phi h)(R_g{}_\ast X_i) \\
	&= \phi(R_g{}_\ast hX_i) \\
	&= R^\ast_g \phi(hX_i) \\
	&= \rho(g^{-1}) \phi(hX_i) \\
	&= \rho(g^{-1}) (\phi h)(X_i)
			\end{split}
		\end{displaymath}
    %
		Furthermore, since $hX_i = 0$ whenever $X_i$ vertical, one 
		has the desired result.
    %
	\item[(b)] The form $d\phi$ is a pseudotensorial $(r+1)$-form 
		of type $(\rho,V)$.

		This is shown to be true, by calculating the following.
    %
		\begin{displaymath}
			\begin{split}
				R_g^\ast(d\phi)(X_i) &= d(R_g^\ast \phi)(X_i) \\
				&= d(\rho(g^{-1}) \phi)(X_i) \\
				&= \rho(g^{-1})d\phi(X_i)
			\end{split}
		\end{displaymath}
    %
		where we used $f^\ast \circ d = d \circ f^\ast$.
    %
	\item[(c)] The \emph{exterior covariant derivative} of $\phi$ 
		is the form $D\phi$ defined by
    %
		\begin{equation}
			D\phi(X_i) \equiv (d\phi h)(X_i) = d\phi(hX_i)
		\end{equation}
    %
		is a tensorial form of type $(\rho,V)$. Hence, $R_g^\ast 
		D\phi = \rho(g^{-1}) D\phi$ and $D\phi(X_i)=0$ whenever one 
		of the $X_i$ vertical. This result follows directly from (a) 
		and (b).
\end{itemize}

Note that by construction, $\phi \in V \otimes \Omega^r(P)$ and 
$D\phi \in V \otimes \Omega^{r+1}(P)$ belong to the same 
representation of $G$: they transform covariantly.

\blankline
From the definition of a connection $1$-form $\omega \in 
\mfrak{g} \otimes \Omega(P)$, it is clear that it is a 
pseudotensorial 1-form of type $(\mrm{Ad},\mfrak{g})$, or simply 
of type $\mrm{Ad}\spc G$. More precisely it is a consequence of 
$R_g^\ast \omega = \mrm{Ad}_{g^{-1}} \omega$. It then follows 
that the exterior derivative $D\omega$ is a tensorial $2$-form of 
type $\mrm{Ad}\spc G$. It is called the \emph{curvature} of 
$\omega$ and it is awarded the symbol $\Omega \equiv D\omega$.
%
\begin{gather*}
	\Omega \in \mfrak{g} \otimes \Omega^2(P) \\
	R_g^\ast \Omega = \mrm{Ad}_{g^{-1}} \Omega
\end{gather*}

Let $X,Y \in TP$. The \emph{Cartan structure equations} are given 
by\footnote{For a proof see e.g.\ \cite{kob1996found}, pg.\ 77.
}
%
\begin{equation}\label{eq:StrucEqCurv}
	d\omega(X,Y) + \tfrac{1}{2}[\omega,\omega](X,Y) = \Omega(X,Y)
\end{equation}
which can be rewritten concisely as
%
\begin{equation}
	d\omega + \omega \wedge \omega = \Omega \spc.
\end{equation}
%
Since the connection and curvature forms are $\mfrak{g}$-valued 
differential forms they can be expanded in a frame $E_a$ as 
$\omega = E_a \cdot \omega^a$ and $\Omega = E_a \cdot \Omega^a$.  
By noting that $[\omega,\omega] \equiv [E_a,E_b]\cdot \omega^a 
\wedge \omega^b$, the structure equations \eqref{eq:StrucEqCurv} 
are expanded in the same frame as
%
\begin{equation}\label{eq:StrucEqCurvComp}
	d\omega^a + \tfrac{1}{2} c_{bc}^{\spc\spc a} \omega^b \wedge 
	\omega^b = \Omega^a
\end{equation}

Taking the exterior derivative of the curvature form and using 
equation \eqref{eq:StrucEqCurvComp}, one obtains
%
\begin{displaymath}
	d\Omega = E_a \cdot \Omega^a
	= E_a \cdot \tfrac{1}{2} c_{bc}^{\spc\spc a}
	(d\omega^b \wedge \omega^c - \omega^b \wedge d\omega^c)
\end{displaymath}
%
Since, $\omega(hX) = 0$ for any vector field $X$ on $P$, we find 
the \emph{Bianchi identity}
%
\begin{equation}
	D\Omega(X,Y,Z) \equiv 0
\end{equation}
for any three vector fields $X, Y, Z$ on $P$.

\blankline
To conclude this section, the curvature form will be be pulled 
back onto the base manifold $M$. Remember that given an open 
covering $\{U_i\}$ of $M$, the local connection forms were given 
by $\mcal{A}_i \equiv \sigma_i^\ast \omega$. Analogously, the 
local curvature forms are defined by
%
\begin{equation}
	\mcal{F}_i \equiv \sigma_i^\ast \Omega \spc.
\end{equation}
%
From Cartan's structure equations it follows that
%
\begin{equation}\label{eq:LocalStrucEqCurv}
	\mcal{F}_i = d\mcal{A}_i + \mcal{A}_i \wedge \mcal{A}_i \spc.
\end{equation}
%
On overlapping regions $U_i \cap U_j \neq \emptyset$ two local 
curvature forms, pulled back by different section $\sigma_i(p)$ 
and $\sigma_j(p) = \sigma_i(p)t_{ij}(p)$, are related by
%
\begin{equation}\label{eq:GaugeTrafo_F}
	\mcal{F}_j = \mrm{Ad}_{t_{ij}^{-1}} \mcal{F}_i \spc.
\end{equation}
%
This result follows basically from the proof for the 
transformation behaviour of the local connection forms, i.e.\ see 
\cite{kob1996found}, pg. 66 and by noting that the curvature form 
$\Omega$ is horizontal. The local version of the Bianchi identity 
is found by taking the exterior derivation of 
\eqref{eq:LocalStrucEqCurv}, i.e.\
%
\begin{displaymath}
	d\mcal{F}_i = d\mcal{A}_i \wedge \mcal{A}_i - \mcal{A}_i \wedge 
	d\mcal{A}_i
\end{displaymath}
%
Since $d\mcal{A}_i$ is a (Lie algebra valued) $2$-form, the RHS 
is just the bracket $[d\mcal{A}_i,\mcal{A}_i]$ and it behaves as 
a commutator. Hence, $[d\mcal{A}_i,\mcal{A}_i] = -[\mcal{A}_i,
d\mcal{A}_i] = [\mcal{A}_i, \mcal{A}_i \wedge \mcal{A}_i] - 
[\mcal{A}_i, \mcal{F}_i]$. Since
%
\begin{displaymath}
	[\mcal{A}_i, \mcal{A}_i \wedge \mcal{A}_i]
	= \mcal{A}_i \wedge \mcal{A}_i \wedge \mcal{A}_i -
	\mcal{A}_i \wedge \mcal{A}_i \wedge \mcal{A}_i = 0
\end{displaymath}
%
the well known form of the Bianchi identity is found, i.e.\
%
\begin{equation}
	d\mcal{F}_i + [\mcal{A}_i, \mcal{F}_i] = 0 \spc.
\end{equation}
%
That this is in a way the pulled back version of the Bianchi 
identity introduced before for the curvature form on the 
principal bundle is not too difficult to understand. In deriving 
the Bianci identity for a principal bundle, the crucial 
ingredient was considering the exterior derivative of Cartan's 
structure equations.  It is an easy exercise to check that 
pulling these back on the base manifold gives you the local 
Bianchi identities.

From now on, the index $i$ will be omitted. Note that this index 
refers to the section $\sigma_i$ which is used to pullback the 
connection and curvature form onto a region $U_i$ of $M$. One 
changes index $i \rightarrow j$ with the given transformation 
rules \eqref{eq:GaugeTrafo_A} and \eqref{eq:GaugeTrafo_F}. This 
is the transformation behaviour when a section is transformed 
into another section (using the transition functions) and it does 
not change the principal bundle at all. This transformation is 
governed by the right action of the structure group on the fibre 
and we call them \emph{gauge transformations}. The connection 
form $\mcal{A}$ is called the gauge field and $G$ the gauge 
group. So although we do ``forget'' about the index we do 
remember its existence: different but equivalent sections are 
related by gauge transformations and $\mcal{A}$ and $\mcal{F}$ 
transform as prescribed.

\subsection{Induced connection on associated bundles}

Let $P(M,G)$ be a principal bundle and $\Gamma$ a connection on 
$P$. Denote by $E$ the associated fibre bundle with fibre $F$, 
i.e.\ $E = P \times_G F$. Consider then the element $w = [u,\xi] 
\in E$ that lies in the fibre $F_p$, hence $\pi_E(w) = \pi(u) = 
p$. The vertical subspace $V_wE \subset T_wE$ is defined as the 
vector space that is tangent to the fibre $F_p$. Since at any 
point $w \in E$ the fibre $F_p$ is a manifold of dimension $\dim 
F$, one has that $\dim V_wE = \dim F$.

Let us now define the horizontal subspaces of $TE$. Consider the 
natural projection $P \times F \rightarrow E$, i.e.\ $(u,\xi) 
\mapsto [u,\xi]$. Fixing $\xi \in F$ then defines a mapping
\begin{equation}
	\xi : P \rightarrow E : u \mapsto [u,\xi]
\end{equation}
Any two elements $v,u \in G_p$ can be connected by an element $a 
\in G$, i.e.\ $v = ua$. This implies that they will be mapped 
into the same fibre $F_p$; $\xi(G_p) \subset F_p$. On the other 
hand, for elements $u,v$ belonging to different fibres in $P$ 
there does not exist an $a \in G$ connecting them and they will 
get mapped in different fibres of $E$. The horizontal subspace 
$H_wE$ is then defined as the image of $H_uP$ under $\xi$, i.e.\ 
$H_wE \equiv \xi_\ast H_uP$ ($w=\xi(u)$). This definition is 
clearly independent of the choice $(u,\xi) \in P \times F$.  
Indeed, choose another pair $(v,\eta)$, such that $\eta(v) = w$.  
This means that $[v,\eta] = [u,\xi]$, hence there exists some $a 
\in G$ such that $(v,\eta) = (ua, a^{-1}\xi)$. It follows that 
$\eta(u) = [u,\eta] = [ua^{-1},\xi] = \xi(ua^{-1})$ or $\eta(ua) 
= \xi(u)$. One then finds
%
\begin{displaymath}
	T_wE \equiv \xi_\ast H_uP = \eta_\ast R_{a\ast} H_uP = 
	\eta_\ast H_vP
\end{displaymath}
which proves our assertion.

\blankline
A curve in $E$ is said to be horizontal if its generating vector 
field is horizontal at each point along the curve. Let $\gamma_t$ 
be a curve in $M$. A curve $\gamma^h_t$ in $E$ is a horizontal 
lift of $\gamma_t$ if it is horizontal and $\pi_E(\gamma^h_t) = 
\gamma_t$. Consider again the mapping $\xi : P \rightarrow E : u 
\mapsto [u,\xi]$ and let $u^h_t$ be the horizontal lift of 
$\gamma_t$ in $P$. By definition, $\xi_\ast$ maps the generating 
vector field of $u^h_t$ into a horizontal field in $TE$. It then 
follows that the curve $\gamma^h_t \equiv [u^h_t, \xi]$ is a 
horizontal lift of $\gamma_t$ in $E$. Because $\xi_\ast$ is an 
isomorphism between $HP$ and $HE$, every horizontal lift in $E$ 
can be obtained in this way.

As in the case of principal bundles, parallel transport in 
associated bundles is also governed by horizontal curves. Let 
$\gamma_t$ be a curve in $M$ and let $\gamma^h_t = [u^h_t, \xi]$ 
be the horizontal lift in $E$. Let $s(p)$ be a section in $E$ 
such that $s(\gamma_0) = \gamma^h_0$. The parallel displacement 
of $s(\gamma_0) \in F_{\gamma_0}$ along $\gamma_t$ is given by 
$\gamma^h_t$, for any $t$. Also, a section $s(\gamma_t) = 
[u^h_t,\xi(t)]$ is parallel transported if $\xi(t)$ is constant 
along $\gamma_t$, as this means that $s(\gamma_t)$ will be a 
horizontal lift of $\gamma_t$.

\section{Linear connections}

\subsection{Soldering}

Let $P(M,G)$ be a principal fibre bundle with $G$ the general 
linear group, so that $P$ is the bundle of linear frames $LM$.  
Let $TM = LM \times_G \mbb{R}^n$. Remember that given a frame $u 
\in P$ and $\pi(u) =p$, $u : \mbb{R}^n \rightarrow T_p M$ is a an 
isomorphism.

The \emph{canonical form} or \emph{solder form} $\theta$ on $LM$ 
is the $\mbb{R}^n$-valued 1-form defined by
%
\begin{equation}
	\theta(X) \equiv u^{-1}(\pi(X)) \quad \mathrm{for}\ X \in T_uLM
\end{equation}
Note that $(R^\ast_a \theta)(X) = \theta(R_{a\ast}X) = 
a^{-1}u^{-1}(\pi(X)) = a^{-1} \theta(X)$. Since $\pi(X) = 0$ for 
$X$ vertical, one concludes that the solder form is a tensorial 
1-form of type $(Gl(n,\mbb{R}),\mbb{R}^n)$. Furthermore, in the 
sense of example \ref{ex:tensorial_skewsymm}, it is the identity 
transformation on $T_pM$. Indeed the associated mapping is given 
by $\tilde{\theta}(X) = u(u^{-1}(\pi(X^\ast)) = X$, for $X \in 
T_pM$.

\subsection{Covariant derivative on associated vector bundles}

In this subsection we focus on the cases where $F$ is a vector 
space $V$ and $G$ a matrix group that acts on $F$ through a 
representation $\rho$.

[The sections of $E$ do form a vector space of infinite 
dimension]

Let $\vphi$ be a section of $E$ defined on $\gamma_t \in M$, so 
that $\pi_E \circ \vphi(\gamma_t) = \gamma_t$. Denote by 
$\dot{\gamma}_t$ the tangent vector field to $\gamma_t$. The 
\emph{covariant derivative} of $\vphi$ in the direction of $X 
\equiv \dot{\gamma}_t$ is given by
%
\begin{equation}
	\nabla_{X} \vphi \equiv \lim_{h \rightarrow 0}
	\tfrac{1}{h}[ \Gamma(\gamma^h)^{t+h}_t (\vphi(\gamma_{t+h}))
	- \vphi(\gamma_t) ]
\end{equation}
%
where $\Gamma(\gamma^h)^s_t : \pi^{-1}_E(\gamma_s) \rightarrow 
\pi^{-1}_E(\gamma_t)$ is the parallel displacement of the fibre 
at $\gamma_s$ onto the fibre at $\gamma_t$. This can be rewritten 
in an equivalent form as follows. Let $\phi(\gamma_t) = 
[\sigma(t), \eta(t)] = [u^h_t a(t), \eta(t)] = [u^h_t, \xi(t)]$, 
where $u^h_t$ is a horizontal lift of $\gamma_t$ in $E$ and $a(t) 
\in G$. Consider the horizontal lift of $\gamma_t$ through the 
element $\vphi(\gamma_{t+h}) = [u^h_{t+h}, \xi(t+h)]$, i.e.\ 
$[u^h_t, \xi|_{t+h}]$. It then follows that
%
\begin{displaymath}
	\nabla_X \vphi = \lim_{h \rightarrow 0} \tfrac{1}{h} ([u^h_t, 
	\xi|_{t+h}] - [u_t, \xi|_t]) = [u^h_t, \tfrac{d}{dt} \xi(t)]
\end{displaymath}

The covariant derivative is ultimately a sum between two elements 
in the fibre $F_{\gamma_t}$, hence it is also an element of this 
same fibre. This happens smoothly along the curve $\gamma_t$ and 
one concludes that given a section $\phi$ in a vector bundle, its 
covariant derivative is a section of the same type. Hence, given 
the fact that $\phi$ transforms through the representation $\rho$ 
under $G$, its covariant derivative transforms through the same 
representation, i.e.\ \emph{covariant}.

Note that the definition of a covariant derivative explicitely 
assumes the sum operation, which is defined in vector bundles.

\blankline
The covariant derivative can be given a local expression as follows.  
Consider therefore a section $\sigma(t) : U \rightarrow P$, such that the 
horizontal lift of $\gamma_t$ in $P$ can be written as
%
\begin{displaymath}
	u^h_t = \sigma(t) a_t
\end{displaymath}
%
where $a_t$ is a curve in $G$. Define a section $e_a(p)$ of $E$ as
%
\begin{displaymath}
	e_a(p) = [\sigma(p), \hat{e}_a]
\end{displaymath}
%
where $\{\hat{e}_a\}$ is a basis for $F$, hence $(\hat{e}_a)^b = 
\delta^a_b$. Then this section along the curve $\gamma_t$ is given by 
$e_a(t) = [\sigma(t), \hat{e}_a] = [u^h_t a^{-1}_t, \hat{e}_a] = [u^h_t, 
a^{-1}_t \hat{e}_a]$. The covariant derivative of $e_a$ in the direction 
$X=\dot{\gamma}_t$ is given by (remember that $G$ acts through a matrix 
representation on $F$)
%
\begin{align*}
	\nabla_X e_a &= [u^h_t, \tfrac{d}{dt}(a_t \hat{e}_a)] \\
	&= [u^h_t, -a^{-1}_t \dot{a}_t a^{-1}_t \hat{e}_a] \\
	&= [u^h_t a^{-1}_t, -\dot{a}_t a^{-1}_t \hat{e}_a]
\end{align*}
%
Since the curve $\sigma(t)a_t$ is horizontal, one has that 
$\dot{a}_t a^{-1}_t = -\mcal{A}(\dot{\gamma}_t)$.\footnote{See 
	\cite{nakahara2003geometry}, pg.\ 69.} This $\mfrak{g}$-valued 
1-form can be explicitely written out as 
$\mcal{A}(\dot{\gamma}_t) = \mcal{A}^a_{\spc b\mu} 
dx^\mu(\dot{\gamma}^\nu \pd_\nu) = \mcal{A}^a_{\spc b\mu} 
\dot{\gamma}^\mu$. Pluging this information into the definition 
of the covariant derivative, one finds
%
\begin{equation}
	\nabla_X e_a = [\sigma(t), \dot{\gamma}^\mu \mcal{A}^b_{\spc 
		a\mu} \hat{e}_b ]=:\dot{\gamma}^\mu \mcal{A}^b_{\spc a\mu} 
	e_b
\end{equation}

In the special case of $\gamma(t)$ being a coordinate curve, 
i.e.\ $\gamma^\mu_t = x^\mu_t$ with $\mu$ a fixed number, $X = 
\pd_\mu$ and covariant differentiation reduces to
%
\begin{displaymath}
	\nabla_\mu e_a = \mcal{A}^b_{\spc a\mu} e_b
\end{displaymath}

Then let us consider a generic section of $E$, i.e.\ $\vphi(p) = 
[\sigma(p), \xi^a(p) \hat{e}_a] = \xi^a(p) e_a$ with $\xi(p) = 
\xi^a(p) \hat{e}_a \in F$.
The covariant derivative along the curve $\gamma_t$ is then found 
to be
%
\begin{displaymath}
	\nabla_X \vphi = \nabla_X (\xi^a(t) e_a) = X(\xi^a(t))e_a + 
	\xi^a(t) \nabla_X e_a
\end{displaymath}
where we used the chain rule for covariant differentiation and 
noted that $\xi^a(p)$ is a set of functions on $M$. Using the 
results obtained before, the covariant derivative is
%
\begin{equation}
	\nabla_X \vphi = \dot{\gamma}^\mu (\pd_\mu \xi^a + 
	\mcal{A}^b_{\spc a\mu} \xi^a) e_b
\end{equation}

The result can be extended to find the covariant differentiation 
of tensor products of sections in $E$. Let $\psi(p) = [\sigma(p), 
\xi^{ab}\hat{e}_a \otimes \hat{e}_b] = \xi^{ab} e_a \otimes e_b$, 
an element of $E \otimes E$.\footnote{The tensor product of two 
	fibre bundles that have the same base manifold is constructed 
	by considering at each base point the tensor product of the 
	fibres. See for example \cite{nakahara2003geometry}.} Then 
invoking the chain rule, one finds
%
\begin{align*}
	\nabla_X \psi &= \nabla_X (\xi^{ab} e_a \otimes e_b) \\
	&= X(\xi^{ab})e_a \otimes e_b + \xi^{ab} \nabla_X e_a \otimes 
	e_b + \xi^{ab} e_a \otimes \nabla_X e_b \\
	&= X^\mu \pd_\mu\xi^{ab} e_a \otimes e_b + \xi^{ab} X^\mu 
	\mcal{A}^c_{~a\mu} e_c \otimes e_b + \xi^{ab} X^\mu 
	\mcal{A}^c_{~b\mu} e_a \otimes e_c~,
\end{align*}
so that
%
\begin{equation}
	\nabla_X \psi = \dot{\gamma}^\mu (\pd_\mu\xi^{ab} + 
	\mcal{A}^a_{~c\mu}\xi^{cb} + \mcal{A}^b_{~c\mu}\xi^{ac}) e_a 
	\otimes e_b~.
\end{equation}
This is straightforwardly generalized for higher order tensor 
products.

Note that to obtain local expressions, one has to choose a 
section in $P$. From the definition of the covariant derivative 
however, it is clear that the latter is independent of the 
section chosen: it is an intrinsic notion, once a connection in 
$P$ is chosen.

\begin{example}
	Linear connection and spin connection
\end{example}

\appendix
\section{Vector valued differential forms}

Let $M$ be an $m$-dimensional manifold and $V$ a vector space 
spanned by $n$ basis elements $E_a$. Let $\zeta$ and $\eta$ be 
respectively a vector valued $p$-form and $q$-form, i.e.\ $\zeta 
\in V \otimes \Omega^p(M)$ and $\eta \in V \otimes \Omega^q(M)$.  
This can be expanded as follows,
%
\begin{gather*}
	\zeta = E_a \cdot \zeta^a \spc;
	\quad \zeta^a \in \Omega^p(M)\\
	\eta = E_a \cdot \eta^a \spc;
	\quad \eta^a \in \Omega^q(M)
\end{gather*}
%
The wedge product of two vector valued differential forms is 
defined as\footnote{The tensor product $E_a \otimes E_b$ will 
	also be denoted as just $E_a E_b$.}
%
\begin{equation}
	\zeta \wedge \eta \equiv
	E_a \otimes E_b \cdot \zeta^a \wedge \eta^b
\end{equation}
%
The exterior derivative is defined to be
%
\begin{equation}
	d\zeta \equiv E_a \cdot d\zeta^a
\end{equation}

From this it follows that
%
\begin{equation}
	\begin{split}
		d(\zeta \wedge \eta) &= E_a \otimes E_b \cdot
		d\zeta^a \wedge \eta^b + (-1)^p \zeta^a \wedge d\eta^b \\
		&= d\zeta \wedge \eta + (-1)^p \zeta \wedge d\eta
	\end{split}
\end{equation}
%
Note that for vector valued differential forms, one does not have 
a relation between $\zeta \wedge \eta$ and $\eta \wedge \zeta$ 
because of the tensor product in $E_a \otimes E_b$.

\blankline
For the following, we consider the case when $V$ is a Lie algebra 
$\mfrak{g}$, hence when $\zeta$ and $\eta$ are $\mfrak{g}$-valued 
differential forms. A Lie bracket is defined as
%
\begin{equation}
	[\zeta, \eta] \equiv [E_a,E_b] \cdot
	\zeta^a \wedge \eta^b
	= c_{ab}^{\spc\spc c}E_c \cdot
	\zeta^a \wedge \eta^b
\end{equation}
%
where $[E_a,E_b]$ denotes the usual Lie bracket on the algebra.
From the definition of the wedge product one has that
%
\begin{equation}\label{App.eq:RelBraWedgeDF}
	[\zeta, \eta] = \zeta \wedge \eta
	-(-1)^{pq} \eta \wedge \zeta
\end{equation}
%
This shows that the bracket $[\cdot, \cdot]$ is a graded 
commutator. It is an anticommutator when $pq$ is odd, being a 
commutator for $pq$ even.

To conclude, consider the example of a Lie algebra valued 
$1$-form $\omega$. From \eqref{App.eq:RelBraWedgeDF} it follows 
that
%
\begin{equation}
	[\omega,\omega] = 2 \omega \wedge \omega
\end{equation}






\bibliographystyle{plain}
\bibliography{../../References/All.bib}

\end{document}
