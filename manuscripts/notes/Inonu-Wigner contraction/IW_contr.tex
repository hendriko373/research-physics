\documentclass[10pt]{article}

%Package declarations
%--------------------
\usepackage{amsmath}
\usepackage{amsfonts}
\usepackage{amssymb}

\usepackage[a4paper]{geometry}
\geometry{left=3cm,right=3cm,top=3cm,bottom=3cm}

\usepackage[english]{babel}

%User defined commands
%---------------------
\newcommand{\mcal}{\mathcal}
\newcommand{\mfrak}{\mathfrak}
\newcommand{\mbb}{\mathbb}
\newcommand{\mrm}{\mathrm}
\newcommand{\pd}{\partial}
\newcommand{\sfr}{\mathfrak{s}}
\newcommand{\blankline}{\vspace{\baselineskip}}
\newcommand{\spc}{\ }
\newcommand{\red}{\color{red}}
\newcommand{\blu}{\color{blue}}

%User defined environments
%-------------------------


\title{In\"on\"u-Wigner contraction}
\author{}
\date{}

\begin{document}
\maketitle

\section{In\"on\"u-Wigner contraction of a Lie algebra}

Consider an $n$-dimensional Lie algebra $\mfrak{g}=(V,[\cdot,\cdot])$, where $V$ 
is the underlying vector space over some field $\mbb{F}$ and $[\cdot,\cdot]$ is 
the Lie bracket, which expressed in a particular basis $X_i$, $i=0\dots n$ is 
characterized by the structure constants $c_{ij}^{\spc\spc k}$,
\begin{equation}
  [X_i,X_j]=c_{ij}^{\spc\spc k} X_k
  \label{eq:Lie_br_X}
\end{equation}
Consider next the continuous function $U:(0,\varepsilon]\rightarrow GL(V)$.  
Hence $U(\varepsilon)$ is a non-singular linear operator on $V$, which 
transforms the basis $X_i$ into a new basis $Y_i$, with corresponding Lie 
bracket
\begin{equation}
  [Y_i,Y_j]=\tilde{c}_{ij}^{~~k} Y_k ~.
  \label{eq:Lie_br_Y}
\end{equation}
It is direct consequence of the transformation properties of the $Y_i = 
U(\varepsilon)_i^{~j}X_j$ that the structure constants transform as a 
$(1,2)$-tensor, i.e.
\begin{equation}
  \tilde{c}_{ij}^{~~k} = U(\varepsilon)_{i}^{~s} U(\varepsilon)_{j}^{~t}
  c_{st}^{~~r} U^{-1}(\varepsilon)_{r}^{~k}
\end{equation}
[QUESTION: The Lie bracket implies that $c_{ij}^{~ ~k} X_k$ is a vector, then 
how is this to be consistent with the transformation properties of $c$, as these 
seem to imply that the Lie bracket gives a (1,1)-tensor.]

It is clear that the Lie algebras defined by \eqref{eq:Lie_br_X} and 
\eqref{eq:Lie_br_Y} are isomorphic, since they are related by a non-singular 
linear transformation. However, if we consider a mapping $U(\varepsilon)$ which 
becomes singular for $\varepsilon$ going to zero, a new algebra might be the 
result. Since an algebra is charecterized completely by its structure constants, 
one has to investigate their behaviour in the limit of a singular 
transformation. More precisely, if the limit
\begin{equation}
  \lim_{\varepsilon \rightarrow 0} \tilde{c}_{ij}^{~ ~k}
  \label{eq:lim_contr}
\end{equation}
exists, the result is a new Lie algebra $\mfrak{g}_0=(V,[\cdot,\cdot]_0)$ which 
may or may not be isomorphic with the original algebra. This process of 
obtaining a new Lie algebra through the limiting procedure \eqref{eq:lim_contr} 
is referred to as \emph{contraction}. The dimension of the contracted algebra 
will be equal to the dimension of the original algebra, as we are assuming that 
the basis elements are well defined for the limiting singular transformation.

There are of course different possibilities to have $U(\varepsilon)$ singular 
when the parameter approaches zero. However, we will limit our interest to the 
case where the function is of the form
\begin{equation}
  U(\varepsilon) = \mrm{diag}(1,\dots,1,\varepsilon,\dots,\varepsilon)
\end{equation}
and the contracting procedure is called \emph{In\"on\"u-Wigner contraction}.

Now, the diagonal form of $U$ implies that the underlying vector space (the 
algebra) has the direct sum structure
\begin{equation}
  \mfrak{g} = \mfrak{h} \oplus \mfrak{p}
\end{equation}
where $\mfrak{h}$ is spanned by $X_\alpha$ and $\mfrak{p}$ is spanned by $X_a$, 
for which we evidently have
\begin{align*}
  U(\varepsilon)X_\alpha &= X_\alpha \\
  U(\varepsilon)X_a &= \varepsilon X_a
\end{align*}
The structure constants, on the other hand, transform as
\begin{equation}
  \tilde{c}_{ij}^{~ ~k} = \varepsilon^p c_{ij}^{~ ~k}
\end{equation}
where $p$ is the difference in covariant and contravariant Latin indexes of the 
structure constant tensor.\footnote{This number $p$ follows automatically from 
the diagonal structure of $U$ and the transformation properties of the structure 
constants.} In contracting the algebra, there will be a problem of convergence 
if $p=-1$, i.e.~if there are two covariant Greek indices and one contravariant 
Latin index. Hence, the contraction procedure will only give a well-defined 
structure constant tensor if for the original algebra $c_{\alpha\beta}^{~ ~ 
~c}=0$, which means that the commutation relations for $\mfrak{g}$ are given by
\begin{align}
  [\mfrak{h},\mfrak{h}] &\subseteq \mfrak{h} \\
  [\mfrak{h},\mfrak{p}] &\subseteq \mfrak{g} \\
  [\mfrak{p},\mfrak{p}] &\subseteq \mfrak{g}
\end{align}
where the first relation, necessary to have the limit well-defined, means that 
$\mfrak{h}$ is a subalgebra of $\mfrak{g}$. Let us now finish the contraction 
procedure and have a look at the resulting algebra $\mfrak{g}_0$. Since
\begin{equation}
  \begin{bmatrix}
    \mfrak{h}_0 \\
    \mfrak{p}_0
  \end{bmatrix} =
  \lim_{\varepsilon \rightarrow 0}
  \begin{bmatrix}
    1 & 0 \\
    0 & \varepsilon
  \end{bmatrix}
  \begin{bmatrix}
    \mfrak{h} \\
    \mfrak{p}
  \end{bmatrix}
\end{equation}
we find the algebra $\mfrak{g}_0 = \mfrak{h} \oplus_s \mfrak{p}_0$ with 
commutation relations
\begin{align}
  [\mfrak{h},\mfrak{h}] &\subseteq \mfrak{h} \\
  [\mfrak{h},\mfrak{p}_0] &\subseteq \mfrak{p}_0 \label{eq:bra_contr2} \\
  [\mfrak{p}_0,\mfrak{p}_0] &= 0
\end{align}
where we used that $\lim_{\varepsilon \rightarrow 0} \varepsilon \mfrak{g} = 
\mfrak{p}_0$ and $\lim_{\varepsilon \rightarrow 0} \varepsilon^2 \mfrak{g} = 0$.  
The algebra $\mfrak{g}$ has been contracted with respect to the subalgebra 
$\mfrak{h}$ into the algebra $\mfrak{g}_0$ which has the same subalgebra 
$\mfrak{h}$, whereas the complement $\mfrak{p}$ of $\mfrak{h}$ in $\mfrak{g}$ 
has become an Abelian ideal $\mfrak{p}_0$ of $\mfrak{g}_0$, such that the 
contracted algebra therefore always is non-semisimple. Since the contraction 
procedure replaces $\mfrak{p}$ by $\mfrak{p}_0$, the Lie bracket 
\eqref{eq:bra_contr2} is inherited from the original algebra (for the 
$\mfrak{p}_0$-terms).


\section{Example: from de Sitter to Poincar\'e}

In this section, the algebra $\mfrak{so}(1,4)$ will be contracted into the 
Poincar\'e algebra $\mfrak{iso}(1,3)$. Therefore, we recall the commutation 
relations for the generators $L_{ij}$, spanning $\mfrak{so}(1,4)$ $(i = 0 \ldots 
4)$
\begin{equation}
  [L_{ij},L_{kl}] = \eta_{il}L_{jk} + \eta_{jk}L_{il} - \eta_{ik}L_{jl} -
  \eta_{jl}L_{ik}
  \label{eq:CommRel_SO}
\end{equation}
This algebra can be contracted with respect to its subalgebra $\mfrak{so}(1,3)$.  
Hence we consider the direct sum structure $\mfrak{so}(1,4) = 
\mfrak{so}(1,3) \oplus \mfrak{p}$, which are spanned respectively by 
$L_{\alpha\beta}$ and $L_{\alpha 4}$ $(\alpha = 0 \ldots 3)$ and for which we 
rewrite the commutation relations \eqref{eq:CommRel_SO} explicitly as
\begin{equation}\label{eq:CommRel_SO_expl}
  \begin{split}
    [L_{\alpha\beta},L_{\kappa\lambda}] &= \eta_{\alpha\lambda}L_{\beta\kappa} + 
    \eta_{\beta\kappa}L_{\alpha\lambda} - \eta_{\alpha\kappa}L_{\beta\lambda} - 
    \eta_{\beta\lambda}L_{\alpha\kappa} \\
    [L_{\alpha 4},L_{\kappa\lambda}] &= \eta_{\alpha\lambda}L_{4\kappa} - 
    \eta_{\alpha\kappa}L_{4\lambda} \\
    [L_{\alpha 4},L_{\beta 4}] &= \eta_{44}L_{\beta\alpha}
  \end{split}
\end{equation}
Consider the following non-singular basis transformation
\begin{equation}
  L_{\alpha\beta}\rightarrow L_{\alpha\beta} \quad \mathrm{and} \quad
  L_{\alpha 4}\rightarrow \Pi_\alpha \equiv \varepsilon L_{\alpha 4}
\end{equation}

We now contract the algebra $\mfrak{so}(1,4)$ with respect to $\mfrak{so}(1,3)$.  
Defining $P_\alpha \equiv lim_{\varepsilon\rightarrow 0} \Pi_\alpha$ we find
\begin{equation}\label{eq:CommRel_P}
  \begin{split}
    [L_{\alpha\beta},L_{\kappa\lambda}] &= \eta_{\alpha\lambda}L_{\beta\kappa} + 
    \eta_{\beta\kappa}L_{\alpha\lambda} - \eta_{\alpha\kappa}L_{\beta\lambda} - 
    \eta_{\beta\lambda}L_{\alpha\kappa} \\
    [P_{\alpha},L_{\kappa\lambda}] &= \eta_{\alpha\kappa}P_\lambda - 
    \eta_{\alpha\lambda}P_\kappa \\
    [P_\alpha,P_\beta] &= 0
  \end{split}
\end{equation}
which are commutation relations, defining the Poincar\'e algebra which has the 
semidirect sum structure $\mfrak{iso}(1,3)=\mfrak{so}(1,3)\oplus_s\mfrak{t}$.
\blankline

The above discussion is equally valid for the case where we start from the anti 
de Sitter algebra $\mfrak{so}(2,3)$, since this algebra also has a Lorentz 
subalgebra. However, this would not be the case for $\mfrak{so}(0,5)$.

\section{From de Sitter to Poincar\'e algebra(s): stereographic coordinates}

In this section we will explicitly consider the IW-contraction of the de Sitter 
into the Poincar\'e algebra. More precisely, we will consider the realization of 
the de Sitter algebra as differential operators in stereographic coordinates, 
after which we will perform an IW-contraction. A first possibility, which gives 
well defined basis elements in the singular limit, is the contraction from de 
Sitter rotations into ordinary translations. Interestingly enough, there will be 
a second possibility for a consistent contraction into a different set of 
differential operators, namely the special conformal transformations.
\blankline

To begin with, consider the basis elements of $\mfrak{so}(1,4)$ realized as 
differential operators on function space, which are given by (in terms of the 
Cartesian coordinates labeling the embedding pseudo-Euclidean space)
\begin{align}
  L_{\mu\nu} &= \eta_{\mu\lambda} \chi^\lambda \pd_\nu - \eta_{\nu\lambda} 
  \chi^{\lambda} \pd_\mu \label{eq:gen_cart1} \\
  L_{\mu 4} &= \eta_{\mu\lambda} \chi^\lambda \pd_4-\eta_{44}\chi^4 \pd_\mu 
  \label{eq:gen_cart2}
\end{align}
and which of course solve the usual commutation relations.

Before contractng the algebra it will be useful to express the differential 
operators in terms of stereographic coordinates $x^\mu$ given by
\begin{equation}
  x^\mu = \Omega^{-1}\chi^\mu \quad \mrm{with} \quad \Omega = -\frac{1}{2} 
  \left( \frac{\chi^4}{l} - 1 \right)
\end{equation}
such that upon invoking the chain rule for partial differentiation, one finds
\begin{align}
  \frac{\pd}{\pd \chi^\mu} &= \Omega^{-1}(x)\frac{\pd}{\pd x^\mu} \\
  \frac{\pd}{\pd \chi^4} &= \frac{\Omega^{-1}(x)}{2l} x^\lambda
  \frac{\pd}{\pd x^\lambda}
\end{align}
Hence, substituting these expressions for \eqref{eq:gen_cart1} and 
\eqref{eq:gen_cart2}, the generators of the de Sitter group are found to be
\begin{align}
  L_{\mu\nu} &= \eta_{\mu\lambda} x^\lambda \pd_\nu - \eta_{\nu\lambda} 
  x^{\lambda} \pd_\mu \\
  L_{\mu 4} &= \sfr l \pd_\mu + \frac{1}{4l} \left(2\eta_{\mu\rho} x^\rho 
  x^\lambda - \sigma^2 \delta_\mu^\lambda \right) \pd_\lambda
\end{align}
Remember that the generators for ordinary translations $P_\mu$ and special 
conformal transformations $K_\mu$ are given by
\begin{equation}
  P_\mu = \frac{\pd}{\pd x^\mu} \quad \mrm{and} \quad
  K_\mu = \left(2\eta_{\mu\rho} x^\rho x^\lambda - \sigma^2 \delta_\mu^\lambda 
  \right) \frac{\pd}{\pd x^\lambda}
\end{equation}
such that we can express the generators of the de Sitter group as
\begin{align}
  L_{\mu\nu} &= \eta_{\mu\lambda} x^\lambda P_\nu - \eta_{\nu\lambda} 
  x^{\lambda} P_\mu \\
  L_{\mu 4} &= \sfr l P_\mu + \frac{1}{4l} K_\mu
\end{align}
which solve the Lie brackets \eqref{eq:CommRel_SO_expl}.

Note that the operators $L_{\mu\nu}$ are still readily interpreted as the 
generators for Lorentz rotations in stereographic coordinates, which leave the 
origin $x^\mu=0$ fixed, and hence span the subalgebra $\mfrak{so}(1,3)$. On the 
other hand, the generators $L_{\mu4}$ will move the origin and define the 
transitivity on de Sitter spacetime. In stereographic coordinates these 
operators are a combination of ordinary translations and special conformal 
transformations, although they essantially are rotations as this follows from 
their Lie brackets.
\blankline

Let us now consider the In\"onu-Wigner contraction of $\mathfrak{so}(1,4)$ with 
respect to $\mfrak{so}(1,3)$, i.e. we will consider an adequate non-singular 
transformation of the base element $L_{\alpha 4}$ and take a well-defined 
singular limit to end up with a new and non-equivalent algebra. As it turns out 
there are two possibilies.
%
\begin{description}
  \item[Poincar\'e algebra]
    Consider the transformation
    \begin{equation}
      L_{\mu4} \rightarrow \Pi_\mu \equiv \frac{L_{4 \mu}}{l} = -\sfr P_\mu - 
      \frac{1}{4l^2} K_\mu
    \end{equation}
    where the $\Pi_\mu$ are called \emph{de Sitter translations}.
    Now we can take the limit where the radius $l$ goes to infinity such that we 
    find the well-defined differential operators
    \begin{equation}
      \lim_{l\rightarrow \infty} \Pi_\mu = -\sfr P_\mu
    \end{equation}
    which together with the unchanged $L_{\mu\nu}$ and the commutation relations 
    \eqref{eq:CommRel_P} span the Poincar\'e algebra 
    $\mfrak{so}(1,3)\oplus_{s}\mfrak{t}$.
  %%
  \item[Conic algebra]
    As a second possibility transform the $L_{\mu4}$ according to
    \begin{equation}
      L_{\mu4} \rightarrow \kappa_\mu \equiv l L_{\alpha4} = \sfr l^2 P_\mu + 
      \frac{1}{4}K_\mu
    \end{equation}
    It is clear that again we can take a limit such that the transformation 
    becomes singular, now however letting the radius $l$ go to zero. Then one 
    finds the differential operators
    \begin{equation}
      \lim_{l\rightarrow 0} \kappa_\mu = \frac{1}{4}K_\mu
    \end{equation}
    which again with $L_{\mu\nu}$ and the Lie brackets \eqref{eq:CommRel_P} span 
    the algebra $\mfrak{so}(1,3)\oplus_{s}\mfrak{q}$.
\end{description}

It is interesting to see how two physically very different limits give rise to 
an identical algebraic structure. From the theoretical discussion on 
In\"on\"u-Wigner contractions however it is clear that these algebras should 
have the same commutation relations. The latter are fixed once the subalgebra is 
chosen to which the algebra is contracted, whilst the basis elements of the 
complement of the subalgebra are assumed merely to exist in the singular limit.  
Which possibilities for these basis elements are available---or also, which 
limits can be taken, after a suitable basis transformation---can be discussed 
only given a representation (realization). Nonetheless every choice will lead to 
the same Lie bracket.


\end{document}
