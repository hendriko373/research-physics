\documentclass[11pt]{article}

%Load preamble files
\usepackage{../../Tex_files/standard}
\usepackage{../../Tex_files/preamble_one}

\title{Reduction of principal bundles}
\author{Hendrik}
\date{\today}

\begin{document}

\maketitle

\begin{abstract}
	We review the reduction of principal bundles and the relation 
	between a Cartan and Ehresmann connection.
\end{abstract}

\section{Reduction of a principal bundle}

In this section the reduction process of a principal bundle is 
reviewed. Since our discussion is mostly based on the works 
\cite{husemoller:1966fibre,husemoller.etal:2008bbt,kob1996found}, 
we would like to refer the reader to these in case a more 
complete and certainly rigorous treatment of the subject is 
desired.

Let $H$ and $G$ be Lie groups and consider an injective group 
homomorphism
\begin{displaymath}
	i : H \to G~.
\end{displaymath}
It will be assumed that $i(H)$ is isomorphic to $H$, from which 
it follows that $H$ is a subgroup of $G$.  This inclusion of $H$ 
into $G$ is generally possible in different ways.  In other 
words, the short exact sequence
\begin{displaymath}
\begin{tikzcd}
	e \rar & H \rar{i} & G \rar & \dfrac{G}{i(H)} \rar & e
\end{tikzcd}
\end{displaymath}
is not canonically given. Of course, any differently chosen 
inclusion results in isomorphic subgroups of $G$.  Therefore it 
might seem overprecise to refer to the specific inclusion by 
using the notation $i(H)$, rather than just denoting any of them 
by the letter $H$. We, however, adhere to this precision, since 
these isomorphic subgroups $i(H)$ may have quite a different 
physical meaning.

Indeed, consider the homogeneous space $S$ that is symmetric 
under the left action of $G$ and for which the isotropy subgroup 
of any point is isomorphic with $H$. The isomorphism $S \simeq 
G/H$ becomes manifest when choosing an origin $o \in S$ so that 
an element $gH_o \in G/H_o$ is identified with $\tau_g(o) \in S$.  
By denoting $H_o = i(H)$ this establishes the isomorphism $S 
\simeq G/i(H)$. If another origin $\xi = \tau_a(o)$ is chosen, 
$S$ will be identified with $G/H_\xi$, where the isotropy group 
of $\xi$ is related to the isotropy group of $o$ through the 
adjoint action
\begin{displaymath}
	H_\xi = a H_o a^{-1} = \Ad(a)(H_o)~.
\end{displaymath}
In both cases the origin singles out a subgroup $i(H)$ in $G$ 
that are evidently isomorphic, nonetheless physically variant, 
being the isotropy subgroups of different points. It is also said 
that by prefering some point as the origin of $S$, the symmetry 
group $G$ is \emph{broken} to a subgroup $i(H)$.

\begin{proposition}\label{prop:corr_section_map}
	Let $Q(M,G)$ be a principal $G$-bundle and let $F$ be a left 
	$G$-space. There is a one-to-one correspondence between 
	sections of $Q[F] = Q \times_G F$ and maps $\varphi : Q \to F$ 
	that are $G$-equivariant, i.e.~they satisfy $R^\ast_g\varphi = 
	g^{-1}\cdot\varphi$.
\end{proposition}

\begin{remark}
A proof of this statement is given in 
\cite{husemoller:1966fibre}, Sec.~4.8 on pg.~46. The bijective
correspondence between such sections and $G$-equivariant mappings 
is as follows. In case $\varphi$ is a map that satisfies 
$\varphi(qg) = g^{-1}\cdot\varphi(q)$, a section $M \to Q 
\times_G F$ is given by $\sigma(\pi(q)) \equiv [q, \varphi(q)]$ 
for any $q \in Q$.  This is a well-defined construction because 
for any $g \in G$
\begin{displaymath}
	\sigma(\pi(qg)) = [qg,\varphi(qg)] = [q,\varphi(q)] = 
	\sigma(\pi(q))~.
\end{displaymath}
Conversely, let $\sigma$ be a section of $Q \times_G F$. There is 
a map $\varphi : Q \to F$ so that
\begin{displaymath}
	\sigma(\pi(q)) = [q,\varphi(q)] = [qg,g^{-1}\cdot\varphi(q)]~.
\end{displaymath}
Since $\sigma(\pi(q)) = \sigma(\pi(qg)) = [qg,\varphi(qg)]$ for 
any $g \in G$, it follows that $\varphi$ is $G$-equivariant.
\end{remark}

Denote by $Q/i(H)$ the space of equivalence classes with respect 
to the right action of $i(H) \subset G$ on $Q$. This quotient 
space can be identified with the associated $G$-bundle $Q[S] = Q 
\times_G S$, where $S$ is the homogeneous space $G/i(H)$. More 
precisely, the correspondence is governed by the 
map\footnote{This map is an isomorphism between fibre bundles; 
	see e.g.~\cite{husemoller:1966fibre}, Sec.~6.1 on pg.~70.}
\begin{displaymath}
	Q/i(H) \to Q[S] : [q] \mapsto [q,\xi_a] = [q,a\xi_o]~,
\end{displaymath}
where $\xi_a$ is the origin of $G/i(H)$, hence $i(H) = H_a = 
aH_oa^{-1}$.  

\begin{definition}
	Let $Q(M,G)$ and $P(M,H)$ be a principal $G$-~and $H$-bundle, 
	respectivily, and for which $i(H)$ is a closed subgroup of 
	$G$. Let $\imath : P \to Q$ be an injection so that 
	$\imath(ph) = \imath(p)i(h)$ for each $p \in P$ and $h \in H$.  
	Then $Q$ is an extension of $P$ and $P$ is a restriction of 
	$Q$. The group $G$ is said to be reduced to the group $H 
	\simeq i(H)$.
\end{definition}

Given a principal $H$-bundle $P$ with $H$ being a subgroup of 
$G$, it is always possible to extend to a principal $G$-bundle.  
Since there is a natural left action of $H$ on $G$, one also has 
the associated bundle $P[G] = P \times_H G$, whose elements are 
the equivalence classes
\begin{displaymath}
	[p,g] = \{(ph,i(h)g)~|~h \in H \}~.
\end{displaymath}
There is a natural right $G$-action on this bundle, given by 
$[p,g]g' = [p,gg']$. Because the left and right actions commute, 
this is a well-defined principal $G$-bundle. We will write 
$Q(M,G) = P \times_{i(H)} G$. The extension of $Q$ is then given 
by the natural injection
\begin{equation}
	\imath : P \to Q : p \mapsto [p,e]~.
\end{equation}

On the other hand, it is not always possible to reduce a 
principal $G$-bundle to a principal $H$-bundle. This is the 
subject of discussion in the following proposition.

\begin{proposition}
	A principal $G$-bundle $Q$ is reducible to a principal 
	$H$-bundle $P$ if and only if the associated bundle $Q[S] = Q 
	\times_G S$, with $S \simeq G/H$, admits a globally defined 
	section.
\end{proposition}
\begin{proof}
	Let $\imath : P \to Q$ be a reduction. The composition 
	$\tilde{\sigma} \equiv \mu \circ \imath$, where $\mu : Q \to 
	Q/i(H) \simeq Q[S]$ is the natural projection $q \mapsto [q] = 
	[q,a\xi_o]$, is constant on the fibres of $P$;
	\begin{displaymath}
		\tilde{\sigma}(ph)  = \mu(\imath(p)i(h)) = 
		\tilde{\sigma}(p)~,\quad h \in H~.
	\end{displaymath}
	Hence, $\tilde{\sigma}$ defines a section $M \to Q \times_G S$ 
	by $\sigma(x) = \tilde{\sigma}(p)$ for any $p \in 
	\pi^{-1}(x)$, since $\pi \circ \sigma = \mathrm{id}_M$.

	Conversely, let $\sigma$ be a section of $Q[S]$. From 
	Proposition~\ref{prop:corr_section_map} it follows that there 
	is a corresponding $G$-equivariant map $\varphi : Q \to S$.  
	Let $\imath(P) \equiv \varphi^{-1}(\xi_a)$, where at each $x 
	\in M$ we have that $i(H)(\xi_a) = \xi_a$. Consider the 
	restriction of $\pi : Q \to M$ to $\imath(P)$ and let 
	$\imath(p_1)$ and $\imath(p_2)$ be two elements in $\imath(P)$ 
	for which
	\begin{displaymath}
		\pi|_{\imath(P)}(\imath(p_1)) = 
		\pi|_{\imath(P)}(\imath(p_2))~.
	\end{displaymath}
	There exists an element $g \in G$ for which $\imath(p_1) = 
	\imath(p_2)g$, so that
	\begin{displaymath}
		\xi_a = \varphi(\imath(p_1)) = g^{-1}\varphi(\imath(p_2)) = 
		g^{-1}\xi_a~.
	\end{displaymath}
	Hence, $g$ must be an element of $i(H)$ and $\imath : P \to Q$ 
	is a reduction from $G$ to $i(H)$.
\end{proof}

To conclude this section, let us summarize the reduction process 
in the following diagram.
\begin{displaymath}
\begin{tikzcd}[column sep=2cm]
	{}			&	& S \simeq G/H									\\
	P(M,H)	\arrow{r}[color=red]{\imath}
				\arrow[end anchor=north west]{ddr}
				&  Q(M,G)
					\arrow[start anchor=north east]{ur}{\varphi}
					\arrow[start anchor=south east]{dr}{\mu}
					\arrow{dd}										\\
				&	& Q \times_G S
						\arrow[end anchor=north east]{dl}	\\
				& M \arrow[dashed,bend right,
						start anchor={[xshift=+2pt]real east},
						end anchor={[xshift=+10pt]south west}
						]{ur}[swap,color=red]{\sigma}
\end{tikzcd}
\end{displaymath}

\section{Induced Cartan connection}

Given a reduction $\imath : P \to Q$ one can wonder how Ehresmann 
connections on $Q$ are related to Cartan connections on the 
reduced bundle $P$. In the following proposition we explain how 
an Ehresmann connection on $Q$ may be interpreted as a Cartan 
connection on $P$, for a certain subclass of reductions $\imath$ 
\cite{sharpe1997diff_geo}. It will be assumed that the dimension 
of $P$ equals the dimension of $G$.

\begin{proposition}
	Let $\gamma \in \Omega(Q,\mathfrak{g})$ be an Ehresmann 
	connection on $Q$. If
	\begin{displaymath}
		\mathrm{ker}~\gamma \cap \imath_\ast(TP) = 0~,
	\end{displaymath}
	then
	\begin{equation}
		\kappa \equiv \imath^\ast\gamma : TP \to \mathfrak{g}
	\end{equation}
	is a Cartan connection on $P$.
\end{proposition}
\begin{proof}
	Because $\mathrm{ker}~\gamma \cap \imath_\ast(TP) = 0$, 
	$\imath^\ast\gamma$ is a $\mathfrak{g}$-valued one-form on $P$ 
	that has no kernel. We verify the three defining properties of 
	a Cartan connection for $\kappa$:
	\begin{itemize}
		\item[(i)] Since $\dim P = \dim G$ and because $\imath$ is an 
			injection, $\imath^\ast \gamma$ is an isomorphism.
		\item[(ii)] Let $\zeta_X$ be the fundamental vector field on 
			$P$ corresponding to $X \in \mathfrak{h}$, i.e.~$\zeta_X f 
			= \dot{f}(p\exp(tX))$ for $f$ a function on $P$. It follows 
			that $\imath_\ast \zeta_X f = \dot{f}(\imath(p) 
			i(\exp(tX)))$, which is a fundamental vector field on $Q$ 
			corresponding to $i(X) \in i(\mathfrak{h}) \simeq 
			\mathfrak{h}$.  It follows that $\kappa(\zeta_X) = i(X)$ 
			for any $X \in \mathfrak{h}$.
		\item[(iii)] Since for any $p \in P$
			\begin{displaymath}
				\imath \circ R_h(p) = [ph,e] = [p,i(h)] = R_{i(h)} \circ 
				\imath(p)~,
			\end{displaymath}
			it follows that
			\begin{displaymath}
				R^\ast_h \imath^\ast \gamma =	\imath^\ast R^\ast_{i(h)} 
				\gamma = \imath^\ast (\Ad(i(h^{-1})) \cdot \gamma) = 
				\Ad(i(h^{-1})) \cdot \imath^\ast \gamma~.
			\end{displaymath}
			One concludes that $R^\ast_h \kappa = \Ad(i(h^{-1}))\cdot 
			\kappa$.
	\end{itemize}
	The proof is completed when identifying $H$ with $i(H)$.
\end{proof}




\bibliographystyle{plain}
\bibliography{../../references/All.bib}
\end{document}

