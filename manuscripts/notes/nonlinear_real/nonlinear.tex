\documentclass[11pt]{article}

%Load preamble files
\usepackage{../../Tex_files/standard}
\usepackage{../../Tex_files/preamble_one}
%\usepackage{showframe} %show frame borders
\usepackage{mathtools}

\usepackage[all]{xy}

\title{Nonlinear de Sitter-Cartan geometry}
\author{Hendrik}
\date{\today}

\begin{document}

\maketitle
\tableofcontents

\section{Motivation}
{\blu \it [Should contain main motivation for using nonlinear 
	realization of Cartan connection: we want reductive splitting 
	(gravity) + local de Sitter invariance]}


The geometric structure encoded in the above {\blu\it (SEC XXX)} 
constructed de Sitter-Cartan connection is useful, since it 
describes a generic spacetime with non-vanishing curvature and 
torsion that locally reduces to de Sitter spaces. In particular 
did the reductive splitting of the de Sitter-Cartan connection 
result in a spin connection and vierbein, mandatory for any 
theory of gravity.  Unfortunately is the situation at this point 
not yet satisfactory. The shortcoming of the given setting is 
that it incorporates only local Lorentz invariance. This follows 
directly from the definition, where the $\mathfrak{g}$-valued de 
Sitter-Cartan connection lives on a principal $H$-bundle.  
Intuitively this can be understood by reconsidering the reduction 
of an Ehresmann $G$-connection on $Q$ to a corresponding Cartan 
connection on $P$. The reduction process consists in singling out 
a section $\xi$ of the associated $G$-bundle of homogeneous de 
Sitter spaces $dS$, namely
\begin{displaymath}
	\xi : M \to Q[dS] : x \mapsto \xi(x)~.
\end{displaymath}
Upon picking out such a global section one breaks the symmetry of 
the de Sitter spaces at any point in spacetime $M$ to the 
isotropy group of the singled out section over these points, 
i.e.~$H_\xi$.

Local Lorentz invariance is necessary but not good enough, for 
the reason that we wish to construct theories that have the 
larger de Sitter group as local symmetries. At first sight we 
arrived at a seeming dead end: fixing a section $\xi$ implied the 
existence of the required spin connection and vierbein but at the 
same time necessarily broke the de Sitter group down to its 
Lorentz subgroup. We would like to retain the existence of a spin 
connection and vierbein but recover the full symmetry group of de 
Sitter rotations. The way out of the apparent impasse is by 
noting that the broken symmetries upon reduction are merely 
hidden. This is so because the section has been chosen completely 
\emph{arbitrary}, so that any section is as good as any other.  
Different but equivalent sections are related by elements in 
$\Omega^0(M,G)$ that are not in $\Omega^0(M,H_\xi)$, so that it 
is readily seen that those local de Sitter transformations that
relate different section are just the broken symmetries.

These transformations can be incorporated in the principal 
Lorentz bundle by nonlinearly realizing them as local Lorentz 
transformations. In this manner, local de Sitter invariance is 
restored while retaining a well-defined reductive splitting of 
the Cartan connection, i.e.~the presence of a spin connection and 
vierbein. Nonlinearly realizing the broken symmetries yields a 
nonlinear de Sitter-Cartan geometry.  Why such a nonlinear Cartan 
geometry is preferred over the geometry in the original 
$SO(1,4)$-bundle will become clear later on in the text. As 
mentioned in this introduction, the crucial reason is that 
nonlinearly realizing the Cartan connection reconciles the 
presence of local de Sitter invariance together with the 
continuance of a well-defined spin connection and vierbein, as 
well as a curvature and torsion.



\section{Introducing nonlinear realizations}

The theory of nonlinear realizations of Lie groups on homogeneous 
manifolds was introduced in the context of spontaneous symmetry 
breaking, see \cite{Coleman:1969sm,Callan:1969sn,Volkov:1973vd}.  
A succinct review on the matter is also exposed in the appendices 
of \cite{wess:1992ss}. In the following subsections we base our 
discussion on these works to introduce the subject of nonlinear 
realization.

{\blu\it Expand this review of literature}


\subsection{Reductive Lie algebra}

In the following let $G$ be a Lie group of dimension $n$ and 
denote by $H$ a $m$-dimensional closed subgroup. It is assumed 
that there is a reductive splitting on the level of the Lie 
algebras, i.e.~$\mathfrak{g} = \mathfrak{h} \oplus \mathfrak{p}$ 
so that $[\mathfrak{h},\mathfrak{h}] \subset \mathfrak{h}$ and 
$[\mathfrak{h},\mathfrak{p}] \subset \mathfrak{p}$.  Let $S$ be a 
homogeneous space $S$ that is symmetric under the left action of 
$G$,
\begin{equation}
	\tau_g : S \to S : p \mapsto gp~,
\end{equation}
and for which the isotropy group of a given point $o$ is given by 
$H_o \simeq H$. Then there is an isomorphism between $S$ and 
$G/H_o$, which comes from identifying $gH_o \in G/H_o$ with 
$\tau_g(o) \in S$. Let the elements $P_a$ form a basis for the 
($d = n - m$)-dimensional vector space 
$\mathfrak{p}$.\footnote{The index $a$ runs from $0$ to $d-1$.}
In some neighborhood of the identity, a group element of $G$ can 
be decomposed uniquely in the form\footnote{This is so if a set 
	of generators for $H$ and the $P_a$ are chosen orthonormal 
	with respect to the Cartan inner product.}
\begin{displaymath}
	g = \exp(i\xi\cdot P) \tilde{h}~,
\end{displaymath}
where $\tilde{h}$ is an element of the stability subgroup $H$ and 
$\xi\cdot P = \xi^a P_a$.  The $\xi^a$ parametrize the coset 
space $G/H$---at least in some neigborhood of the identity---so 
that it is sensible to interpret them constituting a coordinate 
system on the homogeneous space $S$. Next let $g_0$ be some 
element in $G$.  Because a Lie group is naturally closed under 
its own action, the left action of $G$ on itself may equally be 
written as
\begin{displaymath}
	g_0 g = \exp(i\xi'\cdot P)\tilde{h}'
\end{displaymath}
or
\begin{equation}\label{eq:left_action_group}
	g_0 \exp(i\xi\cdot P) = \exp(i\xi'\cdot P)h'~;
	\qquad
	h' := \tilde{h}'\tilde{h}^{-1}~,
\end{equation}
where $\xi'=\xi'(g_0,\xi)$ and $h'=h'(g_0,\xi)$ depend on the 
indicated variables. In case $\lambda$ is a linear representation 
of the subgroup $H$, as in
\begin{displaymath}
	h: \psi \mapsto \lambda(h)\psi~,
\end{displaymath}
a realization of $G$ is constructed through the joint 
transformation law
\begin{equation}\label{eq:nonlin_real}
	g_0 :~\xi \mapsto \xi'~,~\psi \mapsto \lambda(h')\psi~.
\end{equation}
That this gives way to a realization of $G$ can be seen from the 
consecutive action of two elements $g_0$ and $g_1$. The 
transformation of $\xi$ and $h$ is given implicitely 
by~\eqref{eq:left_action_group}, namely
%
\begin{displaymath}
	g_0\exp(i\xi\cdot P) = \exp(i\xi'\cdot P) h' 
	\quad\text{and}\quad
	g_1\exp(i\xi'\cdot P) = \exp(i\xi''\cdot P) h''~.
\end{displaymath}
These transformations are then composed to yield 
$g_1g_0\exp(i\xi\cdot P) = \exp(i\xi''\cdot P)h''h'$. But because 
$g_1 \circ g_0$ also lies in $G$ one resultantly has
\begin{displaymath}
	(g_1g_0)\exp(i\xi\cdot P) = \exp(i\xi'''\cdot P) h'''~.
\end{displaymath}
Combining these considerations readily leads us to conclude that
$h''' = h''h'$ and consequently that 
%
\begin{displaymath}
	\lambda(h''') = \lambda(h'')\lambda(h')~,
\end{displaymath}
because $\lambda$ is a representation of $H$. Moreover has it 
become clear that $\xi''' = \xi''$, from which we deduce the 
equality $(g_1g_0)\xi = g_1(g_0\xi)$. This proves that the 
transformation of $G$ on $\xi$ is a group realization as well.  
Remark that the composition $h''h'$ depends on the transformation 
of $\xi$ so that the realization \eqref{eq:nonlin_real} only is 
meaningful together with the transformation properties of $\xi$.  
Accordingly does the transformation of $\psi$ depends on $\xi$.  
For these reason is the realization~\eqref{eq:nonlin_real} called 
\emph{nonlinear}. In section~\ref{ssec:lin_to_nonlin} we will 
show how any representation of $G$ can be nonlinearly realized in 
the form of~\eqref{eq:nonlin_real}.

Let us for a moment consider the subcase for which the left 
action is given by an element of the isotropy group $H$, say 
$h_0$. The general transformation 
prescription~\eqref{eq:left_action_group} is rewritten in a 
trivial way in the form
%
\begin{displaymath}
	h_0 \exp(i\xi\cdot P) h_0^{-1}h_0 = \exp(i\xi'\cdot P) h'~.
\end{displaymath}
It is a well-known result for Lie groups that the exponential map 
commutes with the adjoint action, by which it is meant that 
\begin{displaymath}
	h_0 \exp(i\xi\cdot P) h_0^{-1} = \exp( \Ad(h_0)(i\xi\cdot 
	P))~.
\end{displaymath}
Owing to the reductive nature of the Lie algebra the adjoint 
action leaves $\mathfrak{p}$ invariant so that
\begin{displaymath}
	\exp(i\xi'\cdot P) = h_0\exp(i\xi\cdot P)h_0^{-1}\quad
	\text{and}\quad
	h' = h_0~,
\end{displaymath}
and where the transformation of $\xi$ is explicitely given by
\begin{displaymath}
	h_{0}: i\xi\cdot P \mapsto i\xi\cdot \mathrm{Ad}(h_0)P =: 
	i\xi'\cdot P~,
\end{displaymath}
The adjoint action being a linear automorphism, one easily 
understands that the coset parameters transform linearly. In 
addition to this we remark that $h'$ does not depend on these 
parameters.  Upon restriction to the subgroup $H$, the 
realization~\eqref{eq:nonlin_real} consequently reduces to a 
linear representation.

There is another subclass of transformations that will be of much 
interest to us in the following subsections. These are the pure 
de Sitter translations, namely elements that are of the form $g_0 
= \exp(i\alpha\cdot P)$.  Acted upon by such translations, the 
coordinates $\xi$ change according to
\begin{equation}\label{eq:left_action_coset}
	\exp(i\alpha\cdot P)\exp(i\xi\cdot P) = \exp(i\xi'\cdot P)h'~.
\end{equation}

\blankline\noindent
To conclude this review of nonlinear realizations on reductive 
Klein geometries, we take a look at the case for which the 
transformation considered in Eq.~\eqref{eq:left_action_group} is 
infinitesimal. Thence let $g_0$ be an element in $G$ that lies 
infinitesimally close to the identity, so that $g_0 = e + \delta 
g_0$ and $\delta g_0 \in \mathfrak{g}$. The coset coordinates 
$\xi$ transform into
\begin{displaymath}
	\xi'(g_0) = \xi'(e) + \pd_g \xi'(g)|_e \delta g_0 + 
	\mathcal{O}((\delta g_0)^2) \simeq \xi + \delta \xi~.
\end{displaymath}
Similarly, it is found up to first order in $\delta g_0$ that
\begin{displaymath}
	\exp(i\xi'\cdot P) = \exp(i\xi\cdot P) + \delta\exp(i\xi\cdot 
	P)~,
	\quad
	h' = (\tilde{h} + \delta \tilde{h})\tilde{h}^{-1} = e + \delta 
	h~.
\end{displaymath}
Note that the variation on the exponential comes from the 
variation $\delta \xi$. Substituting these expansions into 
\eqref{eq:left_action_group} and retaining terms up to first 
order, one finds the equation that determines the infinitesimal 
variations $\delta \xi$ and $\delta h$: 
%
\begin{displaymath}
	\exp(-i\xi\cdot P) \delta g_0 \exp(i\xi\cdot P) - 
	\exp(-i\xi\cdot P) \delta\exp(i\xi\cdot P) = \delta h~.
\end{displaymath}

If the elements are pure translations, hence of the form $g_0 = e 
+ i\epsilon\cdot P$, the transformation parameters satisfy the 
equation
\begin{equation}\label{eq:nonlin_trafo_inf}
	\exp(-i\xi\cdot P) i\epsilon\cdot P \exp(i\xi\cdot P) - 
	\exp(-i\xi\cdot P) \delta\exp(i\xi\cdot P) = \delta h~.
\end{equation}


\subsection{Symmetric Lie algebra}

In this subsection it will be assumed that the Lie algebra 
$\mathfrak{g}$ is not only reductive but also symmetric. This 
means that there is an involutive automorphism $\sigma : 
\mathfrak{g} \to \mathfrak{g}$ such that $\mathfrak{h}$ is an 
eigenspace with eigenvalue $1$, while $\mathfrak{p}$ is an 
eigenspace with eigenvalue $-1$. Group elements of $H$ that are 
obtained by exponentiation of elements in $\mathfrak{h}$ are 
invariant under $\sigma$, while elements generated by elements of 
$\mathfrak{p}$ are mapped into their inverse. The automorphism 
directly implies a third restriction on the commutation relations 
of $\mathfrak{g}$, which is $[\mathfrak{p},\mathfrak{p}] \subset 
\mathfrak{h}$. Then Eq.~\eqref{eq:left_action_group} implies
\begin{displaymath}
	\sigma(g_0) \exp(-i\xi\cdot P) = \exp(-i\xi'\cdot 
	P)\sigma(h')~,
\end{displaymath}
after which $h' = \sigma(h')$ can be eliminated from 
Eq.~\eqref{eq:left_action_group}. One obtains in consequence
%
\begin{equation}
	g_0 \exp(2i\xi\cdot P) \sigma(g_0^{-1}) = \exp(2i\xi'\cdot 
	P)~.
\end{equation}
Written this way, it is manifest that $g_0 : \xi \mapsto \xi'$ is 
a group realization and, when restricted to $H$, this realization 
becomes a linear representation.

\blankline\noindent
Then let us concentrate on the variation of the coset parameters 
caused by infinitisimal transvections, that is to say elements of 
the form $g_0 = e + i\epsilon\cdot P$. Such a variation 
$\delta\xi$ is a solution to Eq.~\eqref{eq:nonlin_trafo_inf}. By 
use of the involutive automorphism one eliminates $\delta h$, so 
that
\begin{displaymath}
\begin{multlined}[b][\linewidth-\multlinegap]
	\quad
	\exp(-i\xi\cdot P)\delta\exp(i\xi\cdot P) - \exp(i\xi\cdot 
	P)\delta\exp(-i\xi\cdot P) \\
	= \exp(-i\xi\cdot P) i\epsilon\cdot P \exp(i\xi\cdot P) + 
	\exp(i\xi\cdot P) i\epsilon\cdot P \exp(-i\xi\cdot P)~.
	\quad
\end{multlined}
\end{displaymath}
Using eqs.~\eqref{eq:Had} and~\eqref{eq:CP}, this is rewritten as 
\begin{displaymath}
\begin{multlined}[b][\linewidth-2\multlinegap]
	\quad\quad
	\frac{1-\exp(-i\xi\cdot P)}{i\xi\cdot P} \wedge 
	i\delta\xi\cdot P - \frac{1-\exp(i\xi\cdot P)}{i\xi\cdot P} 
	\wedge i\delta\xi\cdot P \\
	= \exp(-i\xi\cdot P) \wedge i\epsilon\cdot P + \exp(i\xi\cdot 
	P) \wedge i\epsilon\cdot P~.
	\quad\quad
\end{multlined}
\end{displaymath}
The expression can be solved for $i\delta\xi\cdot P$, leading to
%
\begin{equation}\label{eq:inftrafo_cosetpar}
	i\delta\xi \cdot P = \frac{i\xi\cdot P\,\cosh(i\xi\cdot 
		P)}{\sinh(i\xi\cdot P)} \wedge i\epsilon\cdot P~.
\end{equation}
This result gives the variation of coset parameters due to an 
infinitesimal pure translation $i\epsilon\cdot P$. Remember that 
this is only valid for symmetric Klein geometries. To explicitly 
solve for $\delta \xi ^a$, one needs the specific commutation 
relations of the underlying Lie algebra. In the next section such 
a calculation will be worked for the de Sitter space.

One is equally able to find an expression that relates the 
infinitesimal element $h'(\xi,\epsilon) = e + \delta h$ to the
corresponding transvection $g_0 = e + i\epsilon\cdot P$. The 
element $\delta h = \tfrac{i}{2} \delta h \cdot M \in 
\mathfrak{h}$ is given by Eq.~\eqref{eq:nonlin_trafo_inf}.  
Invoking the identities~\eqref{eq:Had} and~\eqref{eq:CP} one 
obtains\footnote{This result is not only true for symmetric Lie 
	algebras, but for reductive algebras also.}
\begin{equation}
	\tfrac{i}{2} \delta h \cdot M = \exp(-i\xi\cdot P) \wedge 
	i\epsilon\cdot P - \frac{1-\exp(-i\xi\cdot P)}{i\xi\cdot P} 
	\wedge i\delta\xi \cdot P~.
\end{equation}
For symmetric Klein geometries we have already found an 
expression for $\delta\xi$. Let us substitute 
for~\eqref{eq:inftrafo_cosetpar}, after which one gets 
sought-after relation
\begin{equation}\label{eq:inftrafo_h}
	\tfrac{i}{2}\delta h \cdot M = \frac{1-\cosh(i\xi\cdot 
		P)}{\sinh(i\xi\cdot P)} \wedge i\epsilon\cdot P~.
\end{equation}
Upon use of an explicit set of cummutation relations, this 
equation renders the nonlinear realization of an infinitesimal 
transvection.


\subsection{Construction of nonlinear realizations}
\label{ssec:lin_to_nonlin}

In conclusion of this section on nonlinear realizations, the 
construction of such a realization out of a generic linear 
representation of $G$ is established.
Therefore let $V$ be a representation space of $G$ so that the 
action of $G$ is linear and given by
\begin{displaymath}
	\sigma(g) : V \to V~,\quad \forall g  \in G~.
\end{displaymath}
Then consider a section of the associated bundle of $V$-spaces, 
namely
\begin{displaymath}
	\psi : M \to Q[V] = Q \times_G V~.
\end{displaymath}
At any given point $x \in M$ the field $\psi$ transform according 
to the representation $\sigma$,
\begin{displaymath}
	g : \psi(x) \mapsto \psi'(x) = \sigma(g) \psi(x)~.
\end{displaymath}
Next let $\xi$ be a section of the associated bundle of 
homogeneous spaces $Q[G/H] = Q \times_G G/H$. The nonlinear 
realization of $\psi$ is pointwise defined as
\begin{equation}\label{eq:def_nonlinear_field}
	\bar{\psi}(x) \equiv \sigma(\exp(-i\xi\cdot P))\psi(x)~.
\end{equation}
That the field $\bar{\psi}$ indeed transforms nonlinearly in the 
sense of Eq.~\eqref{eq:nonlin_real} \emph{and only with respect 
	to its $H$-indices} under the action of a generic element 
$g_0$ of $G$, is verified as follows:
%
\begin{displaymath}
\begin{split}
	\bar{\psi}'(x) &= \sigma(\exp(-i\xi'\cdot P)) \psi'(x) \\
	&= \sigma(\exp(-i\xi'\cdot P)g_0) \psi(x) \\
	&= \sigma(\exp(-i\xi'\cdot P)g_0\exp(i\xi\cdot P)) 
	\bar{\psi}(x) \\
	&= \sigma(h'(\xi,g_0)) \bar{\psi}(x)~.
\end{split}
\end{displaymath}
It follows that a linear irreducible representation of $G$ 
becomes a nonlinear and reducible representation. The price to be 
paid for getting irreducible $H$-representations is that they 
transform in a nonlinear way. Nonetheless, when restricted to the 
isotropy group $H$, the field~\eqref{eq:def_nonlinear_field} 
transforms according to a linear representation.

\section{An example: de Sitter space}

\subsection{Transformation of group parameters}

{\blu\it This subsection has to be reviewed and updated.}

In this subsection, the change of the group parameters $\xi^a$ 
and $\delta h^{ab}$ due to infinitesimal de Sitter translations 
are calculated. Remember that the coordinates $\xi^a$ are defined 
by the exponentiation of elements of $\mathfrak{p}$. They are 
also referred to as Goldstone fields, because of the resemblance 
of their role in the scheme of spontaneous symmetry breaking in 
field theory. As we have reviewed in the last section, these 
coordinates transform according to a nonlinear realization of the 
full symmetry group $G$. On the other hand, they transform 
linearly when the action is restricted to the subgroup $H$ of 
unbroken symmetries.  One understands that the pure translations 
are the set of transformations that are responsible for the 
nonlinear behaviour.\footnote{In general these elements do not 
	form a group.}

Let us begin by recalling the de Sitter commutation relations 
that involve translations, i.e.\footnote{An element of $SO(4,1)$ 
	is given by $\exp(\tfrac{i}{2}\omega^{ab}M_{ab} + i\xi^a 
	P_a)$.}
%
\begin{equation}\label{eq:comm_rels_dS}
	\begin{split}
		-i[M_{ab},P_c] &= \eta_{ac}P_b - \eta_{bc}P_a \\
		-i[P_a,P_b] &= \mathfrak{s}l^{-2} M_{ab}
	\end{split}
\end{equation}
with $\mathfrak{s} \equiv \eta_{44}$. The de Sitter translations 
were introduced as $P_a \equiv l^{-1}(x)M_{a4}$, whilst the 
$M_{ab}$ span the Lorentz subalgebra $\mathfrak{h} = 
\mathfrak{so}(3,1)$.  It is manifest that the de Sitter algebra 
is symmetric.  In what follows we adhere to the convention 
$\mathfrak{s} = -1$ so that $\eta_{ab} = 
\mathrm{diag}(1,-1,-1,-1)$.

The transformation of the coset parameters~$\xi^a$ under an 
infinitesimal de Sitter translation $\epsilon\cdot P$ is given by 
\eqref{eq:inftrafo_cosetpar}. In the parametrization used in this 
section, this can be rewritten as
%
\begin{equation}\label{eq:inftrafo_cosetpar_dS}
	\delta\xi \cdot P = \frac{i\xi\cdot P\,\cosh(i\xi\cdot 
		P)}{\sinh(i\xi\cdot P)} \wedge \epsilon\cdot P~.
\end{equation}
Recall that the left hand side should be understood as a
power series in the adjoint action (see also Appendix 
\ref{app:nested_comm.not}).  The power series of the relevant 
hyperbolic functions have the form\footnote{The coefficients in 
	the power series for the hyperbolic cosecant are $c_{2n} = 
	2(1-2^{2n-1}) B_{2n}$ with $B_n$ the $n$-th Bernouilli 
	number.}
%
\begin{align*}
	\cosh(i\xi\cdot P) &= \sum_{n=0}^\infty \frac{(i\xi\cdot 
		P)^{2n}}{(2n)!}~, \\
	\mathrm{csch}(i\xi\cdot P) &= (i\xi\cdot P)^{-1} + 
	\sum_{n=1}^\infty \frac{c_{2n}}{(2n)!} (i\xi\cdot P)^{2n-1}~.
\end{align*}
%
Invoking the identity~\eqref{eq:id_dS_comm.1}, one is able to 
work out the cosinus hyperbolicus, i.e.~ 
%
\begin{equation}\label{eq:cosh_transl_dS}
\begin{split}
	\cosh(i\xi\cdot P) \wedge \epsilon\cdot P
	&= \epsilon\cdot P + \sum_{n=1}^\infty 
	\frac{(l^{-1}\xi)^{2n}}{(2n)!} \wedge \left(\epsilon\cdot P - 
		\frac{\xi\cdot\epsilon \xi\cdot P}{\xi^2} \right) \\
	&= \cosh(l^{-1}\xi) \left(\epsilon\cdot P - 
		\frac{\xi\cdot\epsilon \xi\cdot P}{\xi^2} \right) + 
	\frac{\xi\cdot\epsilon \xi\cdot P}{\xi^2}~.
\end{split}
\end{equation}
By equal means, the right hand side 
of~\eqref{eq:inftrafo_cosetpar_dS} is found to be
\begin{displaymath}
\begin{split}
	i\xi\cdot P \,\mathrm{csch}(i\xi\cdot P) &\wedge 
	\cosh(i\xi\cdot P) \wedge \epsilon\cdot P \\
	=& \bigg( \mathds{1} + \sum_{n=1}^\infty \frac{c_{2n}}{(2n)!} 
	(i\xi\cdot P)^{2n} \bigg)\wedge \bigg[\cosh(l^{-1}\xi)
	\bigg(\epsilon\cdot P -\frac{\xi\cdot\epsilon \xi\cdot 
		P}{\xi^2}\bigg) + \frac{\xi\cdot\epsilon \xi\cdot P}{\xi^2} 
	\bigg] \\
	=& \cosh(l^{-1}\xi)\bigg(1 + \sum_{n=1}^\infty 
	\frac{c_{2n}}{(2n)!}(l^{-1}\xi)^{2n} \bigg) 
	\bigg(\epsilon\cdot P -\frac{\xi\cdot\epsilon \xi\cdot 
		P}{\xi^2}\bigg) + \frac{\xi\cdot\epsilon \xi\cdot P}{\xi^2} 
	\\
	=& \cosh(l^{-1}\xi)(l^{-1}\xi)\,\mathrm{csch}(l^{-1}\xi) 
	\bigg(\epsilon\cdot P -\frac{\xi\cdot\epsilon \xi\cdot 
		P}{\xi^2}\bigg) + \frac{\xi\cdot\epsilon \xi\cdot P}{\xi^2} 
	\\
	=& \epsilon\cdot P + \frac{l^{-1}\xi	
		\cosh(l^{-1}\xi)}{\sinh(l^{-1}\xi)} \bigg(\epsilon\cdot P - 
	\frac{\xi\cdot\epsilon \xi\cdot P}{\xi^2}\bigg) + 
	\frac{\xi\cdot\epsilon \xi\cdot P}{\xi^2} - \epsilon\cdot P ~.
\end{split}
\end{displaymath}
The introduction of the extra $\epsilon\cdot P$ terms in the last 
line is just a matter of convention, which allows one to write 
eq.~\eqref{eq:inftrafo_cosetpar_dS} as
\begin{equation}\label{eq:sol_inftrafo_cosetpar_dS}
	\delta\xi \cdot P = \epsilon\cdot P + 
	\bigg(\frac{l^{-1}\xi\cosh(l^{-1}\xi)}{\sinh(l^{-1}\xi)} - 
	1\bigg) \bigg(\epsilon\cdot P - \frac{\xi\cdot\epsilon 
		\xi\cdot P}{\xi^2}\bigg)~.
\end{equation}
This implies that the infinitesimal change of the coset 
parameters is given by
%
\begin{equation}
	\delta\xi^a = \epsilon^a + \Big(\frac{z\cosh z}{\sinh z} - 
	1\Big) \bigg(\epsilon^a - \frac{\xi^a \epsilon_b 
		\xi^b}{\xi^2}\bigg)~,
\end{equation}
where $z = l^{-1}\xi$.

A comment on the derivation of 
eq.~\eqref{eq:sol_inftrafo_cosetpar_dS} is in place. The solution 
was found after use of the power series for the hyperbolic 
cosecant.  In the case of real numbers, the series is only 
defined for values between $-\pi$ and $+\pi$. One could thus 
wonder if this convergence issue inhibits us of trusting the 
solution found above. Remember that 
eq.~\eqref{eq:inftrafo_cosetpar_dS} can be rewritten as
%
\begin{displaymath}
	(i\xi\cdot P)^{-1} \sinh(i\xi\cdot P) \wedge \delta\xi\cdot P 
	= \cosh(i\xi\cdot P) \wedge \epsilon\cdot P~,
\end{displaymath}
which reduces to
%
\begin{displaymath}
	z^{-1}\sinh z \Big(\delta\xi\cdot P - \frac{\xi\cdot \delta\xi 
		\xi\cdot P}{\xi^2}\Big) + \frac{\xi\cdot \delta\xi \xi\cdot 
		P}{\xi^2}
	= \cosh z \Big(\epsilon\cdot P - \frac{\xi\cdot\epsilon 
		\xi\cdot P}{\xi^2}\Big) + \frac{\xi\cdot\epsilon \xi\cdot 
		P}{\xi^2}~.
\end{displaymath}
This result relies on the power series expansion of the 
hyperbolic sine, which is convergent for all values of its 
argument. It is readily checked that the 
solution~\eqref{eq:sol_inftrafo_cosetpar_dS} satisfies the above 
equation. Therefore, we may conclude 
that~\eqref{eq:sol_inftrafo_cosetpar_dS} is the right solution. 

\blankline
Given the de Sitter algebra, it is also possible to compute 
$\delta h = \tfrac{i}{2} \delta h^{ab} M_{ab}$ explicitly.  
From~\eqref{eq:inftrafo_h} it follows that the element of 
$\mathfrak{h}$, corresponding to an infinitesimal de Sitter 
translation, is a solution of
%
\begin{equation}\label{eq:inftrafo_h_dS}
	\tfrac{1}{2} \sinh(i\xi\cdot P) \wedge \delta h \cdot M = 
	(\mathds{1}-\cosh(i\xi\cdot P)) \wedge \epsilon\cdot P~.
\end{equation}
The right hand side is readily found by 
reconsidering~\eqref{eq:cosh_transl_dS}, implying that
%
\begin{displaymath}
	(\mathds{1} - \cosh(i\xi\cdot P)) \wedge \epsilon\cdot P
	= (1 - \cosh z)\bigg(\epsilon\cdot P - \frac{\xi\cdot\epsilon 
		\xi\cdot P}{\xi^2}\bigg)~.
\end{displaymath}
From the power series expansion of the hyperbolic sine,
%
\begin{displaymath}
	\sinh(i\xi\cdot P) = \sum_{n=0}^\infty \frac{(i\xi\cdot 
		P)^{2n+1}}{(2n+1)!}~,
\end{displaymath}
and from~\eqref{eq:id_dS_comm.4}, it follows that
%
\begin{displaymath}
\begin{split}
	\sinh(i\xi\cdot P) \wedge \delta h \cdot M
	&= \delta h^{ab} \sum_{n=0}^\infty \frac{z^{2n}}{(2n+1)!} 
	(\xi_a P_b - \xi_b P_a ) \\
	&= z^{-1} \sinh z\, \delta h^{ab} (\xi_a P_b - \xi_b P_a ) \\
	&= 2 z^{-1} \sinh z\, \delta h^{ab} \xi_a P_b~.
\end{split}
\end{displaymath}
Putting these equations together,~\eqref{eq:inftrafo_h_dS} is 
rewritten as
%
\begin{displaymath}
\begin{split}
	\delta h^{ab} \xi_a P_b &= \frac{z (1-\cosh z)}{\sinh z} 
	\bigg(\epsilon^b - \frac{\xi^a \epsilon_ a \xi^b}{\xi^2} 
	\bigg)P_b \\
	&= (l\xi)^{-1} \frac{1-\cosh z}{\sinh z} (\epsilon^b\xi^a - 
	\epsilon^a\xi^b) \xi_a P_b~,
\end{split}
\end{displaymath}
from which it can be concluded that the sought-after quantities 
are
\begin{equation}
	\delta h = \tfrac{i}{2}\delta h^{ab} M_{ab} = \frac{i}{2l^2} 
	\frac{\cosh z - 1}{z\sinh z} (\epsilon^a\xi^b - 
	\epsilon^b\xi^a) M_{ab}~.
\end{equation}


\subsection{Nonlinear de Sitter-Cartan geometry}
\label{ssec:nonlin_dSC_geom}

As motivated in the introductory section of this chapter, we will 
use the theory of nonlinear realizations to construct a nonlinear 
de Sitter-Cartan geometry. In doing so, it will become clear that 
local de Sitter invariance is restored while preserving the spin 
connection and vierbein.
This treatment closely follows the original work of Stelle and 
West, see \cite{Stelle:1979va,stelle.west:1980ds}. In this 
discussion however, the cosmological constant of the local de 
Sitter spaces is not assumed to be the same at any point in 
spacetime. Rather, we allow for a spacetime varying 
\emph{cosmological function} $\Lambda(x)$.

\blankline
Remember that a de Sitter-Cartan connection on a principal 
$H$-bundle is induced from a $G$-connection on a principal 
$G$-bundle, by reducing the $G$-bundle upon choosing a section 
$\xi$ in the associated bunlde of homogeneous de Sitter spaces 
$G/H$. This induced Cartan connection is to be realized 
nonlinearly. Accordingly let us begin by considering a principal 
de Sitter bundle $Q(M,G)$ over spacetime $M$.  Additionaly a 
geometry is introduced by in the form of a local Ehresmann 
connection $A \in \Omega^1(M, \mathfrak{g})$, that is directly on 
spacetime. The connection is further characterized by its 
curvature 
\begin{displaymath}
F = dA + \tfrac{1}{2}[A,A] \in \Omega^2(M,\mathfrak{g})~.
\end{displaymath}
These $\mathfrak{g}$-valued differential forms can be decomposed 
with respect to their $\mathfrak{h}$-~and $\mathfrak{p}$-valued 
parts, i.e.~
%
\begin{displaymath}
	A = A_\mathfrak{h} + A_\mathfrak{p}
	= \tfrac{i}{2} A^{ab} M_{ab} + iA^a P_a
	\quad\text{and}\quad
	F = F_\mathfrak{h} + F_\mathfrak{p}
	= \tfrac{i}{2} F^{ab} M_{ab} + iF^a P_a~,
\end{displaymath}
where as usual $P_a = M_{a4}/l(x)$. This relation implies that 
$A^a$ and $F^a$ have dimensions of length. Due to the symmetric 
nature of $SO(1,4)$, one may express $F^{ab}$ and $F^{a}$ in 
terms of $A^{ab}$ and $A^a$, which leads to the following 
equations: 
%
\begin{subequations}
	\label{eq:curv_lin_dS}
\begin{align}
	F^{ab} &= dA^{ab} + A\ind{^a_c} \wedge A^{cb} + \frac{1}{l^2} 
	A^a \wedge A^b~,
	\\
	F^a &= dA^a + A\ind{^a_b} \wedge A^b - \frac{1}{l} dl \wedge 
	A^a ~.
\end{align}
\end{subequations}
Although it is not a coincidence that the above expressions 
remind one of the corresponding expressions for curvature and the 
torsion of a Cartan connection, it must be emphasized that the 
quantities~\eqref{eq:curv_lin_dS} are by no means the curvature 
or torsion of some geometric object. At this point there is only 
a curvature $F$ of the Ehresmann connection $A$ in play, while 
for the latter torsion is not defined. Furthermore, remark that 
the decomposition of $F$ ($A$) into $F_\mathfrak{h}$ 
($A_\mathfrak{h}$) and $F_\mathfrak{p}$ ($A_\mathfrak{p}$) is not 
well-defined with respect to the geometry, in the sense that 
local gauge transformations mix up the $\mathfrak{h}$-~and 
$\mathfrak{p}$-valued parts. Stated equivalently, $A$ and $F$ 
each transform irreducibly under $G$.  As a result the 
decompositions are not respected by the symmetries of the 
geometry, so that it would be difficult to atribute it any 
physical meaning.

This observation is very important, since it is the main 
motivation to make use of nonlinear realizations. The symmetric 
splitting of $A$ and $F$ leads directly to the structure of a 
Cartan geometry on a principal Lorentz bundle. This splitting is 
invariant under local Lorentz transformations and the 
corresponding projections make up true geometric objects. We, 
however, want to retain the full symmetry of local de Sitter 
transformations \emph{together} with the symmetric splitting of 
$A$ and $F$. The solution to this situation is using nonlinear 
realizations. After we have constructed the nonlinear versions of 
$A$ and $F$, we will come back once more to the physical reasons 
of why it is necessary to have both the symmetric splitting and 
local de Sitter invariance.
{\blu \it [The motivation for nonlinear realization and the role 
	of symmetry breaking (choosing a section $\xi$) and (vs.?)  
	nonlinear realization should be carefully explained.]}

Remember from our previous discussion on nonlinear realizations 
that, in order to construct such objects, it is necessary to 
single out an arbitrary section $\xi : M \to P[G/H]$ of the 
associated bundle of de Sitter spaces. Given a linearly 
tranforming quantity and such a section $\xi$, the general 
prescription to obtain a nonlinearly transforming object from the 
linear one is summarized in
Eq.~\eqref{eq:def_nonlinear_field}.
Because we are interested in finding the nonlinear versions of 
$A$ and $F$, it is useful to write down their transformation 
behaviour under the action of the de Sitter group. For a local 
gauge transformation $g_0 \in \Omega^0(M,G)$ of an associated 
vector bundle, it is well-known that this behaviour is given 
by\footnote{The fibres of the principal bundle transforms with 
	the inverse.}
%
\begin{subequations}
\begin{equation}
	\label{eq:trafo_conn_dS}
	A \mapsto g_0 A g_0^{-1} + g_0  dg_0^{-1} = \Ad(g_0)\cdot(A + 
	d)~,
\end{equation}
\text{respectevily}
\begin{equation}
	\label{eq:trafo_curv_dS}
	F \mapsto g_0 F g_0^{-1} = \Ad(g_0) \cdot F~.
\end{equation}
\end{subequations}
These transformation laws together with the 
prescription~\eqref{eq:def_nonlinear_field} allow us to find 
their nonlinear counterparts relatively straightforward.

First let us construct the nonlinear version of the 
$G$-connection $A$. Its nonhomogeneous way of 
tranforming~\eqref{eq:trafo_conn_dS} leads one to define the 
corresponding nonlinear de Sitter connection as
%
\begin{equation}
	\label{eq:A_nonlin}
	\bar{A} := \mathrm{Ad}(\exp(-i\xi\cdot P))\cdot(A + d)~.
\end{equation}
The resulting object is once again a $\mathfrak{g}$-valued 
differential form on $M$, for which the symmetric splitting takes 
the form $\bar{A} =\bar{A}_\mathfrak{h} + 
\bar{A}_\mathfrak{p}$.\footnote{It is worthwhile to emphasize 
	that this denotes the splitting of $\bar{A}$, rather than the 
	sum of the nonlinear versions of $A_\mathfrak{h}$ and 
	$A_\mathfrak{p}$. The latter are not even defined, since they 
	separately do not form a linear representation of the de 
	Sitter group.}
The crucial difference with the splitting of the linear 
connection $A$ is that the decomposition for $\bar{A}$ is 
invariant under the whole de Sitter group. This will be the 
subject of a more thorough discussion in the next section, where 
the importance of this fact will be further explained.  
Especially, it will become clear that $\bar{A}_\mathfrak{h}$ and 
$\bar{A}_\mathfrak{p}$ are the genuine spin connection, 
respectively vierbein, for a locally de Sitter invariant theory 
of gravity. Anticipating on this interpretation we introduce the 
notation $\omega := \bar{A}_\mathfrak{h}$ and $e := 
\bar{A}_\mathfrak{p}$.

Next we use the definition of the nonlinear connection $\bar{A}$ 
to explicitely calculate the spin connection $\omega$ and $e$ in 
terms of $\xi$, $A_\mathfrak{h}$ and $A_\mathfrak{p}$. It follows 
from Eq.~\eqref{eq:A_nonlin} that
\begin{displaymath}
	\tfrac{i}{2} \omega^{ab} M_{ab} + i e^a P_a
	= \Ad(\exp(-i\xi\cdot P)) \Big( \tfrac{i}{2} A^{ab} M_{ab} + i 
	A^a P_a + d \Big)~.
\end{displaymath}
Invoking Hadamard's formula~\eqref{eq:Had} and the 
Campbell-Poincar\'e fundamental identity~\eqref{eq:CP} the 
right-hand side can be rewritten as\footnote{The operation 
	$\wedge : \mathfrak{g} \times \mathfrak{g} \to \mathfrak{g} : 
	(X,Y) \mapsto \mathrm{ad}_X (Y)$ should not be confused with 
	the wedge product for differential forms. See also 
	App.~\ref{app:nested_comm.not}.}
\begin{displaymath}
	\exp(-i\xi\cdot P) \wedge \big( \tfrac{i}{2} A^{ab} M_{ab} + i 
	A^a P_a \big) + \frac{1 - \exp(-i\xi\cdot P)}{i\xi\cdot P} 
	\wedge d (i\xi\cdot P)~.
\end{displaymath}
This expression has to be worked out and separated in two parts, 
one valued in the Lorentz algebra $\mathfrak{h}$ and a second 
taking values in the subspace of transvections $\mathfrak{p}$.  
That such a decomposition can be done explicitely follows from 
the symmetric nature of the de Sitter algebra: for any two 
elements $X$ and $Y$ in $\mathfrak{h}$ or $\mathfrak{p}$, the 
element $X \wedge Y$ is in $\mathfrak{h}$ or $\mathfrak{p}$.  In 
order to carry out the calculation, one must use the results of 
App.~\ref{app:nested_comm.dS_results}. Then it is found 
successively:
\begin{gather*}
\begin{multlined}[b][\linewidth-2\multlinegap]
	\exp(-i\xi\cdot P) \wedge \tfrac{i}{2} A^{ab} M_{ab}
	= \tfrac{i}{2} \big( A^{ab} + \frac{\cosh z - 1}{l^2 z^2} 
	\xi_c ( \xi^b A^{ac} - \xi^a A^{bc} ) \big) M_{ab} \\
	+ i \big( z^{-1} \sinh z A\ind{^a_b} \xi^b \big) P_a~,
\end{multlined}
\\
\begin{multlined}[b][\linewidth-2\multlinegap]
	\exp(-i\xi\cdot P) \wedge i A^a P_a = \tfrac{i}{2} \bigg( 
	\frac{\sinh z}{l^2 z} (A^a \xi^b - A^b \xi^a ) \bigg) M_{ab} 
	\\
	+ i \bigg( A^a + (\cosh z - 1) \bigg(A^a - \frac{\xi^b A_b 
		\xi^a}{\xi^2}  \bigg) \bigg) P_a~,
\end{multlined}
\\
\begin{multlined}[b][\linewidth-2\multlinegap]
	\frac{1 - \exp(-i\xi\cdot P)}{i\xi\cdot P} \wedge d (i\xi\cdot 
	P) = \tfrac{i}{2} \bigg( \frac{\cosh z - 1}{l^2 z^2} (d\xi^a 
	\xi^b - d\xi^b \xi^a) \bigg) M_{ab} \\
	+ i \bigg( \frac{\sinh z}{z} \bigg( d\xi^a - \frac{\xi^b 
		d\xi_b \xi^a}{\xi^2} \bigg) + \frac{\xi^b d\xi_b 
		\xi^a}{\xi^2} - \frac{dl}{l} \xi^a \bigg) P_a~.
\end{multlined}
\end{gather*}
Collecting these different contributions and seperating terms 
according to whether they are valued in $\mathfrak{h}$, 
respectively $\mathfrak{p}$, one gets the explicit expressions 
for the spin connection $\omega^{ab}$ and vierbein $e^a$:
\begin{subequations}
\label{eqs:nonlin_spin_vier}
\begin{gather}
\label{eq:nonlin_spinconn}
\begin{multlined}[b][\linewidth-8\multlinegap]
	\omega^{ab} = A^{ab} - \frac{\cosh z - 1}{l^2 z^2} \big( \xi^a 
	(d\xi^b + A\ind{^b_c} \xi^c) - \xi^b (d\xi^a + A\ind{^a_c} 
	\xi^c) \big) \\
	- \frac{\sinh z}{l^2 z} (\xi^a A^b - \xi^b A^a)
\end{multlined}~,
\\
\label{eq:nonlin_vierbein}
\begin{multlined}[b][\linewidth-8\multlinegap]
	e^a = A^a + \frac{\sinh z}{z}( d\xi^a + A\ind{^a_b} \xi^b) - 
	\frac{dl}{l} \xi ^a \\
	+ (\cosh z - 1) \bigg( A^a - \frac{\xi^b A_b \xi^a}{\xi^2} 
	\bigg) - \bigg( \frac{\sinh z}{z} - 1 \bigg) \frac{\xi^b 
		d\xi_b \xi^a}{\xi^2}
\end{multlined}~.
\end{gather}
\end{subequations}
These expressions are almost equal to the corresponding objects 
found by Stelle and West~\cite{stelle.west:1980ds}. The crucial 
difference is that we have a new term $l^{-1} dl\,\xi^a$ in the 
expression~\eqref{eq:nonlin_vierbein} for the vierbein. This term 
is present in the given geometry, since we have allowed the 
tangent de Sitter spaces to have cosmological constants that are 
not necessarily equal along spacetime. More specifically, one has 
to take in account that the length scale defined for the elements 
in $\mathfrak{p}$ may vary. On the other hand, the results 
of~\cite{stelle.west:1980ds} rely on the assumption that the 
local de Sitter spaces have the same pseudo-radius at any point 
in spacetime. The extra contribution is proportional to the 
dimensionless factor $l^{-1} d l$, which will be noticeable only 
if the variation is relatively vast. In case $l$ is a constant 
function, one of course recovers the results 
of~\cite{stelle.west:1980ds}.
As mentioned already before, the main reason for their relevance 
is that $\omega^{ab}$ and $e^a$ transform irreducibily under 
local de Sitter transformations, an argument that will be 
underpinned further in the next section.  

After having constructed the nonlinear connection $\bar{A}$ and 
consequently the spin connection and vierbein of the geometry, we 
turn attention to the nonlinear curvature $\bar{F}$. From the 
definition of $F$ and its transformation 
behaviour~\eqref{eq:trafo_curv_dS} under local de Sitter 
transformations, it follows that the nonlinear curvature equals 
the curvature of the nonlinear connection, i.e.
%
\begin{equation}
\label{eq:F_nonlin}
	\bar{F} := \Ad(\exp(-i\xi\cdot P)) \cdot F = d\bar{A} + 
	\tfrac{1}{2}[\bar{A},\bar{A}]~.
\end{equation}
This is a $\mathfrak{g}$-valued $2$-form on spacetime, which we 
decompose once more according to~$\bar{F} = \bar{F}_\mathfrak{h} 
+ \bar{F}_\mathfrak{p}$. Due to the nonlinear nature of 
$\bar{F}$, the reductive split is invariant under local gauge 
transformations. Being similar to the discussion on the nonlinear 
connection $\bar{A}$, the covariant nature of the decomposition 
suggests that $\bar{F}_\mathfrak{h}$ and $\bar{F}_\mathfrak{p}$ 
must be considered the curvature and torsion of the given 
geometry.  Therefore, let us make use of the suggestive 
notation~$R := F_\mathfrak{h}$ and~$T := F_\mathfrak{p}$.

Let us then look for expressions that give the curvature $R$ and 
torsion $T$ in terms of $\xi$, $F_\mathfrak{h}$ and 
$F_\mathfrak{p}$. Equivalently to the derivation outlined above 
in finding the spin connection and vierbein, one considers the 
definition for the nonlinear curvature $\bar{F}$, i.e.
\begin{displaymath}
	\tfrac{i}{2} R^{ab} M_{ab} + i T^a P_a
	= \Ad(\exp(-i\xi\cdot P)) \Big( \tfrac{i}{2} F^{ab} M_{ab} + i 
	F^a P_a \Big)~,
\end{displaymath}
after which the right-hand side of this equation must be written 
as the sum of an $\mathfrak{h}$-valued and a 
$\mathfrak{p}$-valued part. This calculation is to a large extend 
identical to the one done for $\bar{A}$ and leads to the 
following quantities:
\begin{subequations}
\label{eqs:nonlin_curv_tors}
\begin{align}
\label{eq:nonlin_curv}
	R^{ab} &= F^{ab} - \frac{\cosh z - 1}{l^2 z^2} \xi^c (\xi^a 
	F\ind{^b_c} -  \xi^b F\ind{^a_c}) - \frac{\sinh z}{l^2 z} 
	(\xi^a F^b - \xi^b F^a)~,
\\
\label{eq:nonlin_tors}
	T^a &= \frac{\sinh z}{z} \xi^b F\ind{^a_b} + \cosh z\, F^a + 
	(1 - \cosh z) \frac{\xi_b F^b \xi^a}{\xi^2}~.
\end{align}
\end{subequations}
Note that Eq.~\eqref{eq:F_nonlin} implies that these expressions 
can equally be obtained by calculating $d\bar{A} + 
\tfrac{1}{2}[\bar{A},\bar{A}]$ directly. More precisely, 
decomposing Eq.~\eqref{eq:F_nonlin} according to the reductive 
splitting of the de Sitter algebra, it follows that
\begin{gather*}
	R^{ab} = d\omega^{ab} + \omega\ind{^a_c} \wedge \omega^{cb} + 
	\frac{1}{l^2} e^a \wedge e^b~,
	\\
	T^a = de^a + \omega\ind{^a_b} \wedge e^b - \frac{1}{l} dl 
	\wedge e^a~,
\end{gather*}
after which one can substitute for 
Eqs.~\eqref{eq:nonlin_spinconn} and~\eqref{eq:nonlin_vierbein} to 
obtain $R^{ab}$ and $T^a$. These equations, which express the 
curvature and torsion in terms of the spin connection and 
vierbein, are the ones expected for a Cartan geometry. In this 
manner, we constructed a de Sitter-Cartan geometry on a principal 
Lorentz bundle. Because elements of $SO(1,4)$ that are not in 
$SO(1,3)$ are nonlinearly realized, we actually have $SO(1,4)$ 
invariance of the geometry, while having a well-defined spin 
connection and vierbein at hand. This will be studied in more 
detail in the next section.

To conclude this section, Table~\ref{tbl:not_nonlin_cartan_geo} 
summarizes the various symbols used to denote the linear and 
nonlinear quantities introduced above.
\begin{table}[th]
	\caption{Notation for the linear and nonlinear Cartan 
		connection and curvature and their reductive 
		decomposition.}
	\label{tbl:not_nonlin_cartan_geo}
	\centering
	\renewcommand{\arraystretch}{2}
	\begin{tabular}{cc}
		linear	& nonlinear	\\
		\hline
		\hline
		$A = \tfrac{i}{2}A^{ab} M_{ab} + iA^a P_a$
			& $\bar{A} = \tfrac{i}{2}\omega^{ab} M_{ab} + ie^a P_a$
		\\
		\hline
		$F = \tfrac{i}{2}F^{ab} M_{ab} + iF^a P_a$
			& $\bar{F} = \tfrac{i}{2}R^{ab} M_{ab} + iT^a P_a$
	\end{tabular}
\end{table}

\subsection{Transformation behaviour: linear vs.~nonlinear}
\label{ssec:trafo_beh}

In the last subsection a nonlinear Cartan geometry for the de 
Sitter algebra has been introduced. We claimed that the nonlinear 
de Sitter-Cartan connection gives rise to a genuine spin 
connection and vierbein, while its Cartan curvature breaks up in 
a well-defined Lorentz curvature and torsion tensor. This led us 
to introduce the suggestive notation for the respective objects 
as it is summarized in Table~\ref{tbl:not_nonlin_cartan_geo}. The 
reason for having to use the nonlinear fields is that their 
reductive decomposition is invariant under local de Sitter 
transformations, which should be compared with the merely local 
Lorentz invariance of their linear counterparts. Since we wish 
local de Sitter invariance to be at the heart of any theory of 
gravity, $\omega^{ab}$ and $e^{a}$ are objects that are 
well-defined in light of such a principle. In the following 
paragraphs and at risk of being overprecise, we therefore review 
the transformation properties of the linear and nonlinear fields 
and confirm their particular behaviour under the action of the 
Lorentz and de Sitter groups.

From the general relation between a linear field and its 
nonlinear realization, see Eq.~\eqref{eq:def_nonlinear_field}, 
one deduces that they belong to the same representation space.  
The difference between them is that the nonlinear field becomes 
reducible under the action of the gauge group $G$, even if the 
linear field transforms irreducibly. More precisely, the 
nonlinear field is acted upon only with respect to its 
$H$-indices, since elements in $G$ that are not in the subgroup 
$H$ are nonlinearly realized by elements of the latter.  
Naturally the same is true for the fields at interest, namely the 
Cartan connection and its curvature. Notwithstanding being a 
special case of the situation just described, let us take a 
closer look at the transformation rules of the linear connection 
and curvature, after which we will turn attention to the 
nonlinear ones.

\blankline
Under local de Sitter transformations the connection $A$ changes 
according to Eq.~\eqref{eq:trafo_conn_dS}. If the action is 
restricted to be an element of the Lorentz subgroup $H$, the 
reductive splitting $A = A_\mathfrak{h} + A_\mathfrak{p}$ is 
invariant. Due to the symmetric nature of $\mathfrak{so}(1,4)$ it
is readily inferred that $A_\mathfrak{h}$ and $A_\mathfrak{p}$ 
transform according to
\begin{displaymath}
	A_\mathfrak{h} \mapsto \Ad(h_0) \cdot (A_\mathfrak{h} + d)
	\quad	\text{and} \quad
	A_\mathfrak{p} \mapsto \Ad(h_0) \cdot A_\mathfrak{p}~,
\end{displaymath}
where $h_0 \in H$.
If only local Lorentz invariance were required, these objects 
would be adequate as a spin connection and vierbein.  Since we 
additionally desire local de Sitter invariance, such a 
decomposition will not be satisfying. This is so because the 
symmetric splitting is not respected under a generic 
$G$-transformation. To see this explicitely, consider an 
infinitesimal pure de Sitter translation $e + i \epsilon\cdot P$.
To first order in the transformation parameter $\epsilon$, one 
finds that
\begin{displaymath}
	\delta_\epsilon A
		= i[\epsilon\cdot P,A] -i d(\epsilon\cdot P) \\
		= i[\epsilon\cdot P,A] -i d\epsilon\cdot P + \frac{dl}{l} i 
		\epsilon\cdot P~.
\end{displaymath}
To find the variations of $A^{ab}$ and $A^a$, one further works 
out the right-hand side and separates $\mathfrak{h}$- and 
$\mathfrak{p}$-valued parts. The variations due to the action of 
an infinitesimal de Sitter translation are then concluded to be
\begin{subequations}
	\label{eqs:var_lin_Ah_Ap}
\begin{align}
	\delta_\epsilon A^{ab} &= \frac{1}{l^2}(\epsilon^a A^b - 
	\epsilon^b A^a)~, \\
	\delta_\epsilon A^a &= -d\epsilon^a - A\ind{^a_b}\epsilon^b + 
	\frac{dl}{l}\epsilon^a~.
\end{align}
\end{subequations}
These results show manifestly how $A^{ab}$ and $A^a$ form an 
irreducible multiplet for the de Sitter group.

The variations~\eqref{eqs:var_lin_Ah_Ap} could equally have been 
found by reconsidering the calculation of $\bar{A}$ in 
section~\ref{ssec:nonlin_dSC_geom}. Remember that the definition 
of the nonlinear connection is given by $\Ad(\exp(-i\xi\cdot 
P))\cdot(A + d)$. This is nothing but the transformation of $A$ 
under a pure de Sitter translation with transformation parameter 
$\alpha = -\xi$. The calculation of $\bar{A}$ can hence be used 
here to find the transformations of $A^{ab}$ and $A^a$ under a 
finite pure de Sitter translation $\exp(i\alpha\cdot P)$. More 
precisely and according to the logic just explained, we may copy 
the results of Eqs.~\eqref{eqs:nonlin_spin_vier} together with 
the substitution~$\xi \to -\alpha$. This leads to the finite 
transformations laws:
\begin{subequations}
\begin{gather}
\begin{multlined}[b][\linewidth-8\multlinegap]
	\label{eq:transvec_fin_Ah}
	A^{ab} \mapsto A^{ab} + \frac{1 - \cosh z}{l^2 z^2} \big( 
	\alpha^a (d\alpha^b + A\ind{^b_c} \alpha^c) - \alpha^b 
	(d\alpha^a + A\ind{^a_c} \alpha^c) \big) \\
	+ \frac{\sinh z}{l^2 z} (\alpha^a A^b - \alpha^b A^a)
\end{multlined}~,
	\\
\begin{multlined}[b][\linewidth-8\multlinegap]
	\label{eq:transvec_fin_Ap}
	A^a \mapsto A^a - \frac{\sinh z}{z}( d\alpha^a + A\ind{^a_b} 
	\alpha^b) + \frac{dl}{l} \alpha ^a \\
	+ (\cosh z - 1) \bigg( A^a - \frac{\alpha^b A_b 
		\alpha^a}{\alpha^2} \bigg)
	+ \bigg( \frac{\sinh z}{z} - 1 \bigg) \frac{\alpha^b d\alpha_b 
		\alpha^a}{\alpha^2}
\end{multlined}~.
\end{gather}
\end{subequations}
The infinitesimal variations~\eqref{eqs:var_lin_Ah_Ap} are 
recovered straightforwardly after taking the limit $\alpha \to 
\epsilon$.

The discussion can be extended to the linear curvature $F$ in an 
evident way. This object transforms in a covariant way under 
gauge transformations, as written down in 
Eq.~\eqref{eq:trafo_curv_dS}. In case we only consider elements 
$h_0$ that belong to the subgroup of Lorentz rotations, the 
reductive splitting $F = F_\mathfrak{h} + F_\mathfrak{p}$ is 
invariant and both projections rotate independently according to
\begin{displaymath}
	F_\mathfrak{h} \mapsto \Ad(h_0)\cdot F_\mathfrak{h}
	\quad \text{and} \quad
	 F_\mathfrak{p}\mapsto \Ad(h_0)\cdot F_\mathfrak{p}~.
\end{displaymath}
On the other hand, this reducible behaviour is not present if the 
action is governed by a generic element of the de Sitter group.  
To see this in an explicit way, let us have a look at the 
transformations of $F_\mathfrak{h}$ and $F_\mathfrak{p}$ under a 
pure de Sitter translation $\exp(i\alpha\cdot P)$. Again it is 
possible to recycle the calculation for the nonlinear $\bar{F}$.  
Indeed, the transformed linear curvature $\Ad(\exp(\alpha\cdot 
P))\cdot F$ is equal to the nonlinear $\bar{F}$ with $\xi$ 
replaced by $-\alpha$, which can be seen from 
Eq.~\eqref{eq:F_nonlin}. The transformation of the 
$\mathfrak{h}$-~and $\mathfrak{p}$-valued parts of $F$ are then 
given by considering this replacement in the 
Eqs.~\eqref{eqs:nonlin_curv_tors}, yielding
\begin{subequations}
\label{eqs:transvec_Fh_Fp}
\begin{gather}
\label{eq:transvec_Fh}
	F^{ab} \mapsto F^{ab} - \frac{\cosh z - 1}{l^2 z^2} \alpha^c 
	(\alpha^a F\ind{^b_c} -  \alpha^b F\ind{^a_c}) + \frac{\sinh z}{l^2 
		z} (\alpha^a F^b - \alpha^b F^a)~,
\\
\label{eq:transvec_Fp}
	F^a \mapsto -\frac{\sinh z}{z} \alpha^b F\ind{^a_b} + \cosh 
	z\, F^a + (1 - \cosh z) \frac{\alpha_b F^b 
		\alpha^a}{\alpha^2}~.
\end{gather}
\end{subequations}
To be complete, the infinitesimal variations of $F_\mathfrak{h}$ 
and $F_\mathfrak{p}$ are also written down. These are obtained 
from the finite transformations after taking $\alpha \to 
\epsilon$ and retaining terms up to first order. One directly 
concludes:
\begin{subequations}
\begin{align}
	\delta_\epsilon R^{ab}_{~~\mu\nu} &= \frac{1}{l^2}(\epsilon^a 
	T^b_{~\mu\nu} - \epsilon^b T^a_{~\mu\nu})~, \\
	\delta_\epsilon T^a_{~\mu\nu} &= -\epsilon^b R^a_{~b\mu\nu}~.
\end{align}
\end{subequations}

Consequent to the way in which the reductive projections of $A$ 
and $F$ transform under the de Sitter group, they cannot be given 
well-defined meaning consistent with a principle of local de 
Sitter invariance. Repeatedly mentioned in this text, this is the 
place where the theory of nonlinear realizations shows its 
usefulness. As we will make plain in the following paragraphs, 
the nonlinear fields $\bar{A}$ and $\bar{F}$ possess a reductive 
splitting that is invariant under local de Sitter 
transformations, and which allows us to pin down the real 
geometric objects that can be used in a theory of gravity that is 
locally de Sitter invariant.

Being a generic property of nonlinear realizations, the nonlinear 
connection $\bar{A}$ and curvature $\bar{F}$ belong to the same 
representation space as $A$ and $F$, respectivily.  
Notwithstanding this similarity in the way of transforming, the 
nonlinear fields are reducible, whereas their linear versions 
obviously are not. This is so because any element of the de 
Sitter group $G$ acting on the nonlinear fields is realized 
through an element of its subgroup $H$ of Lorentz 
transformations. As a consequence, the nonlinear connection and 
curvature are acted upon only through their $H$-components, even 
if the element considered belongs to the enclosing de Sitter 
group. To verify the transformation behavior of $\bar{A}$, note 
first that its definition~\eqref{eq:A_nonlin} indicates that 
$\Ad(e)(A+d) = \Ad(\exp(i\xi\cdot P))(\bar{A}+d)$. It follows 
that under the action of an element $g_0$ of the de Sitter group:
%
\begin{displaymath}
\begin{multlined}[b][\linewidth-12\multlinegap]
	\bar{A} \mapsto  \Ad(\exp(-i\xi'\cdot P))\Ad(g_0)(A+d) \\
	\begin{aligned}
	&= \Ad(\exp(-i\xi'\cdot P) g_0) \Ad(\exp(i\xi'\cdot P)) 
	(\bar{A}+d) \\
	&= \Ad(h'(\xi,g_0))(\bar{A}+d)~.
	\end{aligned}
\end{multlined}
\end{displaymath}
This reconfirms the way a nonlinear field transforms, where a 
generic de Sitter transformation is realized by a Lorentz 
rotation. In particular do we conclude that the 
$\mathfrak{g}$-valued $1$-form $\bar{A}$ behaves as a Cartan 
connection on a principal Lorentz bundle.\footnote{[{\blu\it It 
		seems very plausible that it is (the pull-back of) a Cartan 
		connection on $P(M,H)$ Check this: see the base definition 
		of the latter as given in	\cite{sharpe1997diff_geo}].}}
If the element $g_0$ belongs to the Lorentz group $H$, the 
transformation becomes linear.\footnote{Note that a Cartan 
	connection is defined on a principal $H$-bundle, and that is 
	only demanded that the connection transforms in a certain way 
	under the action of $H$ gauge transformations.  Whether the 
	elements of $H$ form a linear or nonlinear realization is not 
	relevant.}
Because any element of the de Sitter group is realized as a 
Lorentz transformation, the reductive splitting $\bar{A} = \omega 
+ e$ is preserved under local de Sitter transformations.
Indeed reconsidering the transformation of $\bar{A}$ explicitely 
for its reductive decompositions, namely 
%
\begin{displaymath}
	\omega + e \mapsto \Ad(h'(\xi,g_0))\cdot (\omega + e + d)~,
\end{displaymath}
it directly follows that
%
\begin{equation}
	\label{eq:trafo_nonlin_omega_e}
	\omega \mapsto \Ad(h'(\xi,g_0)) \cdot (\omega + d)
	\quad\text{and}\quad
	e \mapsto \Ad(h'(\xi,g_0)) \cdot e~,
\end{equation}
as a result of the symmetric nature of $\mathfrak{g} = 
\mathfrak{h} + \mathfrak{p}$.
It is manifest that $\omega$ and $e$ do not mix under local de 
Sitter transformations. Note that $\omega$ is an 
$\mathfrak{h}$-valued spin connection, while $e$ is a 
$\mathfrak{p}$-valued $1$-form that transforms covariantly. In 
case the gauge transfromations are restricted to be elements of 
the Lorentz subgroup, the nonlinear fields transform identical to 
their linear counterparts. 

These conclusions are equally drawn for the 
$\mathfrak{g}$-curvature $\bar{F}$ and its projections $R$ and 
$T$. Both $2$-forms transform covariantly in the adjoint 
representation of the group, and do not mix up in this process.  
Therefore it is clear that $R$ and $T$ are true geometric 
objects, for the splitting is a gauge independent construction.  
They are referred to as the curvature and torsion of the 
geometry.


\subsection{Interpretation of the vielbein $\bar{e}$}

Stelle and West \cite{Stelle:1979va,stelle.west:1980ds} claim 
that the vierbein $\bar{e}$ is a smooth mapping between the 
tangent space to spacetime at any $p \in M$  and the tangent 
space to the internal de Sitter space at $\xi(p)$. Unfortunately, 
a concrete argument did not seem to be included following this 
statement.  Furthermore, under local $H$ gauge transformations 
the vierbein $\bar{e}$ transforms as a vector with an element $h 
\in H_o$, as can be seen from \eqref{eq:trafo_nonlin_omega_e}.  
This indicates that its $SO(3,1)$-indices belong to the tangent 
space at the origin $(\xi = 0)$ of $dS$. To verify its 
transformation behavior under local $G$ transformations, let us 
explicitly reconsider its construction.  

The vierbein is defined as the $\mathfrak{p}$-valued part of the 
Cartan connection $\bar{A} \in \Omega^1(M,\mathfrak{g})$ on 
$P(M,H)$. To give this statement a precise notation, we consider 
the natural projection $\pi : G \to G/H_o : g \mapsto gH_o$. The 
differential of this mapping is a projection of $T_eG = 
\mathfrak{g}$ onto $\mathfrak{p} \simeq T_o dS$. The vierbein is 
obtained from the connection by invoking this projection, 
i.e.~$\bar{e} = \pi_\ast \bar{A}$. This shows clearly that the 
vierbein is a 1-form on $M$ with values in $T_odS$. Nonetheless, 
let us also concentrate on the definition  of $\bar{A}$ itself to 
understand what happens with a tangent vector to spacetime under 
the action of $\bar{e}$, before it ends up in $T_o dS$. The 
definition  was given in Eq.~\eqref{eq:A_nonlin}, which we 
rewrite here for $g = \exp(-\xi\cdot P)$, i.e.~
\begin{displaymath}
	\bar{A} \equiv \Ad(g)\cdot A + (g^{-1})^\ast \theta~.
\end{displaymath}
It should be understood that the adjoint action acts on the 
algebra $\mathfrak{g}$, that $\theta$ is the Maurer-Cartan form 
on $G$ and that $g^\ast$ is the pullback that comes from the 
mapping $g: M \to G : p \mapsto g$. Consider next a vector $X \in 
T_pM$. One then finds,
\begin{align*}
	\bar{A}(X) &= \Ad(g)\cdot A(X) + \theta(g^{-1}_\ast X) \\
		&= L_{g\ast}\Big(R_{g^{-1}\ast}\cdot A(X) + g^{-1}_\ast X 
		\Big)~.
\end{align*}
Denote by $X^\star$ the left invariant vector field on $G$ so 
that
\begin{displaymath}
	X^\star_{g^{-1}} = R_{g^{-1}\ast}\cdot A(X) + g^{-1}_\ast X~.
\end{displaymath}
It follows directly that $\bar{A}(X) = X^\star_e$. Since $\pi 
\circ L_g = \tau_g \circ \pi$, one also has
\begin{displaymath}
	\bar{e}(X) = \pi_\ast L_{g\ast} X^\star_{g^{-1}}
	= \tau_{g\ast} \pi_\ast X^\star_{g^{-1}}~.
\end{displaymath}
Recall that $g = \exp(-\xi\cdot P)$ so that $g^{-1} = 
\exp(\xi\cdot P)$. This implies that $\pi_\ast X^\star_{g^{-1}} 
\in T_\xi dS$, since $\exp(\xi\cdot P)o = \xi \in dS$. The 
element $\bar{e}(X) \in T_o dS$ is the parallel transported 
vector of $\pi_\ast X^\star_{g^{-1}}$, with respect to the 
canonical connection on $G/H_o$ (See Ch. X in 
\cite{kob1996found}).  Therefore, it is understandable that one 
may interpret $\bar{e}$ to be a mapping from the tangent space to 
$M$ at $p$ onto the tangent space to $dS$ at $\xi$, confirming 
the interpretation given by Stelle and West.



\subsection{Discussion}

To conclude let us retrace our steps and try to understand what 
has been going on. We started by introducing a $G$-connection on 
a principal $G$-bundle $P(M,G)$. This contains information about 
a geometry for which the internal symmetry group is $G$. Since 
one is interested in describing a spacetime, whose local geometry 
is invariant under the action of the de Sitter group, this seems 
a good starting point.  However, one does not have a 
canonical---i.e.~consistent with the geometry---spin connection 
and vielbein. This is a crucial shortcoming, as it will not be 
possible to relate the local geometry of the gauge field to the 
geometry of spacetime (no soldering).  By means of a section 
$\xi$, which takes its values in the associated bundle $P 
\times_G G/H$, the principle bundle $P(M,G)$ is reduced to a 
bundle $P(M,H)$.\footnote{For an enlightning proof, see 
	\cite{kob1996found}.} Choosing a section breaks the symmetry 
from $G$ to $H$. As shown in the previous section, $\xi$ can be 
used to construct a Cartan connection $\bar{A}$ on $P(M,H)$ from 
the Ehresmann connection $A$ on $P(M,G)$. This is shown 
schematically in the following diagram:
\begin{displaymath}
	\xymatrix@R=6pt@C=8pt{
			&	\omega & & & \bar{\omega}
		\\
		A
			\ar[ru]^-{\mathfrak{h}}
			\ar[rd]_-{\mathfrak{p}}
			\ar[dddd]
			& & & {\bar{A}}
							\ar[ru]^-{\mathfrak{h}}
							\ar[rd]_-{\mathfrak{p}}
							\ar[dddd]
		\\
			& e & & & \bar{e}
		\\
			& & {\xrightarrow{\hspace*{0.5cm}\xi\hspace*{0.5cm}}}
		\\
			& {R = d\omega + \tfrac{1}{2}[\omega,\omega] + 
				\tfrac{1}{2}[e,e]}
				& & & {\bar{R} = d\bar{\omega} +
							\tfrac{1}{2}[\bar{\omega},\bar{\omega}] +							
							\tfrac{1}{2}[\bar{e},\bar{e}]}
		\\
		F
			\ar[ru]^-{\mathfrak{h}}
			\ar[rd]_-{\mathfrak{p}}
			& & & \bar{F}
							\ar[ru]^-{\mathfrak{h}}
							\ar[rd]_-{\mathfrak{p}}
		\\
			& {T = de + [\omega,e]}
				& & & {\bar{T} = d\bar{e} + [\bar{\omega},\bar{e}]}
	}
\end{displaymath}
The broken symmetries act through a nonlinear realization with 
the elements $h'(\xi,g_0)$, and merely change the point of 
tangency between the local de Sitter fibres and spacetime. On the 
other hand, the unbroken symmetries ($H$) leave the point of 
tangency fixed and act through a linear representation.
Note that the Cartan connection gives rise to a well defined spin 
connection and vierbein, i.e.~they do \emph{not} form an 
irreducible multiplet under the action of $G$. Due to the 
existence of a vierbein $\bar{e}$, spacetime is soldered to the 
de Sitter fibres and one is able to pull back all geometric 
information onto the tangent bundle of spacetime---the arena in 
which takes place gravity. Crucially, one had to make the 
realization nonlinear to have soldering.



\newpage
\appendix
\section{Nested commutators}
\label{app:nested_comm}

\subsection{Notation}
\label{app:nested_comm.not}

For any two elements $X$ and $Y$ of a Lie algebra we define
%
\begin{displaymath}
	X \wedge Y \equiv \mathrm{ad}_X (Y) = [X,Y]
\end{displaymath}
and consequently
\begin{displaymath}
	X^k \wedge Y \equiv \mathrm{ad}_X^k (Y) = 
	[X,[X,\ldots[X,Y]\ldots]]~.
\end{displaymath}
This can be extended to arbitrary functions, where a function is 
considered a power series in $X$, that is
%
\begin{displaymath}
	f(X) \wedge Y = \sum_k c_k X^k \wedge Y~.
\end{displaymath}
Consider a second function $g(X) = \sum_l d_l X^l$. One obtains
%
\begin{displaymath}
	g(X)\wedge f(X)\wedge X = \sum_{kl} c_k d_l 
	\mathrm{ad}_X^l(\mathrm{ad}_X^k(Y)) = \sum_{kl} c_k d_l 
	X^{k+l} \wedge Y = g(X)f(X) \wedge Y~,
\end{displaymath}
where we used the linearity of the adjoint action.
From this result it follows that the equation $f(X) \wedge Y = Z$ 
can be solved for $Y = f(X)^{-1} \wedge Z$. Note that the inverse 
function also is supposed to be expressed as a power series.

To conclude we write down two identities, using the introduced 
notation.  The first is Hadamard's formula
%
\begin{equation}\label{eq:Had}
	\exp(X) Y \exp(-X) = \exp(X) \wedge Y~,
\end{equation}
the other is the Campbell-Poincar\'e fundamental identity,
\begin{equation}\label{eq:CP}
	\exp(-X) \delta\exp(X) = \frac{1-\exp(-X)}{X}\wedge \delta X~.
\end{equation}

\subsection{de Sitter algebra: some results}
\label{app:nested_comm.dS_results}
In this subsection, we compute some intermediary results that are 
used troughout the text. The commutatation relations considered 
are those given by~\eqref{eq:comm_rels_dS}, for the convention 
$\mathfrak{s} = -1$.

The first identity to be verified is
%
\begin{equation}\label{eq:id_dS_comm.1}
	(i\xi\cdot P)^{2n} \wedge \epsilon\cdot P = z^{2n} 
	\bigg(\epsilon\cdot P - \frac{\xi\cdot\epsilon \xi\cdot 
		P}{\xi^2} \bigg)~;\quad n \geqslant 1~.
\end{equation}
Therefore, we compute the sequence
\begin{align*}
	i\xi\cdot P \wedge \epsilon\cdot P
	&= i\xi^a\epsilon^b [P_a,P_b] = l^{-2} \xi^a\epsilon^b 
	M_{ab}~;
	\\
	(i\xi\cdot P)^2 \wedge \epsilon\cdot P
	&= i\xi^c P_c \wedge l^{-2} \xi^a\epsilon^b M_{ab} \\
	&= -il^{-2} \xi^a\epsilon^b\xi^c [M_{ab},P_c] \\
	&= l^{-2}\xi^a\epsilon^b\xi^c (\eta_{ac}P_b - \eta_{bc}P_a) \\
	&= l^{-2} \xi^2 \left(\epsilon\cdot P - \frac{\xi\cdot\epsilon 
			\xi\cdot P}{\xi^2} \right)~;
	\\
	(i\xi\cdot P)^4 \wedge \epsilon\cdot P
	&= l^{-2}\xi^2(i\xi\cdot P)^2 \wedge \left(\epsilon\cdot P - 
		\frac{\xi\cdot\epsilon \xi\cdot P}{\xi^2} \right) \\
	&= l^{-2}\xi^2(i\xi\cdot P)^2 \wedge \epsilon\cdot P \\
	&= (l^{-2}\xi^2)^2 \left(\epsilon\cdot P - 
		\frac{\xi\cdot\epsilon \xi\cdot P}{\xi^2} \right)~; \\
	&~\,\vdots
	\\
	(i\xi\cdot P)^{2n} \wedge \epsilon\cdot P
	&= (l^{-2}\xi^2)^n \left(\epsilon\cdot P - 
		\frac{\xi\cdot\epsilon \xi\cdot P}{\xi^2} \right)~.
\end{align*}
The identity follows by letting $z \equiv 
l^{-1}(\xi^a\xi_a)^{1/2}$.

From~\eqref{eq:id_dS_comm.1} it follows that
%
\begin{displaymath}
\begin{split}
	(i\xi\cdot P)^{2n+1} \wedge \epsilon\cdot P
	&= (i\xi\cdot P) \wedge z^{2n} \bigg(\epsilon\cdot P - 
	\frac{\xi\cdot\epsilon \xi\cdot P}{\xi^2} \bigg) \\
	&= l^{-2}z^{2n} \xi^a \epsilon^b M_{ab}~,
\end{split}
\end{displaymath}
hence, another useful identity is given by
%
\begin{equation}\label{eq:id_dS_comm.2}
	(i\xi\cdot P)^{2n+1} \wedge \epsilon\cdot P = 
	\tfrac{1}{2}l^{-2}z^{2n} (\xi^a\epsilon^b - \xi^b\epsilon^a) 
	M_{ab}~;\quad n \geqslant 0~.
\end{equation}

Finally, the following two identities are derived\footnote{Note 
	that $\delta h \cdot M = \delta h^{ab} M_{ab}$.}
%
\begin{align}
	\label{eq:id_dS_comm.3}
	(i\xi\cdot P)^{2n} \wedge \delta h \cdot M &= \delta h^{ab} 
	l^{-2}z^{2n-2} \xi^c(\xi_b M_{ac} - \xi_a M_{bc})~;\quad n 
	\geqslant 1~, \\
	\label{eq:id_dS_comm.4}
	(i\xi\cdot P)^{2n+1} \wedge \delta h \cdot M &= \delta h^{ab} 
	z^{2n} (\xi_a P_b - \xi_b P_a)~;\quad n \geqslant 0~.
\end{align}
To verify them consider the following series of equations.
%
\begin{align*}
	(i\xi\cdot P)\wedge \delta h \cdot M
	&= \delta h^{ab} \xi^c (-i)[M_{ab},P_c] = \delta h^{ab} (\xi_a 
	P_b - \xi_b P_a)~; \\
	(i\xi\cdot P)^2\wedge \delta h \cdot M
	&= 2 \delta h^{ab} i\xi\cdot P \wedge \xi_a P_b \\
	&= 2 \delta h^{ab} \xi_a \xi^c (-i) [P_b,P_c] \\
	&= 2 \delta h^{ab} l^{-2} \xi_a \xi^c M_{cb} \\
	&= \delta h^{ab} l^{-2} \xi^c (\xi_b M_{ac} - \xi_a M_{bc})~; 
	\\
	(i\xi\cdot P)^4 \wedge \delta h \cdot M
	&= 2\delta h^{ab} l^{-2} \xi_b \xi^c (i\xi\cdot P)^2 \wedge 
	M_{ac} \\
	&= 2\delta h^{ab} l^{-2} \xi_b \xi^c l^{-2} \xi^d (\xi_c 
	M_{ad} - \xi_a M_{cd}) \\
	&= 2 \delta h^{ab} l^{-2} z^2 \xi^d \xi_b M_{ad} \\
	&= \delta h^{ab} l^{-2} z^2 \xi^c (\xi_b M_{ac} - \xi_a 
	M_{bc}) \\
	&~\,\vdots
	\\
	(i\xi\cdot P)^{2n} \wedge \delta h \cdot M
	&= \delta h^{ab} l^{-2} z^{2n-2} \xi^c (\xi_b M_{ac} - \xi_a 
	M_{bc}) \\
	(i\xi\cdot P)^{2n+1} \wedge \delta h \cdot M
	&= 2\delta h^{ab} l^{-2} z^{2n-2} \xi^c \xi_b (-i) 
	[M_{ac},P_d] \\
	&= 2\delta h^{ab} l^{-2} z^{2n-2} (\xi_b\xi_a \xi\cdot P - 
	\xi^2 \xi_b P_a) \\
	&= \delta h^{ab} z^{2n} (\xi_a P_b - \xi_b P_a)~.
\end{align*}


\newpage
\bibliographystyle{ieeetr}
\bibliography{../../references/All}
%\bibliography{/home/hendrik/research/drafts/All.bib}
\end{document}

