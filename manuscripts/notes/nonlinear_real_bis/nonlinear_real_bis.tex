\documentclass[11pt]{article}

%Load preamble files
\usepackage{../../Tex_files/standard}
\usepackage{../../Tex_files/preamble_one}
%\usepackage{showframe} %show frame borders

\usepackage{mathtools}

\usepackage[all]{xy}

\title{Nonlinear realizations bis}
\author{Hendrik}
\date{\today}

\begin{document}

\maketitle

\begin{abstract}
	Omitted appendix of article on cosmological function.
\end{abstract}

\section{Equations}

\begin{equation}
	\label{eq:nonlin_trafo_inf}
	\exp(-i\xi\cdot P) i\epsilon\cdot P \exp(i\xi\cdot P) - 
	\exp(-i\xi\cdot P) \delta\!\exp(i\xi\cdot P) = \tfrac{i}{2} 
	\delta h \cdot M.
\end{equation}

\begin{equation}\label{eq:left_action_group}
	g_0 \exp(i\xi\cdot P) = \exp(i\xi'\cdot P)h'~;
	\qquad
	h' = \tilde{h}'\tilde{h}^{-1},
\end{equation}

\begin{gather}
\label{eq:comm_relations_so(1,4)}
\begin{aligned}
	-i[M_{ab},M_{cd}] &= \eta_{ac}M_{bd} - \eta_{ad}M_{bc} + 
	\eta_{bd}M_{ac} - \eta_{bc}M_{ad}, \\
	-i[M_{ab},P_c] &= \eta_{ac}P_b- \eta_{bc}P_a, \\
	-i[P_a,P_b] &= -l^{-2}M_{ab}.
  \end{aligned}
\end{gather}

\begin{align}
	\label{eq:inf_tr_xi}
	\delta\xi^a &= \epsilon^a + \Big(\frac{z\cosh z}{\sinh z} - 
	1\Big) \bigg(\epsilon^a - \frac{\xi^a \epsilon_b 
		\xi^b}{\xi^2}\bigg), \\
	\label{eq:inf_tr_h}
	\delta h^{ab} &= \frac{1}{l^2} \frac{\cosh z - 1}{z\sinh z} 
	(\epsilon^a\xi^b - \epsilon^b\xi^a),
\end{align}

\begin{subequations}
\label{eqs:nonlin_spin_vier}
\begin{gather}
\label{eq:nonlin_spinconn}
	\omega^{ab} = A^{ab} - \frac{\cosh z - 1}{l^2 z^2} \big[ \xi^a 
	(d\xi^b + A\ind{^b_c} \xi^c) - \xi^b (d\xi^a + A\ind{^a_c} 
	\xi^c) \big] - \frac{\sinh z}{l^2 z} (\xi^a A^b - \xi^b A^a),
%\end{equation}
\\
\begin{multlined}[b][0.851\linewidth]
%\begin{multline}
\label{eq:nonlin_vierbein}
	e^a = A^a + \frac{\sinh z}{z}( d\xi^a + A\ind{^a_b} \xi^b) - 
	\frac{dl}{l} \xi ^a \\
	+ (\cosh z - 1) \bigg( A^a - \frac{\xi^b A_b \xi^a}{\xi^2} 
	\bigg) - \bigg( \frac{\sinh z}{z} - 1 \bigg) \frac{\xi^b 
		d\xi_b \xi^a}{\xi^2}.
\end{multlined}
\end{gather}
\end{subequations}

\begin{equation}
	\label{eq:A_nonlin}
	\bar{A} = \mathrm{Ad}(\exp(-i\xi\cdot P))(A + d).
\end{equation}


\section{Nonlinear realizations}
\label{app:nonlin_real}

In this appendix, some intermediate steps that are useful to 
verify the results of Sec.~4 are explained; see 
also~\cite{Zumino1977189}.

For any two elements of a Lie algebra $\mathfrak{g}$, the adjoint 
action is denoted by
\begin{displaymath}
	\wedge: \mathfrak{g} \times \mathfrak{g} \to \mathfrak{g} : 
	(X,Y) \mapsto X \wedge Y = \mathrm{ad}_X (Y) = [X,Y].
\end{displaymath}
Note that the symbol $\wedge$ for the adjoint action of a Lie 
algebra is exclusive to this section, and should not be confused 
with the wedge product of differential forms, which is exclusive 
to the other sections throughout the text.

Moreover, we write
\begin{displaymath}
	X^k \wedge Y = \mathrm{ad}_X^k (Y) = 
	[X,[X,\ldots[X,Y]\ldots]],
\end{displaymath}
so that for a power series $f(X) = \sum_k c_k X^k$
\begin{displaymath}
	f(X) \wedge Y = \sum_k c_k X^k \wedge Y.
\end{displaymath}
Given a second function $g(X) = \sum_l d_l X^l$, one obtains
\begin{displaymath}
	g(X)\wedge f(X)\wedge X = \sum_{kl} c_k d_l 
	\mathrm{ad}_X^l(\mathrm{ad}_X^k(Y)) = \sum_{kl} c_k d_l 
	X^{k+l} \wedge Y = g(X)f(X) \wedge Y,
\end{displaymath}
because of the linearity of the adjoint action.
From this result it follows that the equation $f(X) \wedge Y = Z$ 
can be solved for $Y = f(X)^{-1} \wedge Z$. Note that the inverse 
function is supposed to be expressed as a power series. The 
following two identities are useful in carrying out the 
calculations of this section. The first is Hadamard's formula, 
namely,
\begin{equation}\label{eq:Had}
	\exp(X) Y \exp(-X) = \exp(X) \wedge Y.
\end{equation}
The other is the Campbell-Poincar\'e fundamental identity, given 
by
\begin{equation}\label{eq:CP}
	\exp(-X) \delta\!\exp(X) = \frac{1-\exp(-X)}{X}\wedge \delta X.
\end{equation}

Let us then solve~\eqref{eq:nonlin_trafo_inf} for $\delta\xi\cdot 
P$ and $\delta h \cdot M$.  Since $\mathfrak{g} = 
\mathfrak{so}(1,4) = \mathfrak{so}(1,3) \oplus \mathfrak{p}$ is 
symmetric, there is an involutive automorphism $\sigma : 
\mathfrak{g} \to \mathfrak{g}$ such that $\mathfrak{so}(1,3)$ is 
an eigenspace with eigenvalue $1$, while $\mathfrak{p} = 
\mathfrak{so}(1,4)/\mathfrak{so}(1,3)$ is an eigenspace with 
eigenvalue $-1$~\cite{kob1996found2}. This automorphism allows 
one to eliminate $\delta h$ from~\eqref{eq:nonlin_trafo_inf}, 
which leads to the expression \begin{displaymath}
\begin{multlined}[b][\linewidth-2\multlinegap]
	\quad\quad
	\frac{1-\exp(-i\xi\cdot P)}{i\xi\cdot P} \wedge 
	i\delta\xi\cdot P - \frac{1-\exp(i\xi\cdot P)}{i\xi\cdot P} 
	\wedge i\delta\xi\cdot P \\
	= \exp(-i\xi\cdot P) \wedge i\epsilon\cdot P + \exp(i\xi\cdot 
	P) \wedge i\epsilon\cdot P,
	\quad\quad
\end{multlined}
\end{displaymath}
and where we made use of the identities~\eqref{eq:Had} 
and~\eqref{eq:CP}. This equation is solved for $\delta \xi^a$, 
which results in
\begin{equation}\label{eq:inftrafo_cosetpar}
	i\delta\xi \cdot P = \frac{i\xi\cdot P\,\cosh(i\xi\cdot 
		P)}{\sinh(i\xi\cdot P)} \wedge i\epsilon\cdot P.
\end{equation}
Substituting~\eqref{eq:inftrafo_cosetpar} 
in~\eqref{eq:nonlin_trafo_inf}, one subsequently solves for 
$\delta h^{ab}$:
\begin{equation}\label{eq:inftrafo_h}
	\tfrac{i}{2}\delta h \cdot M = \frac{1-\cosh(i\xi\cdot 
		P)}{\sinh(i\xi\cdot P)} \wedge i\epsilon\cdot P,
\end{equation}
which relates the infinitesimal element $h'(\xi,\epsilon) = 1 + 
\delta h$ in~\eqref{eq:left_action_group} to the
transvection $g_0 = 1 + i\epsilon\cdot P$. In the 
expressions~\eqref{eq:inftrafo_cosetpar} 
and~\eqref{eq:inftrafo_h}, the hyperbolic functions are given by 
the corresponding power series in $i\xi\cdot P$. They act on 
$i\epsilon\cdot P$ through the adjoint action, which can be 
worked out explicitly for the commutation 
relations~\eqref{eq:comm_relations_so(1,4)}. Doing so, the 
following intermediate results are useful ($z = l^{-1} \xi$ and 
$\xi = (\eta_{ab} \xi^a \xi^b)^{1/2}$):
\begin{align*}
	%\label{eq:id_dS_comm.1}
	(i\xi\cdot P)^{2n} \wedge \epsilon\cdot P &= z^{2n} 
	\bigg(\epsilon\cdot P - \frac{\xi\cdot\epsilon \xi\cdot 
		P}{\xi^2} \bigg)~;\quad n \geqslant 1,
	\\
	%\label{eq:id_dS_comm.2}
	(i\xi\cdot P)^{2n+1} \wedge \epsilon\cdot P &= 
	\tfrac{1}{2}l^{-2}z^{2n} (\xi^a\epsilon^b - \xi^b\epsilon^a) 
	M_{ab}~;\quad n \geqslant 0,
	\\
	%\label{eq:id_dS_comm.3}
	(i\xi\cdot P)^{2n} \wedge \delta h \cdot M &= \delta h^{ab} 
	l^{-2}z^{2n-2} \xi^c(\xi_b M_{ac} - \xi_a M_{bc})~;\quad n 
	\geqslant 1
	\\
	%\label{eq:id_dS_comm.4}
	\text{and}\quad
	(i\xi\cdot P)^{2n+1} \wedge \delta h \cdot M &= \delta h^{ab} 
	z^{2n} (\xi_a P_b - \xi_b P_a)~;\quad n \geqslant 0.
\end{align*}
By means of these relations, the 
variations~\eqref{eq:inftrafo_cosetpar} and~\eqref{eq:inftrafo_h} 
are found to be given by
\begin{displaymath}
	i\delta\xi \cdot P = i\epsilon\cdot P + \bigg(\frac{z\cosh 
		z}{\sinh z} - 1\bigg) \bigg(i\epsilon\cdot P - 
	\frac{\xi\cdot\epsilon \, i\xi\cdot P}{\xi^2}\bigg),
\end{displaymath}
and
\begin{displaymath}
	\tfrac{i}{2}\delta h \cdot M = \frac{i}{2l^2} \frac{\cosh z - 
		1}{z\sinh z} (\epsilon^a\xi^b - \epsilon^b\xi^a) M_{ab},
\end{displaymath}
confirming the equalities~\eqref{eq:inf_tr_xi} 
and~\eqref{eq:inf_tr_h}.

Next we verify that the spin connection and vierbein in the 
nonlinear de~Sitter-Cartan geometry are given 
by~\eqref{eq:nonlin_spinconn} and~\eqref{eq:nonlin_vierbein}.  
Invoking Hadamard's formula~\eqref{eq:Had} and the 
Campbell-Poincar\'e fundamental identity~\eqref{eq:CP}, the 
right-hand side of~\eqref{eq:A_nonlin} can be rewritten as
\begin{displaymath}
	\exp(-i\xi\cdot P) \wedge \big( \tfrac{i}{2} A^{ab} M_{ab} + i 
	A^a P_a \big) + \frac{1 - \exp(-i\xi\cdot P)}{i\xi\cdot P} 
	\wedge d (i\xi\cdot P).
\end{displaymath}
Working out the different terms of this expression, it is found 
successively that
\begin{gather*}
\begin{multlined}[b][0.85\linewidth]
	\exp(-i\xi\cdot P) \wedge \tfrac{i}{2} A^{ab} M_{ab}
	= \tfrac{i}{2} \big( A^{ab} + \frac{\cosh z - 1}{l^2 z^2} 
	\xi_c ( \xi^b A^{ac} - \xi^a A^{bc} ) \big) M_{ab} \\
	+ i \big( z^{-1} \sinh z A\ind{^a_b} \xi^b \big) P_a,
\end{multlined}
\\
\begin{multlined}[b][0.85\linewidth]
	\exp(-i\xi\cdot P) \wedge i A^a P_a = \tfrac{i}{2} \bigg( 
	\frac{\sinh z}{l^2 z} (A^a \xi^b - A^b \xi^a ) \bigg) M_{ab} 
	\\
	+ i \bigg( A^a + (\cosh z - 1) \bigg(A^a - \frac{\xi^b A_b 
		\xi^a}{\xi^2}  \bigg) \bigg) P_a,
\end{multlined}
\\
%\intertext{and}
\begin{multlined}[b][0.85\linewidth]
	\frac{1 - \exp(-i\xi\cdot P)}{i\xi\cdot P} \wedge d (i\xi\cdot 
	P) = \tfrac{i}{2} \bigg( \frac{\cosh z - 1}{l^2 z^2} (d\xi^a 
	\xi^b - d\xi^b \xi^a) \bigg) M_{ab} \\
	+ i \bigg( \frac{\sinh z}{z} \bigg( d\xi^a - \frac{\xi^b 
		d\xi_b \xi^a}{\xi^2} \bigg) + \frac{\xi^b d\xi_b 
		\xi^a}{\xi^2} - \frac{dl}{l} \xi^a \bigg) P_a.
\end{multlined}
\end{gather*}
Collecting these different contributions and separating terms 
according to whether they are valued in~$\mathfrak{so}(1,3)$, 
respectively~$\mathfrak{p}$, one 
recovers~\eqref{eq:nonlin_spinconn}
and~\eqref{eq:nonlin_vierbein}.

%%%%%%%%%%%%%%%%
% BIBLIOGRAPHY %
%%%%%%%%%%%%%%%%

\providecommand{\href}[2]{#2}
\begingroup\raggedright
\begin{thebibliography}{1}

\bibitem{Zumino1977189}
B.~Zumino, ``Non-linear realization of supersymmetry in anti de~{S}itter
  space,'' \href{http://dx.doi.org/10.1016/0550-3213(77)90211-5}{{\em Nucl.
  Phys.} {\bfseries B127} (1977) 189--201}.

\bibitem{kob1996found2}
S.~Kobayashi and K.~Nomizu, {\em Foundations of Differential Geometry}, vol.~2.
\newblock Wiley-Interscience, New York, 1996.
\newblock Reprint of the 1969 original.

\end{thebibliography}
\endgroup

\end{document}

