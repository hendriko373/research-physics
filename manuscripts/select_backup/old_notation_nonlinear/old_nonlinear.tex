\documentclass[11pt]{article}

%Load preamble files
\usepackage{../../Tex_files/standard}
\usepackage{../../Tex_files/preamble_one}
%\usepackage{showframe} %show frame borders

\usepackage[all]{xy}

\title{Nonlinear realizations}
\author{Hendrik}
\date{\today}

\begin{document}

%\begin{abstract}
%	An introduction to non-linear realizations is given and the 
%	example for the de Sitter algebra is worked out explicitly.
%\end{abstract}

\maketitle
\tableofcontents

\section{Introduction}

In the quest for a theory of gravity whose local kinematical 
structure is invariant under the de Sitter group, we try to find 
an appropriate geometrical language. Indeed, if one is of the 
opinion that kinematics are described by geometry and dynamics by 
Lagrangians, \emph{a priori} it is the former that may have to be 
given another underlying algebraic structure. In more precise 
terms, the desired geometry describes a 
manifold---spacetime---that locally reduces to a de Sitter space 
whose cosmological constant is allowed to vary along the 
manifold. Loosely stated, geometry is the description of an 
Ehresmann connection on a principal $G$ bundle.\footnote{For a 
	nuanced discussion on this statement, see the introduction to 
	\cite{sharpe1997diff_geo}.} The structure group $G$ contains 
the symmetry transformations of the internal (associated) fibres.  
Consequently, for the case at hand it makes sense to let $G = 
SO(1,4)$ and to consider the associated bundle of homogeneous de 
Sitter spaces, i.e.~$E = P \times_G G/H$.  A nonvanishing 
curvature of the gauge field reflects the nonhomogeneity of the 
underlying base manifold, i.e.~spacetime.  However, this is not 
the end of the story.  Gravitational theories live in the tangent 
structure of spacetime $M$, which naturally is lacking any 
information about the geometry of the principal bundle. The 
structure that is needed to be able to see the geometry in the 
tangent bundle is a vielbein.  Such a well defined field
would solder the internal de Sitter fibres to spacetime, but is 
missing in the general case of Ehresmann connections. This makes 
it clear that these type of connections are not appropriate for 
the kind of geometry to be described in theories of gravity.  
Rather, one has to specify for \emph{Cartan connections}. These 
objects are $\mathfrak{g}$-valued connections on a principal 
bundle $P(M,H)$ such that the dimension of 
$\mathfrak{g}/\mathfrak{h}$ equals the dimension of $M$.  Given 
this equality in dimension, one understands that a vielbein 
directly is at hand.

Nonetheless, Cartans connections are Ehresmann connections---but 
not \emph{vice versa}, so that one should be able to reconstruct 
a useful Cartan connection from the right Ehresmann connection.
This is the place where symmetry breaking and nonlinear 
realizations come into play.\footnote{For a concise review on the 
	role of symmetry breaking in gravity, 
	see~\cite{Wise:2011-sym.br}.} It is a standard result in 
differential geometry that sections $\xi$ of $P \times_G G/H$ are 
in one--to--one correspondence to reductions $\imath$ of $P(M,G)$ 
to $P(M,H)$, schematically expressed as\footnote{A proof of this 
	statement is given in~\cite{kob1996found}.}
\begin{displaymath}
\begin{tikzcd}[row sep=large]
	P(M,H)	\arrow{r}[color=red]{\imath}
				\arrow[end anchor=north west]{dr}
				& P(M,G)
					\arrow{r}
					\arrow{d}
				& P \times_G G/H
					\arrow[end anchor=north east]{dl}	\\
				& M \arrow[dashed,bend right,
						start anchor={[xshift=+2pt]real east},
						end anchor={[xshift=+15pt]south west}
						]{ur}[swap,color=red]{\sigma}
\end{tikzcd}
\end{displaymath}
The reduction can be understood as a symmetry breaking process.  
At any point $p$ of spacetime, the section singles out some point 
of the homogeneous de Sitter spaces. In this manner the symmetry 
group $G$ of the homogeneous space is broken to the isotropy 
group $H_\xi$ of the point $\xi(p)$. The broken symmetries 
translate the section $\xi(p) \mapsto g(p)\xi(p)$ for which the 
isotropy groups are related by the adjoint action, i.e.~$H_{g\xi} 
= \Ad(g)\, H_\xi$. Going along the same line, a 
$\mathfrak{g}$-valued Ehresmann connection gives way to a 
$\mathfrak{g}$-valued Cartan connection on the reduced bundle 
\cite{sharpe1997diff_geo}.  If the algebra $\mathfrak{g}$ is 
reductive, the broken directions $\mathfrak{g}/\mathfrak{h}_\xi 
\simeq T_\xi dS$ can identified with the tangent structure at 
$p$, effectively introducing a vielbein. Since the algebra is 
reductive, the identification remains consistent under local 
gauge transformations. In the remainder of this text, we will 
explicitly construct such a Cartan geometry on a principal 
$SO(3,1)$ bundle starting from an Ehresmann connection on a 
principal $SO(4,1)$ bundle. To this purpose we use the techniques 
of~\cite{Stelle:1979va,stelle.west:1980ds}, however slightly 
generalizing for a spacetime dependent de Sitter length scale.


\section{Nonlinear realizations}

The theory of nonlinear realizations of Lie groups on homogeneous 
manifolds was introduced in the context of spontaneous symmetry 
breaking \cite{Coleman:1969sm,Callan:1969sn,Volkov:1973vd} and 
has been applied in the context of supersymmetry and supergravity 
[citations]. 


\subsection{Nonlinear realizations}

Consider a Lie group $G$ of dimension $n$ for which $H$ is a 
$d$-dimensional closed subgroup. It is assumed that there is a 
reductive splitting on the level of the Lie algebras, that is 
$\mathfrak{g} = \mathfrak{h} \oplus \mathfrak{p}$ so that 
$[\mathfrak{h},\mathfrak{h}] \subset \mathfrak{h}$ and 
$[\mathfrak{h},\mathfrak{p}] \subset \mathfrak{p}$. Given a 
homogeneous space $S$ that is symmetric under the left action of 
$G$,
\begin{equation}
	\tau_g : S \to S : p \mapsto gp~,
\end{equation}
and for which the isotropy group of a given point $o$ is given by 
$H_o \simeq H$, there is an isomorphism between $S \simeq G/H_o$ 
due to $g \in G \leftrightarrow \tau_g(o) \in S$.

Let $\mathfrak{p} = \mathrm{span}\{P_a\}$ $(a = 1\ldots n-d)$.  
Within some neighborhood of the identity, a group element of $G$ 
can be represented (uniquely(?)) in the form
%
\begin{displaymath}
	g = \exp(\xi\cdot P) \tilde{h}~,
\end{displaymath}
where $\tilde{h}$ is an element of the stability subgroup $H$ and 
$\xi\cdot P = \xi^a P_a$.  The $\xi^a$ parametrize the coset 
space $G/H$---at least in some neigborhood of the identity---and 
can be considered a coordinate system of the manifold $M$. It 
follows that the left action of $G$ on itself can be written as
%
\begin{displaymath}
	g_0 g = \exp(\xi'\cdot P)\tilde{h}'
\end{displaymath}
or
\begin{equation}\label{eq:left_action_group}
	g_0 \exp(\xi\cdot P) = \exp(\xi'\cdot P)h'~;
	\qquad
	h' := \tilde{h}'\tilde{h}^{-1}~,
\end{equation}
where $\xi'=\xi'(g_0,\xi)$ and $h'=h'(g_0,\xi)$ depend on the 
indicated variables. Given a linear representation of $H$,
\begin{displaymath}
	h: \psi \mapsto D(h)\psi~,
\end{displaymath}
a nonlinear realization of $G$ can be constructed through
\begin{equation}\label{eq:nonlin_real}
	g_0 :~\xi \mapsto \xi'~,~\psi \mapsto D(h')\psi~.
\end{equation}
To show that it is a realization let us consider
%
\begin{align*}
	g_0\exp(\xi\cdot P) &= \exp(\xi'\cdot P) h' \\
	g_1\exp(\xi'\cdot P) &= \exp(\xi''\cdot P) h'' \\
	(g_1g_0)\exp(\xi\cdot P) &= \exp(\xi'''\cdot P) h'''
\end{align*}
Since the first two equations also imply that
$g_1g_0\exp(\xi\cdot P) = \exp(\xi''\cdot P)h''h'$, we see that 
$h''' = h''h'$ and hence
%
\begin{displaymath}
	D(h''') = D(h'')D(h')~,
\end{displaymath}
because $D$ is a representation of $H$. Furthermore, one 
concludes that $\xi''' = \xi''$ so that $(g_1g_0)\xi = 
g_1(g_0\xi)$, which implies that the transformation of $G$ on 
$\xi$ can also be considered a group realization. Remark that the 
composition $h''h'$ depends on the transformation of $\xi$ so 
that the realization \eqref{eq:nonlin_real} only is meaningful 
together with the transformation properties of $\xi$. {\blu The 
	latter transform in a nonlinear way under the $G$ action.  
	Therefore, the realization on $\psi$ is also said to be 
	nonlinear and the $D(h')$ constitute a nonlinear realization 
	of $G$.}

Consider the case for which $g_0 = h_0$ is an element of the 
isotropy subgroup $H$. From the general action of $G$ as given in 
\eqref{eq:left_action_group} one obtains
%
\begin{displaymath}
	h_0 \exp(\xi\cdot P) h_0^{-1}h_0 = \exp(\xi'\cdot P) h'~.
\end{displaymath}
On the other hand, it is well-known that the derivative of $h_0$ 
at the identity of $G/H$ is a linear automorphism given by the 
adjoint representation,\footnote{This result is valid only for 
	reductive algebras, the ones considered here.}
\begin{displaymath}
	h_{0\ast}: \xi\cdot P \mapsto \xi\cdot \mathrm{Ad}(h_0)P =: 
	\xi'\cdot P~,
\end{displaymath}
so that $h_0 : \xi \mapsto \xi'$ is a linear transformation. For 
matrix groups this amounts to conjugation so that we also have
\begin{displaymath}
	\exp(\xi'\cdot P) = h_0\exp(\xi\cdot P)h_0^{-1}\quad
	\text{and}\quad
	h' = h_0~.
\end{displaymath}
It is manifest that $h'$ does not depend on $\xi$. Consequently, 
for these elements the realization~\eqref{eq:nonlin_real} is a 
linear representation.  One concludes that the nonlinear action 
of $G$ reduces to a linear action when restricted to the isotropy 
subgroup $H$.

On the other hand, consider the case where the elements are of 
the form $g_0 = \exp(\xi_0\cdot P)$. 
Then~\eqref{eq:left_action_group} becomes
%
\begin{equation}\label{eq:left_action_coset}
	\exp(\xi_0\cdot P)\exp(\xi\cdot P) = \exp(\xi'\cdot P)h'~.
\end{equation}


\subsection{Infinitesimal transformations}

Let us focus on the case where an element of $G$ lies 
infinitesimally close to the identity, namely $g_0 = e + \delta 
g_0$ with $\delta g_0 \in \mathfrak{g}$.  Then to first order in 
$\delta g_0$ we have
%
\begin{displaymath}
	\exp(\xi'\cdot P) = \exp(\xi\cdot P) + \delta\exp(\xi\cdot P)~,
	\quad	h' = (\tilde{h} + \delta \tilde{h})\tilde{h}^{-1} = e + 
	\delta h~.
\end{displaymath}
Substituting this into \eqref{eq:left_action_group} and retaining 
terms up to first order in $\delta g_0$, we get
%
\begin{displaymath}
	\exp(-\xi\cdot P) \delta g_0 \exp(\xi\cdot P) - \exp(-\xi\cdot 
	P) \delta\exp(\xi\cdot P) = \delta h~.
\end{displaymath}

If the elements are pure translations, i.e.~of the form $g_0 = 
\exp(\xi_0\cdot P)$, the transformation parameters satisfy the 
equation (let $\delta g_0 = \epsilon\cdot P$)
\begin{equation}\label{eq:nonlin_trafo_inf}
	\exp(-\xi\cdot P) \epsilon\cdot P \exp(\xi\cdot P) - 
	\exp(-\xi\cdot P) \delta\exp(\xi\cdot P) = \delta h~.
\end{equation}


\subsection{Symmetric Lie algebra}

Assume the Lie algebra $\mathfrak{g}$ is not only reductive but 
also symmetric. This means that there is an involutive 
automorphism $\sigma : \mathfrak{g} \to \mathfrak{g}$ such that 
$\mathfrak{h}$ is an eigenspace with eigenvalue $1$, while 
$\mathfrak{p}$ is an eigenspace with eigenvalue $-1$. Group 
elements of $H$ that are obtained by exponentiation of elements 
in $\mathfrak{h}$ are invariant under $\sigma$, while elements 
generated by elements of $\mathfrak{p}$ are mapped into their 
inverse. The automorphism directly implies a third restriction on 
the commutation relations of $\mathfrak{g}$, 
i.e.~$[\mathfrak{p},\mathfrak{p}] \subset \mathfrak{h}$.

We can consider the automorphism $\sigma$ acting 
on~\eqref{eq:left_action_group} together with the original 
equation, after which $h' = \sigma(h')$ can be eliminated. Doing 
so, this yields
%
\begin{equation}
	g_0 \exp(2\xi\cdot P) \sigma(g_0^{-1}) = \exp(2\xi'\cdot P)~.
\end{equation}
Written this way, it is manifest that $g_0 : \xi \mapsto \xi'$ is 
a group realization and, when restricted to $H$, this realization 
becomes a linear representation.

In the following, we concentrate on infinitisimal transformations 
under pure translations, namely those that are of the form $g_0 = 
e + \epsilon\cdot P$.  Therefore, we first use the information 
that $\mathfrak{g}$ is symmetric to eliminate $\delta h$ from 
eq.~\eqref{eq:nonlin_trafo_inf}, which leads to
%
\begin{multline*}
	\quad
	\exp(-\xi\cdot P)\delta\exp(\xi\cdot P) - \exp(\xi\cdot 
	P)\delta\exp(-\xi\cdot P) \\= \exp(-\xi\cdot P) \epsilon\cdot 
	P \exp(\xi\cdot P) + \exp(\xi\cdot P) \epsilon\cdot P 
	\exp(-\xi\cdot P)~.
	\quad
\end{multline*}
Using eqs.~\eqref{eq:Had} and~\eqref{eq:CP} one finds
%
\begin{multline*}
	\quad\quad
	\frac{1-\exp(-\xi\cdot P)}{\xi\cdot P} \wedge \delta\xi\cdot P 
	- \frac{1-\exp(\xi\cdot P)}{\xi\cdot P} \wedge \delta\xi\cdot 
	P \\
	= \exp(-\xi\cdot P) \wedge \epsilon\cdot P + \exp(\xi\cdot P) 
	\wedge \epsilon\cdot P~.
	\quad\quad
\end{multline*}
The expression can be solved for $\delta\xi\cdot P$, leading to
%
\begin{equation}\label{eq:inftrafo_cosetpar}
	\delta\xi \cdot P = \frac{\xi\cdot P\,\cosh(\xi\cdot 
		P)}{\sinh(\xi\cdot P)} \wedge \epsilon\cdot P~.
\end{equation}
This result gives the infinitesimal change in coset parameters 
due to an infinitesimal pure translation $\epsilon\cdot P$.  
Remember that it is only valid for symmetric spaces. To 
explicitly solve for $\delta \xi ^a$, one needs the specific 
commutation relations of the underlying algebra.

\blankline
Given the nonlinear transformation~\eqref{eq:inftrafo_cosetpar}, 
one can work out the corresponding 
transformation~\eqref{eq:nonlin_trafo_inf} for $\delta h$. In 
other words, for symmetric algebras we may specify further the 
transformation $h'(\xi,\epsilon)$ under a pure translation 
$\delta g_0 = \epsilon\cdot P$.

Using eqs.~\eqref{eq:Had} and~\eqref{eq:CP}, we find 
from~\eqref{eq:nonlin_trafo_inf} that $\delta h \equiv \delta h 
\cdot M \in \mathfrak{h}$ is equal to\footnote{This result is 
	true for all reductive algebras, that is symmetric \emph{and} 
	non-symmetric.}
%
\begin{equation}
	\delta h \cdot M = \exp(-\xi\cdot P) \wedge \epsilon\cdot P - 
	\frac{1-\exp(-\xi\cdot P)}{\xi\cdot P} \wedge \delta\xi \cdot 
	P~.
\end{equation}
In the case of symmetric Lie algebras one may substitute 
for~\eqref{eq:inftrafo_cosetpar}. After some algebra it is found 
that
%
\begin{equation}\label{eq:inftrafo_h}
	\delta h \cdot M = \frac{1-\cosh(\xi\cdot P)}{\sinh(\xi\cdot 
		P)} \wedge \epsilon\cdot P~.
\end{equation}


\subsection{From linear to nonlinear realizations}

Let $\psi$ be a field belonging to some linear representation 
$\sigma$ of $G$, that is
%
\begin{equation}
	g : \psi(x) \mapsto \psi'(x) = \sigma(g) \psi~.
\end{equation}
Given a section $\xi$ of $P \times_G G/H$, as considered above, a 
nonlinear realization of $G$ is constructed by
%
\begin{equation}\label{eq:def_nonlinear_field}
	\bar{\psi}(x) \equiv \sigma(\exp(-\xi\cdot P))\psi~.
\end{equation}
This field transforms nonlinearly \emph{and only with respect to 
	its $H$-indices} under the action of a generic element $g_0$ 
of $G$:
%
\begin{displaymath}
\begin{split}
	\bar{\psi}'(x) &= \sigma(\exp(-\xi'\cdot P)) \psi'(x) \\
	&= \sigma(\exp(-\xi'\cdot P)g_0) \psi(x) \\
	&= \sigma(\exp(-\xi'\cdot P)g_0\exp(\xi\cdot P)) \bar{\psi}(x) 
	\\
	&= \sigma(h'(\xi,g_0)) \bar{\psi}(x)~.
\end{split}
\end{displaymath}
It follows that a linear irreducible representation of $G$ 
becomes a nonlinear and reducible representation. The price to be 
paid for getting irreducible $H$-representations is that they 
transform in a nonlinear way. Nonetheless, when restricted to the 
isotropy group $H$, the field~\eqref{eq:def_nonlinear_field} 
transforms according to a linear representation.

\section{An example: de Sitter space}

\subsection{Transformation of group parameters}

In this subsection, the change of the group parameters $\xi^a$ 
and $\delta h^{ab}$ due to infinitesimal de Sitter translations 
are calculated. Remember that the coordinates $\xi^a$ are defined 
by the exponentiation of elements of $\mathfrak{p}$. They are 
also referred to as Goldstone fields, because of the resemblance 
of their role in the scheme of spontaneous symmetry breaking in 
field theory. As we have reviewed in the last section, these 
coordinates transform according to a nonlinear realization of the 
full symmetry group $G$. On the other hand, they transform 
linearly when the action is restricted to the subgroup $H$ of 
unbroken symmetries.  One understands that the pure translations 
are the set of transformations that are responsible for the 
nonlinear behaviour.\footnote{In general these elements do not 
	form a group.}

Let us begin by recalling the de Sitter commutation relations 
that involve translations, i.e.\footnote{An element of $SO(4,1)$ 
	is given by $\exp(\tfrac{i}{2}\omega^{ab}M_{ab} + i\xi^a 
	P_a)$.}
%
\begin{equation}\label{eq:comm_rels_dS}
	\begin{split}
		-i[M_{ab},P_c] &= \eta_{ac}P_b - \eta_{bc}P_a \\
		-i[P_a,P_b] &= \mathfrak{s}l^{-2} M_{ab}
	\end{split}
\end{equation}
with $\mathfrak{s} \equiv \eta_{44}$. The de Sitter translations 
were introduced as $P_a \equiv l^{-1}(x)M_{a4}$, whilst the 
$M_{ab}$ span the Lorentz subalgebra $\mathfrak{h} = 
\mathfrak{so}(3,1)$.  It is manifest that the de Sitter algebra 
is symmetric.  In what follows we adhere to the convention 
$\mathfrak{s} = -1$ so that $\eta_{ab} = 
\mathrm{diag}(1,-1,-1,-1)$.

The transformation of the coset parameters~$\xi^a$ under an 
infinitesimal de Sitter translation $\epsilon\cdot P$ is given by 
\eqref{eq:inftrafo_cosetpar}. In the parametrization used in this 
section, this can be rewritten as
%
\begin{equation}\label{eq:inftrafo_cosetpar_dS}
	\delta\xi \cdot P = \frac{i\xi\cdot P\,\cosh(i\xi\cdot 
		P)}{\sinh(i\xi\cdot P)} \wedge \epsilon\cdot P~.
\end{equation}
Recall that the left hand side should be understood as a
power series in the adjoint action (see also Appendix 
\ref{app:nested_comm.not}).  The power series of the relevant 
hyperbolic functions have the form\footnote{The coefficients in 
	the power series for the hyperbolic cosecant are $c_{2n} = 
	2(1-2^{2n-1}) B_{2n}$ with $B_n$ the $n$-th Bernouilli 
	number.}
%
\begin{align*}
	\cosh(i\xi\cdot P) &= \sum_{n=0}^\infty \frac{(i\xi\cdot 
		P)^{2n}}{(2n)!}~, \\
	\mathrm{csch}(i\xi\cdot P) &= (i\xi\cdot P)^{-1} + 
	\sum_{n=1}^\infty \frac{c_{2n}}{(2n)!} (i\xi\cdot P)^{2n-1}~.
\end{align*}
%
Invoking the identity~\eqref{eq:id_dS_comm.1}, one is able to 
work out the cosinus hyperbolicus, i.e.~ 
%
\begin{equation}\label{eq:cosh_transl_dS}
\begin{split}
	\cosh(i\xi\cdot P) \wedge \epsilon\cdot P
	&= \epsilon\cdot P + \sum_{n=1}^\infty 
	\frac{(l^{-1}\xi)^{2n}}{(2n)!} \wedge \left(\epsilon\cdot P - 
		\frac{\xi\cdot\epsilon \xi\cdot P}{\xi^2} \right) \\
	&= \cosh(l^{-1}\xi) \left(\epsilon\cdot P - 
		\frac{\xi\cdot\epsilon \xi\cdot P}{\xi^2} \right) + 
	\frac{\xi\cdot\epsilon \xi\cdot P}{\xi^2}~.
\end{split}
\end{equation}
By equal means, the right hand side 
of~\eqref{eq:inftrafo_cosetpar_dS} is found to be
\begin{displaymath}
\begin{split}
	i\xi\cdot P \,\mathrm{csch}(i\xi\cdot P) &\wedge 
	\cosh(i\xi\cdot P) \wedge \epsilon\cdot P \\
	=& \bigg( \mathbb{1} + \sum_{n=1}^\infty \frac{c_{2n}}{(2n)!} 
	(i\xi\cdot P)^{2n} \bigg)\wedge \bigg[\cosh(l^{-1}\xi)
	\bigg(\epsilon\cdot P -\frac{\xi\cdot\epsilon \xi\cdot 
		P}{\xi^2}\bigg) + \frac{\xi\cdot\epsilon \xi\cdot P}{\xi^2} 
	\bigg] \\
	=& \cosh(l^{-1}\xi)\bigg(1 + \sum_{n=1}^\infty 
	\frac{c_{2n}}{(2n)!}(l^{-1}\xi)^{2n} \bigg) 
	\bigg(\epsilon\cdot P -\frac{\xi\cdot\epsilon \xi\cdot 
		P}{\xi^2}\bigg) + \frac{\xi\cdot\epsilon \xi\cdot P}{\xi^2} 
	\\
	=& \cosh(l^{-1}\xi)(l^{-1}\xi)\,\mathrm{csch}(l^{-1}\xi) 
	\bigg(\epsilon\cdot P -\frac{\xi\cdot\epsilon \xi\cdot 
		P}{\xi^2}\bigg) + \frac{\xi\cdot\epsilon \xi\cdot P}{\xi^2} 
	\\
	=& \epsilon\cdot P + \frac{l^{-1}\xi	
		\cosh(l^{-1}\xi)}{\sinh(l^{-1}\xi)} \bigg(\epsilon\cdot P - 
	\frac{\xi\cdot\epsilon \xi\cdot P}{\xi^2}\bigg) + 
	\frac{\xi\cdot\epsilon \xi\cdot P}{\xi^2} - \epsilon\cdot P ~.
\end{split}
\end{displaymath}
The introduction of the extra $\epsilon\cdot P$ terms in the last 
line is just a matter of convention, which allows one to write 
eq.~\eqref{eq:inftrafo_cosetpar_dS} as
\begin{equation}\label{eq:sol_inftrafo_cosetpar_dS}
	\delta\xi \cdot P = \epsilon\cdot P + 
	\bigg(\frac{l^{-1}\xi\cosh(l^{-1}\xi)}{\sinh(l^{-1}\xi)} - 
	1\bigg) \bigg(\epsilon\cdot P - \frac{\xi\cdot\epsilon 
		\xi\cdot P}{\xi^2}\bigg)~.
\end{equation}
This implies that the infinitesimal change of the coset 
parameters is given by
%
\begin{equation}
	\delta\xi^a = \epsilon^a + \Big(\frac{z\cosh z}{\sinh z} - 
	1\Big) \bigg(\epsilon^a - \frac{\xi^a \epsilon_b 
		\xi^b}{\xi^2}\bigg)~,
\end{equation}
where $z = l^{-1}\xi$.

A comment on the derivation of 
eq.~\eqref{eq:sol_inftrafo_cosetpar_dS} is in place. The solution 
was found after use of the power series for the hyperbolic 
cosecant.  In the case of real numbers, the series is only 
defined for values between $-\pi$ and $+\pi$. One could thus 
wonder if this convergence issue inhibits us of trusting the 
solution found above. Remember that 
eq.~\eqref{eq:inftrafo_cosetpar_dS} can be rewritten as
%
\begin{displaymath}
	(i\xi\cdot P)^{-1} \sinh(i\xi\cdot P) \wedge \delta\xi\cdot P 
	= \cosh(i\xi\cdot P) \wedge \epsilon\cdot P~,
\end{displaymath}
which reduces to
%
\begin{displaymath}
	z^{-1}\sinh z \Big(\delta\xi\cdot P - \frac{\xi\cdot \delta\xi 
		\xi\cdot P}{\xi^2}\Big) + \frac{\xi\cdot \delta\xi \xi\cdot 
		P}{\xi^2}
	= \cosh z \Big(\epsilon\cdot P - \frac{\xi\cdot\epsilon 
		\xi\cdot P}{\xi^2}\Big) + \frac{\xi\cdot\epsilon \xi\cdot 
		P}{\xi^2}~.
\end{displaymath}
This result relies on the power series expansion of the 
hyperbolic sine, which is convergent for all values of its 
argument. It is readily checked that the 
solution~\eqref{eq:sol_inftrafo_cosetpar_dS} satisfies the above 
equation. Therefore, we may conclude 
that~\eqref{eq:sol_inftrafo_cosetpar_dS} is the right solution. 

\blankline
Given the de Sitter algebra, it is also possible to compute 
$\delta h = \tfrac{i}{2} \delta h^{ab} M_{ab}$ explicitly.  
From~\eqref{eq:inftrafo_h} it follows that the element of 
$\mathfrak{h}$, corresponding to an infinitesimal de Sitter 
translation, is a solution of
%
\begin{equation}\label{eq:inftrafo_h_dS}
	\tfrac{1}{2} \sinh(i\xi\cdot P) \wedge \delta h \cdot M = 
	(\mathbb{1}-\cosh(i\xi\cdot P)) \wedge \epsilon\cdot P~.
\end{equation}
The right hand side is readily found by 
reconsidering~\eqref{eq:cosh_transl_dS}, implying that
%
\begin{displaymath}
	(\mathbb{1} - \cosh(i\xi\cdot P)) \wedge \epsilon\cdot P
	= (1 - \cosh z)\bigg(\epsilon\cdot P - \frac{\xi\cdot\epsilon 
		\xi\cdot P}{\xi^2}\bigg)~.
\end{displaymath}
From the power series expansion of the hyperbolic sine,
%
\begin{displaymath}
	\sinh(i\xi\cdot P) = \sum_{n=0}^\infty \frac{(i\xi\cdot 
		P)^{2n+1}}{(2n+1)!}~,
\end{displaymath}
and from~\eqref{eq:id_dS_comm.4}, it follows that
%
\begin{displaymath}
\begin{split}
	\sinh(i\xi\cdot P) \wedge \delta h \cdot M
	&= \delta h^{ab} \sum_{n=0}^\infty \frac{z^{2n}}{(2n+1)!} 
	(\xi_a P_b - \xi_b P_a ) \\
	&= z^{-1} \sinh z\, \delta h^{ab} (\xi_a P_b - \xi_b P_a ) \\
	&= 2 z^{-1} \sinh z\, \delta h^{ab} \xi_a P_b~.
\end{split}
\end{displaymath}
Putting these equations together,~\eqref{eq:inftrafo_h_dS} is 
rewritten as
%
\begin{displaymath}
\begin{split}
	\delta h^{ab} \xi_a P_b &= \frac{z (1-\cosh z)}{\sinh z} 
	\bigg(\epsilon^b - \frac{\xi^a \epsilon_ a \xi^b}{\xi^2} 
	\bigg)P_b \\
	&= (l\xi)^{-1} \frac{1-\cosh z}{\sinh z} (\epsilon^b\xi^a - 
	\epsilon^a\xi^b) \xi_a P_b~,
\end{split}
\end{displaymath}
from which it can be concluded that the sought-after quantities 
are
\begin{equation}
	\delta h = \tfrac{i}{2}\delta h^{ab} M_{ab} = \frac{i}{2l^2} 
	\frac{\cosh z - 1}{z\sinh z} (\epsilon^a\xi^b - 
	\epsilon^b\xi^a) M_{ab}~.
\end{equation}


\subsection{Towards a Cartan geometry}

Invoking the theory of nonlinear realizations, an Ehresmann 
connection $A$ and its corresponding curvature $F = dA + 
\tfrac{1}{2}[A,A]$ on $P(M,G)$ can be pulled back to an 
associated Cartan geometry on the reduced bundle 
$P(M,H)$.\footnote{$G = SO(1,4)$; $H = SO(1,3)$.} In what follows
such a Cartan connection will be constructed explicitly.

Under local gauge transformations $g_0$ on some associated vector 
bundle (the fibre of the principal bundle transforms with the 
inverse), the connection and its curvature transform as 
%
\begin{subequations}
\begin{align}
	\label{eq:trafo_conn_dS}
	A &\mapsto g_0 A g_0^{-1} + g_0  dg_0^{-1} = 
	\mathrm{Ad}(g_0)\cdot(A + d)  \\
	\label{eq:trafo_curv_dS}
	F &\mapsto g_0 F g_0^{-1} = \mathrm{Ad}(g_0) \cdot F~.
\end{align}
\end{subequations}
These $\mathfrak{g}$-valued differential forms can be decomposed 
with respect to their $\mathfrak{h}$-~and $\mathfrak{p}$-valued 
parts, i.e.~
%
\begin{align*}
	A_\mu &= \tfrac{i}{2}\omega^{ab}_{~~\mu}M_{ab} + ie^a_{~\mu} 
	P_a \\
	F_{\mu\nu} &= \tfrac{i}{2} R^{ab}_{~~\mu\nu} M_{ab} + 
	iT^a_{~\mu\nu} P_a~,
\end{align*}
with $P_a = M_{a4}/l(x)$ and
%
\begin{align}
	R^{ab}_{~~\mu\nu} &= \pd_\mu \omega^{ab}_{~~\nu} - 
		\pd_\nu \omega^{ab}_{~~\mu} + 
		\omega^a_{~c\mu}\omega^{cb}_{~~\nu} - 
		\omega^a_{~c\nu}\omega^{cb}_{~~\mu} + l^{-2} 
		(e^a_{~\mu}e^b_{~\nu} - e^a_{~\nu}e^b_{~\mu}) \\
		\nonumber
		&\equiv B^{ab}_{~~\mu\nu} + l^{-2} (e^a_{~\mu}e^b_{~\nu} - 
		e^a_{~\nu}e^b_{~\mu}) \\
	T^a_{~\mu\nu} &= \pd_\mu e^a_{~\nu} - \pd_\nu e^a_{~\mu} + 
	\omega^a_{~b\mu} e^b_{~\nu} - \omega^a_{~b\nu} e^b_{~\mu}
	- l^{-1}( \pd_\mu l\, e^a_{~\nu} - \pd_\nu l\, e^a_{~\mu})~.
\end{align}
Although it is not a coincidence that the above notation may 
remind of the curvature and the torsion tensors, it must be 
emphasized that these quantities are by no means the curvature or 
torsion of some geometric object. First of all and for the moment 
being, there is only a curvature $F$ of the connection $A$ in 
play, while for the latter torsion is not defined. Secondly, 
remark that the decomposition is not canonical in the sense that 
local gauge transformations mix up the $\mathfrak{h}$-~and 
$\mathfrak{p}$-valued parts.  Stated equivalently, $A$ and $F$ 
transform irreducibly under $G$. As a result, the decomposition 
is not respected by the symmetries of the geometry, from which it 
would be difficult to atribute it any physical meaning. It is 
here where symmetry breaking assumes an important role: by means 
of a section $\xi$ of the associated bundle of homogeneous 
spaces, one reduces the principal $G$ bundle to a principal $H$ 
bundle.  From the general 
prescription~\eqref{eq:def_nonlinear_field} and its particular 
transformation behaviour \eqref{eq:trafo_conn_dS}, the 
corresponding nonlinear de Sitter connection is defined as
%
\begin{equation}
	\label{eq:def_A_bar}
\begin{split}
	\bar{A} &= \mathrm{Ad}(\exp(-i\xi\cdot P))\cdot(A + d) \\
	\tfrac{i}{2}\bar{\omega}^{ab}_{~~\mu}M_{ab} + 
	i\bar{e}^a_{~\mu} P_a
	&\equiv \exp(-i\xi\cdot P)(\pd_\mu +
	\tfrac{i}{2}\omega^{ab}_{~~\mu}M_{ab} + ie^a_{~\mu} P_a) 
	\exp(i\xi\cdot P)
\end{split}
\end{equation}
The left hand side can be calculated using the techniques of 
nested commutators. Carefully doing so, one is led to the 
expressions
%
\begin{equation}
	\label{eq:nonlin_spinconn}
\begin{split}
	\bar{\omega}^{ab}_{~~\mu} = \omega^{ab}_{~~\mu} &- \frac{\cosh 
		z - 1}{l^2 z^2} \big(\xi^a (\pd_\mu \xi^b + 
	\omega^b_{~c\mu}\xi^c) - \xi^b (\pd_\mu \xi^a + 
	\omega^a_{~c\mu}\xi^c) \big) \\
	&~~- \frac{\sinh z}{l^2 z} (\xi^a e^b_{~\mu} - \xi^b 
	e^a_{~\mu})
\end{split}
\end{equation}
and
\begin{multline}
	\label{eq:nonlin_vierbein}
	\bar{e}\ind{^a_\mu} = \cosh z\, e\ind{^a_\mu} - (\cosh z - 1) 
	\frac{\xi\ind{_b} e\ind{^b_\mu} \xi^a}{\xi^2}
	\\
	+ \frac{\sinh z}{z}(\pd_\mu\xi^a + \omega\ind{^a_{b\mu}}\xi^b) 
	- \frac{\pd_\mu l}{l} \xi^a - \bigg( \frac{\sinh z}{z} - 1 
	\bigg) \frac{\xi_b \pd_\mu \xi^b \xi^a}{\xi^2}~,
\end{multline}
These expressions coincide with those found by Stelle and 
West~\cite{stelle.west:1980ds}, aside from the term $l^{-1} 
\pd_\mu l \xi^a$ in Eq.~\eqref{eq:nonlin_vierbein}.  The latter 
is new for the geometry constructed here, since it is allowed for 
the tangent de Sitter spaces to have a varying cosmological 
constant.  More specifically, one has to take in account that the 
length scale defined for the elements of $\mathfrak{p}$ may 
change along spacetime.  On the other hand, the derivation of the 
result found in~\cite{stelle.west:1980ds} relies on a constant de 
Sitter radius at any basepoint. This extra contribution is 
proportional to the dimensionless factor $l^{-1}\pd_\mu l$, which 
will be noticeable only if the variation is relatively vast.
As will be made manifest in the following section, these barred 
objects transform with respect to their $H$-indices only, 
although nonlinearly for elements not belonging to the stabilizer 
of $G$.

From the definition of $F$ and its transformation 
behaviour~\eqref{eq:trafo_curv_dS} under local $G$ 
transformations, it follows that the nonlinear curvature equals 
the curvature of the nonlinear connection, i.e.
%
\begin{equation}
	\bar{F} = d\bar{A} + \tfrac{1}{2}[\bar{A},\bar{A}]~.
\end{equation}
Decomposing the curvature of $\bar{A}$ according to the reductive 
splitting of $\mathfrak{g}$
%
\begin{displaymath}
	\bar{F}_{\mu\nu} = \tfrac{i}{2} \bar{R}^{ab}_{~~\mu\nu} M_{ab} 
	+ i\bar{T}^a_{~\mu\nu} P_a~,
\end{displaymath}
while remembering that $\bar{F} = \mathrm{Ad}(\exp(-i\xi\cdot 
P))\cdot F$, it is a matter of algebra to conclude that
%
\begin{subequations}
\begin{equation}
	\label{eq:Rnonlin}
\begin{split}
	\bar{R}^{ab}_{~~\mu\nu} = R^{ab}_{~~\mu\nu} - \frac{\cosh z - 
		1}{l^2z^2} \xi^c (\xi^a R^b_{~c\mu\nu} &-  \xi^b 
	R^a_{~c\mu\nu}) \\ &- \frac{\sinh z}{l^2 z} (\xi^a 
	T^b_{~\mu\nu} - \xi^b T^a_{~\mu\nu})~,
\end{split}
\end{equation}
\begin{equation}
	\label{eq:Tnonlin}
	\bar{T}^a_{~\mu\nu} = \frac{\sinh z}{z} \xi^c R^a_{~c\mu\nu} + 
	\cosh z\, T^a_{~\mu\nu} + (1 - \cosh z) \frac{\xi_b 
		T^b_{~\mu\nu}\xi^a}{\xi^2}~.
\end{equation}
\end{subequations}
Of course, one could also obtain this result by calculating 
$d\bar{A} + \tfrac{1}{2}[\bar{A},\bar{A}]$ directly from 
$\bar{\omega}$ and $\bar{e}$. The quantities $\bar{R}$ and 
$\bar{T}$ are the ($\mathfrak{h}$-)curvature and torsion, 
respectively, of the Cartan connection $\bar{A}$.

\subsection{Transformation behavior}
\label{ssec:trafo_beh}

In this subsection we discuss in more detail, the way in which 
the different fields introduced above transform. The general 
relation between linear and nonlinear fields indicates that the 
latter belong to the same representation space as the former, 
although the nonlinear fields become reducible under the action 
of the full gauge group $G$. Let us first review how the linear 
fields transform under elements of $G$, after which we turn 
attention to their nonlinear counterparts.

Under $G$ gauge transformations $A$ transforms as in 
Eq.~\eqref{eq:trafo_conn_dS}. When restricted to the subgroup 
$H$, the reductive splitting $A = \omega + e$ remains intact. It 
is easily inferred that $\omega$ and $e$ transform as
\begin{displaymath}
	\omega \mapsto \Ad(h_0) \cdot (\omega + d)
	\quad	\text{and} \quad
	e \mapsto \Ad(h_0) \cdot e~.
\end{displaymath}
As has been mentioned before, this symmetric splitting is not 
respected under a generic $G$-transformation. Consider for 
example an infinitesimal pure de Sitter translation $e + i 
\epsilon\cdot P$, under which to first order in the 
transformation parameter $\epsilon$
\begin{align*}
	\delta_\epsilon A
		&= i[\epsilon\cdot P,A] -i d(\epsilon\cdot P) \\
		&= i[\epsilon\cdot P,A] -i d\epsilon\cdot P + \frac{dl}{l} 
		i \epsilon\cdot P~.
\end{align*}
After applying the commutation relations for the de Sitter 
algebra, one discovers the variations of $\omega$ and $e$, namely
\begin{subequations}
\begin{align}
	\delta_\epsilon \omega^{ab} &= \frac{1}{l^2}(\epsilon^a e^b - 
	\epsilon^b e^a)~, \\
	\delta_\epsilon e^a &= -d\epsilon^a - \omega^a_{~b}\epsilon^b 
	+ \frac{dl}{l}\epsilon^a~.
\end{align}
\end{subequations}
It is manifest how $\omega$ and $e$ form an irreducible multiplet 
for the full de Sitter group.

To obtain the variations of $\omega$ and $e$, we could have saved 
energy by reconsidering the definition of $\bar{A}$. The latter 
equals $\Ad(\exp(-i\xi\cdot P))\cdot(A + d)$, which is equal to a 
transformation of $A$ under a pure de Sitter translation with 
transformation parameter $\alpha = -\xi$. One thus concludes, 
that under a \emph{finite} translation $\exp(i\alpha\cdot P)$ the 
fields $\omega$ and $e$ transform as ($z \equiv l^{-1} (\alpha 
\cdot \alpha)^{1/2}$)
\begin{subequations}
	\label{eq:transvec_fin_omega}
\begin{equation}
\begin{split}
	\omega^{ab} \mapsto
		\omega^{ab} - \frac{\cosh z - 1}{l^2 z^2} \Big[\alpha^a 
		(d\alpha^b &+ \omega^b_{~c}\alpha^c) \\
		&- \alpha^b (d\alpha^a + \omega^a_{~c}\alpha^c) \Big] + 
		\frac{\sinh z}{l^2 z} (\alpha^a e^b - \alpha^b e^a)~,
\end{split}
\end{equation}
\begin{equation}
	\label{eq:transvec_fin_e}
\begin{split}
	e^a \mapsto
		e^a &- d\alpha^a - \frac{\sinh z}{z} \omega^a_{~b} \alpha^b 
		+ \frac{dl}{l}\alpha^a \\
		&+ (\cosh z -1) \bigg( e^a - \frac{\alpha_b e^b 
			\alpha^a}{\alpha^2} \bigg) - \Big(\frac{\sinh z}{z} - 
		1\Big) \bigg( d\alpha^a - \frac{\alpha_b d\alpha^b 
			\alpha^a}{\alpha^2} \bigg)~.
\end{split}
\end{equation}
\end{subequations}
The infinitesimal transformations are recovered by taking the 
limit $\alpha \to \epsilon$.

The curvature $F$ transforms in a covariant way under gauge 
transformations, as was written down in 
Eq.~\eqref{eq:trafo_curv_dS}. Upon restriction to the subgroup 
$H$ of Lorentz rotations, the reductive splitting $F = R + T$ is 
invariant and $R$ and $T$ transform according to
\begin{displaymath}
	R \mapsto \Ad(h_0)\cdot R
	\quad \text{and} \quad
	T \mapsto \Ad(h_0)\cdot T~.
\end{displaymath}
On the other hand, under a pure de Sitter translation 
$\exp(i\alpha\cdot P)$ they form an irreducible multiplet:
\begin{subequations}
\begin{equation}
	R^{ab}_{~~\mu\nu} \mapsto R^{ab}_{~~\mu\nu} - \frac{\cosh z - 
		1}{l^2z^2} \alpha^c (\alpha^a R^b_{~c\mu\nu} -  \alpha^b 
	R^a_{~c\mu\nu}) + \frac{\sinh z}{l^2 z} (\alpha^a 
	T^b_{~\mu\nu} - \alpha^b T^a_{~\mu\nu})~,
\end{equation}
\begin{equation}
	T^a_{~\mu\nu} \mapsto -\frac{\sinh z}{z} \alpha^c 
	R^a_{~c\mu\nu} + \cosh z\, T^a_{~\mu\nu} + (1 - \cosh z) 
	\frac{\alpha_b T^b_{~\mu\nu}\alpha^a}{\alpha^2}~.
\end{equation}
\end{subequations}
For an infinitesimal translation $e + i\epsilon\cdot P$, the 
variations reduce to
\begin{subequations}
\begin{align}
	\delta_\epsilon R^{ab}_{~~\mu\nu} &= \frac{1}{l^2}(\epsilon^a 
	T^b_{~\mu\nu} - \epsilon^b T^a_{~\mu\nu})~, \\
	\delta_\epsilon T^a_{~\mu\nu} &= -\epsilon^b R^a_{~b\mu\nu}~.
\end{align}
\end{subequations}

Under the action of $G$-transformations the nonlinear fields 
$\bar{A}$ and $\bar{F}$ behave in a similar manner as the linear 
fields $A$ and $F$, respectivily.  More precisely, the linear and 
the nonlinear field belongs to the same representation space. The 
crucial difference between the two is that while $A$ and $F$ are 
irreducible under their respective action of $G$, the barred 
fields $\bar{A}$ and $\bar{F}$ are reducible with respect to the 
full gauge group $G$. As we have seen, any element of $G$ acts on 
the nonlinear fields through an element of $H$, so that only the 
$H$-components of these fields get mixed up under local 
$G$-transformations.
Although a consequence of the general scheme, we discuss this in 
more detail for the nonlinear fields $\bar{A}$ and $\bar{F}$. In  
doing this, it may become obvious that the thus obtained 
construction is natural for the kind of geometry we are wishing 
to describe.

To verify the transformation behavior of $\bar{A}$, note first 
that its definition~\eqref{eq:def_A_bar} indicates that 
$\Ad(e)(A+d) = \Ad(\exp(i\xi\cdot P))(\bar{A}+d)$. It follows 
that under the action of $g_0 \in G$
%
\begin{displaymath}
\begin{split}
	\bar{A} \mapsto \bar{A}'
		&= \Ad(\exp(-i\xi'\cdot P))\Ad(g_0)(A+d) \\
		&= \Ad(\exp(-i\xi'\cdot P) g_0) \Ad(\exp(i\xi'\cdot P)) 
		(\bar{A}+d) \\
		&= \Ad(h'(\xi,g_0))(\bar{A}+d)~.
\end{split}
\end{displaymath}
This confirms that $\bar{A}$ transforms only with elements 
belonging to $H$ and that it is a $\mathfrak{g}$-valued 
connection. It seems very plausible that it is (the pull-back of) 
a Cartan connection on $P(M,H)$[{\blu Check this: see the base 
	definition of the latter as given in	
	\cite{sharpe1997diff_geo}}]. In case $g_0$ is en element of 
$H$, the transformation becomes linear.\footnote{Note that a 
	Cartan connection is defined on a principal $H$-bundle, and 
	that is only demanded that the connection transforms in a 
	certain way under the action of $H$ gauge transformations.  
	Whether the elements of $H$ form a linear or nonlinear 
	realization is not relevant.}  Remember that the reductive 
decomposition $A = \omega + e$ is not preserved under $G$ gauge 
transformations, which made us conclude that the splitting does 
not make any sense with respect to a local de Sitter geometry.  
For the nonlinear connection the story changes completely. Since 
any element of $G$ acts through an element of $H$, the reductive 
splitting $\bar{A} = \bar{\omega} + \bar{e}$ will be invariant 
under $G$-transformations. Indeed, writing out the reductive 
splitting explicitely, $\bar{A}$ transforms as
%
\begin{displaymath}
	\tfrac{i}{2} \bar{\omega}'^{ab} M_{ab} + i\bar{e}^a P_a = 
	\Ad(h'(\xi,g_0))(	\tfrac{i}{2} \bar{\omega}^{ab} M_{ab} + 
	i\bar{e}^a P_a + d)~,
\end{displaymath}
Since \emph{both $\mathfrak{h}$ and $\mathfrak{p}$ form a 
	representation of $H$}, it follows that
%
\begin{subequations}
\begin{align}
	\label{eq:trafo_ebar}
	\bar{e}' &= \Ad(h'(\xi,g_0)) \cdot \bar{e} \\
	\bar{\omega}' &= \Ad(h'(\xi,g_0)) \cdot (\bar{\omega}+d)~.
\end{align}
\end{subequations}
It is manifest that $\bar{\omega}$ and $\bar{e}$ do not mix under 
local $G$ transformations. Note that $\bar{\omega}$ is an 
$\mathfrak{h}$-valued spin connection, which is (the pull-back 
of) an Ehresmann connection on $P(M,H)$. When the gauge 
transfromations are restricted to the subgroup $H$, the nonlinear 
fields transform the same as their linear counterparts. 

These conclusions are equally drawn for the 
$\mathfrak{g}$-curvature $\bar{F}$ and its projections $\bar{R}$ 
and $\bar{T}$. They all transform covariantly in the adjoint 
representation of the group, with respect to their $H$-indices 
under generic elements of $G$. Therefore it becomes manifest that 
$\bar{R}$ and $\bar{T}$ are true geometric objects, for the 
splitting is a gauge independent construction.  They are referred 
to as the curvature and torsion of the geometry.


\subsection{Interpretation of the vielbein $\bar{e}$}

Stelle and West \cite{Stelle:1979va,stelle.west:1980ds} claim 
that the vierbein $\bar{e}$ is a smooth mapping between the 
tangent space to spacetime at any $p \in M$  and the tangent 
space to the internal de Sitter space at $\xi(p)$. Unfortunately, 
a concrete argument did not seem to be included following this 
statement.  Furthermore, under local $H$ gauge transformations 
the vierbein $\bar{e}$ transforms as a vector with an element $h 
\in H_o$, as can be seen from \eqref{eq:trafo_ebar}.  This 
indicates that its $SO(3,1)$-indices belong to the tangent space 
at the origin $(\xi = 0)$ of $dS$. To verify its transformation 
behavior under local $G$ transformations, let us explicitly 
reconsider its construction.  

The vierbein is defined as the $\mathfrak{p}$-valued part of the 
Cartan connection $\bar{A} \in \Omega^1(M,\mathfrak{g})$ on 
$P(M,H)$. To give this statement a precise notation, we consider 
the natural projection $\pi : G \to G/H_o : g \mapsto gH_o$. The 
differential of this mapping is a projection of $T_eG = 
\mathfrak{g}$ onto $\mathfrak{p} \simeq T_o dS$. The vierbein is 
obtained from the connection by invoking this projection, 
i.e.~$\bar{e} = \pi_\ast \bar{A}$. This shows clearly that the 
vierbein is a 1-form on $M$ with values in $T_odS$. Nonetheless, 
let us also concentrate on the definition  of $\bar{A}$ itself to 
understand what happens with a tangent vector to spacetime under 
the action of $\bar{e}$, before it ends up in $T_o dS$. The 
definition  was given in Eq.~\eqref{eq:def_A_bar}, which we 
rewrite here for $g = \exp(-\xi\cdot P)$, i.e.~
\begin{displaymath}
	\bar{A} \equiv \Ad(g)\cdot A + (g^{-1})^\ast \theta~.
\end{displaymath}
It should be understood that the adjoint action acts on the 
algebra $\mathfrak{g}$, that $\theta$ is the Maurer-Cartan form 
on $G$ and that $g^\ast$ is the pullback that comes from the 
mapping $g: M \to G : p \mapsto g$. Consider next a vector $X \in 
T_pM$. One then finds,
\begin{align*}
	\bar{A}(X) &= \Ad(g)\cdot A(X) + \theta(g^{-1}_\ast X) \\
		&= L_{g\ast}\Big(R_{g^{-1}\ast}\cdot A(X) + g^{-1}_\ast X 
		\Big)~.
\end{align*}
Denote by $X^\star$ the left invariant vector field on $G$ so 
that
\begin{displaymath}
	X^\star_{g^{-1}} = R_{g^{-1}\ast}\cdot A(X) + g^{-1}_\ast X~.
\end{displaymath}
It follows directly that $\bar{A}(X) = X^\star_e$. Since $\pi 
\circ L_g = \tau_g \circ \pi$, one also has
\begin{displaymath}
	\bar{e}(X) = \pi_\ast L_{g\ast} X^\star_{g^{-1}}
	= \tau_{g\ast} \pi_\ast X^\star_{g^{-1}}~.
\end{displaymath}
Recall that $g = \exp(-\xi\cdot P)$ so that $g^{-1} = 
\exp(\xi\cdot P)$. This implies that $\pi_\ast X^\star_{g^{-1}} 
\in T_\xi dS$, since $\exp(\xi\cdot P)o = \xi \in dS$. The 
element $\bar{e}(X) \in T_o dS$ is the parallel transported 
vector of $\pi_\ast X^\star_{g^{-1}}$, with respect to the 
canonical connection on $G/H_o$ (See Ch. X in 
\cite{kob1996found}).  Therefore, it is understandable that one 
may interpret $\bar{e}$ to be a mapping from the tangent space to 
$M$ at $p$ onto the tangent space to $dS$ at $\xi$, confirming 
the interpretation given by Stelle and West.



\subsection{Discussion}

To conclude let us retrace our steps and try to understand what 
has been going on. We started by introducing a $G$-connection on 
a principal $G$-bundle $P(M,G)$. This contains information about 
a geometry for which the internal symmetry group is $G$. Since 
one is interested in describing a spacetime, whose local geometry 
is invariant under the action of the de Sitter group, this seems 
a good starting point.  However, one does not have a 
canonical---i.e.~consistent with the geometry---spin connection 
and vielbein. This is a crucial shortcoming, as it will not be 
possible to relate the local geometry of the gauge field to the 
geometry of spacetime (no soldering).  By means of a section 
$\xi$, which takes its values in the associated bundle $P 
\times_G G/H$, the principle bundle $P(M,G)$ is reduced to a 
bundle $P(M,H)$.\footnote{For an enlightning proof, see 
	\cite{kob1996found}.} Choosing a section breaks the symmetry 
from $G$ to $H$. As shown in the previous section, $\xi$ can be 
used to construct a Cartan connection $\bar{A}$ on $P(M,H)$ from 
the Ehresmann connection $A$ on $P(M,G)$. This is shown 
schematically in the following diagram:
\begin{displaymath}
	\xymatrix@R=6pt@C=8pt{
			&	\omega & & & \bar{\omega}
		\\
		A
			\ar[ru]^-{\mathfrak{h}}
			\ar[rd]_-{\mathfrak{p}}
			\ar[dddd]
			& & & {\bar{A}}
							\ar[ru]^-{\mathfrak{h}}
							\ar[rd]_-{\mathfrak{p}}
							\ar[dddd]
		\\
			& e & & & \bar{e}
		\\
			& & {\xrightarrow{\hspace*{0.5cm}\xi\hspace*{0.5cm}}}
		\\
			& {R = d\omega + \tfrac{1}{2}[\omega,\omega] + 
				\tfrac{1}{2}[e,e]}
				& & & {\bar{R} = d\bar{\omega} +
							\tfrac{1}{2}[\bar{\omega},\bar{\omega}] +							
							\tfrac{1}{2}[\bar{e},\bar{e}]}
		\\
		F
			\ar[ru]^-{\mathfrak{h}}
			\ar[rd]_-{\mathfrak{p}}
			& & & \bar{F}
							\ar[ru]^-{\mathfrak{h}}
							\ar[rd]_-{\mathfrak{p}}
		\\
			& {T = de + [\omega,e]}
				& & & {\bar{T} = d\bar{e} + [\bar{\omega},\bar{e}]}
	}
\end{displaymath}
The broken symmetries act through a nonlinear realization with 
the elements $h'(\xi,g_0)$, and merely change the point of 
tangency between the local de Sitter fibres and spacetime. On the 
other hand, the unbroken symmetries ($H$) leave the point of 
tangency fixed and act through a linear representation.
Note that the Cartan connection gives rise to a well defined spin 
connection and vierbein, i.e.~they do \emph{not} form an 
irreducible multiplet under the action of $G$. Due to the 
existence of a vierbein $\bar{e}$, spacetime is soldered to the 
de Sitter fibres and one is able to pull back all geometric 
information onto the tangent bundle of spacetime---the arena in 
which takes place gravity. Crucially, one had to make the 
realization nonlinear to have soldering.

\section{de Sitter Teleparallel Gravity}

\subsection{Introduction}

In this section we require that the $\mathfrak{h}$-valued part of 
the Cartan curvature $\bar{F}$ vanishes. In other words, the 
geometry outlined in the last section should at all times satisfy 
the following condition, namely
\begin{equation}
	\label{eq:cond_Rnonlin0}
	\bar{R} \equiv 0~.
\end{equation}
From the discussion on transformation behavior in 
Subsection~\ref{ssec:trafo_beh} it is clear that this condition 
is consistent with the geometry, i.e.~invariant under $G$-gauge 
transformations. This construction may result in the 
mathemathical structure of Teleparallel Gravity in the 
corresponding limit, i.e.~a diverging length scale $l(x) \to 
\infty$ at any point in spacetime. In that case, and because of 
the naturality of the given Cartan geometry together with a 
vanishing curvature~\eqref{eq:cond_Rnonlin0}, the thus obtained 
geometry could be seen as the generalization of Teleparallel 
Gravity, where the local kinematics are those governed by the de 
Sitter algebra.

\blankline
Let us begin by taking a closer look at the condition of 
vanishing curvature, given in Eq.~\eqref{eq:cond_Rnonlin0}.  
Combining this requirement with Eq.~\eqref{eq:Rnonlin}, one finds 
that
\begin{displaymath}
	R^{ab}_{~~\mu\nu} = \frac{\cosh z - 1}{\xi^2} \xi^c (\xi^a 
	R^b_{~c\mu\nu} - \xi^b R^a_{~c\mu\nu}) + \frac{\sinh z}{l^2 z} 
	(\xi^a T^b_{~\mu\nu} - \xi^b T^a_{~\mu\nu})~.
\end{displaymath}
This expression can be contracted with $\xi$, which results in
\begin{equation}
	\cosh z\, \xi^c R^a_{~c\mu\nu} = \frac{\sinh z}{l^2 z} (\xi^a 
	\xi_b	T^b_{~\mu\nu} - \xi^2 T^a_{~\mu\nu})~.
\end{equation}
Substituting this equations into the torsion $\bar{T}$, see 
Eq.~\eqref{eq:Tnonlin}, one obtains
\begin{equation}
	\label{eq:Tnonlinear_Rnl0}
	\bar{T}^a_{~\mu\nu} = \frac{1}{\cosh z} T^a_{~\mu\nu} + 
	\bigg(1-\frac{1}{\cosh z}\bigg) \frac{\xi^a\xi_b 
		T^b_{~\mu\nu}}{\xi^2}~.
\end{equation}
Contracting both sides with $\xi$ additionally shows 
that\footnote{This equality holds also for a non-vanishing 
	curvature $\bar{R}$, as can be seen from 
	Eq.~\eqref{eq:Tnonlin}.}
\begin{displaymath}
	\xi_a \bar{T}^a_{~\mu\nu} = \xi_a T^a_{~\mu\nu}~.
\end{displaymath}
\begin{remark}
	It is interesting to have a look at the limiting situations 
	for a vanishing, respectively diverging cosmological constant.  
	In the case of $l(x) \to \infty$, $z$ vanishes and 
	from~\eqref{eq:Tnonlinear_Rnl0} it is found that
	\begin{displaymath}
		\lim_{\Lambda \to 0} \bar{T}^a_{~\mu\nu} = T^a_{~\mu\nu}~,
	\end{displaymath}
	while on the other hand for $l(x) \to 0$, $z$ diverges and
	\begin{displaymath}
		\lim_{\Lambda \to \infty} \bar{T}^a_{~\mu\nu} = 
		\frac{\xi^a\xi_b T^b_{~\mu\nu}}{\xi^2}~.
		\qedhere
	\end{displaymath}
\end{remark}

Subsequently let us investigate the additional restriction of a 
vanishing torsion, i.e.\footnote{To be clear: the condition of a 
	vanishing curvature is \emph{not} relaxed.}
\begin{equation}
	\label{eq:cond_Tnonlin0}
	\bar{T} = 0~.
\end{equation}
From Eq.~\eqref{eq:Tnonlinear_Rnl0} and observing that 
$\xi\cdot\bar{T} = \xi\cdot T$ one infers that $T$ vanishes. An 
obvious choice of gauge corresponding to such a geometry is $e = 
0$, that is
\begin{displaymath}
	e = 0 \quad\Rightarrow\quad \bar{T} = 0~.
\end{displaymath}
On the other hand, a vanishing torsion does not necessarily imply 
that $e$ is equal to zero. To find the most general $e$ 
consistent with the condition~\eqref{eq:cond_Tnonlin0}, it is 
worthwhile to note that the latter is invariant under local gauge 
transformations and spacetime diffeomorphisms.  These 
transformations are the most general at hand and their effect on 
$e$ will exhaust its values, corresponding to a vanishing 
torsion.  Since $e$ transforms in a homogeneous way under both 
spacetime diffeomorphisms and local Lorentz transformations, 
these will leave $e=0$ invariant. Therefore it is sufficient to 
consider the action on $e$ due to de Sitter transvections 
$\exp(i\alpha\cdot P)$ solely.  From the transformation 
rule~\eqref{eq:transvec_fin_e} for $e$ one finds that $e^a = 0$ 
transforms into
\begin{equation}
	\label{eq:e_trivial}
	e'^a = -\frac{\sinh z}{z} (d\alpha^a + \omega^a_{~b}\alpha^b) + 
	\frac{dl}{l} \alpha^a + \bigg( \frac{\sinh z}{z} - 1 \bigg) 
	\frac{\alpha_b d\alpha^b \alpha^a}{\alpha^2}~.
\end{equation}
The vierbein $\bar{e}'$ Lorentz rotates according 
to
\begin{multline*}
	\bar{e}'^a = \cosh z' e'^a - (\cosh z' -1) \frac{\xi'_b e'^b 
		\xi'^a}{\xi'^2}
	\\
	+ \frac{\sinh z'}{z'}(d\xi'^a + \omega'^a_{~b}\xi'^b) - 
	\frac{dl}{l} \xi'^a - \bigg( \frac{\sinh z'}{z'} - 1 \bigg) 
	\frac{\xi'_b d\xi'^b \xi'^a}{\xi'^2}~,
\end{multline*}
while the torsion $\bar{T}'^a$ remains zero. We call a vierbein 
$\bar{e}$ \emph{trivial} if and only if $e$ is of the 
form~\eqref{eq:e_trivial}. In the case of a trivial vierbein, it 
is clear that some gauge transformation will render $e = 0$. But 
the relevant action is given by de Sitter transvections, which 
correspond to a shift in the section $\xi$.
Since $\xi$ is arbitrary, it is then without loss of generality 
to assume that a trivial vierbein is of the form
\begin{equation}
	\bar{e}^a = \frac{\sinh z}{z}(d\xi^a + \omega^a_{~b}\xi^b) - 
	\frac{dl}{l} \xi^a - \bigg( \frac{\sinh z}{z} - 1 \bigg) 
	\frac{\xi_b d\xi^b \xi^a}{\xi^2}~.
\end{equation}
Hence, the vanishing of torsion entails the triviality of the 
vierbein. Conversely, in case the vielbein is trivial, the 
torsion will be equal to zero.


\subsection{Equations of motion for a particle}

Given the vielbein $\bar{e}$, it is possible to construct a line 
element on spacetime that is invariant under local de Sitter 
transformations.  The quadratic line element is defined as
\begin{displaymath}
	d\tau^2 = \bar{e}\ind{^a_\mu} \bar{e}\ind{_{a\nu}} dx^\mu 
	dx^\nu~,
\end{displaymath}
from which the square root is extracted, resulting in
\begin{equation}
	d\tau = \bar{u}_a \bar{e}^a~.
\end{equation}
In the last expression the nonlinear four-velocity has been 
introduced, which is given by
\begin{displaymath}
	\bar{u}^a = \bar{e}\ind{^a_\mu} u^\mu~.
\end{displaymath}

The line element has the dimension of length, implying that a 
possible action for the worldline $x^\mu(\tau)$ of a particle 
with mass $m$ equals
\begin{equation}
	\label{eq:action_particle}
	\mathcal{S} = -mc \int_{\tau_1}^{\tau_2} d\tau
		= -mc \int_{\tau_1}^{\tau_2} \bar{u}_a \bar{e}^a~.
\end{equation}
The action attains an extremum for the worldline being the 
physical one. This means that the equations of motion correspond 
to $\delta \mathcal{S} = 0$, where an infinitisemal variation of 
the worldline $x^\mu(\tau) \to x^\mu + \delta x^\mu(\tau)$ is 
considered. Under this deviation, the 
action~\eqref{eq:action_particle} varies according to
\begin{displaymath}
	\delta\mathcal{S} = -mc \int_{\tau_1}^{\tau_2} \delta\bar{u}_a 
	\bar{e}^a + \bar{u}_a \delta\bar{e}^a
	= -mc \int_{\tau_1}^{\tau_2} \bar{u}_a \delta\bar{e}^a~.
\end{displaymath}
After a rather lengthy calculation, which we wrote down in 
Appendix~\eqref{ssec:dSTG_var_ue}, one finds
\begin{displaymath}
	\delta\mathcal{S} = mc \int_{\tau_1}^{\tau_2} d\tau \delta 
	x^\mu \bigg\{ \bar{e}\ind{^a_\mu} \bigg( 
	\frac{d\bar{u}_a}{d\tau} - \bar{\omega}\ind{^b_{a\rho}} 
	\bar{u}\ind{_b} u^\rho + u^\rho \frac{\pd_\rho l}{l} \bar{u}_a 
	\bigg) - \bar{T}\ind{^a_{\mu\rho}} \bar{u}\ind{_a} u^\rho - 
	\frac{\pd_\mu l}{l} \bigg\}~.
\end{displaymath}
This quantity should vanish for an arbitrary variation, a 
condition that leads to the equations of motion:
\begin{displaymath}
	u^\rho \bar{D}_\rho (l \bar{u}^a) = l\bar{e}^{a\mu} 
	\bigg(\bar{T}\ind{^b_{\mu\rho}} \bar{u}\ind{_b} u^\rho + 
	\frac{\pd_\mu l}{l} \bigg)~,
\end{displaymath}
where $\bar{D} \equiv d + \bar{\omega}$ is the covariant 
derivative with respect to the spin connection $\bar{\omega}$.
The equations of motion can be rewritten in the form
\begin{equation}
	\label{eq:eom_particle_dSTG}
	u^\rho \bar{D}_\rho \bar{u}^a = \bar{e}\ind{^{a\mu}} 
	\bar{T}\ind{^b_{\mu\rho}} \bar{u}\ind{_b} u^\rho + 
	(\bar{e}\ind{^{a\mu}} - \bar{u}^a u^\mu) \frac{\pd_\mu l}{l}~.
\end{equation}

It is interesting to have a closer look at this equation. First 
note that in the appropriate limit of a vanishing cosmological 
function $(l \to \infty)$, the  equation of motion of 
Teleparallel Gravity for a spinless particle in a gravitational 
field is recovered \cite{aldrovandi:2012tele}. Similar to the 
equation there, we still have a force equation at hand in which 
both the terms on the right-hand side are indeed genuine 
relativistic forces, being orthogonal to the four-velocity 
$\bar{u}^a$. The first force is the obvious generalization to the 
given geometry of the gravitational force in ordinary 
Teleparallel gravity. The second force term however is new, and 
will be noticeable only in spacetime regions where the 
cosmological function varies relativily strong. Observe that the 
operator $\bar{e}\ind{^{a\mu}} - \bar{u}^a u^\mu$ is a projector, 
since
\begin{displaymath}
	(\bar{e}\ind{^{b\rho}} - \bar{u}^b u^\rho) \bar{e}_{a\rho} 
	(\bar{e}\ind{^{a\mu}} - \bar{u}^a u^\mu) = \bar{e}\ind{^{b\mu}} 
	- \bar{u}^b u^\mu~.
\end{displaymath}

\subsection{Field equations}

In this subsection we look for the equations of motions that 
specify for the geometry in de Sitter Teleparallel Gravity.  In a 
first attempt, the free action is the one given by replacing $T 
\to \bar{T}$ in the action that describes free Poincar\'e 
Teleparallel Gravity {\blu [Citations]}. It is thus proposed that  
\begin{equation}
	\label{eq:action_dSTG}
	\mathcal{S} = \frac{c^3}{16\pi G} \int \mathrm{Tr}~ \bar{T} 
	\wedge \star \bar{T} = \frac{c^3}{16\pi G} \int d^4 x \, 
	\bar{e}\, \mathcal{L}~,
\end{equation}
where
\begin{equation}
	\label{eq:lagrangian_dSTG}
	\mathcal{L} = \tfrac{1}{4} \bar{T}\ind{^a_{\mu\nu}} 
	\bar{T}\ind{_a^{\mu\nu}} + \tfrac{1}{2} 
	\bar{T}\ind{^a_{\mu\nu}} \bar{T}\ind{^{b\mu}_\lambda} 
	\bar{e}\ind{_a^\lambda} \bar{e}\ind{_b^\nu} - 
	\bar{T}\ind{^a_{\mu\nu}} \bar{T}\ind{^{b\mu}_\lambda} 
	\bar{e}\ind{_a^\nu} \bar{e}\ind{_b^\lambda}~.
\end{equation}
The corresponding field equations are found by 
extremizing~\eqref{eq:action_dSTG} with respect to the vierbein 
$\bar{e}\ind{^a_\mu}$, i.e.~
\begin{displaymath}
\begin{split}
	0 = \delta\mathcal{S} &= \int d^4x \, \delta\bar{e} \,
	\mathcal{L} + \int d^4x \, \bar{e} \, \delta\mathcal{L} \\
	&= \int d^4x \, \bar{e} \, \bar{e}\ind{_a^\mu} \mathcal{L} 
	\delta\bar{e}\ind{^a_\mu} + \int d^4x \, \bar{e} \bigg( 
	\frac{\pd \mathcal{L}}{\pd \bar{e}\ind{^a_\mu}} \delta 
	\bar{e}\ind{^a_\mu} + \frac{\pd \mathcal{L}}{\pd \pd_\rho 
		\bar{e}\ind{^a_\mu}} \delta \pd_\rho \bar{e}\ind{^a_\mu} 
	\bigg)~.
\end{split}
\end{displaymath}
After invoking Stokes' theorem together with the assumption that 
the fields go to zero when approaching infinity,\footnote{{\blu 
		Is it the fields that go to zero that legitimate the 
		omitting of the boundary terms, or is it the vanishing of 
		the variation?}} the equations of motion are
\begin{equation}
	\pd_\mu \bigg( \bar{e} \frac{\pd \mathcal{L}}{\pd \pd_\mu 
		\bar{e}\ind{^a_\nu}} \bigg) - \bar{e} \frac{\pd 
		\mathcal{L}}{\pd \bar{e}\ind{^a_\nu}} - \bar{e}\, 
	\bar{e}\ind{_a^\nu} \mathcal{L} = 0~.
\end{equation}
For the given Lagrangian it is shown in 
Appendix~\ref{app:field_eqs} that these equations reduce to
\begin{displaymath}
	\pd_\mu (\bar{e}\, \bar{W}\ind{_a^{\mu\nu}}) - \bar{e}\, 
	\bar{\omega}\ind{^b_{a\mu}} \bar{W}\ind{_b^{\mu\nu}} + \bar{e} 
	\frac{\pd_\mu l}{l} \bar{W}\ind{_a^{\mu\nu}} + \bar{e}\, 
	\bar{T}\ind{^b_{\mu a}} \bar{W}\ind{_b^{\mu\nu}} - \bar{e}\, 
	\bar{e}\ind{_a^\nu} \mathcal{L} = 0~,
\end{displaymath}
where we introduced the notation
\begin{equation}
	\bar{W}\ind{_a^{\mu\nu}} \equiv \bar{T}\ind{_a^{\mu\nu}} + 
	\bar{T}\ind{^{\nu\mu}_a} - \bar{T}\ind{^{\mu\nu}_a} - 
	2\bar{e}\ind{_a^\nu} \bar{T}\ind{^{\lambda\mu}_\lambda} + 
	2\bar{e}\ind{_a^\mu} \bar{T}\ind{^{\lambda\nu}_\lambda}~.
\end{equation}

The field equations can be rewritten in a manifestly covariant 
form as
\begin{equation}
	\label{eq:field_eqs_dSTG}
	\bar{D}_\mu (\bar{e}\, \bar{W}\ind{_a^{\mu\nu}}) + \bar{e} 
	\frac{\pd_\mu l}{l} \bar{W}\ind{_a^{\mu\nu}} + \bar{e}\, 
	\bar{t}\ind{_a^\nu} = 0~,
\end{equation}
where we denoted the expression
\begin{displaymath}
	\bar{t}\ind{_a^\nu} = \bar{T}\ind{^b_{\mu a}} 
	\bar{W}\ind{_b^{\mu\nu}} - \bar{e}\ind{_a^\nu} \mathcal{L}~.
\end{displaymath}
Note further that
\begin{displaymath}
	\bar{D}_\nu \bar{D}_\mu (\bar{e}\, \bar{W}\ind{_a^{\mu\nu}})
	= \frac{1}{2} [\bar{D}_\nu, \bar{D}_\mu] (\bar{e}\, 
	\bar{W}\ind{_a^{\mu\nu}}) 
	= \frac{1}{2} \bar{e}\, \bar{B}\ind{_a^b_{\nu\mu}} 
	\bar{W}\ind{_b^{\mu\nu}}~.
\end{displaymath}
For the given geometry one has that $\bar{B}^{ab} = -l^{-2} 
\bar{e}^a \wedge \bar{e}^b$ so that
\begin{displaymath}
	\bar{D}_\nu \bar{D}_\mu (\bar{e}\, \bar{W}\ind{_a^{\mu\nu}})
	= \frac{1}{l^2} \bar{e}\, \bar{e}\ind{_{a\mu}} 
	\bar{e}\ind{^b_\nu} \bar{W}\ind{_b^{\mu\nu}}
	= \frac{1}{l^2} \bar{e}\, \bar{W}\ind{_{ba}^b}~.
\end{displaymath}
Since $\bar{W}\ind{_{ba}^b} = -4 \bar{T}\ind{_{ba}^b}$ and 
because one can infer from the second Bianchi identity that the 
trace of $\bar{T}$ vanishes, one concludes that
\begin{equation}
	\bar{D}_\nu \bar{D}_\mu (\bar{e}\, \bar{W}\ind{_a^{\mu\nu}})
	= 0~.
\end{equation}
It should be emphasized that this result is particular to the 
geometry at hand. More specifically is it a consequence of the 
condition $\bar{R} \equiv 0$. From the field 
equations~\eqref{eq:field_eqs_dSTG} one observes that
\begin{equation}
	\bar{D}_\mu(\bar{e}\, \bar{t}\ind{_a^\mu}) = \frac{\pd_\mu 
		l}{l} \bar{e}\, \bar{t}\ind{_a^\mu}~.
\end{equation}


\newpage
\appendix
\section{Nested commutators}

\subsection{Notation}

\label{app:nested_comm.not}

For any two elements $X$ and $Y$ of a Lie algebra we define
%
\begin{displaymath}
	X \wedge Y \equiv \mathrm{ad}_X (Y) = [X,Y]
\end{displaymath}
and consequently
\begin{displaymath}
	X^k \wedge Y \equiv \mathrm{ad}_X^k (Y) = 
	[X,[X,\ldots[X,Y]\ldots]]~.
\end{displaymath}
This can be extended to arbitrary functions, where a function is 
considered a power series in $X$, that is
%
\begin{displaymath}
	f(X) \wedge Y = \sum_k c_k X^k \wedge Y~.
\end{displaymath}
Consider a second function $g(X) = \sum_l d_l X^l$. One obtains
%
\begin{displaymath}
	g(X)\wedge f(X)\wedge X = \sum_{kl} c_k d_l 
	\mathrm{ad}_X^l(\mathrm{ad}_X^k(Y)) = \sum_{kl} c_k d_l 
	X^{k+l} \wedge Y = g(X)f(X) \wedge Y~,
\end{displaymath}
where we used the linearity of the adjoint action.
From this result it follows that the equation $f(X) \wedge Y = Z$ 
can be solved for $Y = f(X)^{-1} \wedge Z$. Note that the inverse 
function also is supposed to be expressed as a power series.

To conclude we write down two identities, using the introduced 
notation.  The first is Hadamard's formula
%
\begin{equation}\label{eq:Had}
	\exp(X) Y \exp(-X) = \exp(X) \wedge Y~,
\end{equation}
the other is the Campbell-Poincar\'e fundamental identity,
\begin{equation}\label{eq:CP}
	\exp(-X) \delta\exp(X) = \frac{1-\exp(-X)}{X}\wedge \delta X~.
\end{equation}

\subsection{de Sitter algebra: some results}

In this subsection, we compute some intermediary results that are 
used troughout the text. The commutatation relations considered 
are those given by~\eqref{eq:comm_rels_dS}, for the convention 
$\mathfrak{s} = -1$.

The first identity to be verified is
%
\begin{equation}\label{eq:id_dS_comm.1}
	(i\xi\cdot P)^{2n} \wedge \epsilon\cdot P = z^{2n} 
	\bigg(\epsilon\cdot P - \frac{\xi\cdot\epsilon \xi\cdot 
		P}{\xi^2} \bigg)~;\quad n \geqslant 1~.
\end{equation}
Therefore, we compute the sequence
\begin{align*}
	i\xi\cdot P \wedge \epsilon\cdot P
	&= i\xi^a\epsilon^b [P_a,P_b] = l^{-2} \xi^a\epsilon^b 
	M_{ab}~;
	\\
	(i\xi\cdot P)^2 \wedge \epsilon\cdot P
	&= i\xi^c P_c \wedge l^{-2} \xi^a\epsilon^b M_{ab} \\
	&= -il^{-2} \xi^a\epsilon^b\xi^c [M_{ab},P_c] \\
	&= l^{-2}\xi^a\epsilon^b\xi^c (\eta_{ac}P_b - \eta_{bc}P_a) \\
	&= l^{-2} \xi^2 \left(\epsilon\cdot P - \frac{\xi\cdot\epsilon 
			\xi\cdot P}{\xi^2} \right)~;
	\\
	(i\xi\cdot P)^4 \wedge \epsilon\cdot P
	&= l^{-2}\xi^2(i\xi\cdot P)^2 \wedge \left(\epsilon\cdot P - 
		\frac{\xi\cdot\epsilon \xi\cdot P}{\xi^2} \right) \\
	&= l^{-2}\xi^2(i\xi\cdot P)^2 \wedge \epsilon\cdot P \\
	&= (l^{-2}\xi^2)^2 \left(\epsilon\cdot P - 
		\frac{\xi\cdot\epsilon \xi\cdot P}{\xi^2} \right)~; \\
	&~\,\vdots
	\\
	(i\xi\cdot P)^{2n} \wedge \epsilon\cdot P
	&= (l^{-2}\xi^2)^n \left(\epsilon\cdot P - 
		\frac{\xi\cdot\epsilon \xi\cdot P}{\xi^2} \right)~.
\end{align*}
The identity follows by letting $z \equiv 
l^{-1}(\xi^a\xi_a)^{1/2}$.

From~\eqref{eq:id_dS_comm.1} it follows that
%
\begin{displaymath}
\begin{split}
	(i\xi\cdot P)^{2n+1} \wedge \epsilon\cdot P
	&= (i\xi\cdot P) \wedge z^{2n} \bigg(\epsilon\cdot P - 
	\frac{\xi\cdot\epsilon \xi\cdot P}{\xi^2} \bigg) \\
	&= l^{-2}z^{2n} \xi^a \epsilon^b M_{ab}~,
\end{split}
\end{displaymath}
hence, another useful identity is given by
%
\begin{equation}\label{eq:id_dS_comm.2}
	(i\xi\cdot P)^{2n+1} \wedge \epsilon\cdot P = 
	\tfrac{1}{2}l^{-2}z^{2n} (\xi^a\epsilon^b - \xi^b\epsilon^a) 
	M_{ab}~;\quad n \geqslant 0~.
\end{equation}

Finally, the following two identities are derived\footnote{Note 
	that $\delta h \cdot M = \delta h^{ab} M_{ab}$.}
%
\begin{align}
	\label{eq:id_dS_comm.3}
	(i\xi\cdot P)^{2n} \wedge \delta h \cdot M &= \delta h^{ab} 
	l^{-2}z^{2n-2} \xi^c(\xi_b M_{ac} - \xi_a M_{bc})~;\quad n 
	\geqslant 1~, \\
	\label{eq:id_dS_comm.4}
	(i\xi\cdot P)^{2n+1} \wedge \delta h \cdot M &= \delta h^{ab} 
	z^{2n} (\xi_a P_b - \xi_b P_a)~;\quad n \geqslant 0~.
\end{align}
To verify them consider the following series of equations.
%
\begin{align*}
	(i\xi\cdot P)\wedge \delta h \cdot M
	&= \delta h^{ab} \xi^c (-i)[M_{ab},P_c] = \delta h^{ab} (\xi_a 
	P_b - \xi_b P_a)~; \\
	(i\xi\cdot P)^2\wedge \delta h \cdot M
	&= 2 \delta h^{ab} i\xi\cdot P \wedge \xi_a P_b \\
	&= 2 \delta h^{ab} \xi_a \xi^c (-i) [P_b,P_c] \\
	&= 2 \delta h^{ab} l^{-2} \xi_a \xi^c M_{cb} \\
	&= \delta h^{ab} l^{-2} \xi^c (\xi_b M_{ac} - \xi_a M_{bc})~; 
	\\
	(i\xi\cdot P)^4 \wedge \delta h \cdot M
	&= 2\delta h^{ab} l^{-2} \xi_b \xi^c (i\xi\cdot P)^2 \wedge 
	M_{ac} \\
	&= 2\delta h^{ab} l^{-2} \xi_b \xi^c l^{-2} \xi^d (\xi_c 
	M_{ad} - \xi_a M_{cd}) \\
	&= 2 \delta h^{ab} l^{-2} z^2 \xi^d \xi_b M_{ad} \\
	&= \delta h^{ab} l^{-2} z^2 \xi^c (\xi_b M_{ac} - \xi_a 
	M_{bc}) \\
	&~\,\vdots
	\\
	(i\xi\cdot P)^{2n} \wedge \delta h \cdot M
	&= \delta h^{ab} l^{-2} z^{2n-2} \xi^c (\xi_b M_{ac} - \xi_a 
	M_{bc}) \\
	(i\xi\cdot P)^{2n+1} \wedge \delta h \cdot M
	&= 2\delta h^{ab} l^{-2} z^{2n-2} \xi^c \xi_b (-i) 
	[M_{ac},P_d] \\
	&= 2\delta h^{ab} l^{-2} z^{2n-2} (\xi_b\xi_a \xi\cdot P - 
	\xi^2 \xi_b P_a) \\
	&= \delta h^{ab} z^{2n} (\xi_a P_b - \xi_b P_a)~.
\end{align*}


\section{de Sitter Teleparallel Gravity: intermediate results}

In this section we work out to some extend, results related to de 
Sitter Teleparallel gravity that were used in the main body of 
the text.

\subsection{Variation of $\int \bar{u}_a \bar{e}^a$}
\label{ssec:dSTG_var_ue}

In this calculation the variation of $\int \bar{u}_a \bar{e}^a$ 
will be verified.  To begin with let us rewrite the expression 
for a non-trivial vierbein, i.e.
\begin{multline*}
	\bar{e}^a = \cosh z\, e^a - (\cosh z -1) \frac{\xi_b e^b 
		\xi^a}{\xi^2} \\
		+ \frac{\sinh z}{z}(d\xi^a + \omega^a_{~b}\xi^b) - 
		\frac{dl}{l} \xi^a - \bigg( \frac{\sinh z}{z} - 1 \bigg) 
		\frac{\xi_b d\xi^b \xi^a}{\xi^2}~.
\end{multline*}
We compute:
\begin{displaymath}
\begin{split}
	\int \bar{u}_a &\delta\bar{e}^a \\
	&= \int \bar{u}_a \bigg\{ \delta\!\cosh z\, e^a + \cosh z\, 
	\delta e\ind{^a_\rho} dx^\rho + \cosh z\, e\ind{^a_\rho} 
	\delta dx^\rho - \delta\!\cosh z \frac{\xi_b e^b \xi^a}{\xi^2} 
	\\
	&\nla{} + (\cosh z - 1) \bigg[ \frac{2\delta\xi}{\xi} 
	\frac{\xi_b e^b \xi^a}{\xi^2} - \frac{\delta\xi_b e^b 
		\xi^a}{\xi^2} -\frac{\xi\ind{_b} \delta e\ind{^b_\rho} 
		\xi^a}{\xi^2} dx^\rho - \frac{\xi_b e^b \delta\xi^a}{\xi^2} 
	\\
	&\nla{} - \frac{\xi\ind{_b} e\ind{^b_\rho} \xi^a}{\xi^2} 
	\delta dx^\rho \bigg] + \delta \Big( \frac{\sinh z}{z} \Big) ( 
	d\xi^a + \omega\ind{^a_b}\xi^b )
	+ \frac{\sinh z}{z} (\delta d\xi^a + \delta 
	\omega\ind{^a_{b\rho}} \xi^b dx^\rho
	\\
	&\nla{} + \omega\ind{^a_b} \delta\xi^b + 
	\omega\ind{^a_{b\rho}} \xi^b \delta dx^\rho) + \frac{\delta l 
		dl}{l^2} \xi^a - \frac{\delta dl}{l} \xi^a - 
	\frac{1}{l}\delta\xi^a - \delta \Big( \frac{\sinh z}{z} \Big) 
	\frac{\xi_b d\xi^b \xi^a}{\xi^2}
	\\
	&\nla{} + \Big( \frac{\sinh z}{z} -1 \Big) \bigg[ 
	\frac{2\delta\xi}{\xi} \frac{\xi_b d\xi^b \xi^a}{\xi^2} - 
	\frac{\delta\xi_b d\xi^b \xi^a}{\xi^2} - \frac{\xi_b \delta 
		d\xi^b \xi^a}{\xi^2} - \frac{\xi_b d\xi^b \xi^a}{\xi^2} 
	\bigg]\bigg\}
\end{split}
\end{displaymath}
For any function on $M$, note that $df \to df + d\delta f$ so 
that $\delta (df) = d(\delta f)$, i.e.
\begin{equation}
	[\delta,d] f = 0~.
\end{equation}
The following variations also are useful:
\begin{gather}
	\delta z = \delta (l^{-1} \xi) = - l^{-2} \delta l\, \xi + 
	l^{-1} \delta\xi~, \\
	\delta \xi = \xi^{-1} \xi_a \delta \xi^a~.
\end{gather}
The variation is assumed to vanish at the endpoints of the 
particle's worldline, so that a total derivative over a term 
containing $\delta x^\rho$ integrates to zero. One first 
integrates by parts the terms containing variations of the 
differentials $\delta dx^\rho = d\delta x^\rho$, after which the 
boundary integrals render zero. Doing so, one obtains
\begin{displaymath}
\begin{split}
	\int \bar{u}_a &\delta\bar{e}^a \\
	&= \int \bigg[ -d\bar{u}\ind{_a} \bigg\{ \cosh z\, 
	e\ind{^a_\mu} - (\cosh z - 1) \frac{\xi\ind{_b} e\ind{^b_\mu} 
		\xi^a}{\xi^2} + \frac{\sinh z}{z}(\pd_\mu \xi^a + 
	\omega\ind{^a_{b\mu}} \xi^b)
	\\
	&\nla{} - \frac{\pd_\mu l}{l} \xi^a - \Big( \frac{\sinh z}{z} 
	- 1 \Big) \frac{\xi_b \pd_\mu \xi^b \xi^a}{\xi^2} \bigg\} 
	\delta x^\mu + \bar{u}_a \delta x^\mu dx^\rho \bigg\{ \bigg[ 
	\frac{\sinh z}{z} \omega\ind{^a_{b\rho}} \Big( \pd_\mu \xi^b
	\\
	&\nla{} - \xi^b \frac{\xi_c \pd_\mu \xi^c}{\xi^2} \Big) + 
	\cosh z\, ( \pd_\rho \xi^a + \omega\ind{^a_{b\rho}} \xi^b )  
	\frac{\xi_c \pd_\mu \xi^c}{\xi^2} - \pd_\rho \xi^a \frac{\xi_b 
		\pd_\mu \xi^b}{\xi^2} - \cosh z \Big( \pd_\rho \xi^a
	\\
	&\nla{} + \omega\ind{^a_{b\rho}} \xi^b - \xi^a \frac{\xi_b 
		\pd_\rho \xi^b}{\xi^2} \Big) \frac{\pd_\mu l}{l} + \pd_\rho 
	\xi^a \frac{\pd_\mu l}{l} - 2(\cosh z - 1) \xi^a \frac{\xi_b 
		\pd_\rho \xi^b}{\xi^2} \frac{\xi\ind{_c} 
		e\ind{^c_\mu}}{\xi^2}
	\\
	&\nla{} + (\cosh z - 1) \pd_\rho \xi^a \frac{\xi\ind{_b} 
		e\ind{^b_\mu}}{\xi^2} + (\cosh z - 1) \xi^a \frac{\pd_\rho 
		\xi\ind{_b} e\ind{^b_\mu}}{\xi^2} + z\sinh z \frac{\xi_c 
		\pd_\mu \xi^c}{\xi^2} \Big( e\ind{^a_\rho}
	\\
	&\nla{} - \xi^a \frac{\xi\ind{_b} e\ind{^b_\rho}}{\xi^2} \Big) 
	- z \sinh z \Big( e\ind{^a_\rho} - \xi^a \frac{\xi\ind{_b} 
		e\ind{^b_\rho}}{\xi^2} \Big) \frac{\pd_\mu l}{l} \bigg] - 
	\bigg[ \rho \leftrightarrow \mu \bigg] \bigg\} + \bar{u}_a 
		\delta x^\mu dx^\rho \bigg\{
	\\
	&\nla{} -\bigg[ \frac{\pd_\rho l}{l} \frac{\sinh z}{z} \Big( 
	\pd_\mu \xi^a + \omega\ind{^a_{b\mu}} \xi^b - \xi^a 
	\frac{\xi_b \pd_\mu \xi^b}{\xi^2} \Big) + \frac{\pd_\rho l}{l} 
	\frac{\pd_\mu l}{l} \xi^a + \cosh z\, \pd_\rho e\ind{^a_\mu}
	\\
	&\nla{} - (\cosh z - 1) \xi^a \frac{\xi_b \pd_\rho 
		e\ind{^b_\mu}}{\xi^2} + \frac{\sinh z}{z} \pd_\rho 
	\omega\ind{^a_{b\rho}} \xi^b	\bigg] + \bigg[ \rho 
	\leftrightarrow \mu \bigg] \bigg\} \bigg]
\end{split}
\end{displaymath}
The terms between the first pair of curly brackets is just the 
vierbein $\bar{e}\ind{^a_\mu}$, while those between the second 
pair of curly brackets equal
\begin{multline*}
	\bigg[ \bar{\omega}\ind{^a_{b\rho}} \bar{e}\ind{^b_\mu} - 
	\frac{\sinh z}{z} \omega\ind{^a_{b\rho}} \omega\ind{^b_{c\mu}} 
	\xi^c - \frac{\pd_\rho l}{l} \xi^a \frac{\xi_b \pd_\mu 
		\xi^b}{\xi^2} - \cosh z\, \omega\ind{^a_{b\rho}} 
	e\ind{^b_\mu}
	\\
	- (\cosh z - 1) \xi^a \frac{\omega\ind{_{bc\rho}} \xi^c 
		e\ind{^b_\mu}}{\xi^2} - z \sinh z\, e\ind{^a_\rho} 
	\frac{\xi\ind{_c} e\ind{^c_\mu}}{\xi^2} \bigg] - \bigg[ \rho 
	\leftrightarrow \mu \bigg]~.
\end{multline*}
This permits us to further work out
\begin{displaymath}
\begin{split}
	\int \bar{u}_a &\delta\bar{e}^a \\
	&= \int \bigg[ -d\bar{u}\ind{_a} \bar{e}\ind{^a_\mu} \delta 
	x^\mu + \bar{u}_a \delta x^\mu dx^\rho \bigg\{ \bigg[ 
	\bar{\omega}\ind{^a_{b\rho}} \bar{e}\ind{^b_\mu} - \frac{\sinh 
		z}{z} \xi^c \Big( \pd_\rho \omega\ind{^a_{c\rho}} + 
	\omega\ind{^a_{b\rho}} \omega\ind{^b_{c\mu}}
	\\
	&\nla{} + \frac{1}{l^2} e\ind{^a_\rho} e\ind{_{c\mu}} \Big) - 
	\cosh z \Big( \pd_\rho e\ind{^a_\mu} + \omega\ind{^a_{b\rho}} 
	e\ind{^b_\mu} - \frac{\pd_\rho l}{l} e\ind{^a_\mu} \Big) - (1 
	- \cosh z) \frac{\xi^a \xi_b}{\xi^2} \Big( \pd_\rho 
	e\ind{^b_\mu}
	\\
	&\nla{} + \omega\ind{^b_{c\rho}} e\ind{^c_\mu} - 
	\frac{\pd_\rho l}{l} e\ind{^b_\mu} \Big) - \frac{\pd_\rho 
		l}{l} \bigg( \cosh z\, e\ind{^a_\mu} - (\cosh z - 1) 
	\frac{\xi\ind{_b} e\ind{^b_\mu} \xi^a}{\xi^2} + \frac{\sinh 
		z}{z}(\pd_\mu \xi^a
	\\
	&\nla{} + \omega\ind{^a_{b\mu}} \xi^b)- \frac{\pd_\mu l}{l} 
	\xi^a - \Big( \frac{\sinh z}{z} - 1 \Big) \frac{\xi_b \pd_\mu 
		\xi^b \xi^a}{\xi^2} \bigg) \bigg] - \bigg[ \rho 
	\leftrightarrow \mu \bigg] \bigg\} \bigg]
	\\
	&= \int \delta x^\mu \bigg[ -d\bar{u}\ind{_a} 
	\bar{e}\ind{^a_\mu} + \bar{u}\ind{_a} 
	\bar{\omega}\ind{^a_{b\rho}} \bar{e}\ind{^b_\mu} dx^\rho - 
	\bar{u}\ind{_a} \bar{\omega}\ind{^a_{b\mu}} 
	\bar{e}\ind{^b_\rho} dx^\rho - \bar{u}_a dx^\rho \bigg( 
		\frac{\sinh z}{z} \xi^c R\ind{^a_{c\rho\mu}}
	\\
	&\nla{} + \cosh z\, T\ind{^a_{\rho\mu}} + (1 - \cosh z) 
	\frac{\xi^a \xi\ind{_b} T\ind{^b_{\rho\mu}}}{\xi^2} \bigg) - 
	\frac{\pd_\rho l}{l} \bar{e}\ind{^a_\mu} \bar{u}\ind{_a} 
	dx^\rho + \frac{\pd_\mu l}{l} \bar{e}\ind{^a_\rho} 
	\bar{u}\ind{_a} dx^\rho \bigg]
\end{split}
\end{displaymath}
In this expression one recognizes the torsion $\bar{T}$, as given 
in Eq.~\eqref{eq:Tnonlin}. This leads to the end of this 
calculation:
\begin{equation}
	\int \bar{u}_a \delta\bar{e}^a = \int d\tau \delta x^\mu 
	\bigg\{ - \bar{e}\ind{^a_\mu} \bigg( \frac{d\bar{u}_a}{d\tau} 
	- \bar{\omega}\ind{^b_{a\rho}} \bar{u}\ind{_b} u^\rho + u^\rho 
	\frac{\pd_\rho l}{l} \bar{u}\ind{_a} \bigg) + 
	\bar{T}\ind{^a_{\mu\rho}} \bar{u}\ind{_a} u^\rho + 
	\frac{\pd_\mu l}{l} \bigg\}~.
\end{equation}

\subsection{Variation of $\mathcal{L}$ with respect to 
	$\bar{e}$.}
\label{app:field_eqs}

In this subsection we work out some intermediary results that 
lead to the functional variation of the 
Lagrangian~\eqref{eq:lagrangian_dSTG} with respect to the 
vierbein. More precisely, we calculate the derivatives of 
$\mathcal{L}$ with respect to $\bar{e}\ind{^c_\sigma}$ and 
$\pd_\rho \bar{e}\ind{^c_\sigma}$ in turn.  
From~\eqref{eq:lagrangian_dSTG}:
\begin{displaymath}
\begin{split}
	\frac{\pd \mathcal{L}}{\pd \bar{e}\ind{^c_\sigma}}
	&= \frac{1}{4} \frac{\pd \bar{T}\ind{^a_{\mu\nu}}}{\pd 
		\bar{e}\ind{^c_\sigma}} \bar{T}\ind{_a^{\mu\nu}}
	+ \frac{1}{4} \bar{T}\ind{^a_{\mu\nu}} \frac{\pd 
		\bar{T}\ind{_a^{\mu\nu}}}{\pd \bar{e}\ind{^c_\sigma}}
	+ \frac{1}{2} \frac{\pd \bar{T}\ind{^a_{\mu\nu}}}{\pd 
		\bar{e}\ind{^c_\sigma}} \bar{T}\ind{^{b\mu}_\lambda} 
	\bar{e}\ind{_a^\lambda} \bar{e}\ind{_b^\nu}
	+ \frac{1}{2} \bar{T}\ind{^a_{\mu\nu}} \frac{\pd 
		\bar{T}\ind{^{b\mu}_\lambda}}{\pd \bar{e}\ind{^c_\sigma}} 
	\bar{e}\ind{_a^\lambda} \bar{e}\ind{_b^\nu}
	\\
	&\nla{} + \frac{1}{2} \bar{T}\ind{^a_{\mu\nu}} 
	\bar{T}\ind{^{b\mu}_\lambda} \frac{\pd( 
		\bar{e}\ind{_a^\lambda} \bar{e}\ind{_b^\nu})}{\pd 
		\bar{e}\ind{^c_\sigma}}
	- \frac{\pd \bar{T}\ind{^a_{\mu\nu}}}{\pd 
		\bar{e}\ind{^c_\sigma}} \bar{T}\ind{^{b\mu}_\lambda} 
	\bar{e}\ind{_a^\nu} \bar{e}\ind{_b^\lambda}
	- \bar{T}\ind{^a_{\mu\nu}} \frac{\pd 
		\bar{T}\ind{^{b\mu}_\lambda}}{\pd \bar{e}\ind{^c_\sigma}} 
	\bar{e}\ind{_a^\nu} \bar{e}\ind{_b^\lambda}
	\\
	&\nla{} - \bar{T}\ind{^a_{\mu\nu}} 
	\bar{T}\ind{^{b\mu}_\lambda} \frac{\pd( \bar{e}\ind{_a^\nu} 
		\bar{e}\ind{_b^\lambda})}{\pd \bar{e}\ind{^c_\sigma}}
\end{split}
\end{displaymath}
It is thus useful to consider first the following equalities:
\begin{gather*}
	\frac{\pd \bar{T}\ind{^a_{\mu\nu}}}{\pd \bar{e}\ind{^c_\sigma}}
	= \big[ \bar{\omega}\ind{^a_{c\mu}} \delta^\sigma_\nu - 
	\frac{\pd_\mu l}{l} \delta^a_c \delta^\sigma_\nu \big] - \big[ 
	\mu \leftrightarrow \nu \big]~,
	\\
	\frac{\pd g_{\rho\lambda}}{\pd \bar{e}\ind{^c_\sigma}} = 
	\frac{\pd (\bar{e}\ind{^a_\rho} \bar{e}\ind{_{a\lambda}})}{\pd 
		\bar{e}\ind{^c_\sigma}} = \bar{e}\ind{_{c\lambda}} 
	\delta^\sigma_\rho + \bar{e}\ind{_{c\rho}} 
	\delta^\sigma_\lambda~,
	\\
	\frac{\pd g^{\rho\lambda}}{\pd \bar{e}\ind{^c_\sigma}} =
	- g^{\sigma\rho} \bar{e}\ind{_c^\lambda} - g^{\sigma\lambda} 
	\bar{e}\ind{_c^\rho}~.
\end{gather*}
Subsequently it is possible to obtain
\begin{gather*}
	\frac{\pd \bar{e}\ind{_a^\lambda}}{\pd \bar{e}\ind{^c_\sigma}} 
	= - \bar{e}\ind{_a^\sigma} \bar{e}\ind{_c^\lambda}~,
	\\
	\begin{split}
	\frac{\pd \bar{T}\ind{_a^{\mu\nu}}}{\pd \bar{e}\ind{^c_\sigma}} 
	&= \big[ \eta_{ab} \bar{\omega}\ind{^b_{c\alpha}} g^{\alpha\mu} 
	g^{\sigma\nu} - \eta_{ac} \frac{\pd_\lambda l}{l} 
	g^{\lambda\mu} g^{\sigma\nu} + \bar{T}\ind{_a^{\sigma\mu}} 
	\bar{e}\ind{_c^\nu} \\
	&\nla{}+ \bar{T}\ind{_{a\lambda}^\mu} \bar{e}\ind{_c^\lambda} 
	g^{\sigma\nu} \big] - \big[ \mu \leftrightarrow \nu \big]~,
	\end{split}
	\\
	\begin{split}
	\frac{\pd \bar{T}\ind{^{b\mu}_\lambda}}{\pd 
		\bar{e}\ind{^c_\sigma}}
	&= g^{\rho\mu} \bar{\omega}\ind{^b_{c\rho}} 
	\delta^\sigma_\lambda - g^{\sigma\mu} 
	\bar{\omega}\ind{^b_{c\lambda}} - g^{\rho\mu} \frac{\pd_\rho 
		l}{l} \delta^b_c \delta^\sigma_\lambda + g^{\sigma\mu} 
	\frac{\pd_\lambda l}{l} \delta^b_c \\
	&\nla{}- \bar{T}\ind{^b_{\rho\lambda}} \bar{e}\ind{_c^\mu} 
	g^{\sigma\rho} - \bar{T}\ind{^b_{\rho\lambda}} 
	\bar{e}\ind{_c^\rho} g^{\sigma\mu}~.
	\end{split}
\end{gather*}
Substituting these equations for $\pd \mathcal{L}/\pd 
\bar{e}\ind{^c_\sigma}$, it takes some algebra to get the 
following:
\begin{equation}
	\frac{\pd \mathcal{L}}{\pd \bar{e}\ind{^c_\sigma}}
	= \bar{\omega}\ind{^a_{c\mu}} \bar{W}\ind{_a^{\mu\sigma}} + 
	\bar{T}\ind{^a_{\mu c}} \bar{W}\ind{_a^{\sigma\mu}} - 
	\frac{\pd_\mu l}{l} \bar{W}\ind{_c^{\mu\sigma}}~.
\end{equation}

It is a simpler exercise to find the derivative of the Lagrangian 
with respect to the first order derivatives of the vierbein. One 
only needs the expression
\begin{displaymath}
	\frac{\pd \bar{T}\ind{^a_{\mu\nu}}}{\pd\pd_\rho 
		\bar{e}\ind{^c_\sigma}} = \delta^\rho_\mu \delta^\sigma_\nu 
	\delta^a_c - \delta^\rho_\nu \delta^\sigma_\mu \delta^a_c~.
\end{displaymath}
This is sufficient since the derivative operator annihilates the 
metric $g_{\mu\nu} = \bar{e}\ind{^a_\mu}\bar{e}\ind{_{a\nu}}$ and 
we can freely raise and lower spacetime indices. Using this 
information, it is readily found that
\begin{equation}
	\frac{\pd \mathcal{L}}{\pd\pd_\rho \bar{e}\ind{^c_\sigma}} = 
	\bar{W}\ind{_c^{\rho\sigma}}~.
\end{equation}

\newpage
\bibliographystyle{ieeetr}
\bibliography{../../references/All}
%\bibliography{/home/hendrik/research/drafts/All.bib}
\end{document}

