\documentclass[11pt]{article}

%Load preamble files
\usepackage{../../Tex_files/standard}
\usepackage{../../Tex_files/preamble_one}
%\usepackage{showframe} %show frame borders

\usepackage{mathtools}
\mathtoolsset{centercolon}



\usepackage[all]{xy}

\title{Contracted Bianchi identities -- Field equations}
\author{Hendrik}
\date{\today}

\begin{document}

\maketitle

\section{Teleparallel Gravity}

As discussed in the document \verb+cartan_geo_ext.pdf+ the 
geometric setting underlying Teleparallel Gravity is that of a 
Riemann-Cartan geometry with vanishing curvature. The Bianchi 
identities for a generic Riemann-Cartan\footnote{RC geometry.}  
geometry are given by
\begin{gather*}
	dR + [\omega,R] \equiv 0~,\\
	dT + [\omega,T] + [e,R] \equiv0~,
\end{gather*}
which in the case of Teleparallel Gravity reduce to
\begin{gather*}
	0 \equiv 0~,\\
	dT + [\omega,T] \equiv0~.
\end{gather*}
Apart from these Bianchi identities for the torsion $T$, there 
are two identities at hand for the corresponding contortion $K$.  
The latter is defined through the splitting $\omega = 
\mathring{\omega} + K$, where $\mathring{\omega}$ is the unique 
Levi-Civita spin connection, i.e., the spin connection without 
torsion.  Introducing the $\mathfrak{so}(1,3)$-valued two form $Q 
:= dK + [\omega,K] - \tfrac{1}{2}[K,K]$, these identities are of 
the form (for a generic RC geometry)
\begin{gather*}
	dQ + [\omega, Q] - [K,Q] + [K,R] \equiv 0~,\\
	[e,Q] - [e,R] \equiv 0~.
\end{gather*}
Specializing for Teleparallel Gravity, one finds that
\begin{subequations}
\begin{gather}
	\label{eq:bianchi_ISO_DQ}
	dQ + [\omega, Q] - [K,Q] \equiv 0~,\\
	\label{eq:bianchi_ISO_eQ}
	[e,Q] \equiv 0~.
\end{gather}
\end{subequations}
Of course, the two sets of Bianchi identities for $\omega$ and 
$K$ are related by considering the identities for the Levi-Civita 
connection $\mathring{\omega}$.

Introducing the notation $\mathring{D} := d + \mathring{\omega}$, 
the first identity~\eqref{eq:bianchi_ISO_DQ} can be expanded as 
$\mathring{D}_{[\rho} Q\ind{^{ab}_{\mu\nu]}} \equiv 0$.
Contracting this equation twice with the vielbein results in
\begin{displaymath}
	\mathring{D}_\rho \mathcal{Q} - 2 \mathring{D}_\rho 
	e\ind{_a^\mu} Qi\ind{^a_\mu} + 2 e\ind{_a^\mu} 
	\mathring{D}_\mu Qi\ind{^a_\rho} - 2 e\ind{_a^\mu} 
	\mathring{D}_\mu e\ind{_b^\nu} Q\ind{^{ab}_{\nu\rho}} \equiv 
	0~,
\end{displaymath}
where the notation $\mathcal{Q} := Q\ind{^{ab}_{\mu\nu}}
e\ind{_a^\mu} e\ind{_b^\nu}$ and $Qi\ind{^a_\mu} := 
Q\ind{^{ab}_{\mu\nu}} e\ind{_b^\nu}$ has been used.\footnote{Let 
	us note, \emph{en passant}, that due to the 
	identity~\eqref{eq:bianchi_ISO_eQ}, $Qi_{\mu \nu}$ is a
	symmetric tensor.}
The vielbein postulate $D_\rho e\ind{_a^\mu} = 
-\Gamma\ind{^\mu_{\nu\rho}} e\ind{_a^\nu}$ allows us to eliminate 
the covariant derivatives on the tetrads, which renders
\begin{align*}
	0 &\equiv
	e\ind{_a^\mu} \mathring{D}_\mu Qi\ind{^a_\rho} - e\ind{_a^\mu} 
	\mathring{\Gamma}\ind{^\sigma_{\mu \rho}} Qi\ind{^a_\sigma} - 
	\tfrac{1}{2} \pd_\rho \mathcal{Q} \\
	&\equiv \mathring{\nabla}_\mu( Qi\ind{^\mu_\rho} - 
	\tfrac{1}{2} \delta^\mu_\rho \mathcal{Q} )~.
\end{align*}
This equation can be rewritten as follows:
\begin{align*}
	0 &\equiv
	\mathring{\nabla}_\mu Qi\ind{^\mu_a} e\ind{^a_\rho} + 
	Qi\ind{^\mu_a} \mathring{\nabla}_\mu e\ind{^a_\rho} - 
	\tfrac{1}{2} \mathring{\nabla}_\mu e\ind{^a_\rho} 
	e\ind{_a^\mu} \mathcal{Q} - \tfrac{1}{2} e\ind{^a_\rho} 
	\mathring{\nabla}_\mu e\ind{_a^\mu} \mathcal{Q} - 
	\tfrac{1}{2}e\ind{^a_\rho} e\ind{_a^\mu} \pd_\mu \mathcal{Q}\\
	&= \pd_\mu Qi\ind{^\mu_a} e\ind{^a_\rho} + \pd_\mu \ln e\, 
	Qi\ind{^\mu_a} e\ind{^a_\rho} - Qi\ind{^\mu_a} 
	\mathring{\omega}\ind{^a_{b\mu}} e\ind{^b_\rho} + \tfrac{1}{2} 
	\mathring{\omega}\ind{^a_{b\mu}} e\ind{^b_\rho} e\ind{_a^\mu} 
	\mathcal{Q} \\
	&\nla{}- \tfrac{1}{2} e\ind{^a_\rho} \pd_\mu e\ind{_a^\mu} 
	\mathcal{Q} - \tfrac{1}{2} e\ind{^a_\rho} \pd_\mu \ln e\, 
	e\ind{_a^\mu} \mathcal{Q} - \tfrac{1}{2}e\ind{^a_\rho} 
	e\ind{_a^\mu} \pd_\mu \mathcal{Q} \\
	&= e\ind{^a_\rho} \big( \pd_\mu \ln e\, Qi\ind{^\mu_a} + 
	\mathring{D}_\mu Qi\ind{^\mu_a} - \tfrac{1}{2} \pd_\mu \ln e\, 
	e\ind{_a^\mu} \mathcal{Q} - \tfrac{1}{2} \mathring{D}_\mu 
	e\ind{_a^\mu} \mathcal{Q} - \tfrac{1}{2} e\ind{_a^\mu} \pd_\mu 
	\mathcal{Q} \big)~,
\end{align*}
which, given that the vierbein be invertible, is true if and only 
if
\begin{equation}
	\label{eq:contr_bianchi_DQ}
	\mathring{D}_\mu \big( e\, Qi\ind{^\mu_a} - \tfrac{1}{2} e\, 
	e\ind{_a^\mu} \mathcal{Q} \big) \equiv 0~.
\end{equation}

To conclude what conditions the contracted Bianchi 
identity~\eqref{eq:contr_bianchi_DQ} puts on the field equations 
of Teleparallel Gravity, we further work out the tensors 
$Qi\ind{^\mu_a}$ and $\mathcal{Q}$. These were defined as 
contractions of the Lorentz algebra valued two-form
\begin{displaymath}
	Q\ind{^{ab}_{\mu\nu}} = D_\mu K\ind{^{ab}_\nu} - D_\nu 
	K\ind{^{ab}_\mu} - K\ind{^a_{c\mu}} K\ind{^{cb}_\nu} + 
	K\ind{^a_{c\nu}} K\ind{^{cb}_\mu}~.
\end{displaymath}
Let us first calculate the scalar function $\mathcal{Q} = 
Q\ind{^{ab}_{\mu\nu}} e\ind{_a^\mu} e\ind{_b^\nu}$:
\begin{align*}
	\mathcal{Q} &= \pd_\mu K\ind{^{\mu\nu}_\nu} - D_\mu 
	e\ind{_a^\mu} K\ind{^{a\nu}_\nu} - D_\mu e\ind{_b^\nu} 
	K\ind{^{\mu b}_\nu} - \pd_\nu K\ind{^{\mu\nu}_\mu} + D_\nu 
	e\ind{_a^\mu} K\ind{^{a\nu}_\mu}  \\
	&\nla{}+ D_\nu e\ind{_b^\nu} K\ind{^{\mu b}_\mu} - 
	K\ind{^\mu_{\rho \mu}} K\ind{^{\rho\nu}_\nu} + 
	K\ind{^\mu_{\rho \nu}} K\ind{^{\rho\nu}_\mu} \\
	&= 2 \pd_\mu K\ind{^{\mu\nu}_\nu} + \Gamma\ind{^\mu_{\rho\mu}} 
	e\ind{_a^\rho} K\ind{^{a\nu}_\nu} + \Gamma\ind{^\nu_{\rho\mu}} 
	e\ind{_b^\rho} K\ind{^{\mu b}_\nu} - 
	\Gamma\ind{^\mu_{\rho\nu}} e\ind{_a^\rho} K\ind{^{a \nu}_\mu} 
	\\
	&\nla{}- \Gamma\ind{^\nu_{\rho\nu}} e\ind{_b^\rho}
	K\ind{^{\mu b}_\mu} - K\ind{^\mu_{\rho \mu}} 
	K\ind{^{\rho\nu}_\nu} + K\ind{^\mu_{\rho \nu}} 
	K\ind{^{\rho\nu}_\mu} \\
	&= 2 \pd_\mu K\ind{^{\mu\nu}_\nu} + 2 
	\mathring{\Gamma}\ind{^\mu_{\rho\mu}} K\ind{^{\rho\nu}_\nu} + 
	2 K\ind{^\mu_{\rho \mu}} K\ind{^{\rho\nu}_\nu} + 2 
	\mathring{\Gamma}\ind{^\nu_{\rho\mu}} K\ind{^{\mu\rho}_\nu} + 
	2 K\ind{^\nu_{\rho\mu}} K\ind{^{\mu\rho}_\nu} \\
	&\nla{}- K\ind{^\mu_{\rho \mu}} K\ind{^{\rho\nu}_\nu} + 
	K\ind{^\mu_{\rho \nu}} K\ind{^{\rho\nu}_\mu} \\
	&= \frac{2}{e} \pd_\mu( e K\ind{^{\mu\nu}_\nu}) + 
	K\ind{^\nu_{\rho\mu}} K\ind{^{\mu\rho}_\nu} - K\ind{^\mu_{\rho 
			\mu}} K\ind{^{\nu\rho}_\nu}~.
\end{align*}
Subsequently we take a better look at $Qi\ind{^\mu_a} =
Q\ind{^{cb}_{\rho\nu}} e\ind{_c^\mu} e\ind{_a^\rho} 
e\ind{_b^\nu}$:
\begin{align*}
	Qi\ind{^\mu_a} &= D_\rho( e\ind{_a^\rho}  K\ind{^{\mu\nu}_\nu} 
	) -  D_\rho e\ind{_c^\mu} e\ind{_a^\rho} K\ind{^{c\nu}_\nu} - 
	D_\rho e\ind{_a^\rho} K\ind{^{\mu\nu}_\nu} - e\ind{_a^\rho} 
	D_\rho e\ind{_b^\nu} K\ind{^{\mu b}_\nu} - D_\nu 
	K\ind{^{\mu\nu}_a} \\
	&\nla{}+  D_\nu e\ind{_c^\mu} K\ind{^{c\nu}_a} + D_\nu 
	e\ind{_a^\rho} K\ind{^{\mu\nu}_\rho} + D_\nu e\ind{_b^\nu} 
	K\ind{^{\mu b}_a} - K\ind{^\mu_{\rho a}} K\ind{^{\rho\nu}_\nu} 
	+ K\ind{^\mu_{\rho \nu}} K\ind{^{\rho\nu}_a}
	\\
	&= D_\nu K\ind{^{\nu\mu}_a} - \Gamma\ind{^\nu_{\rho\nu}} 
	e\ind{_b^\rho} K\ind{^{\mu b}_a} + D_\rho( e\ind{_a^\rho}  
	K\ind{^{\mu\nu}_\nu} ) + \Gamma\ind{^\rho_{\sigma\rho}} 
	e\ind{_a^\sigma} K\ind{^{\mu\nu}_\nu} + 
	\Gamma\ind{^\mu_{\sigma\rho}} e\ind{_c^\sigma} e\ind{_a^\rho} 
	K\ind{^{c\nu}_\nu} \\
	&\nla{} + e\ind{_a^\rho} \Gamma\ind{^\nu_{\sigma\rho}} 
	e\ind{_b^\sigma} K\ind{^{\mu b}_\nu} - 
	\Gamma\ind{^\mu_{\rho\nu}} e\ind{_c^\rho} K\ind{^{c\nu}_a} - 
	\Gamma\ind{^\rho_{\sigma\nu}} e\ind{_a^\sigma} 
	K\ind{^{\mu\nu}_\rho} - K\ind{^\mu_{\rho a}} 
	K\ind{^{\rho\nu}_\nu} \\
	&\nla{}+ K\ind{^\mu_{\rho \nu}} K\ind{^{\rho\nu}_a}
	\\
	&= e^{-1} D_\nu (e K\ind{^{\nu\mu}_a} ) + 
	K\ind{^\nu_{\rho\nu}} K\ind{^{\rho\mu}_a} + e^{-1} D_\nu ( e\,
	e\ind{_a^\nu} K\ind{^{\mu\rho}_\rho} ) + K\ind{^\rho_{a\rho}} 
	K\ind{^{\mu\nu}_\nu}	\\
	&\nla{}+ e\ind{_a^\rho} \Gamma\ind{^\mu_{\sigma\rho}} 
	K\ind{^{\sigma\nu}_\nu} + e\ind{_a^\rho} ( 
	\Gamma\ind{^\nu_{\sigma\rho}} - \Gamma\ind{^\nu_{\rho\sigma}} 
	) K\ind{^{\mu\sigma}_\nu} - K\ind{^\mu_{\rho\nu}} 
	K\ind{^{\rho\nu}_a} - K\ind{^\mu_{\rho a}} 
	K\ind{^{\rho\nu}_\nu} \\
	&\nla{}+ K\ind{^\mu_{\rho \nu}} K\ind{^{\rho\nu}_a}
	\\
	&= e^{-1} D_\nu( e K\ind{^{\nu\mu}_a} + e\, e\ind{_a^\nu} 
	K\ind{^{\mu\rho}_\rho} ) +  T\ind{^\nu_{\sigma a}}	
	K\ind{^{\sigma\mu}_\nu} + e\ind{_a^\rho} 
	\Gamma\ind{^\mu_{\sigma\rho}} K\ind{^{\sigma\nu}_\nu} + 
	K\ind{^\rho_{a\rho}} K\ind{^{\mu\nu}_\nu}~.
\end{align*}
One may consider the difference
\begin{align*}
	Qi\ind{^\mu_a} - &\tfrac{1}{2} e\ind{_a^\mu} \mathcal{Q}
	\\
	&= e^{-1} D_\nu( e K\ind{^{\nu\mu}_a} + e\, e\ind{_a^\nu} 
	K\ind{^{\mu\rho}_\rho} ) +  T\ind{^\nu_{\sigma a}}	
	K\ind{^{\sigma\mu}_\nu} + e\ind{_a^\rho} 
	\Gamma\ind{^\mu_{\sigma\rho}} K\ind{^{\sigma\nu}_\nu} + 
	K\ind{^\rho_{a\rho}} K\ind{^{\mu\nu}_\nu} \\
	&\nla{}- e^{-1} \pd_\nu( e\, e\ind{_a^\mu} 
	K\ind{^{\nu\rho}_\rho} ) + \pd_\nu e\ind{_a^\mu} 
	K\ind{^{\nu\rho}_\rho} - \tfrac{1}{2} e\ind{_a^\mu} 
	K\ind{^\nu_{\rho\sigma}} K\ind{^{\sigma\rho}_\nu} + 
	\tfrac{1}{2} e\ind{_a^\mu}	K\ind{^\sigma_{\rho \sigma}} 
	K\ind{^{\nu\rho}_\nu}
	\\
	&= e^{-1} D_\nu( e K\ind{^{\nu\mu}_a} + e\, e\ind{_a^\nu} 
	K\ind{^{\mu\rho}_\rho} - e\, e\ind{_a^\mu} 
	K\ind{^{\nu\rho}_\rho} ) +  T\ind{^b_{\sigma a}}	
	K\ind{^{\sigma\mu}_b} + K\ind{^\rho_{a\rho}} 
	K\ind{^{\mu\nu}_\nu} \\
	&\nla{}+ e\ind{_a^\rho}( \Gamma\ind{^\mu_{\sigma\rho}} - 
	\Gamma\ind{^\mu_{\rho\sigma}} ) K\ind{^{\sigma\nu}_\nu} - 
	\tfrac{1}{2} e\ind{_a^\mu} \mathcal{L}_\mathrm{tg} \\
	&= \tfrac{1}{2} e^{-1} D_\nu (e\, W\ind{_a^{\nu\mu}} )
	+ T\ind{^b_{\sigma a}} K\ind{^{\sigma\mu}_b} - 
	T\ind{^\mu_{\sigma a}} K\ind{^{\sigma\nu}_\nu} + 
	K\ind{^\rho_{a\rho}} K\ind{^{\mu\nu}_\nu} - \tfrac{1}{2} 
	e\ind{_a^\mu} \mathcal{L}_\mathrm{tg}~,
\end{align*}
where the notation $W\ind{_a^{\nu\mu}} = 2( K\ind{^{\nu\mu}_a} + 
e\ind{_a^\nu} K\ind{^{\mu\rho}_\rho} - e\ind{_a^\mu} 
K\ind{^{\nu\rho}_\rho} )$ is used in the last line. Since 
$K\ind{^{a\nu}_\nu} = T\ind{^{\nu a}_\nu}$, one further concludes 
that
\begin{align*}
	Qi\ind{^\mu_a} - &\tfrac{1}{2} e\ind{_a^\mu} \mathcal{Q}
	\\
	&= \tfrac{1}{2} e^{-1} D_\nu (e\, W\ind{_a^{\nu\mu}} ) + 
	T\ind{^b_{\nu a}} K\ind{^{\nu\mu}_b} - T\ind{^b_{\nu a}} 
	e\ind{_b^\mu} K\ind{^{\nu\rho}_\rho} + T\ind{^\rho_{\rho a}} 
	K\ind{^{\mu\nu}_\nu} - \tfrac{1}{2} e\ind{_a^\mu} 
	\mathcal{L}_\mathrm{tg}
	\\
	&= \tfrac{1}{2} e^{-1} D_\nu (e\, W\ind{_a^{\nu\mu}} ) + 
	T\ind{^b_{\nu a}} ( K\ind{^{\nu\mu}_b} + e\ind{_b^\nu} 
	K\ind{^{\mu\rho}_\rho} - e\ind{_b^\mu} K\ind{^{\nu\rho}_\rho} 
	) - \tfrac{1}{2} e\ind{_a^\mu} \mathcal{L}_\mathrm{tg}
	\\
	&= \tfrac{1}{2} \big[ e^{-1} D_\nu (e\, W\ind{_a^{\nu\mu}} ) + 
	T\ind{^b_{\nu a}} W\ind{_b^{\nu\mu}} -
	e\ind{_a^\mu} \mathcal{L}_\mathrm{tg} \big]~.
\end{align*}
Finally, the contracted Bianchi 
identities~\eqref{eq:contr_bianchi_DQ} imply that
\begin{equation}
	\mathring{D}_\mu \Big[ D_\nu (e\, W\ind{_a^{\nu\mu}} ) + e\, 
	T\ind{^b_{\nu a}} W\ind{_b^{\nu\mu}} - e\, e\ind{_a^\mu} 
	\mathcal{L}_\mathrm{tg} \Big] \equiv 0~.
\end{equation}

%\bibliography{/home/hendrik/research/drafts/references/All.bib}
%\bibliographystyle{plain}
%\bibliography{../../references/All}

\end{document}
