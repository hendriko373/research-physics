\documentclass[11pt]{article}

%Load preamble files
\usepackage{../../Tex_files/standard}
\usepackage{../../Tex_files/preamble_one}
%\usepackage{showframe} %show frame borders

\usepackage{mathtools}

\usepackage[all]{xy}

\title{Cartan geometry}
\author{Hendrik}
\date{\today}

\begin{document}

\maketitle

\section{Lie algebra-valued differential forms}
\label{app:Lie_val_forms}

Let $\mathcal{M}$ be an $m$-dimensional smooth manifold and let 
$\mathfrak{g}$ be a Lie algebra with Lie bracket $[\cdot,\cdot] : 
\mathfrak{g} \times \mathfrak{g} \to \mathfrak{g}$. The space of 
$\mathfrak{g}$-valued differential $p$-forms on $\mathcal{M}$, 
i.e.,~$\mathfrak{g} \otimes \Omega^p(\mathcal{M})$, is denoted by 
$\Omega^p(\mathcal{M},\mathfrak{g})$. An element $\eta \in 
\Omega^p(\mathcal{M},\mathfrak{g})$ may be expanded in a basis 
$E_a$ for $\mathfrak{g}$ as $\eta = \eta^a \otimes E_a$, also 
written as $\eta^a E_a$, and where each $\eta^a \in 
\Omega^p(\mathcal{M})$. 

For any two elements $\eta \in \Omega^p(\mathcal{M},\mathfrak{g})$ and 
$\theta \in \Omega^q(\mathcal{M},\mathfrak{g})$ with $p + q \leq m$, a Lie 
bracket is defined by
\begin{displaymath}
	[.,.] : \Omega^p(\mathcal{M},\mathfrak{g}) \times 
	\Omega^q(\mathcal{M},\mathfrak{g}) \to \Omega^{p+q}(\mathcal{M},\mathfrak{g})
	: (\eta,\theta) \mapsto [\eta,\theta] = \eta^a \wedge \theta^b 
	\otimes [E_a,E_b],
\end{displaymath}
where the bracket in the last term is of course the ordinary Lie 
bracket of $\mathfrak{g}$. It is easily verified that this 
operation is a graded commutator, namely,
\begin{displaymath}
	[\eta,\theta] = (-1)^{pq + 1} [\theta,\eta].
\end{displaymath}
The Jacobi identity for $\mathfrak{g}$ consequently generalizes 
to a graded Jacobi identity (let $\omega \in 
\Omega^r(\mathcal{M},\mathfrak{g})$:
\begin{displaymath}
	(-1)^{rp}[[\eta,\theta],\omega] + 
	(-1)^{pq}[[\theta,\omega],\eta] +
	(-1)^{qr}[[\omega,\eta],\theta] \equiv 0.
\end{displaymath}

The exterior derivative $d : \Omega^p(\mathcal{M}) \to 
\Omega^{p+1}(\mathcal{M})$ of differential forms can equally be 
defined an operation on the space of $\mathfrak{g}$-valued 
differential forms, by limiting its action on the form parts, 
i.e.,
\begin{displaymath}
	d : \Omega^p(\mathcal{M},\mathfrak{g}) \to \Omega^{p+1}(\mathcal{M},\mathfrak{g})
	: \eta \mapsto d\eta = d\eta^a \otimes E_a.
\end{displaymath}
This derivative respects a graded Leibniz rule:
\begin{displaymath}
	d[\eta,\theta] = [d\eta,\theta] + (-1)^p [\eta,d\theta].
\end{displaymath}

\end{document}
