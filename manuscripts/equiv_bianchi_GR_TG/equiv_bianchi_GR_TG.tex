\documentclass[11pt]{article}

%Load preamble files
\usepackage{../../Tex_files/standard}
\usepackage{../../Tex_files/preamble_one}
%\usepackage{showframe} %show frame borders

\usepackage{mathtools}

\usepackage[all]{xy}

\title{Equivalence Bianchi identities GR - TG}
\author{Hendrik}
\date{\today}

\begin{document}

\maketitle

\begin{abstract}
	In this document we show that the Bianchi identities for 
	Teleparallel Gravity, as they are derived 
	in~\cite{aldrovandi:2012tele}, i.e., from the Bianchi 
	identities for General Relativity together with the dictionary 
	$\omega = \mathring{\omega} + K$, are just the Bianchi 
	identities for a Riemann-Cartan geometry for the connection 
	$\omega$.
\end{abstract}

A generic connection $\omega$ can be written as the sum (see 
e.g.~\cite{kob1996found})
\begin{equation}
	\label{eq:decomp_contortion}
	\omega = \mathring{\omega} + K~,
\end{equation}
where $K \in \Omega^1(M,\mathfrak{so}(1,3))$ is called the 
\emph{contortion} of $\omega$. This connection fulfills the 
Bianchi identities
\begin{subequations}
\label{eq:bianchiISO_hp}
\begin{alignat}{3}
	\label{eq:bianchiISO_h}
	dR + [\omega,R] &\equiv 0
		&\quad&\text{or}
		&\quad
		dR^{ab} + \omega\ind{^a_c} \wedge R^{cb} + \omega\ind{^b_c} 
		\wedge R^{ac} &\equiv 0~, \\
	\label{eq:bianchiISO_p}
	dT + [\omega,T] + [e,R] &\equiv 0
		&\quad&\text{or}
		&\quad
		dT^a + \omega\ind{^a_b} \wedge T^b + e^c \wedge R\ind{_c^a} 
		&\equiv 0~.
\end{alignat}
\end{subequations}
We next show explicitely that the decomposition in 
Eq.~\eqref{eq:decomp_contortion} is consistent with the overal 
structure of the Cartan geometry and derive the geometry of $A = 
\omega + e$ from the geometry of $\mathring{A} = 
\mathring{\omega} + e$ by use of the dictionary 
$\mathring{\omega} = \omega - K$. Consider therefore first the 
curvature and torsion of $\mathring{\omega}$:
\begin{align*}
	\mathring{R} &= d\mathring{\omega} + \frac{1}{2} 
	[\mathring{\omega},\mathring{\omega}] \\
	&= d\omega - dK + \frac{1}{2} [\omega,\omega] - [\omega,K] + 
	\frac{1}{2}[K,K] \\
	&= R - Q~,
\end{align*}
where we denoted $Q \coloneqq dK + [\omega,K] - 
\tfrac{1}{2}[K,K]$. For the torsion one finds
\begin{displaymath}
	0 \equiv \mathring{T} = de + [\omega,e] - [K,e]~,
\end{displaymath}
which implies that the torsion and the contortion of $\omega$ are 
related by
\begin{equation}
	T = [K,e]~.
\end{equation}
Let us then turn attention to the Bianchi identities. The one for 
the curvature leads to
\begin{align*}
	0 &\equiv d\mathring{R} + [\mathring{\omega},\mathring{R}] \\
	&= dR - dQ + [\omega,R] - [\omega,Q] - [K,R] + [K,Q] \\
	&= dR + [\omega,R] - [K,R] - [d\omega,K] + [\omega,dK] + 
	[dK,K] - [\omega,dK] \\
	&\nla{} - [\omega,[\omega,K]] + \tfrac{1}{2} [\omega,[K,K]] + 
	[K,dK] + [K,[\omega,K]] - \tfrac{1}{2} [K,[K,K]] \\
	&= dR + [\omega,R] - [K,R] - [d\omega,K] + \tfrac{1}{2} 
	[K,[\omega,\omega]] + \tfrac{1}{2} [\omega,[K,K]] - 
	\tfrac{1}{2} [\omega,[K,K]] \\
	&= dR + [\omega,R]~.
\end{align*}
The Bianchi identity for the torsion gives result to
\begin{align*}
	0 &\equiv [e,\mathring{R}] \\
	&= [e,R] - [e,Q] \\
	&= [e,R] - [e,dK] - [e,[\omega,K]] + \tfrac{1}{2} [e,[K,K]] \\
	&= [e,R] + d[e,K] - [de,K] - [[K,e],\omega] - [[e,\omega],K] + 
	[[e,K],K] \\
	&= [e,R] + dT - [T,K] - [T,\omega] + [T,K] \\
	&= dT + [\omega,T] + [e,R]~.
\end{align*}
These results indeed are identical to the Bianchi identities of 
$\omega$.


\bibliography{/home/hendrik/research/drafts/references/All.bib}
\bibliographystyle{plain}

\end{document}

