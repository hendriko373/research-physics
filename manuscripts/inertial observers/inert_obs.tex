\documentclass[10pt]{article}

%Package declarations
%--------------------
\usepackage{setspace}
\setstretch{1.2}
\usepackage[a4paper]{geometry}
\geometry{left=3.5cm,right=3.5cm,top=3cm,bottom=3cm}

\usepackage[english]{babel}
\usepackage{color}

\usepackage{../Tex_files/preamble_one}

\numberwithin{equation}{section}

\author{}
\title{Inertial observers in de Sitter special relativity}
\date{}

\begin{document}
\maketitle

\begin{abstract}
	In this document we try to find a notion of inertial observers 
	in de Sitter special relativity. In doing so, we look for 
	inspiration in ordinary special relativity and try to give a 
	mathematical rigorous treatment. Also, we try to find a 
	definition for geodesics inside de Sitter special relativity.
\end{abstract}

\section{Inertial observers in ordinary special relativity}

Inertial observers in OSR\footnote{Ordinary Special Relativity} 
are those who describe Minkowski space with Cartesian coordinates 
$x^\mu$. Every one of them can relate its frame of reference with 
any other one through an element of the Poincar\'e group. The 
normal subgroup of Lorentz rotation $\mcal{L}$ relates observers 
which have a relative velocity. The quotient group 
$\mcal{P}/\mcal{L}$ of translations relates observers that are 
seperated in spacetime. Translations (left) are generated by the 
Killing fields $K_a = \delta^\mu_a \pd_\mu$. Not only are these 
vector fields Killing, they also form a tetrad or Vierbein 
$\{e_a\}$, i.e.\ at any point $p$ one has $e^a (e_b) = 
\delta^a_b$. A generic vector $\xi = \xi^a e_a$ is expressed with 
respect to this Vierbein. Since the latter are left invariant 
under translations, the notion of \emph{global} inertial 
observers emerges. Observers seperated in spacetime can compare 
their measurements (components) of $\xi$ as if they are residing 
in the same point, because a translation cannonically connects 
the frames considered. More precisely, if $\xi_t = \xi^a_t e_a$ 
is a field along a curve $\gamma_t$


\end{document}
