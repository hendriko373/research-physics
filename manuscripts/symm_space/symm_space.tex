\documentclass[11pt]{article}

%Package declarations
%--------------------
\usepackage{amsmath,amsfonts,amssymb}
\usepackage{setspace}
\setstretch{1.1}
\usepackage[a4paper]{geometry}
\geometry{left=3.5cm,right=3.5cm,top=3cm,bottom=3cm}

\usepackage[english]{babel}
\usepackage{color}

\usepackage{../Tex_files/preamble_one}

\title{de Sitter space as a homogeneous space}
\date{}

\begin{document}
\maketitle

Throughout this document, let $G$ be a connected Lie group and 
$H$ a closed subgroup.

\section{Introduction: Klein geometries}

In this section \emph{Klein geometries} are introduced. We 
discuss the relation between geometry and a corresponding Lie 
group structure, after which the relation to homogeneous spaces 
becomes evident.

Felix Klein realized that a geometry can be understood as 
consisting of a connected manifold $M$ together with a Lie group 
$G$ of ``motions" acting transitively on $M$ and which leaves all 
the ``properties of the figures" studied in the geometry 
invariant.  Being the simplest example, Euclidean geometry is 
given by the manifold (as an affine space) $M = \mbb{R}^n$ on 
which the group $G = \mbb{R}^n \rtimes SO(n)$ acts effectively 
and transitively, that leaves invariant the properties angle and 
length of figures (sets of points).  Since from Klein's point of 
view geometry is the study of the properties of the figures being 
invariant under the action of $G$, it is possible to shift 
emphasis from the space $M$ to the group $G$ in the following 
rigorous manner.

Fix a point $p$ in $M$. Then one may define a map $\lambda_p : G 
\to M : g \mapsto gp$. Since $G$ acts transitively on $M$, this 
mapping is surjective---however not necessarily 
one-to-one.\footnote{This would require a free action.}  
Furthermore, we assume in the remainder of the text that $G$ acts 
effectively on $M$. Consider the subset $H_p \subset G$ that maps 
into $p$ under $\lambda_p$, i.e.\
%
\begin{equation}
	H_p \equiv \lambda^{-1}_p(p) = \{g \in G~|~gp = p \}
\end{equation}
%
and which is called the \emph{stabilizer of $p$}. It is easily 
verified that $H_p$ is a closed subgroup of $G$. Let $q \in M$ 
and consider $\lambda^{-1}_p(q) = \{ g \in G~|~gp = q \}$. It 
follows that $\lambda^{-1}_p(q) = g_0 H_p$ where $g_0 \in 
\lambda^{-1}_p(q)$ is arbitrary. Different points in $M$ are the 
image of elements in $G$ that are not connected by $H$. Hence, 
$\lambda_p : G/H_p \to M$ is an isomorphism.  If another origin 
$\tilde{p} = g_1 p$ were chosen, $H_{\tilde{p}} = \{ g \in 
	G~|~gg_1p= g_1p \}$ and $H_{\tilde{p}} = \mrm{Ad}_{g_1}H_p$.  
Choosing different origins gives rise to conjugate stabilizers, 
implying they are isomorphic to eachother. Therefore it is common 
to just talk about \emph{the} stabilizer $H$, although a 
calculation will force you to pick any of them. It becomes 
evident that the choice of origin is arbitrary which explains why 
all the points in a Klein geometry are said to be equivalent. The 
above considerations have established the isomorphism between 
$G/H$ and $M$, where $H$ is the maximal isotropy group of a point 
in $M$.

Let us remark that while $M$ has no preferred origin, since the 
space is transitive under the symmetry group $G$, $G/H$ does have 
a preferred basepoint by picking out the identity coset. This is 
of course not a contradiction, since an element in $M$ was 
promoted to origin in order to establish the isomorphism.  As 
already mentioned, any other point could have been chosen which 
would correspond to a stabilizer conjugate to $H$. The knowledge 
that this choice is arbitrary restores this hidden symmetry in 
$G/H$. 

To recapitulate, a Klein geometry is defined.
%
\begin{definition}
	A \emph{Klein geometry} is a pair $(G,H)$, where $G$ is a Lie 
	group and $H$ a closed subgroup such that $G/H \equiv \{gH\}$ 
	is connected. The connected coset space $M \equiv G/H$ is 
	called \emph{the space of the Klein geometry}.

	A Klein geometry is called reductive if there is an 
	$\mrm{Ad}(H)$-module decomposition $\mfrak{g} = \mfrak{h} 
	\oplus \mfrak{p}$, where $\mfrak{g}$ and $\mfrak{h}$ are the 
	Lie algebras of $G$ and $H$, respectively.
   
	A metric Klein geometry is a reductive Klein geometry equipped 
	with an $\mathrm{Ad}(H)$-invariant metric $\eta$ on 
	$\mfrak{g}/\mfrak{h}$.
\end{definition}
The power of this definition is that the starting point is the 
Lie group structure, rather than a space. Given a Lie group $G$, 
a different Klein space is given for any closed subgroup 
$H$.\footnote{Here we mean of course subgroups that are not 
	isomorphic to eachother.}
The \emph{points} are the coset elements and their actual 
interpretation will depend on the subgroup $H$.

Since a homogeneous space $(X,G)$ is a space $X$ that is 
transitive under a symmetry group $G$, it is now clear that the 
Klein coset spaces are homogeneous.\footnote{Since the transitive 
	action of $G$ on a Klein space preserves the properties of the 
	geometry, it is impossible to distinguish a point from any 
	other by means of these properties of the geometry.}
The converse is not true in general, since $X$ and $G$ may have 
less structure than the objects defined above. In the remainder 
we will not make the distinction and assume a homogeneous space 
to have the mathematical structure of a Klein space, so that both 
terms may be used interchangeably.

\section{Symmetric homogeneous spaces}

\section{Example: de Sitter space in four dimensions}

\subsection{$dS$ as a Klein geometry}

In this section, de Sitter space will be studied from the 
algebraic point of view.

Consider the $n+1$-dimensional vector space $\mbb{R}^{n+1}$ with 
basis $(e_0,\ldots,e_n)$ and a metric structure, defined through 
the symmetric bilinear form
\begin{equation}
	F(x,y) = x^\tau \eta y
	\quad\mrm{with}~
	\eta =
	\begin{pmatrix}
		1 & 0 \\
		0 & -\mbb{I}_n
	\end{pmatrix}
\end{equation}
for $x$ and $y$ column vectors in $\mbb{R}^{n+1}$.
The orthogonal group $O(n,1)$ leaving invariant $F$ is a subgroup 
of the general linear group, namely
%
\begin{equation}
	O(n,1) = \{ A \in Gl(n+1,\mbb{R})~|~ A^\tau \eta A = \eta\}
\end{equation}
The componenent connected to the identity has determinant 1, and 
is denoted by $SO(n,1)$.  The elements of the Lie algebra 
$\mfrak{o}(n,1)$ are the tangent vectors at the identity of 
$SO(n,1)$. Therefore, consider a curve $A(t)$ through the 
identity $\mbb{I}$, such that $A(t) = \mbb{I} + tX + O(t^2)$ 
where $X = \dot{A}(0)$. A-priori, $X \in \mfrak{gl}(n+1,\mbb{R})$ 
but given the restriction that $A \in SO(n,1)$, it is directly 
found that  $X^\tau \eta + \eta X = 0$ and $\mrm{Tr}~X = 0$.

Let us make the isomorphism between $SO(n,1)/SO(n-1,1)$ and 
$n$-dimensional de Sitter space $dS_n$ manifest. The latter is 
the hypersurface in $\mbb{R}^{n+1}$ fulfilling
%
\begin{equation}
	dS_n : x^\tau \eta x = -l^2
\end{equation}
Consider $o = le_n \in dS_n$ and let $H$ be the isotropy group of 
$o$, that is all matrices $A \in SO(n,1)$ of the form $(Ae_n = 
e_n)$
%
\begin{displaymath}
	A =
	\begin{pmatrix}
		B & 0 \\
		0 & 1
	\end{pmatrix}
\end{displaymath}
and where $B \in SO(n-1,1)$ since we rotated around a spacelike 
axis. Let $\lambda_o$ be the mapping such that for $A \in 
SO(n,1)$, $\lambda_o(\pi(A)) = Ae_n$, where $\pi : SO(n,1) \to 
SO(n,1)/SO(n-1,1)$ is the natural projection. Since $SO(n,1)$ 
acts transitively on $dS_n$, $\lambda_o$ constitutes an 
isomorphism between $SO(n,1)/SO(n-1,1)$ and the $n$-dimensional 
de Sitter space $dS_n$.\footnote{Note how it is the transitivity 
	of $dS_n$ under $G$ which guarantees the isomorphism between 
	$G/H$ and $dS_n$: $gH$ acts transitivily on $dS_n$. This does 
	not seem to imply that ``pure translations" act transitively on 
	$dS_n$.}

An involutive automorphism $\sigma$ on $SO(n,1)$ is given by
\begin{displaymath}
	\sigma(A) = SAS^{-1}\quad\mrm{with}~
	S =
	\begin{pmatrix}
		\mbb{I}_n & 0 \\
		0 & -1
	\end{pmatrix}
\end{displaymath}
So $(SO(n,1),SO(n-1,1),\sigma)$ is a symmetric space and the 
canonical decomposition of the Lie algebra is given by 
$\mfrak{o}(n,1) = \mfrak{o}(n-1,1) \oplus \mfrak{p}$.

In the following let us focus on the four-dimensional case $dS 
\simeq SO(4,1)/SO(3,1)$. A generic element of $\mfrak{o}(4,1)$ in 
its fundamental representation is canonically decomposed with 
respect to $\sigma$ as
\begin{displaymath}
	\underbrace{\begin{pmatrix}
		0				& x_{01}	& x_{02}	& x_{03}	& x_{04} \\
		x_{01}	& 0				& x_{12}	& x_{13}	& x_{14} \\
		x_{02}	&	-x_{12}	& 0				& x_{23}	& x_{24} \\
		x_{03}	&	-x_{13}	& -x_{23}	& 0				& x_{34} \\
		x_{04}	&	-x_{14}	& -x_{24}	& -x_{34}	& 0	
	\end{pmatrix}}_{\mfrak{o}(4,1)}
	=
	\underbrace{\begin{pmatrix}
		B & 0 \\
		0 & 0
	\end{pmatrix}}_{\mfrak{o}(3,1)}
	+
	\underbrace{\begin{pmatrix}
		0								& \xi \\
		\bar{\xi}^\tau	& 0
	\end{pmatrix}}_{\mfrak{p}}
\end{displaymath}
where $\xi^a \in \mbb{R}^{4}$ and we used the shorthand notation 
$\bar{\xi} = \eta\xi$ with $\eta = (+,-,-,-)$. A generic element 
of $\mfrak{p}$ is written as $\xi = \xi^a P_a$ with $P_a$ the 
generators of de Sitter translations. We now use the Killing form 
on $\mfrak{g}$ to define a non-degenerate bilinear symmetric 
metric on the Lie algebra. We then define a metric
%
\begin{equation}
	\langle \xi,\eta \rangle := \tfrac{1}{2} \mathrm{tr}\, 
	(\xi\eta)~,\quad \xi,\eta \in \mfrak{g}~.
\end{equation}
The restriction of this metric on the $\mfrak{p}$-subspace gives 
a metric on $\mfrak{p}$. A quick calculation shows that $\langle 
\xi,\eta \rangle = \xi^a \eta^b \eta_{ab}$, for $\xi,\eta \in 
\mfrak{p}$.

The isomorphism $G/H \simeq dS$ induces an isomorphism between 
the corresponding tangent spaces, i.e.~an isomorphism $\mfrak{p} 
\simeq T_odS$. Let us make this manifest. Consider de Sitter 
space $dS$ with radius $\sqrt{3/\Lambda}$ and choose the origin 
$o = \sqrt{3/\Lambda}\,e_4 = (0,0,0,0,\sqrt{3/\Lambda})$.  Given 
this origin, the above introduced isomomorphism $\lambda_o : G/H 
\to dS$ was defined by $\lambda_o(\pi(A)) = Ao$ for any $A \in 
G$.  Let $\tfrac{1}{l}\xi^a P_a \in \mfrak{p}$ and 
$\exp(\tfrac{t}{l}\xi^a P_a)$ the corresponding one-parameter 
subgroup of $G$, where $l$ is a yet to be fixed length scale 
\cite{Wise:2006sm}. An isomorphism between $\mfrak{p}$ and 
$T_odS$ is then induced by $\lambda_o$, namely by considering its 
differential
%
\begin{displaymath}
	\lambda_{o\ast}\tfrac{d}{dt}f[\exp\Big(\frac{t}{l}\xi^a 
	P_a\Big)]|_{t=0} = \tfrac{d}{dt}[\exp\Big(\frac{t}{l}\xi^a 
	P_a\Big)o]^\mu \pd_\mu f = 
	\frac{1}{l}\sqrt{\frac{3}{\Lambda}}\xi^\mu\pd_\mu f
\end{displaymath}
for an arbitrary function $f$. Fixing the length scale 
\cite{Wise:2006sm}
\begin{displaymath}
	l = \sqrt{\frac{3}{\Lambda}}
\end{displaymath}
gives
%
\begin{equation}
	\lambda_{o\ast} : \mfrak{p} \to T_odS : \frac{1}{l}\xi^a P_a 
	\mapsto \xi^\mu \pd_\mu~.
\end{equation}
This mapping explains the choice for the length scale and not 
surpisingly is related to the fact that $G/H$ is a Klein 
geometry. Given an element $\xi^a P_a \in \mfrak{p}$, a length 
scale is introduced such that it may be identified with $\xi^\mu 
\pd_\mu$ on the tangent space of $dS$. In this case, the metric 
defined through the Killing form $\eta_{ab}\xi^a\xi^b$ equals the 
de Sitter metric applied to the corresponding elements in its 
tangent space at the origin, i.e.\ $\eta_{\mu\nu}\xi^\mu\xi^\nu$.  

\subsection{Geodesics: completeness but no connectedness}

The geodesics with respect to the canonical connection through 
the point $p=-le_4$ are given by the orbits of $p$ under $\exp 
\mfrak{p}$. More precisely, considering the isomorphism between 
$\mfrak{p}$ and $\mbb{R}^4$ given by
%
\begin{displaymath}
	X=
	\begin{pmatrix}
		0								& \xi \\
		\bar{\xi}^\tau	& 0
	\end{pmatrix}\leftrightarrow
	\xi
\end{displaymath}
the geodesic through $o \in G/H$ in the direction $\xi \in 
T_oG/H$ is the curve $\exp (tX) o$.

We construct these geodesics for $dS_4$ through $-le_4$ as 
parametrized curves in $\mbb{R}^5$. In fact, given $\xi \in 
T_0G/H$, $\exp (tX)$ is a rotation in the plane $P$ spanned by 
the the orthogonal basis $e_\ast = \xi^\mu e_\mu$ and $e_4$ $(\mu 
= 0\ldots 3)$.  It is then clear that the orbit through $p$ is 
the intersection of $P = \{a e_\ast + b e_4~|~a,b \in \mbb{R}\}$ 
and $dS_4$.  Since $e_4$ is spacelike $(F(e_4,e_4) = -1)$, there 
are three cases possible. Namely, $e_\ast$ timelike, spacelike or 
lightlike.
%
\begin{enumerate}
	\item $F(e_\ast,e_\ast) = -\alpha^2$. In this case, all 
		elements in $P$ are spacelike. The intersection $P \cap dS_4$ 
		is given by the elements in $P$ that satisfy $F(a e_\ast + b 
		e_4, a e_\ast + b e_4) = -l^2$, that is $\alpha^2 a^2 + b^2 = 
		l^2$. It is clear that $a= \pm l\alpha^{-1} \sin t$ and $b= 
		\pm l\cos t$. Given the fact that the curve goes through $p = 
		-le_4$ we find the family of geodesics
		%
		\begin{equation}\label{eq:geo_space}
			\gamma(t) = \alpha^{-1}l\sin t~e_\ast - l \cos t~e_4~,
			\quad \forall t \in \mbb{R}
		\end{equation}
		which are spacelike ellipses, since 
		$F(\dot{\gamma},\dot{\gamma}) = -l^2$.
		%
	\item $F(e_\ast,e_\ast) = \alpha^2$. In this case, part of the 
		elements in $P$ are timelike. The curves sought-after satisfy 
		$F(a e_\ast + b e_4, a e_\ast + b e_4) = -l^2$, hence 
		$-\alpha^2 a^2 + b^2 = l^2$. Given the hyperbolic identity 
		$-\sinh^2t + \cosh^2t = 1$, it follows that $a= \pm 
		l\alpha^{-1} \sinh t$ and $b= \pm l\cosh t$. The intersection 
		are thus hyperbolae.  The branch going through $p$ is the 
		geodesic that we looked for, i.e.\
		%
		\begin{equation}\label{eq:geo_time}
			\gamma(t) = \alpha^{-1}l\sinh t~e_\ast - l\cosh t~e_4~,
			\quad \forall t \in \mbb{R}
		\end{equation}
		%
		It is directly verified that these are timelike everywhere, 
		$F(\dot{\gamma},\dot{\gamma}) = l^2$.
	\item $F(e_\ast,e_\ast) = 0$. This case is the degenerate limit 
		of the first two $(\alpha \to 0)$. The intersection of $P$ 
		with $dS_4$ are the elements of $P$ for which $0a^2 - b^2 = 
		-l^2$. The solutions are $a \in \mbb{R}$ and $b = \pm l$.  
		Again choosing the ones through $p$ singles out the family
		%
		\begin{equation}\label{eq:geo_null}
			\gamma(t) = t~e_\ast - le_4~,\quad \forall t \in \mbb{R}
		\end{equation}
\end{enumerate}
These three cases exhaust all posibilities. Note also that these 
geodesics are manifestly complete. However, as we will prove now, 
$dS_4$ is not geodesically connected. Therefore, we consider 
again the orbits of ``pure" de Sitter translations through $p = 
-le_4$ and check whether these connect with any point $q \in 
dS_4$. First, let us assume $p$ and $q$ are non-antipodal, that 
is $p \neq -q$. We discuss the mutually exclusive cases for which 
the vector $q-p$ is timelike, null-like or spacelike.

\begin{enumerate}
	\item $p,q$ timelike seperated. This means that $F(q-p,q-p) > 
		0$ or
		\begin{equation}
			F(p,q) < -l^2
		\end{equation}
		%
		In this case, only timelike geodesics through $p$ could 
		connect with $q$. Therefore consider a generic timelike 
		geodesic \eqref{eq:geo_time} and the inner product 
		$F(p,\gamma(t)) = -l^2 \cosh t,~\forall t \in \mbb{R}$.  
		Hence, all timelike seperated points $q$ can be connected by 
		a	timelike geodesic, for some choice $e_\ast \in \mbb{R}^4$.
		%
	\item $p,q$ lightlike seperated. In this case $F(q-p,q-p) = 0$ 
		or
		\begin{equation}
			F(p,q) = -l^2
		\end{equation}
		%
		The inner product $F(p,\gamma(t)) = -l^2$ for lightlike 
		geodesics \eqref{eq:geo_null} through $p$. Hence, they 
		connect with all lightlike seperated points $q$.
		%
	\item $p,q$ spacelike seperated. For $q-p$ to be spacelike one 
		has that
		\begin{equation}
			F(p,q) > -l^2
		\end{equation}
		Again consider the inner product $p$ and spacelike seperated 
		points, connected by spacelike geodesics 
		\eqref{eq:geo_space}, that is $F(p,\gamma(t)) = -l^2 \cos 
		t,~\forall t \in \mbb{R}$. It follows that these spacelike 
		geodesics only will be able to connect with points $q$ for 
		which $-l^2 < F(p,q) < l^2$.
\end{enumerate}
From this it follows that one can only connect points $p,q \in 
dS_4$ with pure de Sitter translations if $F(p,q) < l^2$. Note 
that this means that de Sitter space is not transitive under the 
exponentiation of $\mfrak{p}$, $\Pi = \exp \mfrak{p}$. It is 
transitive under a finite composition of such group elements 
$\Pi$.  However, this in general will not be a pure de Sitter 
translation--and it will be certainly \emph{not} a de Sitter 
translation if it connects to the above excluded region.  
Therefore, de Sitter space is transitive under elements of the 
full de Sitter group only.

\subsection{Geodesics in stereographic coordinates}

Let us project the above constructed geodesics on $dS$ into the 
stereographic hyperplane $\chi^4 = -l$. Before deducing their 
explicit form, one may understand that these geodesics through 
the origin are straight lines for the stereographic observer.  
Remember that the orbits of pure de Sitter translations are the 
intersections of the two-planes through the origin and the south 
pole with the embedded de Sitter space in $\mbb{R}^5$ and take 
notice that the stereographic projection of a point of $dS$ is 
given by the intersection of the line through this point and the 
north pole with the stereographic hyperplane. Since all points of 
any given geodesic lie in a same plane that also goes through the 
north pole, the stereographic projection of such a geodesic is 
just the intersection of the respective plane with the 
stereographic hyperplane---that is, a straight line.

As we are to derive the explicit form of these geodesics through 
the origin $x^\mu = 0 \leftrightarrow \chi^4 = -l$, let us remind 
that the stereographic projection onto the hyperplane $\chi^4 = 
-l$ through the south pole is given by $x^\mu = 
\Omega^{-1}\chi^\mu$, where $\Omega = -\tfrac{1}{2} (\chi^4/l 
-1)$.

The parametrized curves representing timelike geodesics 
containing the south pole were shown to be given by $\gamma(t) = 
\alpha^{-1}l\sinh t~e_\ast - l\cosh t~e_4$ or explicitly in 
Cartesian coordinates,
%
\begin{displaymath}
	\chi^A(t) = l
	\begin{pmatrix}
		\alpha^{-1}\xi^\mu \sinh t \\
		-\cosh t
	\end{pmatrix}~,
	\quad \alpha^2 = \xi^\mu \xi_\mu \equiv \eta_{\mu\nu} \xi^\mu 
	\xi^\nu~.
\end{displaymath}
%
Projecting this family of geodesics onto $\chi^4 = -l$ gives 
their form in stereographic coordinates, that is
%
\begin{equation}
	x^\mu(t) = 2l \frac{\xi^\mu}{\sqrt{\xi^\lambda\xi_\lambda}}
	\frac{\sinh t}{1 + \cosh t}~,
\end{equation}
which is clearly a straight line through the origin.  Regarding 
these curves some remarks are in place. Note that the function
%
\begin{displaymath}
	\frac{\sinh t}{1 + \cosh t}
\end{displaymath}
monotonically increases from $-1 \to 1$ for $t$ going from 
$-\infty \to \infty$. Hence, the curves $x^\mu(t)$ are finite 
straight lines through the origin. More precisely, they start and 
stop at
%
\begin{displaymath}
	x^\mu(\pm \infty) = \pm 2l \frac{\xi^\mu}{\sqrt{\xi^\lambda 
			\xi_\lambda}}
\end{displaymath}
Considering all possible timelike directions $\xi^\mu$, the sets 
of start and end point gives the two-sheeted hyperboloid 
$\eta_{\mu\nu} x^\mu x^\nu = 4l^2$. Keeping in mind the singular 
character of the stereographic projection, this is consistent 
behaviour.  Indeed, the infinite branches of the hyperbolae in 
$dS$ get projected onto finite lines in stereographic coordinates 
as future and past null infinity get projected onto the given 
surfaces $(\sigma^2 = 4l^2)$.  Notwithstanding the finite 
coordinate range, these timelike geodesics have infinite length 
and it takes an infinite amount of proper time for the free 
particle to reach the above defined hyperboloid. To see this, let 
us consider the proper time along the geodesic
%
\begin{displaymath}
	\tau(t) = \int_0^t \sqrt{g_{\mu\nu} \dot{x}^\mu \dot{x}^\nu} dt 
	= lt
\end{displaymath}
The result follows since $t \to \infty$ in approaching the given 
hyperboloid.

In the case of spacelike geodesics $\gamma(t) = \alpha^{-1}l\sin 
t~e_\ast - l \cos t~e_4$, the parametrization for Cartesian 
coordinates is given by
%
\begin{displaymath}
	\chi^A(t) = l
	\begin{pmatrix}
		\alpha^{-1}\xi^\mu \sin t \\
		-\cos t
	\end{pmatrix}~,
	\quad \alpha^2 = -\xi^\mu \xi_\mu \equiv -\eta_{\mu\nu} \xi^\mu 
	\xi^\nu~.
\end{displaymath}
%
In stereographic coordinates these curves are represented by
%
\begin{equation}
	x^\mu(t) = 2l \frac{\xi^\mu}{\sqrt{-\xi^\lambda\xi_\lambda}}
	\frac{\sin t}{1 + \cos t}~,
\end{equation}
which again are straight lines through the origin. Here, singular 
behaviour seems to appear for $t = \pi$. However, a quick look to 
the Cartesian system shows that this corresponds to the north 
pole, the point discarded for the stereographic projection to be 
well-defined. In this case, we are confronted with a finite range 
of the parameter $t \in ]-\pi, \pi[$ and an infinite range for 
$x^\mu(t)$, since
%
\begin{displaymath}
	\lim_{t \to \pm \pi} \frac{\sin t}{1 + \cos t} = \pm \infty~.
\end{displaymath}
As it turns out, the (finite) ellipses are projected onto 
infinite straight lines.\footnote{In a sense, they are cut open 
	at the north pole after which the conformal factor stretches 
	them out infinitely.} Let us emphasize that the proper length 
of these geodesics is finite.  Indeed, since
%
\begin{displaymath}
	\sigma(t) = \int_0^t \sqrt{g_{\mu\nu} \dot{x}^\mu \dot{x}^\nu} 
	dt = lt
\end{displaymath}
the proper length of the curve originating at the origin $(t=0)$ 
and ending at infinity $(t=\pi)$ is given by $\sigma(\pi) = 
l\pi$. Consistently, this is the arclength between the south pole 
and the north pole of $dS$ embedded in $\mbb{R}^5$.

Lightlike geodesics were found to be given by straight lines in 
$\mbb{R}^5$, i.e.\ $\gamma(t) = \xi^\mu t~e_\mu - le_4$. In 
Cartesian coordinates,
%
\begin{displaymath}
	\chi^A(t) =
	\begin{pmatrix}
		\xi^\mu t \\
		-l
	\end{pmatrix}~,
	\quad \xi^\mu\xi_\mu = \eta_{\mu\nu} \xi^\mu \xi^\nu = 0~.
\end{displaymath}
In stereographic coordinates these are also a straight lines, 
because
%
\begin{equation}
	x^\mu(t) = \xi^\mu t~.
\end{equation}

\bibliographystyle{plain}
\bibliography{../../References/All.bib}
\end{document}

