\documentclass[11pt]{article}

%Load preamble files
\usepackage{../../Tex_files/standard}
\usepackage{../../Tex_files/preamble_one}
%\usepackage{showframe} %show frame borders

\usepackage{mathtools}

\usepackage[all]{xy}

\title{Cartan geometry}
\author{Hendrik}
\date{\today}

\begin{document}

\maketitle

\begin{abstract}
	We give an extended discussion on Cartan geometry.
\end{abstract}

\section{Introduction and definition}

{\blu[

	Introduction has to be expanded and must include
	\begin{itemize}
		\item A note on the relation to Klein geometries, which 
			should make our motivation for using Cartan geometries 
			clear. The reduction of the latter to the former must be 
			pointed out in the definitions.
		\item References used or interesting for further readings: 
			including Sharpe's book, Michor's and Wise's papers.
		\item A short note on history: Cartan's original articles.  
			This is related to the first point.
	\end{itemize}]
}

Let $G \supset H$ be a Lie group with $H$ a closed subgroup and 
let $P(M,H)$ denote the principal bundle $\pi : P \to M$ with 
typical fibre $H$. We begin this short review by writing down the 
formal definition of a Cartan geometry as given in 
\cite{sharpe1997diff_geo}.

\subsection{Definition}

\begin{definition}[Cartan geometry]
	\label{def:cartan_geo}
	A \textbf{Cartan geometry $(P,\kappa)$ modeled on 		
		$(\mathfrak{g},H)$} consists of a principal bundle $P(M,H)$ 
	together with a $\mathfrak{g}$-valued 1-form $\kappa$ on $P$, 
	satisfying the following properties:
	\begin{itemize}
		\item[(i)] for each $p \in P$, the linear map $\kappa_p : 
			T_pP \to \mathfrak{g}$ is an isomorphism;
		\item[(ii)] $\kappa(\zeta_X) = X$, for each fundamental 
			vector field $\zeta_X$ corresponding to $X \in 
			\mathfrak{h}$;
		\item[(iii)] $R_h^\ast \kappa = \mathrm{Ad}(h^{-1}) \cdot 
			\kappa$ for each $h \in H$.
	\end{itemize}
	The 1-form $\kappa$ is called the \textbf{Cartan connection} of 
	the geometry.
\end{definition}

Since $\kappa : TP \to \mathfrak{g}$ defines an isomorphism at 
any point in $P$, it follows that $\dim P = \dim G$ and one 
obtains that
\begin{displaymath}
	\dim M = \dim G/H~.
\end{displaymath}
The second property in the definition implies that a Cartan 
connection $\kappa$ restricts to the Maurer-Cartan form on the 
fibers of $P$, i.e.~$\kappa|_H = \kappa_H$, a proof of which can 
be found in \cite{sharpe1997diff_geo}.
The bijective mapping $\kappa_p$ can be inverted, i.e.
\begin{displaymath}
	\kappa^{-1}_p : \mathfrak{g} \to T_pP~,
\end{displaymath}
and for which the equivariance property of $\kappa$ implies that
\begin{displaymath}
	\kappa^{-1}_{uh}(X) = R_{h\ast}\,\kappa^{-1}_u(\mathrm{Ad}(h) 
	X)~.
\end{displaymath}
This can be understood by considering the commuting diagram that 
represents the equivariance property of $\kappa$, namely
%
\begin{displaymath}
\begin{tikzcd}[row sep=1.75cm,column sep=2.8cm]
	T_u P		\arrow{r}{R_{h \ast}}
				\arrow{d}{\kappa_u}
		& T_{uh} P
				\arrow{d}{\kappa_{uh}}
	\\
	\mathfrak{g}
				\arrow{r}{\Ad(h^{-1})}
		& \mathfrak{g}
\end{tikzcd}~,
\end{displaymath}
which is easily inverted to obtain the equivariance of 
$\kappa^{-1}$.

The \textbf{Cartan curvature} $K$ of a Cartan connection is the 
$\mathfrak{g}$-valued 2-form on $P$ defined by
\begin{equation}\label{eq:cartan-curvature}
	K = d\kappa + \frac{1}{2}[\kappa,\kappa]~.
\end{equation}
This curvature form is horizontal, which means that it vanishes 
when any of its arguments is tangent to the fibers of $P$. This 
may be understood by noting that $\kappa$ restricts to $\kappa_H$ 
in the direction of the fibers, for which the curvature is 
nothing but the structure equation \cite{kob1996found}.  
Explicitly this can be checked as follows.  Because $K(\xi,\eta)$ 
is antisymmetric in its arguments, it is sufficient to show that 
$K(\xi,\eta) = 0$ where $\xi$ is a vertical and $\eta$ a vector 
field on $P$. Being vertical, at any given point one has that 
$\xi = \kappa^{-1}(X) = \zeta_X$ for some $X \in \mathfrak{h}$.  
One is able to verify that 
%
\begin{displaymath}
	\begin{split}
		K(\kappa^{-1}(X),\eta) &= 
		d\kappa(\kappa^{-1}(X),\eta) + 
		[\kappa(\kappa^{-1}(X)),\kappa(\eta)] 
		\\
		&= i_{\kappa^{-1}(X)} \circ d\kappa(\eta) + 
		[X,\kappa(\eta)]~,
	\end{split}
\end{displaymath}
where $i_{\kappa^{-1}(X)} \circ d\kappa(\eta) = d 
\circ i_{\kappa^{-1}(X)}\kappa(\eta) - 
\mathcal{L}_{\kappa^{-1}(X)}\kappa(\eta) = 
-\mathcal{L}_{\kappa^{-1}(X)}\kappa(\eta)$. The 
last equality is due to the constancy of the function 
$i_{\kappa^{-1}(X)}\kappa$. Next consider the equalities
%
\begin{displaymath}
	\begin{split}
		\mathcal{L}_{\kappa^{-1}(X)}\kappa(\eta)
		&= \lim_{t \to 0} \frac{1}{t} (R^\ast_{x_t}\kappa - \kappa) 
		\\
		&= \lim_{t \to 0} \frac{1}{t} (\Ad(x^{-1}_t) \cdot \kappa - 
		\kappa) \\
		&=	\mathrm{ad}_X \kappa~,
	\end{split}
\end{displaymath}
where $x_t = \exp(tX)$ is the one-parameter subgroup of $H$ that 
is generated by $X$.  Gathering this information we find that the 
curvature is indeed horizontal, that is
%
\begin{displaymath}
	K(\xi,\eta) = -\mathrm{ad}_X \kappa(\eta) 
	+ \mathrm{ad}_X \kappa(\eta) = 0~.
\end{displaymath}

The \textbf{torsion} $\Theta$ of the Cartan connection is a 
$\mathfrak{g}/\mathfrak{h}$-valued 2-form on $P$ obtained by 
composing the curvature form with the canonical projection 
$\mathfrak{g} \to \mathfrak{g}/\mathfrak{h}$, that is
\begin{displaymath}
\begin{tikzcd}[column sep=1.2cm]
	T^{(2,0)} P
				\arrow{r}{K}
				\arrow[bend right,swap]{rr}{\Theta}
		& \mathfrak{g}
				\arrow{r}
		& \mathfrak{g}/\mathfrak{h}
\end{tikzcd}
\end{displaymath}

By taking the exterior derivative of the curvature, one finds the 
\textbf{Bianchi identity}
\begin{equation}\label{eq:cartan_bianchi}
	dK + [\kappa,K] = 0~.
\end{equation}

\subsection{Reductive Cartan geometry}

A Cartan geometry $(P,\kappa)$ modeled on $(\mathfrak{g},H)$ is 
reductive if there is an $H$-module decomposition $\mathfrak{g} = 
\mathfrak{h} \oplus \mathfrak{p}$, i.e.~a splitting of the 
algebra $\mathfrak{g}$ in $\mathrm{Ad}(H)$-invariant subspaces.  
Corresponding to the reductive splitting, the Cartan connection 
is expanded as a sum of an $\mathfrak{h}$-valued and a 
$\mathfrak{p}$-valued part:
%
\begin{displaymath}
	\kappa = \omega + e~;\quad
	\text{with}
	\quad\left\{
	\begin{aligned}
		\omega &:= \mathrm{pr}_\mathfrak{h}\circ\kappa \in 
		\Omega^1(P,\mathfrak{h})~, \\
		e &:= \mathrm{pr}_\mathfrak{p}\circ\kappa \in 
		\Omega^1(P,\mathfrak{p})~.
	\end{aligned}
	\right.
\end{displaymath}
These projections make sense, since by definition the reductive 
splitting is preserved under gauge transformations. The issue is 
further discussed in the following proposition.

\begin{proposition}
	Let $(P,\kappa)$ be a reductive Cartan geometry with a 
	corresponding splitting of the connection $\kappa = \omega + 
	e$. The one-form $\omega$ is an Ehresmann connection on $P$, 
	while $e$ is a displacement form, i.e.~$H$-equivariant and 
	strictly horizontal.
\end{proposition}
%%%
\begin{proof}
	By the $H$-equivariance of $\kappa$, i.e.~$R_h^\ast \kappa = 
	\Ad(h^{-1})\cdot\kappa$, it follows that
	\begin{displaymath}
		R_h^\ast\omega + R_h^\ast e = \Ad(h^{-1}) \cdot \omega + 
		\Ad(h^{-1}) \cdot e~,
	\end{displaymath}
	since the splitting is reductive.
	Rearranging this equation,
	\begin{displaymath}
		R_h^\ast \omega - \Ad(h^{-1}) \cdot \omega = -R_h^\ast e + 
		\Ad(h^{-1}) \cdot e~,
	\end{displaymath}
	the left-hand side is $\mathfrak{h}$-valued, while the     
	right-hand side is $\mathfrak{p}$-valued. The equality makes 
	sense only when both sides equal zero. This proves the 
	equivariance of both $\omega$ and $e$.
	
	Next let $\zeta_X$ be the fundamental vector field 
	corresponding to $X \in \mathfrak{h}$. Then
	\begin{displaymath}
		X = \kappa(\zeta_X) = \omega(\zeta_X) + e(\zeta_X)~,
	\end{displaymath}
	or
	\begin{displaymath}
		X - \omega(\zeta_X) = e(\zeta_X)~.
	\end{displaymath}
	The left-hand side is valued in $\mathfrak{h}$, while the 
	right-hand side is valued in $\mathfrak{p}$. This implies that 
	$\omega(\zeta_X) = X$ and $e(\zeta_X) =0$ for any $X \in 
	\mathfrak{h}$.
\end{proof}

Note that a vector field $X$ on $P$, which at every point maps 
into $\mathfrak{p}$ under $\kappa$, is horizontal with respect to 
the Ehresmann connection $\omega$. This is easily shown to be 
true by considering $\omega(X) = \mathrm{pr}_\mathfrak{h} \circ 
\kappa(X) = 0$. 

\blankline
The Cartan curvature can likewise consistently be written as the 
sum of an $\mathfrak{h}$-, respectively $\mathfrak{p}$-valued 
differential form:
%
\begin{displaymath}
	K = R + T~;\quad
	\text{with}
	\quad\left\{
	\begin{aligned}
		R &:= \mathrm{pr}_\mathfrak{h} \circ K \in 
		\Omega^2(P,\mathfrak{h})~,\\
		T &:= \mathrm{pr}_\mathfrak{p} \circ K\in 
		\Omega^2(P,\mathfrak{p})~.
	\end{aligned}
	\right.
\end{displaymath}
%
The definition for the Cartan curvature allows us to find 
explicit expressions for $R$ and $T$ in terms of $\omega$ and 
$e$.  Therefore we substitute $\kappa$ in 
Eq.~\eqref{eq:cartan-curvature} for the latter two and take into 
account the reductive nature of the geometry, which results in 
%
\begin{displaymath}
	K = \underbrace{d\omega + 
		\frac{1}{2}[\omega,\omega]}_{\mathfrak{h}\text{-valued}}
		+ \underbrace{\vphantom{\frac{1}{2}} de + 
			[\omega,e]}_{\mathfrak{p}\text{-valued}} +
	\underbrace{\frac{1}{2}[e,e]}_{\mathfrak{g}\text{-valued}}~.
\end{displaymath}
For a reductive algebra two elements in $\mathfrak{p}$ commute 
into a generic element of $\mathfrak{g}$, so that the last term 
generally contributes in part to both $R$ and $T$. However, in 
case $\mathfrak{g}$ is not only a reductive but also a 
\emph{symmetric} Lie algebra, elements of the form $[e,e]$ are 
$\mathfrak{h}$-valued and it follows that
\begin{align*}
	R &= d\omega + \frac{1}{2}[\omega,\omega]  + 
	\frac{1}{2}[e,e]
	= B + \tfrac{1}{2}[e,e]~,\\
	T &= de + [\omega,e]~.
\end{align*}
Here we recognized $B \equiv d\omega + 
\tfrac{1}{2}[\omega,\omega]$ as the exterior covariant derivative 
of the Ehresmann connection $\omega$. One therefore concludes 
that a flat Cartan geometry $(K=0)$ does not necessarily imply 
that the curvature of $\omega$ vanishes $(B=0)$, but rather that 
$B = -\tfrac{1}{2}[e,e]$.
A Cartan geometry for which $T$ vanishes is said to be 
\textbf{torsion-free}.

Given a reductive Cartan connection, the corresponding Bianchi 
identity~\eqref{eq:cartan_bianchi} may also be seperated in an 
$\mathfrak{h}$-valued and a $\mathfrak{p}$-valued part. This 
results in two Bianchi identities, namely
%
\begin{subequations}
\begin{alignat}{2}
	dB + [\omega,B] &\equiv 0 
	&\qquad\quad&(\text{1st identity})~,\\
	dT + [\omega,T] + [e,B] &\equiv 0 &&(\text{2nd identity})~.
\end{alignat}
\end{subequations}
The first identity is $\mathfrak{p}$-valued, while the second is 
$\mathfrak{h}$-valued.
Note that the first Bianchi identity is just the usual identity 
for the Ehresmann connection $\omega$.

\section{The coframe field\dots just the Vielbein!}

{\blu [This section must be reviewed. Especially, notation should 
	be made consistent with other parts of the documents.]
}

The above introduced solder form or coframe field $\theta : TP 
\to \mathfrak{p}$ seems to be a quiet different mapping then the 
coframe or \emph{vielbein} generally used by physicists. Given an 
associated bundle $E = P \times_H \mathfrak{p}$, the Vielbein is 
a $E$-valued 1-form on $M$, as in the following diagram
%
\begin{displaymath}
	\xymatrix{
		TM
			\ar[rr]^-{e}
			\ar[dr]
		& & E
			\ar[dl]^-{\pi_E}
		\\
		& M &
	}
\end{displaymath}
Locally, $e$ maps a vector of $T_pM$ into $\pi^{-1}_E(p) \simeq
\mathfrak{p}$, such that it gives a local isomorphism $e : T_pM 
\to \mathfrak{p}$, for any $p \in M$. In the following it is 
shown that $\theta$ and $e$ are in fact two different 
descriptions of the same mapping \cite{Wise:2010sm}.

Suppose a tetrad $e:TM \to E$ is given so that a corresponding 
solder form $\theta:TP \to \mathfrak{p}$ is constructed as 
follows.  For any $v \in T_uP$, $e(\pi(v)) \in \pi^{-1}_E(p)$, 
where $\pi(u) = p$.  Hence, one can write $e(\pi(v)) = [u',X]$ 
for $u' \in \pi^{-1}(p)$ and $X \in 
\mathfrak{p}$.\footnote{$[\cdot,\cdot]$ an element of the 
	equivalence classes $E$, that is $(u,X) \sim 
	(uh,Ad_{h^{-1}}X)$.} Then define $\theta(v)$ to be the unique 
element in $\mathfrak{p}$ such that
%
\begin{displaymath}
	e(\pi(v)) = [u,\theta(v)]~.
\end{displaymath}
This construction is well-defined if $\theta$ is $H$-equivariant.  
Indeed, for
%
\begin{displaymath}
	e(\pi(v)) = [u,\theta(v)] = [uh,\mathrm{Ad}_{h^{-1}} 
	\theta(v)]~,
\end{displaymath}
but also
\begin{displaymath}
	e(\pi(v)) = e(\pi(R_h(v)) = [uh,\theta(R_h(v))]~,
\end{displaymath}
so that consistency requires $R^\ast_h \theta = 
\mathrm{Ad}_{h^{-1}} \theta$. Furthermore, $\theta$ is strictly 
horizontal because if $v \in V_uP$, $e(\pi(v)) = 0 = 
[u,\theta(v)]$, hence $\theta(v) = 0$. Since the thus constructed 
form $\theta: TP \to \mathfrak{p}$ is a $H$-equivariant 
horizontal form, it is a legitimate solder form.

Next we start with a given solder form $\theta$ and define a 
corresponding frame field $e$ as follows. Remember also that a 
reductive Cartan geometry canonically incorporates a connection 
$\omega$ on $P$. Let $\pi(v) \in T_pM$ and for any $u \in 
\pi^{-1}(p)$ let $v \in T_uP$ be its horizontal lift with respect 
to $\omega$. Define then $e$ trough (set $w = \pi(v))$
%
\begin{displaymath}
	e(w) = \pi(v)) = [u,\theta(v)]~,
\end{displaymath}
which is clearly an element of $\pi^{-1}_E(p)$. Note how it is 
manifestly the inverse construction of the one before, but that 
now we had to make use of a connection to unambiguously pick out 
a $v \in H_uP$. We check again consistency of this construction.  
If $uh$ were chosen then $v_{uh} = R_h v_u$, due to the right 
invariance of a horizontal lift, so that
%
\begin{displaymath}
	e(w) = [uh,\theta(R_h v_u)] = [uh,\mathrm{Ad}_{h^{-1}} 
	\theta(v)] = [u,\theta(v)]~.
\end{displaymath}

These results are nicely summarized in the following commuting 
diagram:
\begin{displaymath}
	\xymatrix{
		T_uP
			\ar[r]^-{\theta}
			\ar[d]^-{\pi}
		& \mathfrak{p}
			\ar[d]^-{[u,\cdot]}
		\\
		T_pM
			\ar[r]^-{e}
		& \pi^{-1}_E(p)
	}
\end{displaymath}


\section{Riemann-Cartan geometry}
\label{sec:RC_geometry}

In this section we discuss in some detail Riemann-Cartan 
geometry. This geometry is the mathematical framework that 
underlies Einstein-Cartan theory, General Relativity in the 
Palatini formalism and Teleparellel Gravity, although the latter 
two only use part of its available structure. It does not include 
the description of the so-called metric-affine theories of 
gravity, since Riemann-Cartan geometry is automatically metric, 
as will become clear in the following.

The relevant Cartan geometry\footnote{Let us make clear that in 
	what follows we will work directly on the base manifold $M$.  
	More precisely, given some section $\sigma : M \to P$, the 
	Cartan connection is a 1-form $A = \sigma^\ast \kappa$ on $M$, 
	while we will denote the pulled back Cartan curvature by $F = 
	\sigma^\ast K = dA + \tfrac{1}{2} [A,A]$.}
is modeled on $(\mathfrak{iso}(1,3),SO(1,3))$, where
\begin{displaymath}
	\mathfrak{iso}(1,3) = \mathfrak{so}(1,3) \oplus 
	\mathbb{R}^{3,1}
\end{displaymath}
is the Poincar\'e algebra,\footnote{Therefore, a correcter 
	nomenclature for the Riemann-Cartan geometry could be 
	\emph{Poinar\'e-Cartan geometry}.} whose commutation relations 
are given by~\eqref{eq:comm_relations_so(1,4)} for the 
contraction $l \to \infty$. The corresponding Cartan connection 
is a 1-form $A$ on spacetime that is valued in the Poincar\'e 
algebra,\footnote{Note that the relevant Cartan connection for 
	metric-affine theories of gravity is valued in the affine 
	algebra, which directly results in loosing the metricity of 
	the geometry.} which we split according to the symmetric 
nature of the algebra as
\begin{displaymath}
	A = \omega + e = \tfrac{i}{2} \omega^{ab} M_{ab} + i e^a P_a~.
\end{displaymath}

Since the Poincar\'e algebra is reductive, the 1-form 
$\omega^{ab}$ is an Ehresmann connection for the Lorentz algebra, 
i.e. a \emph{spin connection}. A corresponding covariant 
derivative $D := d + \omega$ is readily defined.  Furthermore, 
the 1-forms $e^{a}$ define a symmetric 2-form $g$ on $M$---the 
metric---given by
\begin{equation}
	g_{\mu \nu} := e\ind{^a_\mu} e\ind{^b_\nu} \eta_{ab}~.
\end{equation}
The forms $e\ind{^a_\mu} dx^\mu$ are called the coframe 
fields---there is one for each $a$. It is then useful to define a 
set of dual vector fields $\vartheta\ind{_a^\mu} \pd_\mu$ through 
$\vartheta_a \ip e^b = \delta^b_a$, which generally are given the 
name of a \emph{vierbein} or tetrad. From the definition of the 
metric it follows that $g_{\mu\nu} \vartheta\ind{_a^\nu} = 
e_{a\mu}$.  Therefore in the case of the matrix $e\ind{^a_\mu}$ 
being non-degenerate, an inverse of the metric exists and the 
coframe field and the vielbein are related by raising or lowering 
spacetime indices. The use of $\vartheta_a$ to denote the 
vielbein is then obsolete and will be referred to in the 
following by $e_a$. It then follows directly that $e\ind{^a_\mu} 
e\ind{_a^\nu} = \delta^\nu_\mu$ and that $g^{\mu\nu} = 
e\ind{^{a\mu}} e\ind{_a^\nu}$.

Furthermore, given the spin covariant derivative $D$ and the 
vierbein, it is possible to define covariant differentiation of 
world tensors. The corresponding derivative $\nabla = d + \Gamma$ 
is defined such that $\nabla_\mu V^\rho = e\ind{_a^\rho} D_\mu 
V^a$. This implies that the relation between the spin and linear 
connections is given by
\begin{displaymath}
	\Gamma\ind{^\mu_{\nu\rho}} = e\ind{_a^\mu} D_\rho 
	e\ind{^a_\nu} \quad \mathrm{and} \quad
	\omega\ind{^a_{b\rho}} = e\ind{^a_\mu} \nabla_\rho 
	e\ind{_b^\mu}~.
\end{displaymath}
and that the vierbein postulate is true, i.e. $D_\rho 
e\ind{^a_\mu} - \Gamma\ind{^\nu_{\mu \rho}} e\ind{^a_\nu} \equiv 
0$. Furthermore, for such a geometry one has that
\begin{align*}
	\nabla_\rho g_{\mu\nu} &= -\omega\ind{^a_{b\rho}} 
	e\ind{^b_\mu} e\ind{_{a\nu}} - \omega\ind{_{ab\rho}} 
	e\ind{^b_\nu} e\ind{_{a\mu}}
	\\
	&= (\omega\ind{_{ba\rho}} - \omega\ind{_{ab\rho}} ) 
	e\ind{^b_\mu} e\ind{^a_\nu} \equiv 0~,
\end{align*}
since $\omega$ is valued in the Lorentz algebra. One concludes 
that Riemann-Cartan geometry is \emph{metric}, as would be any 
reductive Cartan geometry where the subalgebra $\mathfrak{h}$ is 
the Lorentz algebra.

\blankline
The Cartan curvature can equally be written as a sum of a 
Lorentz-, respectively translation-valued part, i.e.
\begin{displaymath}
	F = R + T = \tfrac{i}{2} R^{ab} M_{ab} + i T^a P_a~.
\end{displaymath}
The curvature and torsion of the geometry are given by
\begin{subequations}
\begin{alignat}{3}
	\label{eq:curv_RC}
	R &= d\omega + \tfrac{1}{2} [\omega,\omega]
		&\quad&\text{or}
		&\quad
		R^{ab} &= d\omega^{ab} + \omega\ind{^a_c} \wedge 
		\omega^{cb}~,\\
	\label{eq:tors_RC}
	T &= de + [\omega,e]
		&\quad&\text{or}
		&\quad
		T^a &= de^a + \omega\ind{^a_b} \wedge e^b~,
\end{alignat}
\end{subequations}
while the Bianchi idendities for the Cartan connection are 
expressed as functions of the curvature and torsion:
\begin{subequations}
\label{eq:bianchiISO_hp}
\begin{alignat}{3}
	\label{eq:bianchiISO_h}
	dR + [\omega,R] &\equiv 0
		&\quad&\text{or}
		&\quad
		dR^{ab} + \omega\ind{^a_c} \wedge R^{cb} + \omega\ind{^b_c} 
		\wedge R^{ac} &\equiv 0~, \\
	\label{eq:bianchiISO_p}
	dT + [\omega,T] + [e,R] &\equiv 0
		&\quad&\text{or}
		&\quad
		dT^a + \omega\ind{^a_b} \wedge T^b + e^c \wedge R\ind{_c^a} 
		&\equiv 0~.
\end{alignat}
\end{subequations}

The expression~\eqref{eq:tors_RC} for the torsion can be solved 
for the spin connection, which yields
\begin{equation}
	\label{eq:decomp_contortion}
	\omega\ind{^a_{b\mu}} = \tfrac{1}{2} e_{c\mu} 
	(\Omega\ind{_b^{ca}} + \Omega\ind{_b^{ac}} - 
	\Omega\ind{^{ac}_b} ) + K\ind{^a_{b\mu}}
	= \mathring{\omega}\ind{^a_{b\mu}} + K\ind{^a_{b\mu}}~.
\end{equation}
In this expression $\mathring{\omega}$ is the torsionless 
Levi-Civita connection, while we introduced the coefficients of 
anholonomy $\Omega_{abc} := e_b \ip e_a \ip de_c = e\ind{_a^\mu} 
e\ind{_b^\nu} (\pd_\mu e\ind{_{c\nu}} - \pd_\nu e\ind{_{c\mu}})$ 
and the \emph{contortion} $K$ of $\omega$:
\begin{equation}
	K\ind{^a_{b\mu}} := \tfrac{1}{2} ( T\ind{^a_{\mu b}} + 
	T\ind{_\mu^a_b} + T\ind{_b^a_\mu} )~.
\end{equation}

\begin{remark}
	 Cartan's beautiful language to describe a Riemann-Cartan 
	 geometry makes the distinction between local Lorentz 
	 transformations and diffeomorphisms manifest. Mathematically, 
	 the former are given by sections of the principal Lorentz 
	 bundle over spacetime, while the latter are 1-to-1 mappings 
	 of spacetime to itself. Physically, only local Lorentz 
	 transformations have a non-trivial meaning. At any point in 
	 spacetime they relate the reference frames of observers that 
	 have a relative velocity. Diffeomorphisms, on the other hand, 
	 represent coordinate transformations of spacetime. They 
	 merely relate different ways of labeling space and time, and 
	 as such are lacking any deeper meaning.
\end{remark}

\begin{remark}
	The geometric setting used to describe the Palatini formalism 
	of General Relativity singles out the zero-torsion Levi-Civita 
	spin connection $\mathring{\omega}$. The contortion vanishes 
	naturally and the connection is determined fully by the 
	vierbein, i.e. the gravitational field. The presence of a 
	gravitational interaction is quantified by the curvature 
	$\mathring{R}^{ab}$. This is what is meant when saying that 
	the gravitational interaction is geometrized.  There is a 
	price that has to be paid, however. The spin connection does 
	not only represent the gauge freedom of local Lorentz 
	transformations but also contains information whether there is 
	a gravitational field present or not, so that it is impossible 
	to seperate inertial effects from gravitation. There are 
	arguments to see this as a shortcoming from a conceptual point 
	of view, since they might be quite different 
	\emph{physically}.  Inertial forces certainly can be created 
	and get rid of by considering local Lorentz transformations.  
	On the contrary, the presence of a gravitational field, 
	encoded in the covariant object $\mathring{R}^{ab}$, is 
	objective.

	It is usually argued that choosing a reference frame for which 
	the Levi-Civita connection vanishes at some point yields a 
	\emph{local inertial reference frame}.  The reason for such 
	nomenclature is of course the reduction of covariant 
	derivatives to ordinary derivatives at that point, so that the 
	laws of physics recover their special relativistic form.  
	Since inertial and gravitational motion is identical for 
	infinitesimal objects, the gravitational interaction is said 
	to be inertial, and as such the gravitational field determines 
	inertial motion.\footnote{For a clear discussion on this and 
		related points, see Chapter~2 in~\cite{rovelli2004-QG}} 
	Having a look at processes that take place on a finite region 
	in spacetime however, the freely falling observer will notice 
	the consequences of tidal effects, present when 
	$\mathring{R}^{ab} \neq 0$ and which cannot be accounted for 
	by changing reference frame. Mathematically, the spin 
	connection cannot be gauged away over a finite region in 
	spacetime, so that this notion of inertial frames in General 
	Relativity can at most be a \emph{local} concept.	

	{\blu[[} In this manner, one has to give up the idea of 
	inertial effects being a non-local manifestation of the 
	observer's motion. Otherwise, the identification of 
	gravitational and inertial forces is not tenable, since 
	curvature cannot be gauged away and tidal effects will be 
	present for any class of observers.  One can wonder about the 
	usefulness of this modified notion for inertial motion and 
	desire to understand inertial forces as a result of one's 
	motion, which can be eliminated everywhere by a suitable 
	Lorentz rotation. These fictious forces are very different in 
	nature of the covariant gravitational interaction. As 
	mentioned already, such point of view cannot be taken in 
	General Relativity, since the spin connection includes both 
	types of ineraction at any point in spacetime. In fact, the 
	possibility to seperate them consistently at the mathematical 
	level lies at the heart of Teleparallel Gravity.
	{\blu]]}
	
	To conclude, let us specify the Bianchi identities for the 
	geometry underlying General Relativity, i.e.
	\begin{subequations}
	\begin{gather*}
		d\mathring{R} + [\mathring{\omega},\mathring{R}] \equiv 
		0~,\\
		[e,\mathring{R}] \equiv 0~.
	\qedhere
	\end{gather*}
	\end{subequations}
\end{remark}

\begin{remark}
	In Teleparallel Gravity the connection considered is flat and 
	is related to the Levi-Civita spin connection through 
	Eq.~\eqref{eq:decomp_contortion}. This relation together with 
	the geometry of General Relativity gives rise to the geometric 
	structure of Teleparallel Gravity, as it is derived 
	in~\cite{aldrovandi:2012tele}. From the discussion 
	in~\verb+equiv_bianchi_GR_TG.pdf+ it follows that this 
	geometric framework is a special case of Riemann-Cartan 
	geometry, with vanishing curvature but non-zero torsion.

	The spin connection is flat, i.e. $d\omega\ind{^{ab}} + 
	\omega\ind{^a_c} \wedge \omega\ind{^{cb}} = 0$, a condition 
	very different in nature of the one for vanishing torsion in 
	General Relativity. By the fundamental theorem of calculus,  
	it follows that the spin connection is pure gauge, which means 
	that $\omega\ind{^{ab}} = \Lambda\ind{_c^a} 
	d\Lambda\ind{^{cb}}$ in which $\Lambda\ind{^a_b} \in 
	\Omega^0(M, \mathfrak{so}(1,3))$ is a local Lorentz 
	transformation. The spin connection then accounts only for 
	inertial effects and one can choose a class of inertial 
	observers, so that $\omega^{ab} = 0$ everywhere. In this 
	theory, the presence of a gravitational field is encoded in a 
	non-vanishing torsion. The obvious advantage is that inertial 
	and gravitational effects are not mixed up in the spin 
	connection so that both the physically very important concepts 
	of inertial forces and the gravitational interaction have a 
	clear well-distincted description.
\end{remark}

Finally, the geometry of Einstein-Cartan theory is 
\emph{a-priori} a generic Riemann-Cartan geometry, i.e.~having 
non-vanishing curvature and torsion. Also note that the geometry 
only fixes the kinematics of the respective theories, and that 
their ultimate content is determined further once an action 
describing their dynamics is postulated.


\section{de Sitter-Cartan geometry}
\label{sec:dSC_geometry}

The Cartan geometry that is modeled on $(\mathfrak{so}(1,4), 
SO(1,3))$, will be given the name of a \emph{de Sitter-Cartan 
	geometry}. It is a slight generalization of Riemann-Cartan 
geometry, in the sense that in some well-defined limit the latter 
can be recovered from it. The Cartan connection for this geometry 
is a 1-form valued in the de Sitter algebra $\mathfrak{so}(1,4)$, 
which is characterized by its commutation relations\footnote{We 
	adhere to the convention $\eta_{ab} = (+,-,-,-)$.}
\begin{gather}
\label{eq:comm_relations_so(1,4)}
\begin{aligned}
	-i[M_{ab},M_{cd}] &= \eta_{ac}M_{bd} - \eta_{ad}M_{bc} + 
	\eta_{bd}M_{ac} - \eta_{bc}M_{ad} \\
	-i[M_{ab},P_c] &= \eta_{ac}P_b- \eta_{bc}P_a\\
	-i[P_a,P_b] &= -l^{-2}M_{ab}~,
  \end{aligned}
\end{gather}
The de Sitter transvections are defined by $P_a \equiv M_{a4}/l$, 
where it should be emphasized that the length scale $l(x)$ is 
\emph{a-priori} allowed to be a function on $M$. From these 
brackets it is manifest that the algebra is symmetric, hence 
reductive.  The corresponding reductive splitting reads as
\begin{displaymath}
	\mathfrak{so}(1,4) = \mathfrak{so}(1,3) \oplus \mathfrak{p}~,
\end{displaymath}
where $\mathfrak{so}(1,3) = \mathrm{span}\{M_{ab}\}$ is the 
Lorentz subalgebra and $\mathfrak{p} = \mathrm{span}\{P_a\}$ the 
subspace of de Sitter transvections, or de Sitter translations.  
This Cartan decomposition can be carried through to the 
connection and curvature forms, denoted by
\begin{subequations}
\begin{gather}
	A = \omega + e = \tfrac{i}{2} \omega^{ab} M_{ab} + i e^a P_a~, 
	\\
	F = R + T = \tfrac{i}{2} R^{ab} M_{ab} + i T^a P_a~.
\end{gather}
\end{subequations}
Due to the reductive nature of the de Sitter algebra, 
$\omega^{ab}$ is a spin connection. Identical to the discussion 
in Section~\ref{sec:RC_geometry}, $SO(1,3)$-covariant 
differentiation can be defined and a metric structure 
constructed. We would like to refer to the discourse there, which 
is equally valid here.

Given the commutation 
relations~\eqref{eq:comm_relations_so(1,4)}, one is able to 
compute the curvature~$R^{ab}$ and the torsion~$T^{a}$ for de 
Sitter-Cartan geometry, namely
\begin{subequations}
\begin{gather}
	\label{eq:curv_dSC}
	R^{ab} = d\omega^{ab} + \omega\ind{^a_c} \wedge \omega^{cb} + 
	\frac{1}{l^2} e^a \wedge e^b
		=: B^{ab} + \frac{1}{l^2} e^a \wedge e^b ~,
	\\
	\label{eq:tors_dSC}
	T^a = de^a + \omega\ind{^a_b} \wedge e^b - \frac{1}{l} dl 
	\wedge e^a =: G^a - \frac{1}{l} dl \wedge e^a~.
\end{gather}
\end{subequations}
In these equations the exterior covariant derivative of the spin 
connection and vierbein are denoted by $B^{ab}$ and $G^a$, 
respectively. Some further remarks concerning these results are 
in place. Note that in the limit of a diverging length scale $l$, 
the expressions reduce to the curvature~$B^{ab}$ and 
torsion~$G^{a}$ for a Riemann-Cartan geometry. In the generic 
case however, the curvature and torsion are not given by the 
exterior covariant derivative of the spin connection and 
vierbein. The extra term in the expression~\eqref{eq:curv_dSC} 
represents the curvature of the local de Sitter space. This 
contribution is present because the commutator of two de Sitter 
transvections equals a Lorentz rotation. In the 
expression~\eqref{eq:tors_dSC} for the torsion, there is a new 
term when the length scale is not a constant function. This term 
comes about as follows. Remember that torsion is the 
$\mathfrak{p}$-valued $2$-form $de + [\omega,e]$.  The first 
contribution to this expression really means
\begin{displaymath}
	de = d(i e^a P_a) = d(i l^{-1} e^a M_{a4} ) = ide^a P_a - i 
	(l^{-1} dl \wedge e^a) P_a~,
\end{displaymath}
since $l$ is allowed to change along $M$.

As in the case of a Riemann-Cartan geometry, 
expression~\eqref{eq:tors_dSC} can be solved for the spin 
connection. It is found that $\omega\ind{^a_{b\mu}} = 
\mathring{\omega}\ind{^a_{b\mu}} + K\ind{^a_{b\mu}}$, where 
$\mathring{\omega}^{ab}$ is the Levi-Civita spin connection for 
which the covariant exterior derivative $\mathring{G}^{a}$ 
vanishes, while the contortion is given by
\begin{displaymath}
	K\ind{^a_{b\mu}} := \tfrac{1}{2} ( G\ind{^a_{\mu b}} + 
	G\ind{_\mu^a_b} + G\ind{_b^a_\mu} )~.
\end{displaymath}
These results are identical to the corresponding in a 
Riemann-Cartan geometry, since the torsion there is just the 
exterior covariant derivative of the vierbein, which here is 
denoted by $G^a$. Note that the torsion $\mathring{T}^a$ does not 
vanish for the Levi-Civita connection in a de Sitter-Cartan 
geometry, rather is its torsion equal to~$\mathring{T}^a = 
-l^{-1} dl \wedge e^a$.

\blankline
The Bianchi identities for the given de Sitter-Cartan geometry 
are of the form
\begin{subequations}
\begin{gather}
	\begin{gathered}
	\label{eq:bianchidS_expl_h}
	dR^{ab} + \omega\ind{^a_c} \wedge R^{cb} + \omega\ind{^b_c} 
	\wedge R^{ac} + \frac{1}{l^2} e^{[a} \wedge T^{b]} \equiv 0 \\
	\text{or}\quad
	dB^{ab} + \omega\ind{^a_c} \wedge B^{cb} + \omega\ind{^b_c} 
	\wedge B^{ac} \equiv 0~,
	\end{gathered}
	\\
	\label{eq:bianchidS_expl_p}
	dT^a + \omega\ind{^a_b} \wedge T^b + e^c \wedge R\ind{_c^a} - 
	\frac{1}{l} dl \wedge T^a \equiv 0~.
\end{gather}
\end{subequations}
The first Bianchi identity~\eqref{eq:bianchidS_expl_h} is the 
usual identity for the Ehresmann connection~$\omega$.


\begin{remark}
	Consider for a moment the gauge invariant condition
	\begin{displaymath}
		R \equiv 0 \quad \text{or} \quad B \equiv -\frac{1}{2} 
		[e,e]~.
	\end{displaymath}
	The Bianchi identity for the 
	torsion~\eqref{eq:bianchidS_expl_p} clearly reduces to
	\begin{displaymath}
		dT^a + \omega\ind{^a_b} \wedge T^b - \frac{1}{l} dl \wedge 
		T^a = 0~.
	\end{displaymath}
	From the first Bianchi identity~\eqref{eq:bianchidS_expl_h} 
	one concludes that $[e,T] \equiv 0$ or
	\begin{equation}
		\label{eq:bianchidSTG_expl_h}
		\frac{1}{l^2} e^{[a} \wedge T^{b]} = 0~.
	\end{equation}
	Note that this restriction does carry through upon taking the 
	contraction limit to the Riemann-Cartan geometry, since there 
	$[e,T] = 0$ so that~\eqref{eq:bianchidS_expl_h} would be 
	trivially satisfied.

	Next consider the interior product\footnote{For any vector 
		field $X$ on $M$ the interior product of a differential 
		form with respect to $X$ will be denoted by
		\begin{displaymath}
			X \ip \omega := i_X \omega~.
		\end{displaymath}
		It then follows that $e_a \ip e^b = \delta^b_a$.}
	of the left-hand side of Eq.~\eqref{eq:bianchidSTG_expl_h}:
	\begin{align*}
		e_a \ip \Big( e^{[a} \wedge T^{b]} \Big)
		&= 4 T^b - e^a \wedge e_a \ip T^b - T^b + e^b \wedge e_a 
		\ip T^a \\
		&= T^b + e^b \wedge e_a \ip T^a~.
	\end{align*}
	Here we made use of the Leibniz rule $e_a \ip (\alpha \wedge 
	\beta) = e_a \ip \alpha \wedge \beta + (-)^p \alpha \wedge e_a 
	\ip \beta$, where $\alpha$ is assumed to be a $p$-form. It is 
	also implied that $e^a \wedge e_a \ip T^b = 2T^b$. A second 
	contraction gives
	\begin{displaymath}
		e_b \ip e_a \ip \Big( e^{[a} \wedge T^{b]} \Big)
		=  e_b \ip T^b + 4 e_a \ip T^a - e^b \wedge e_b \ip e_a \ip 
		T^a = 4 e_a \ip T^a~.
	\end{displaymath}
	In this result one should consider that $e_a \ip T^a = e^b 
	\wedge e_b \ip e_a \ip T^a$, which can be verified by a direct 
	calculation. Gathering these results one finds that the second 
	Bianchi identity~\eqref{eq:bianchidSTG_expl_h} leads to
	\begin{displaymath}
		l^{-2} e_a \ip T^a = l^{-2} T\ind{^a_{a\mu}} dx^\mu = 0~,
	\end{displaymath}
	hence that the torsion is \emph{traceless}. Furthermore, from 
	the first contraction it followed that
	\begin{displaymath}
		T^b = -e^b \wedge e_a \ip T^a = 0~.
	\end{displaymath}

	It is a quite remarkable result, but setting the curvature $R$ 
	to zero forces the geometry to have vanishing torsion. The 
	resulting geometry describes a Klein geometry 
	$SO(1,4)/SO(1,3)$.
\end{remark}

\begin{remark}
	It is equally possible to consider a de Sitter-Cartan geometry 
	for which the spin curvature $B^{ab}$ is constrained to be 
	zero, i.e.
	\begin{displaymath}
		B^{ab} \equiv 0~.
	\end{displaymath}
	This means that the exterior covariant derivative of 
	$\omega^{ab}$ is zero, a condition well known to render the 
	Weitzenb\"ock connection. Note that in de Sitter-Cartan 
	geometry the Weitzenb\"ock connection has a non-vanishing 
	curvature $R^{ab} = l^{-2} e^a \wedge e^b$. This is similar to 
	the conclusion that the torsion of the Levi-Civita connection 
	is non-zero in a de Sitter-Cartan geometry. Nonetheless, as 
	explained above, both connection are identical to those 
	considered in a Riemann-Cartan geometry.
	
	Note that the condition that the spin curvature should be 
	equal to zero is gauge invariant, since $B^{ab}$ transform 
	covariantly under local Lorentz transformations. The first 
	Bianchi identity~\eqref{eq:bianchidS_expl_h} is trivially 
	satisfied, while the second reduces 
	to~\eqref{eq:bianchidS_expl_p} for $e\ind{^b} \wedge 
	R\ind{_b^a} = e\ind{^b} \wedge B\ind{_b^a} = 0$. It can be 
	seen from these expression that they do not put any further 
	restriction on the given geometry.

	This kind of geometry will be used in formulating de Sitter 
	Teleparallel Gravity.
\end{remark}


%\bibliography{/home/hendrik/research/drafts/references/All.bib}
\bibliographystyle{plain}
\bibliography{../../references/All}

\end{document}
